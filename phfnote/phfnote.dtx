% \iffalse meta-comment
%
% Copyright (C) 2016 by Philippe Faist <philippe.faist@bluewin.ch>
% -------------------------------------------------------
% 
% This file may be distributed and/or modified under the
% conditions of the LaTeX Project Public License, either version 1.3
% of this license or (at your option) any later version.
% The latest version of this license is in:
%
%    http://www.latex-project.org/lppl.txt
%
% and version 1.3 or later is part of all distributions of LaTeX 
% version 2005/12/01 or later.
%
% \fi
%
% \iffalse
%<*driver>
\ProvidesFile{phfnote.dtx}
%</driver>
%<package>\NeedsTeXFormat{LaTeX2e}[2005/12/01]
%<package>\ProvidesPackage{phfnote}
%<*package>
    [2016/08/14 v1.0 phfnote package]
%</package>
%
%<*driver>
\documentclass{ltxdoc}
\usepackage{xcolor}
\usepackage{lipsum}
\usepackage[preset=xpkgdoc]{phfnote}

\usepackage{needspace}

\makeatletter
\newsavebox\phfnoteDocVirtualPage@contents
\newenvironment{phfnoteDocVirtualPage}{%
  \par%
  \begingroup%
  \makeatletter%
  \begin{lrbox}{\phfnoteDocVirtualPage@contents}%
  \begin{minipage}{12cm}\vspace*{0.5cm}\relax%
    \def\rmdefault{cmr}\def\sfdefault{cmbr}\normalfont%
    \def\shortlipsum{Lorem ipsum dolor sit amet, consectetuer
      adipiscing elit. Ut purus elit, vestibulum ut, placerat ac,
      adipiscing vitae, felis. Curabitur dictum gravida
      mauris. Nam arcu libero, nonummy eget, consectetuer id,
      vulputate a, magna.}%
    \def\@title{Notes on Lambda-Majorization}%
    \def\@author{Ph. Faist}%
    \def\@date{23.12.2011}%
    \def\notetitletopspace{0pt}%
    \def\notetitlefont{\sffamily\bfseries}%
    \def\morepagecontents{\par\vspace{1em}\centering\ldots}%
    \ignorespaces%
  }%
  {%
  \end{minipage}%
  \end{lrbox}%
  %%\centering%
  \begin{tcolorbox}[text width=6cm,sharp corners,%
    before={\par\vspace{5pt}\centering\nopagebreak\parindent=0pt},after={\par\vspace{5pt}},%
    leftrule=0.4pt,toprule=0.4pt,rightrule=0.6pt,bottomrule=0.6pt,%
    colframe=black,colback=white]%
    \scalebox{0.5}{\usebox{\phfnoteDocVirtualPage@contents}}%
  \end{tcolorbox}%
  %%\par%
  \endgroup%
}
\makeatother

\def\RevTeX{{\small R\raise-0.2ex\hbox{\textsc{ev}}}\TeX}

\EnableCrossrefs
\CodelineIndex
\RecordChanges

\begin{document}
  \DocInput{phfnote.dtx}
\end{document}
%</driver>
% \fi
%
% \CheckSum{0}
%
% \CharacterTable
%  {Upper-case    \A\B\C\D\E\F\G\H\I\J\K\L\M\N\O\P\Q\R\S\T\U\V\W\X\Y\Z
%   Lower-case    \a\b\c\d\e\f\g\h\i\j\k\l\m\n\o\p\q\r\s\t\u\v\w\x\y\z
%   Digits        \0\1\2\3\4\5\6\7\8\9
%   Exclamation   \!     Double quote  \"     Hash (number) \#
%   Dollar        \$     Percent       \%     Ampersand     \&
%   Acute accent  \'     Left paren    \(     Right paren   \)
%   Asterisk      \*     Plus          \+     Comma         \,
%   Minus         \-     Point         \.     Solidus       \/
%   Colon         \:     Semicolon     \;     Less than     \<
%   Equals        \=     Greater than  \>     Question mark \?
%   Commercial at \@     Left bracket  \[     Backslash     \\
%   Right bracket \]     Circumflex    \^     Underscore    \_
%   Grave accent  \`     Left brace    \{     Vertical bar  \|
%   Right brace   \}     Tilde         \~}
%
%
% \changes{v1.0}{2016/04/20}{Initial version}
%
% \GetFileInfo{phfnote.dtx}
%
% \iffalse Bypass indexing for following commands: \fi
% \DoNotIndex{\newcommand,\newenvironment,\renewcommand,\long,\def,\edef,\gdef,\xdef,\if,\else,\fi,\par,\relax,\vspace,\vskip,\hspace,\hskip,\vbox,\hbox}
% 
% \title{\phfqitltxPkgTitle{phfnote}}
% \author{Philippe Faist\quad\email{philippe.faist@bluewin.ch}}
%
% \maketitle
%
% \begin{abstract}
%   \pkgname{phfnote}---A handy \LaTeX{} class for typesetting short notes and
%   medium-length reports, full of goodies to make it look just right.
% \end{abstract}
%
% \phantomsection\label{sec:toc}
% \inlinetoc
%
% \section{Introduction}
%
% Have you ever thought, ``let me write up these short notes using \LaTeX,'' but
% then disliked the default style of the |article| class?  Have you ever asked
% yourself why half the page should be taken up by the title?  Yes?  Then
% welcome to \pkgname{phfnote}.
%
% The package \pkgname{phfnote} provides basic formatting for short
% documents, such as notes on a specific topic, short documentation, or
% quick memos.  It aims to cover all basic needs for such purposes: include
% a standard set of relevant packages, a nice title which doesn't take up
% too much space, better page margin sizes, and some basic styling to make
% the note look nicer.  At the same time, it is highly configurable so that
% nothing is really unchangeable.  And all overridden features can be
% restored individually to their class-provided defaults.
%
% This package has been designed to work optimally along with the |article|
% document class, but in principle any relatively standard \LaTeX{} class should
% work.  Notes can be typeset in \index{two-column} two-column mode with the
% |twocolumn| option of for example the |article| class.  Settings such as the
% page margins and font goodies are automatically adapted to look best according
% to the standard document font size (10pt, 11pt, or 12pt).
%
% Be aware that this package is not meant as a full-fledged formatting class for
% complicated articles.  For that, you should use a specialized class such as
% \RevTeX.\footnote{See \url{https://journals.aps.org/revtex}}
%
% In the following, we detail individual features of this class, and explain how
% to activate, deactivate, and customize them.
%
%
% \section{Basic Usage}
%
% \subsection{Loading the Package}
%
% You can get started with the minimal template:
%
% \begin{verbatim}
% \documentclass[11pt,a4paper]{article}
% \usepackage{phfnote}
%
% \begin{document}
% \title{Title of my notes}
% \author{Me}
% \date{\today}
% \maketitle
%
% ...
%
% \end{document}
% \end{verbatim}
%
% The package \pkgname{phfnote} introduces its default note formatting
% style, with a more compact title, and some formatting adjustments in the
% text and section headings.
%
% \subsection{Presets}
% \label{sec:presets}
%
% There are a number of package options which can be provided to activate,
% deactivate or adjust the formatting.  The most straightforward way of changing
% the formatting is to use \emph{presets}.
%
% Presets are processed immediately when given in the package option list,
% meaning that their position in the list is meaningful.  For example, the
% option list
% \begin{verbatim}
%   \usepackage[title=small,preset=article,par=skip]{phfnote}
% \end{verbatim}
% will set |title=small| only if it is not overridden by the |article| preset,
% but will enforce |par=skip| in any case.  You may in theory load several
% presets, e.g. |preset=sfnote,preset=article|, but this is essentially useless
% since presets tend to set a wide range of settings such that in any case the
% last preset specified is effectively applied.
%
% First, there is a set of presets which are different alternative ``note''
% styles.  All the following define the note to have spacing between paragraphs
% and no first line indentation, use the default note title style, and use a
% wider page geometry.
%
% \begin{pkgoptions}
% \item[preset=sfnote] {\fontfamily{cmbr}\selectfont Format the note in \LaTeX'
%   sans-serif ``Computer Modern Bright'' font.  This is a nice, light, font for
%   short notes, but I find it more difficult to read at smaller font sizes or
%   in longer paragraphs.}
% \item[preset=sfssnote] {\fontfamily{cmss}\selectfont Format the note in
%   \LaTeX' default sans-serif font.  A very nice sans serif font.  It might
%   look heavy though, depending on your taste.}
% \item[preset=opensansnote] {\fontfamily{fosj}\selectfont Format the note
%   in Open Sans font (using the `\pkgname{opensans}' package with some
%   default options).  A very beautiful and readable sans serif font.}
% \item[preset=utopianote] {\fontfamily{futs}\selectfont Format the note in
%   Utopia font (by using the \pkgname{fourier} package).  Perfect to my
%   taste for documenting code for example, but I find it a bit heavy for
%   scientific documents.}
% \item[preset=mnmynote] Format the note in Minion Pro font, with sans
%   serif text formatted with the Myriad Pro font (professional fonts by
%   Adobe which can be used in \LaTeX{} with the \pkgname{MinionPro} and
%   \pkgname{MyriadPro} packages\footnote{See
%   \url{https://github.com/sebschub/FontPro}; the fonts themselves ship
%   with some Adobe products}).  These beautiful fonts can be used for any
%   purpose.
% \end{pkgoptions}
% 
% Based essentially on |utopianote|, the preset |pkgdoc| sets up the
% document to look nice for a \LaTeX{} package documentation.  The preset
% |xpkgdoc| adds additional definitions to aid in documenting \LaTeX{}
% packages on top of |pkgdoc|.
% \begin{pkgoptions}
% \item[preset=pkgdoc] Basic formatting and settings for documenting \LaTeX{}
%   packages. This preset was used for the current document.
% \item[preset=xpkgdoc] Same as |preset=pkgdoc|, but in addition a set of
%   useful commands are also provided, the \pkgname{tcolorbox} package is
%   loaded along with some default boxes.  Also some commands are patched
%   to achieve some fixes.  This preset is used for the documentation of
%   packages in the \pkgname{phfqitltx} package suite.  (For details see
%   the implementation of |\phfnote@preset@xpkgdoc| below.)
% \end{pkgoptions}
%
%
% The following preset makes the document look more like an article.  There are
% some slight minor differences with respect to the default |article| class'
% title in the choice of formatting the title and text.
%
% \begin{pkgoptions}
% \item[preset=article] Sets a more title style closer to |article|'s default
%   title style (but slightly more compact) and sets paragraphs to indent with
%   no skip.
% \end{pkgoptions}
%
% The last preset, |reset|, guarantees that including this package is
% non-invasive, meaning that only new \LaTeX{} macros are made available without
% altering any appearance.  This is useful if you want to use a small feature
% provided by this package, but you already have all the page geometry, title,
% etc.\@ set up and want to make sure those aren't touched.
%
% \begin{pkgoptions}
% \item[preset=reset] Deactivates all features of this package by default.
%   Individual settings can still later be switched on via specific package
%   options.  Use this to activate only a specific set of features:
%   |[preset=reset,...]| will ensure that only the additional given features are
%   set.
%
%   This is safer than deactivating individually all other features, because in
%   the future we may add new features which may be on by default.  In this
%   case, the preset |reset| will guarantee all features to be deactivated.
% \end{pkgoptions}
%
% \section{Features}
%
% This package provides a large collection of small features, which, put all
% together, make the document look nicer (hopefully).  Let's go through these
% features, one by one.
%
% Note also that some features provided in the presets, such changing the
% document font, are not provided as individual features here.  This is because
% they may be set and customized directly using few lines of \LaTeX{} code or
% directly by including an external package.  In those cases, you may have a
% look at the preset's definition for inspiration (see \autoref{impl:presets}).
%
% For a summary of package options, see \autoref{sec:package-options}.
%
%
% \subsection{Title Formatting}
%
% \subsubsection{Title Styles}
% \label{sec:title-styles}
% \label{sec:main-default-title-style}
%
% The \pkgname{phfnote} package allows a set of alternative title styles.
% By default, the |default| title style is used.  You may change this
% setting with the |title=...| package option.
%
% \begin{pkgoptions}
% \item[title=default] The default title style displays the title in large bold
%   sans serif font, left-aligned.  Below the title appears the information
%   about author and date, indented, followed by a horizontal rule.  It looks
%   like this:
%   \begin{phfnoteDocVirtualPage}
%     \notetitle@style@default\null \shortlipsum \morepagecontents
%   \end{phfnoteDocVirtualPage}
%   As you can see, it saves more space on the page compared to the default
%   article title.
%
% \item[title=small] A smaller title style which displays all the relevant
%   information on a single line.  This is useful for when even the default
%   title style appears too large.  It looks like this:
%   \begin{phfnoteDocVirtualPage}
%     \notetitle@style@small\null \shortlipsum \morepagecontents
%   \end{phfnoteDocVirtualPage}
%
% \item[title=article] Mimics the default title style from the |article| class,
%   but saves a little more space.  It looks like this:
%   \begin{phfnoteDocVirtualPage}
%     \vspace*{3em}\notetitle@style@article\null \shortlipsum \morepagecontents
%   \end{phfnoteDocVirtualPage}
%
% \item[title=] An empty argument to |title| instructs \pkgname{phfnote} not to
%   override any title definition, thus preserving the default class title
%   style.
%
%   Beware that some other title goodies, such as our more advanced |\thanks|
%   notes, or spacing adjustments for the abstract, will probably not work.
% \end{pkgoptions}
%
%
% \subsubsection{Customizing the style of the \phfverb{default} and \phfverb{small} title styles}
% 
% You may customize the appearance of the |default| and |small| title styles by
% overriding some macros.
%
% \needspace{3\baselineskip}
% \DescribeMacro{\notetitlefont}\DescribeMacro{\notetitleauthorfont}
% \DescribeMacro{\notetitledatefont} The macros
% |\notetitlefont|, |\notetitleauthorfont|, and |\notetitledatefont| set the
% default main font title, author text and date text.  You may override these
% settings with, for instance:
% \begin{verbatim}
% \renewcommand{\notetitlefont}{\sffamily\bfseries}
% \renewcommand{\notetitleauthorfont}{\itshape}
% \renewcommand{\notetitledatefont}{\footnotesize}
% \end{verbatim}
% The spacing of the title may be adjusted with the macros
% \DescribeMacro{\notetitlebelowspace} |\notetitlebelowspace| and
% \DescribeMacro{\notetitletopspace} |\notetitletopspace|.  Override these with
% e.g.:
% \begin{verbatim}
% \renewcommand{\notetitlebelowspace}{4mm}
% \renewcommand{\notetitletopspace}{-1.2cm}
% \end{verbatim}
% Finally, you may override the command \DescribeMacro{\notetitlehrule}
% |\notetitlehrule| which draws the rule below the title:
% \begin{verbatim}
% \renewcommand{\notetitlehrule}{\hrule height 0.8pt}
% \end{verbatim}
% 
%
%
% \subsubsection{Title notes: \phfverb\thanks{} and \phfverb\thanksmark}
%
% \DescribeMacro{\thanks}
% Notes in the title can be introduced with the |\thanks| macro.  You may use
% this to specify an e-mail address, an affiliation, or any other more specific
% information.  |\thanks| may appear in all three title, authors and date.
%
% The appearance of this additional information depends on the title style.  In
% the default note title style, such thanks-notes appear directly below the
% title.  For example, with
% {\ttfamily|\|author\{Ph.\@ Faist|\|thanks\{|\|itshape Institute for Theoretical Physics, ETH Zurich\}\}}, you get:
% \begin{phfnoteDocVirtualPage}
%   \author{Ph. Faist\thanks{\itshape Institute for Theoretical Physics, ETH Zurich}}\relax
%   \notetitle@style@default\null \morepagecontents
% \end{phfnoteDocVirtualPage}
% whereas with the other styles, this information is typeset as regular footnotes.
%
% \leavevmode\marginpar{\raggedleft \phfverb\thanks\phfverb{[N]}}\relax
% You may specify an optional argument to |\thanks|, forcing the footnote to a
% specific number (it must be a number).  For example, with
% {\ttfamily|\|author\{Ph.\@ Faist|\|thanks[9]\{|\|itshape Institute for
% Theoretical Physics, ETH Zurich\}\}}, you get:
% \begin{phfnoteDocVirtualPage}
%   \author{Ph. Faist\thanks[9]{\itshape Institute for Theoretical Physics, ETH Zurich}}\relax
%   \notetitle@style@default\null \morepagecontents
% \end{phfnoteDocVirtualPage}
%
% \DescribeMacro{\thanksmark} |\thanksmark[N]| works with |\thanks| as
% |\footnotemark| works with |\footnote|.  It just displays the given number as
% a footnote mark.  In this way, you can have for example several shared
% affiliations:
% \begin{phfnoteDocVirtualPage}
%   \title{Notes about Stuff}
%   \date{25.12.2015}
%   \author{First Author\thanks[1]{\itshape Institute ABC}, Second Author\thanks[2]{\itshape Somewhere else},
%      and Third Author\thanksmark[1]}\relax
%   \notetitle@style@default\null \morepagecontents
% \end{phfnoteDocVirtualPage}
% the author code was:
% \begin{verbatim}
% \author{First Author\thanks[1]{\itshape Institute ABC},
%     Second Author\thanks[2]{\itshape Somewhere else},
%     and Third Author\thanksmark[1]}
% \end{verbatim}
%
% Unfortunately, you still have to provide the numbering manually.  On the other
% hand, this package is not meant to replace \RevTeX, so if you're writing a
% complicated article with many authors and affiliations, you probably shouldn't
% be using \pkgname{phfnote} in the first place.
%
% \begin{pkgwarning}
%   The optional argument to |\thanks|, as well as the command |\thanksmark|,
%   are not made available if you don't use one of |\phfnote|'s title styles.
%
%   This behavior is such as to prevent interference with more advanced class
%   mechanisms, such as \RevTeX's.
% \end{pkgwarning}
%
%
% \subsection{Abstract}
% \label{sec:abstract-attributes}
%
% \DescribeEnv{abstract} The |abstract| environment renders indented text aimed
% to provide a short summary of the document.  We might use, for example, the
% following code:
% \begin{verbatim}
% \begin{abstract}
% Lorem ipsum dolor sit amet, consectetuer adipiscing elit. Ut purus
% elit, vestibulum ut, placerat ac, adipiscing vitae, felis.
% Curabitur dictum gravida mauris. Nam arcu libero, nonummy eget,
% consectetuer id, vulputate a, magna.
% \end{abstract}
% \end{verbatim}
%
% which would look like this:
% \begin{phfnoteDocVirtualPage}
%   \title{Notes about Stuff}\relax
%   \date{25.12.2015}\relax
%   \author{Me}\relax
%   \notetitle@style@default
%   \def\noteabstract@nameline{{\parskip=0pt\relax\par\centering\noteabstractnamefont\abstractname\par}\vskip 1ex\relax}%
%   \def\noteabstracttextwidth{0.8\textwidth}%
%   \begin{abstract}
% Lorem ipsum dolor sit amet, consectetuer adipiscing elit. Ut purus
% elit, vestibulum ut, placerat ac, adipiscing vitae, felis.
% Curabitur dictum gravida mauris. Nam arcu libero, nonummy eget,
% consectetuer id, vulputate a, magna.
%   \end{abstract}
%   \morepagecontents
% \end{phfnoteDocVirtualPage}
%
% The |abstract| environment should be given \emph{after} the |\maketitle|
% command.  (In contrast to, e.g., \RevTeX.)
%
% You may customize the appearance of the abstract via a list of attributes
% given as argument to a package option.  When you combine arguments, make sure
% to put them in a braced group: |[abstract={wide,noname,it}]|.
%
% \begin{pkgoptions}
% \item[abstract=\pkgoptattrib{wide}] The abstract should not be indented, and
%   should instead be aligned to the rest of the text.
% \item[abstract=\pkgoptattrib{narrow}] The abstract should be indented narrower
%   then by default.
% \item[abstract=\pkgoptattrib{noname}] The title ``Abstract.'' above the text
%   will not be typeset.  The abstract text is typeset directly instead.
% \item[abstract=\pkgoptattrib{original}] Revert to the class' default
%   implementation of the |abstract| environment.  The class' implementation is
%   restored and no longer tampered with.
% \item[abstract=\pkgoptattrib{small}] Use a smaller font for the abstract text
%   (|\small| font).
% \item[abstract=\pkgoptattrib{compact}] Reduce spacing before and after the
%   abstract.  If the abstract is short, this might look slightly better.
% \item[abstract=\pkgoptattrib{it}] Typeset the abstract text using an italic
%   typeface.
% \end{pkgoptions}
%
% 
% The abstract environment's appearance can be customized more finely
% by redefining some macros.  (In fact, this is what the package
% options |abstract=...| actually do.)  The font used for the text of
% the abstract is set by \DescribeMacro{\noteabstracttextfont}
% |\noteabstracttextfont|.  This macro should expand to font selection
% commands, such as |\itshape|, |\bfseries|, |\small|, etc.  The title
% of the abstract (the word ``Abstract.\@'') is typeset in the font set
% by \DescribeMacro\noteabstractnamefont |\noteabstractnamefont|.  The
% width of the whole abstract text is determined by
% \DescribeMacro{\noteabstracttextwidth} |\noteabstracttextwidth|.
% Observe that |\noteabstracttextwidth| is a macro, and not a proper
% length, so that it can determine more dynamically the length.  The
% spacing below \DescribeMacro{\noteabstractafterspacing}
% (|\noteabstractafterspacing|) and above
% \DescribeMacro{\noteabstractbeforespacing}
% (|\noteabstractbeforespacing|) the abstract can further be
% specified, also as macros.
%
% \subsection{Table of Contents}
% \label{sec:inline-toc}
%
% \DescribeMacro{\inlinetoc} The package \pkgname{phfnote} also provides a
% table of contents typeset with reduced spacing to be more compact, and
% with horizontal rules before and after.  You can insert the table of
% contents with the command |\inlinetoc|.  It looks like this:
%
% \begin{phfnoteDocVirtualPage}
%   \title{Notes about Stuff}\relax
%   \date{25.12.2015}\relax
%   \author{Me}\relax
%   \notetitle@style@default
%   \def\noteabstract@nameline{{\parskip=0pt\relax\par\centering\noteabstractnamefont\abstractname\par}\vskip 1ex\relax}%
%   \def\noteabstracttextwidth{0.8\textwidth}%
%   \begin{abstract}
% Lorem ipsum dolor sit amet, consectetuer adipiscing elit. Ut purus
% elit, vestibulum ut, placerat ac, adipiscing vitae, felis.
%   \end{abstract}
%   \def\@starttoc#1{\relax
%     \contentsline {section}{\numberline {1}Introduction}{2}{section.1}
%     \contentsline {section}{\numberline {2}Basic Usage}{3}{section.2}
%     \contentsline {subsection}{\numberline {2.1}Loading the Package}{3}{subsection.2.1}
%     \contentsline {subsection}{\numberline {2.2}Presets}{3}{subsection.2.2}
%   }\relax
%   \inlinetoc
%   \morepagecontents
% \end{phfnoteDocVirtualPage}
%
%
% \subsection{Predefined Package Sets}
% \label{sec:package-sets}
%
% The \pkgname{phfnote} package also provides sets of standard \LaTeX{}
% packages to load.  You may choose between a varying degree of
% ``richness'' of packages included.
%
% \begin{pkgoptions}
% \item[pkgset=none] Do not include any package set.
%
% \item[pkgset=minimal] Include some basic minimal set useful for
%   scientific notes: the \AmS{} packages \pkgname{amsmath},
%   \pkgname{amssymb}, \pkgname{amsfonts}, and \pkgname{amsthm}.  The
%   \pkgname{xcolor} package is also loaded.
%
% \item[pkgset=rich] Include a fair amount of packages which may be useful.
%   On top of the |minimal| package set, this set includes the packages
%   \pkgname{enumitem}, \pkgname{graphicx}, \pkgname{microtype},
%   \pkgname{caption}, \pkgname{setspace}, as well as \pkgname{inputenc}
%   with the |utf8| option and \pkgname{fontenc} with the |T1| option.
%
%   This package set is loaded by default.
%
% \item[pkgset=extended] Additionally, include packages \pkgname{float},
%   \pkgname{verbdef}, \pkgname{csquotes}, \pkgname{dsfont}, \pkgname{bbm}
%   and \pkgname{mathtools}.
%
% \end{pkgoptions}
%
%
% \subsection{Page Geometry}
% \label{sec:pagegeomdefs}
%
% Another important aspect of \pkgname{phfnote} is the handling of page
% margins.  Often the default page margins of the \pkgname{article} class
% are quite narrow.  While it is a good typographical practice to avoid
% long lines, on occasion we prefer to have notes typeset with wider text.
% The general answer is the \pkgname{geometry} package, which allows to set all
% margins in full detail.
%
% The \pkgname{phfnote} package provides some standard choices of options for the
% \pkgname{geometry} package, which are adjusted according to the document font size,
% and whether the document is typeset in two columns.
%
% If you want anything more complicated than what is provided by a default
% setting here, just use |pagegeomdefs=false| and invoke the \pkgname{geometry} package
% directly with your preferred set of options.
%
% \begin{pkgoptions}
% \item[pagegeomdefs=true] Include the \pkgname{geometry} package, using the default
%   settings or whatever is specified with the |pagegeom| option.
% \item[pagegeomdefs=false] Do not attempt to change the document margins, and
%   don't load the \pkgname{geometry} package.
% \item[nopagegeomdefs] Same as |pagegeomdefs=false|.
% \end{pkgoptions}
%
% The page geometry predefined settings are the following.
%
% \begin{pkgoptions}
% \item[pagegeom=default] Default settings. Not too wide, not too
%   narrow. Settings vary according to single or double column setting, and
%   according to default font point size.
% \item[pagegeom=narrow] Narrower style.  For single-column documents, this is
%   closer to the typographically-advertised-optimal of 50--80 characters per
%   line, but it might look narrow to some.
% \item[pagegeom=wide] Wide, comfortable style.  Wastes less paper.
% \item[pagegeom=xwide] Extra wide.  Use if you pity trees.
% \item[pagegeom=bigmargin] Makes the margins asymmetric, so that a wide margin
%   note can fit.  This style is used in this package documentation, for
%   example.
% \end{pkgoptions}
% 
%
% \subsection{Section Headers Styling}
% \label{sec:secfmt}
%
% The \pkgname{phfnote} package provides some limited styling of section headers.  The
% font, size and ``compactness'' of the headers can be adjusted with title options.
% But really, these options are quite basic.  You should use \pkgname{titlesec} or
% \pkgname{sectsty} directly if you want anything serious.
%
% The section headings are customized using the \pkgname{sectsty} package.  If this
% conflicts in your document, then use the |[secfmt={}]| package option to
% indicate that section headings should NOT be styled by this package.  Then take
% care of section styling manually.
%
% Package options may be used to customize the appearance of the section
% headings by specifying a list of attributes.  When you combine arguments, make
% sure to put them in a braced group: |[secfmt={section,compact}]|.  Beware that
% attributes are not merged between different occurrences of the |secfmt|
% keyword in the package options; the last occurrence defines all set
% attributes.  If the |secfmt| package option is not given, then by default only
% the |section| attribute is set.
%
% \begin{pkgnote}
%   Don't forget to include the attribute `|section|' and/or `|paragraph|'
%   depending on which type of heading you want your settings to apply to.  For
%   example, \pkgoptionfmt{secfmt=\pkgoptattribnodots{sffamily}} has no effect,
%   you need to use e.g.\@
%   \pkgoptionfmt{secfmt=\pkgoptattribnodots{section,sffamily}}.
% \end{pkgnote}
% 
% Available attributes are the following:
%
% \begin{pkgoptions}
% \item[secfmt=\pkgoptattrib{section}] Use the |section| attribute to activate
%   the styling of section-level headings, that is, |\section|, |\subsection|
%   and |\subsubsection|.
% \item[secfmt=\pkgoptattrib{paragraph}] This attribute indicates that the
%   styling should apply to paragraph-level headings as well (|\paragraph| and
%   |\subparagraph|).
% \item[secfmt=\pkgoptattrib{compact}] Reduce the sizes of the section headings
%   (if the section-level headings are styled, i.e.\@ you need to specify the
%   |section| attribute), giving the document a more ``compact'' appearance.
% \item[secfmt=\pkgoptattrib{larger}] Increase the sizes of the section
%   headings.  Suitable for longer documents or for small document font sizes.
% \item[secfmt=\pkgoptattrib{secsquares}] Display black squares on the left side
%   of |\section|-level commands, making them stand out better.  This is useful
%   for documents (such as the present one) with several layers of sub-sections.
% \item[secfmt=\pkgoptattrib{secnummargin}] Display the section, subsection, and
%   subsubsection numbering in the left margin and have the title occupy the
%   full width of the text (such as for this document).  If you want both
%   \pkgoptionfmt{secsquares} and \pkgoptionfmt{secnummargin}, you must specify
%   them in that order, or the black square may end up overlapping with the
%   number.
% \item[secfmt=\pkgoptattrib{rmfamily}] Typeset headings in the regular
%   roman font of the document, instead of trying to apply the
%   {\fontfamily{ppl}\fontseries{bx}\selectfont Palatino font}.  This
%   applies to section-level and/or paragraph-level headings, depending on
%   which of the attributes |section| and/or |paragraph| have been
%   specified.
% \item[secfmt=\pkgoptattrib{sffamily}] Typeset headings in a sans-serif font.
%   The default document sans serif font is used.  This applies to section-level
%   and/or paragraph-level headings, depending on which of the attributes
%   |section| and/or |paragraph| have been specified.
% \item[secfmt=\pkgoptattrib{itpar}] Typeset paragraph-level headings in italic.
% \item[secfmt=\pkgoptattrib{blockpar}] Change the paragraph-level headings not
%   to be in ``run-in'' style, but to be typeset on their own line like section
%   headings.
% \item[secfmt=\pkgoptattribempty] Leave the argument empty to keep the original class styling;
%   nothing will be overridden and the \pkgname{sectsty} package is not loaded.
% \end{pkgoptions}
%
% You can also directly modify the section heading style by redefining some
% macros.  Note that these macros only affect those sectioning commands which we
% have decided to style, which is specified by the |section| and |paragraph|
% attributes to be specified in the |secfmt={...}| package option.
%
% \DescribeMacro{\notesectionallfont} The macro |\notesectionallfont| is invoked
% for every sectioning command (for those which are styled, see the |section|
% and |paragraph| attributes). The macro |\notesectionallfont| internally
% invokes \DescribeMacro{\notesectionallfontfamily} |\notesectionallfontfamily|
% to select which font family to use.  The family should be given as the font
% code, e.g.: |pbk| = {\fontfamily{pbk}\selectfont Bookman}; |bch| =
% {\fontfamily{bch}\selectfont Charter}; |ppl| = {\fontfamily{ppl}\selectfont
% Palatino}; |ptm| = {\fontfamily{ptm}\selectfont Adobe Times}; |phv| =
% {\fontfamily{phv}\selectfont Adobe Helvetica}; |pcr| =
% {\fontfamily{pcr}\selectfont Adobe Courier}; |put| =
% {\fontfamily{put}\selectfont Utopia}; |cmr| = {\fontfamily{cmr}\selectfont
% Computer Modern Roman}; |cmss| = {\fontfamily{cmss}\selectfont CM Sans Serif};
% |cmbr| = {\fontfamily{cmbr}\selectfont CM Bright}; google many more or look
% directly into the source of corresponding \LaTeX{} packages.
%
% You may customize these either via attributes or by redefining them directly.
% Beware that if you redefine |\notesectionallfont| then you are responsible for
% honoring, or ignoring, the value of |\notesectionallfontfamily|.
%
% \DescribeMacro{\notesectionfont} \DescribeMacro{\notesubsectionfont}
% \DescribeMacro{\notesubsubsectionfont} \DescribeMacro{\noteparagraphfont}
% \DescribeMacro{\notesubparagraphfont} These macros define the font commands to
% apply for the section heading corresponding to the given sectioning command.
% This macro is invoked after |\notesectionallfont|, which means that font
% definitions in these macros take precedence over those in
% |\notesectionallfont|.
%
% \DescribeMacro{\notesectionsetfonts} The macro |\notesectionsetfonts| is a
% shorthand to set all section font definitions for the section-level commands
% |\section|, |\subsection|, and |\subsubsection|.  For example,
% \begin{verbatim}
%   \notesectionsetfonts{\Large}{\large}{\normalsize}
% \end{verbatim}
% will set the font sizes for |\section|, |\subsection| and |\subsubsection| in
% this order.
%
% \DescribeMacro{\noteparagraphsetfonts} The macro |\noteparagraphsetfonts| is
% the corresponding shorthand for the paragraph-level commands.  It takes two
% arguments, the font definitions to apply for headings of level |\paragraph|
% and |\subparagraph|.
%
% \subsection{Appearance of Paragraphs}
% \label{sec:par-defs}
%
% Several presets may be set to define the appearance of paragraphs.  
%
% \begin{pkgoptions}
% \item[par=indent] Paragraphs are indented, bearing some
%   similarity to the \pkgname{article} class' default paragraph style.
% \item[par=skip] Paragraphs are separated by additional spacing,
%   and not indented.
% \item[par=indentminiskip] Paragraphs are indented, but there is also a small
%   space between each paragraph.
% \item[par=original] Do not modify the appearance of paragraphs, and leave the
%   class default.
% \end{pkgoptions}
%
% \subsection{Adjusting Spacing of Lines and Words}
% \label{sec:spacingdefs}
%
% The \pkgname{phfnote} package also provides definitions to adjust spacing of lines and
% words.
%
% This includes definitions to avoid overflowing words in the margin in case of
% long words.
% 
% \begin{pkgoptions}
% \item[spacingdefs=true] Apply adjustments to line and word spacing.
% \item[spacingdefs=false] Do not attempt any adjustments of line or word
%   spacing.
% \item[nospacingdefs] Alias for |spacingdefs=false|.
% \end{pkgoptions}
%
%
% \subsection{Adjustments for Fonts}
% \label{sec:fontdefs}
%
% The \pkgname{phfnote} package provides as well some adjustments for fonts to make some
% fonts look nicer.
%
% Concretely, the {\fontfamily{cmbr}\selectfont Computer Modern Bright} font is
% used as sans serif font instead of {\fontfamily{cmss}\selectfont \LaTeX's
% default sans serif font}, and the more universal |T1| font encoding is used
% instead of the default |OT1|.
%
% \begin{pkgoptions}
% \item[fontdefs=true] Apply adjustments to fonts.
% \item[fontdefs=false] Do not apply adjustments to fonts.
% \item[nofontdefs] Same as |fontdefs=false|.
% \end{pkgoptions}
%
% \subsection{Footnote Figure Style}
% \label{sec:footnotedefs}
%
% The footnotes' appearance can also be slightly enhanced.
%
% \begin{pkgoptions}
% \item[footnotedefs=true] Changes the symbol appearance a little bit---the
%   footnote number is smaller and typeset in boldface.
% \item[footnotedefs=false] Do not change the footnote appearance.
% \item[nofootnotedefs] Same as |footnotedefs=false|.
% \end{pkgoptions}
% 
%
% \subsection{Hyperref Loading}
% \label{sec:hyperrefdefs}
%
% There are many options for setting up the \pkgname{hyperref} package, and often, the
% defaults (with boxed links) are pretty ugly in my opinion.  Enable the
% |hyperrefdefs| feature of \pkgname{phfnote} to alter the defaults to something I
% personally like better (dark blue links as in this document).
%
% \begin{pkgoptions}
% \item[hyperrefdefs=true] Load the \pkgname{hyperref} package, and set some sensible
%   settings.  Also ensures the |\email| and |\url| commands are made available.
% \item[hyperrefdefs=false] Do not load the \pkgname{hyperref} package, do not set
%   sensible settings.
% \item[nohyperrefdefs] Same as |hyperrefdefs=false|.
% \end{pkgoptions}
%
% \DescribeMacro{\url} In order to typeset URLs, the |\url| command is made
% available from the package \pkgname{url} (which is then linkified by \pkgname{hyperref}).  For
% example, you can type |\url{https://github.com/phfaist/}|.
%
% \DescribeMacro{\email} A similar command allows to typeset e-mail addresses.
% The text is displayed as a hyperlink, which when clicked opens a e-mail
% composer to that address (via a |mailto:XXX| link).  For example, try
% |\email{pulp_fiction@tarantino.com}|.
%
% \DescribeMacro{\phfnotePdfLinkColor} The command |\phfnotePdfLinkColor| may by
% used to set the color of the links.  It takes one argument, a color
% specification understood by the \pkgname{xcolor} package.  For example:
% \begin{verbatim}
%   \phfnotePdfLinkColor{green!50!black}
% \end{verbatim}
%
% \begin{pkgnote}
%   The package \pkgname{xcolor} must be loaded for |\phfnotePdfLinkColor|
%   to work.  (The \pkgname{xcolor} package is automatically loaded as part
%   of a package set as long as you're not using the option |pkgset=none|;
%   see \autoref{sec:package-sets}.)
% \end{pkgnote}
%
%
% \subsection{Bibliography Definitions}
% \label{sec:bibliographydefs}
%
% This package also provides some definitions for the bibliography.
%
% It sets the |naturemagdoi| style by default, which is a hacked (by yours
% truly) version of the |naturemag| style to include the journal name as a
% hyperlink (as in APS bibliography styles).
%
% The bibliography is also typeset in a smaller font.
%
% Finally, an entry in the table of contents is generated.
%
% \begin{pkgoptions}
% \item[bibliographydefs=true] Load the \pkgname{hyperref} package, and set some sensible
%   settings.  Also ensures the |\email| and |\url| commands are made available.
% \item[bibliographydefs=false] Do not load the \pkgname{hyperref} package, do not set
%   sensible settings.
% \item[nobibliographydefs] Same as |bibliographydefs=false|.
% \end{pkgoptions}
%
% \DescribeMacro{\bibliography} \DescribeMacro{\bibliographystyle} The
% |\bibliographystyle| and |\bibliography| macros can be used as usual, for
% example:
% \begin{verbatim}
%   \bibliographystyle{apsrmp4-1} % optional
%   \bibliography{mybibfile}
% \end{verbatim}
% bearing in mind that if the |\bibliographystyle| command is not present, our
% custom |naturemagdoi| bibliography style is used.
%
%
% \subsection{URL Styles}
% \label{sec:url-styles}
%
% As a bonus, the \pkgname{phfnote} package provides an alternative set of URL styles to
% use with the |\url| and |\email| commands (see \autoref{sec:hyperrefdefs}).
%
% All the styles described below typeset the URL in a slightly smaller size, so
% as to avoid a common issue with URLs that they tend to appear too large.
% Also, the tilde character is fixed so that it appears nicely, as in:\\
% \url{https://people.phys.ethz.ch/~pfaist/}.
%
% The URL style can be set with the command |\urlstyle|\marg{name of style}.
%
% \begin{description}[font=\ttfamily,labelwidth=8em]
% \item[notett] typewriter font
% \item[notesf] default sans serif font
% \item[notesfss] Computer Modern Sans Serif font
% \item[noteitsf] italic using default sans serif font 
% \item[noterm] normal roman typeface
% \item[noteit] just italic typeface
% \item[notesml] just smaller than surrounding text
% \end{description}
%
%
% \subsection{A \phfverb{\notesmaller} Command}
%
% This general-purpose command is handy to typeset text smaller than its
% surrounding text, for when you don't know what size the surrounding text is
% typeset at.  In some sense, this is a very very lightweight analogue of what
% the \pkgname{relsize} package does.  (This is used, for example, in our implementation
% of URL styles introduced in \autoref{sec:url-styles}.)
%
% \DescribeMacro{\notesmaller}\DescribeMacro{\notesmaller[0.8]} Set the font
% size to a fraction of the surrounding font size.  The fraction may be
% specified as an optional argument.  A fraction of 0.8 makes the text size 0.8
% times that of the surrounding text, that is, smaller than the surrounding
% text.  A value of 1 does not change the font size.  If the fraction is not
% specified, the value stored in |\notesmallerfrac| is used.
%
% \DescribeMacro{\notesmallerfrac} The fraction by which |\notesmaller| typesets
% smaller text when no optional argument is given.  You may redefine this
% command to set the default ``smaller'' size fraction.
%
% \subsection{Tools Mostly for Hackers}
%
% The \pkgname{phfnote} package also provides some small hacks.  They are documented
% further in \autoref{sec:impl-other-stand-alone-defs}.  These are:
% \DescribeMacro{\phfnoteHackSectionStarWithTOC} a macro
% |\phfnoteHackSectionStarWithTOC| to hack into a command which generates a
% |\section*|, in order for that command to also generate a corresponding entry
% in the table of contents; and
% \DescribeMacro{\phfnoteSaveDefs}\DescribeMacro{\phfnoteRestoreDefs} a pair of
% commands to save and restore \LaTeX{} definitions.
%
% 
%
%
% \section{Summary of Package Options}
% \label{sec:package-options}
%
% \begin{pkgoptions}
% \item[preset=\meta{preset name}] Load a preset specifying a predefined set of
%   options for the general appearance of the document.  See documentation in
%   \autoref{sec:presets}
% \item[title=\meta{title style}] Set the title style.  Documentation in
%   \autoref{sec:title-styles}
% \item[abstract=\meta{abstract attributes}] Set the abstract style by
%   specifying a comma-separated list of attributes.  Don't forget to
%   put the list of attributes within braces,
%   |[abstract={wide,noname,it}]|.  Documentation in
%   \autoref{sec:abstract-attributes}
% \item[pkgset=\meta{package set}] Specify a standard set of \LaTeX{}
%   packages to load.  See \autoref{sec:package-sets}.
% \item[pagegeomdefs=\meta{true or false}] Whether to care about page
%   margins. |nopagegeomdefs| is synonym for |pagegeomdefs=false|.
% \item[pagegeom=\meta{geom style}] Set a page margin style.  Only has effect if
%   |pagegeomdefs=true|.  Options are documented in \autoref{sec:pagegeomdefs}.
% \item[secfmt=\meta{section formatting attributes}] A list of attributes
%   defining how section (and possibly paragraph) headings should look like.
%   See \autoref{sec:secfmt}.
% \item[par=\meta{par style}] Define how paragraphs should be spaced.  Refer to
%   \autoref{sec:par-defs}.
% \item[spacingdefs=\meta{true or false}] Adjust spacing of lines and words
%   (\autoref{sec:spacingdefs}).
% \item[fontdefs=\meta{true or false}] Adjust some fonts
%   (\autoref{sec:fontdefs}).
% \item[footnotedefs=\meta{true or false}] Adjust slighly the appearance of
%   footnotes.  See \autoref{sec:footnotedefs}.
% \item[hyperrefdefs=\meta{true or false}] Load the \pkgname{hyperref} package, and set
%   some defaults settings.  See \autoref{sec:hyperrefdefs}.
% \item[bibliographydefs=\meta{true or false}] Adjust the appearance and style
%   of the bibliography.  See \autoref{sec:bibliographydefs}.
% \end{pkgoptions}
%
% \begin{pkgtip}
%   To activate only a subset of features, use \pkgoptionfmt{preset=reset} and then
%   enable only the features required.  In this way, you can ensure that only those
%   features which are explicitly specified are enabled.
% \end{pkgtip}
%
%
% \StopEventually{\PrintChangesAndIndex}
%
% \section{Implementation}
%
% Here comes the gory code.
%
% Let's start by loading the \pkgname{kvoptions} package, which we need to
% parse the package options.  It's better to use \pkgname{xkeyval} as
% backend, because the |\setkeys| by \pkgname{keyval} is a little fragile:
% for example, it gets confused if, within a preset, we include a package
% or run a command which itself parses key-vals.
%
%    \begin{macrocode}
\RequirePackage{xkeyval}
\RequirePackage{kvoptions}
%    \end{macrocode}
%
% Also load \pkgname{etoolbox}, for various utilities.
%    \begin{macrocode}
\RequirePackage{etoolbox}
%    \end{macrocode}
%
% \subsection{Internal Generic Code}
%
% \begin{macro}{\phfnote@internal@execattribbs}
%   An internal general-purpose macro to execute all definitions given in list of
%   attributes.
%
%   Often, a list of attributes are given via a package option (e.g.\@ for the abstract),
%   and these attributes need to be executed, or implemented, in the order they are given.
%   This macro takes care of that.  Each possible attribute must be defined as a macro
%   with a common prefix, to which the attribute is appended.
%
%   The arguments are:
%   \begin{itemize}
%   \item |#1| = prefix to look for attributes (e.g.\@ |noteabstract@attr@|);
%   \item |#2| = a human-readable name of what |#1| represents, which is used in an error
%     message in case the required attribute is not found (e.g.\@ |{abstract attribute}|);
%   \item |#3| = the list of attributes specified by the user.
%   \end{itemize}
% 
%   For example, |\phfnote@internal@execattribs{noteabstract@attr@}|\hskip0pt\relax
%   |{abstract attribute}|\hskip0pt\relax |{noname,small}| causes the commands
%   |\noteabstract@attr@noname| and |\noteabstract@attr@small| to be invoked, in this
%   order.
%
%    \begin{macrocode}
\def\phfnote@internal@execattribs#1#2#3{%
  \@for\next:=#3\do{%
    \ifcsname #1\next\endcsname%
      \csname #1\next\endcsname%
    \else%
      \PackageWarning{phfnote}{Unknown #2: '\next'. Ignoring.}
    \fi
  }
}
%    \end{macrocode}
% \end{macro}
% 
% \subsection{Title Styling}
%
% See \autoref{sec:title-styles} for a description of the styles and which
% features are available.
%
% \subsubsection{First, some common simple definitions for our different styles}
%
% \needspace{3\baselineskip}
% \begin{macro}{\notetitlefont}
% \begin{macro}{\notetitleauthorfont}
% \begin{macro}{\notetitledatefont}
%   These may be redefined to adapt the font of the title, author and date.
%
%    \begin{macrocode}
\newcommand{\notetitlefont}{\sffamily\bfseries}
\newcommand{\notetitleauthorfont}{}
\newcommand{\notetitledatefont}{\footnotesize}
%    \end{macrocode}
% \end{macro}
% \end{macro}
% \end{macro}
% 
% \begin{macro}{\notetitlebelowspace}
% \begin{macro}{\notetitletopspace}
%   These macros may be redefined to adjust spacing above and after the title.  They are
%   macros, not lengths, so they can be adjusted dynamically on the spot.
%    \begin{macrocode}
\newcommand{\notetitlebelowspace}{4mm}
\newcommand{\notetitletopspace}{-1.2cm}
%    \end{macrocode}
% \end{macro}
% \end{macro}
%
% \begin{macro}{\notetitlehrule}
%   Allow customization of the horizontal rule below the title.  The macro
%   |\notetitlehrule| expands to commands which generate the rule, such as
%   ``|\hrule height 1pt|''.
%    \begin{macrocode}
\newcommand{\notetitlehrule}{\hrule}
%    \end{macrocode}
% \end{macro}
% 
% \begin{macro}{\notetitle@title}
%   Provide a ``long'' definition for |\title|, so that the title can have several
%   paragraphs.  Our style handles this by putting the title on several lines, and it can
%   be useful depending on how you want to format the title.
%
%   This macro will replace |\title| when a title style is actually selected in
%   |\phfnote@do@notetitle|.
%    \begin{macrocode}
\long\def\notetitle@title#1{\long\gdef\@title{#1}}
%    \end{macrocode}
% \end{macro}
% 
% \begin{macro}{\phfnote@title@checksetspace}
%   Some of our title styles require the \pkgname{setspace} package.  This utility checks that
%   this package is loaded, and generates an error otherwise.
%
%   |#1| = the current title style name; this is required only for the error message.
%    \begin{macrocode}
\def\phfnote@title@checksetspace#1{%
  \ifdefined\singlespace\else%
    \PackageError{phfnote}{Note title style `#1' requires the
      `setspace' package to be loaded!  Please load it, or use a
      pkgset which loads it automatically}%
  \fi%
}
%    \end{macrocode}
% \end{macro}
%
% \subsubsection{Implementation of \phfverb\thanks{} and \phfverb\thanksmark}
%
% Here we provide a few fixes for the implementation of |\thanks|, both for our main
% `default' title style as well as for other simpler styles.  Our implementation supports
% |\thanks[N]{...}| and |\thanksmark[N]| as for footnotes.
%
% These newer implementations are only applied if one of our title styles is set.
% Otherwise, the class defaults are left (which may be needed, e.g., for \RevTeX).
%
% \paragraph{Implementation of \phfverb{\thanks} and friends for our main
% `default' title style}
%
% \begin{macro}{\phfnote@setupthanksmpfootnote}
%   Internal---called at the beginning of a |minipage| environment, it sets up necessary
%   stuff to support |\thanks| notes within the minipage, in a single paragraph.
%
%   Some of this code was taken or really inspired directly from |latex.ltx|.
%    \begin{macrocode}
\def\phfnote@setupthanksmpfootnote{%
%    \end{macrocode}
% 
% The |\thanks| macro is implemented as a |\footnote| in a minipage.  So we hack into the
% `mpfootnote' mechanism.
%    \begin{macrocode}
  \def\thempfootnote{\arabic{mpfootnote}}%
  \let\footnoterule\relax%
  \let\thanks\footnote%
%    \end{macrocode}
%
% All footnote material is stored in a macro |\phfnote@mpfootmaterial|, initially
% empty:\footnote{NOTE: this differs from how footnotes are usually treated (directly
% typeset into a vbox I think).  Not sure what the side-effects might be.  Because this is
% just for simple email/institute info/etc. in the title, hopefully this shouldn't have
% any serious consequences.}
%    \begin{macrocode}
  \def\phfnote@mpfootmaterial{}%
%    \end{macrocode}
% and locally define |\@mpfootnotetext| to store the footnote content into that buffer,
%    \begin{macrocode}
  \long\def\@mpfootnotetext##1{%
    \protected@edef\@currentlabel%
         {\csname p@mpfootnote\endcsname\@thefnmark}%
    \protected@edef\@tmpa{\protect\phfnote@mympfootnotemark{\@thefnmark}{##1}%
      \protect\phfnote@mpfootnoteglue}%
    \expandafter\g@addto@macro\expandafter\phfnote@mpfootmaterial%
      \expandafter{\@tmpa}%
  }%
%    \end{macrocode}
% 
% Also provide |\thanksmark|, so that we can refer to other thanks/footnote-marks.
%    \begin{macrocode}
  \def\thanksmark[##1]{\phfnote@mympfootnotemark{##1}}%
}
%    \end{macrocode}
% \end{macro}
%
% \begin{macro}{\phfnote@finalizempfootnotes}
%   Macro to call at the end of a |minipage| environment, to ensure that all |\footnote|'s
%   (and thus |\thanks|'s) are properly formatted.
%
%   This simply takes all the tokens collected in |\phfnote@mpfootmaterial| (see just
%   above), and typesets it in the |\@mpfootins| box.  The latter is automatically typeset
%   by the minipage in |\end{minipage}|.
%  \begin{macrocode}
\def\phfnote@finalizempfootnotes{%
  \global\setbox\@mpfootins=\vbox{%
    \parskip=0pt\parindent=0pt\parshape 1 0.04\textwidth 0.96\textwidth\relax%
    \noindent\leavevmode%
    \reset@font\footnotesize%
    \phfnote@fmt@titlefootnotes%
    \phfnote@mpfootmaterial}%
}
%    \end{macrocode}
% \end{macro}
%
% \needspace{3\baselineskip}
% \begin{macro}{\phfnote@fmt@titlefootnotes}
% \begin{macro}{\phfnote@mympfootnotemark}
% \begin{macro}{\phfnote@mpfootnoteglue}
%   Some formatting utilities which can be overridden if you know what you're doing.
%   |\phfnote@fmt@titlefootnotes| allows you to override the font in which the
%   title-footnotes/thanks are typeset.  |\phfnote@mympfootnotemark| is responsible for
%   formatting its argument as a footnote mark, usually in superscript.
%   |\phfnote@mpfootnoteglue| is the glue which is used between two footnote texts (as
%   they are typeset in a single paragraph).
%    \begin{macrocode}
\def\phfnote@fmt@titlefootnotes{}
\def\phfnote@mympfootnotemark#1{\@textsuperscript{\normalfont#1}}
\def\phfnote@mpfootnoteglue{\hskip 1.2em plus 2em minus 0.5em\relax}
%    \end{macrocode}
% \end{macro}
% \end{macro}
% \end{macro}
% 
%
% \paragraph{For those not using the main `default' title style}
%
% We use \LaTeX's own |\thanks| mechanism, however we patch on the possibility for using
% |\thanks[N]{text}| and |\thanksmark[N]| for overriding the number which is used.
%
% \begin{macro}{\notetitle@thanksmark}
%   The |\thanksmark| is trivially implemented by |\footnotemark|.  Very handy indeed.
%
%   Again, this macro is only made available as |\thanksmark| when a title style is set in
%   |\phfnote@do@notetitle|.
%    \begin{macrocode}
\def\notetitle@thanksmark{\footnotemark}
%    \end{macrocode}
% \end{macro}
% 
% Start by saving the old |\thanks| macro, just in case.
%    \begin{macrocode}
\let\phfnote@old@thanks\thanks
%    \end{macrocode}
%
% \begin{macro}{\notetitle@thanks}
%   Now, we need to extend \LaTeX's |\thanks| to allow an optional argument as for
%   footnotes.  This macro will be renamed |\thanks| in |\phfnote@do@notetitle|.
%
%   Check whether there is an optional argument; if there is none we execute \LaTeX's
%   original thanks code (replicated here), otherwise, we specify the optional argument
%   explicitly at the relevant location in \LaTeX's implementation:
%    \begin{macrocode}
\def\notetitle@thanks{\@ifnextchar[\phfnote@thanks{\phfnote@thanks[]}}%]
\long\def\phfnote@thanks[#1]#2{%
  \if\relax\detokenize{#1}\relax%
%    \end{macrocode}
% 
% The optional argument is empty---just execute \LaTeX's original |\thanks| code,
% replicated here:
%    \begin{macrocode}
    \footnotemark%
    \protected@xdef\@thanks{\@thanks\protect\footnotetext[\the\c@footnote]{#2}}%
%    \end{macrocode}
% 
% Otherwise, execute \LaTeX's original |\thanks| code, but with the optional argument
% inserted wherever needed:
%    \begin{macrocode}
  \else% argument, pass on to sub-commands:
    \footnotemark[#1]%
    \protected@xdef\@thanks{\@thanks\protect\footnotetext[#1]{#2}}%
  \fi%
}
%    \end{macrocode}
% \end{macro}
% 
%
% \subsubsection{Title Styles Definition}
%
% The title styles are documented in \autoref{sec:title-styles}.
%
% \paragraph{Title style: `default'}
%
% Implementation our main `default' title style.  See \autoref{sec:main-default-title-style}.
%
% \begin{macro}{\notetitle@style@default}
%    The default title style.  Nothing mysterious, hopefully.
%    \begin{macrocode}
\newcommand{\notetitle@style@default}{%
  \begingroup\par\raggedright%
    \phfnote@setupthanksmpfootnote%
    \vspace*{\notetitletopspace}%
    \phfnote@title@checksetspace{default}%
    \begin{minipage}{\textwidth}%
      \begin{singlespace}%
        \parskip=0pt\parindent=0pt\relax%
        {\let\phfnote@old@par\par%
          \def\par{\phfnote@old@par%
            \parskip=1.5ex\relax\parshape 1 0pt \textwidth\relax%
            \noindent}%
          \par%
          \Large  {\notetitlefont \@title}\par}%
        \vskip 2mm\relax
        \if\relax\detokenize\expandafter{\@author}\relax\else%
          \par\parshape 1 0.04\textwidth 0.96\textwidth\relax%
          {\notetitleauthorfont \@author}%
          \vskip 2mm\relax%
        \fi
        \if\relax\detokenize\expandafter{\@date}\relax\else%
          \par\parshape 1 0.04\textwidth 0.96\textwidth\relax%
          {\notetitledatefont \@date}
          \vskip 2mm\relax%
        \fi
        \global\let\@thanks\@empty%
        \phfnote@finalizempfootnotes%
      \end{singlespace}%
    \end{minipage}\par%
    \vspace*{2mm}%
    \notetitlehrule\relax%
    \par%
  \endgroup%
  \vskip\notetitlebelowspace\relax% don't change this, abstract needs to \removelastskip
}
%    \end{macrocode}
% \end{macro}
%
%
% \paragraph{Title style: `small'}
%
% Implementation an alternate `small' title style.
%
% \begin{macro}{\notetitle@style@small}
%    The default title style.  Nothing mysterious, hopefully.
%    \begin{macrocode}
\newcommand{\notetitle@style@small}{%
  \begingroup\par\raggedright%
    \let\footnote\thanks%
    \vspace*{\notetitletopspace}%
    {\notetitlefont \@title}%
    \hfill\makebox{\fontsize{9pt}{10pt}\selectfont {\notetitleauthorfont \@author}%
      \hspace*{2mm}--\hspace*{2mm}{\emph{\notetitledatefont \@date}}}%
    \vspace*{1mm}\notetitlehrule\relax\vspace*{1mm}%
    \par%
  \endgroup%
  \vskip\notetitlebelowspace\relax% don't change this, abstract needs to \removelastskip
}
%    \end{macrocode}
% \end{macro}
%
% 
% \paragraph{Title style: `article'}
%
% Implementation the `article' title style.
%
% \begin{macro}{\notetitle@style@article}
%    The title style definition.  Nothing mysterious, hopefully.
%    \begin{macrocode}
\newcommand{\notetitle@style@article}{%
  \vspace*{-3em}%
  \begingroup
    \centering
    \let\footnote\thanks%
    {\LARGE \@title \par}%
    \vskip 1.5em%
    {\large%
      \lineskip .5em%
      \begin{tabular}[t]{c}%
        \@author%
      \end{tabular}\par}%
    \vskip 1.5em%
    {\large \@date}%
    \par%
  \endgroup%
  \par%
  \vskip 2.5em\relax%
}
%    \end{macrocode}
% \end{macro}
% 
% \subsubsection{Plugging into \phfverb{\maketitle}}
%
% Actually perform the definitions to make |\maketitle| produce the
% title with the given style.  Specifically, we override
% |\@maketitle|. The latter is called internally by |\maketitle|, and
% the advantage of overriding |\@maketitle| only is that we inherit
% the mechanism provided by the style class to deal with two-column
% layouts.
%
% \begin{macro}{\phfnote@do@notetitle}
%   This macro takes care of installing the correct title into the
%   document, by overriding |\@maketitle|.
%
%   This macro is called later after processing the package options.
%   Its argument |#1| is the style name, e.g., |default|.
%
%    \begin{macrocode}
\def\phfnote@do@notetitle#1{
%    \end{macrocode}
% If we have an empty title style, then we leave default title provided by the class. 
%    \begin{macrocode}
  \if\relax\detokenize\expandafter{#1}\relax
  \else
%    \end{macrocode}
% Otherwise, we have a title style to set.  Do some checks that the given style is indeed
% defined.
%    \begin{macrocode}
    \ifcsname notetitle@style@#1\endcsname
      \def\phfnote@tmp@titsty{#1}%
    \else
      \PackageError{phfnote}{Unknown title style: '#1'.}{Unknown title
        style: '#1'. Please consult the package documentation for available
        styles.}
      \def\phfnote@tmp@titsty{default}%
    \fi
%    \end{macrocode}
% Apply new (default) definitions of |\thanks|, |\thanksmark| and |\title|.  Do this here
% only, because this can clash with more complicated versions from, e.g., \RevTeX.
%    \begin{macrocode}
    \let\title\notetitle@title
    \let\thanks\notetitle@thanks
    \let\thanksmark\notetitle@thanksmark
%    \end{macrocode}
% Now, actually overload the title style by redefining |\@maketitle|.
%    \begin{macrocode}
    \def\@maketitle{\csname notetitle@style@\phfnote@tmp@titsty\endcsname}
  \fi
}
%    \end{macrocode}
% \end{macro}
%
%
%
% \subsection{Abstract}
%
% Now we can take care of the abstract.  Unlike the title styles, the abstract has a base
% implementation.  Then, we may have attributes which change some parameters.
%
% 
% \begin{environment}{notedefaultabstract}
%   First, save the old environment |\begin{abstract}...\end{abstract}| provided by the
%   class (if any).
%    \begin{macrocode}
\let\notedefaultabstract\abstract
\let\endnotedefaultabstract\endabstract
%    \end{macrocode}
% \end{environment}
%
% \begin{macro}{\noteabstracttextfont}
% \begin{macro}{\noteabstractnamefont}
% \begin{macro}{\noteabstracttextwidth}
% \begin{macro}{\noteabstractafterspacing}
% \begin{macro}{\noteabstractbeforepacing}
%   Macros which can be overridden to customize the abstract.  See
%   \autoref{sec:abstract-attributes}.
%    \begin{macrocode}
\newcommand{\noteabstracttextfont}{}
\newcommand{\noteabstractnamefont}{\bfseries\small}
\if@twocolumn
  \newcommand\noteabstracttextwidth{\hsize}
\else
  \newcommand{\noteabstracttextwidth}{0.9\hsize}
\fi
\newcommand\noteabstractafterspacing{1.5em}
\newcommand\noteabstractbeforespacing{1.5em}
%    \end{macrocode}
% \end{macro}
% \end{macro}
% \end{macro}
% \end{macro}
% \end{macro}
% 
% \begin{macro}{\noteabstract@nameline}
%   Create the line which contains the title of the abstract, that is, the word
%   ``Abstract.''  This can be overloaded, of course, for customization.
%    \begin{macrocode}
\def\noteabstract@nameline{
  {\parskip=0pt\relax\par\centering\noteabstractnamefont%
    \abstractname%
    \par}\vskip 1ex\relax%
}
%    \end{macrocode}
% \end{macro}
% 
% \begin{environment}{noteabstract}
%   The proper |noteabstract| environment.
%
%    \begin{macrocode}
\newenvironment{noteabstract}{%
  \removelastskip%
  \vspace{\noteabstractbeforespacing}%
  \begingroup%
    \par\noindent\centering%
    \begin{minipage}{\noteabstracttextwidth}%
      \noteabstract@nameline%
      \noteabstracttextfont%
    }%
    {%
    \end{minipage}%
    \par%
  \endgroup%
  \vspace{\noteabstractafterspacing}%
}
%    \end{macrocode}
% \end{environment}
% 
% The abstract can be customized by the attributes.  Here we define them:
%    \begin{macrocode}
\def\noteabstract@attr@wide{%
  \def\noteabstracttextwidth{\textwidth}%
}
\def\noteabstract@attr@narrow{%
  \if@twocolumn
  \else
    \def\noteabstracttextwidth{0.8\textwidth}%
  \fi
}
\def\noteabstract@attr@noname{%
  \def\noteabstract@nameline{}%\vspace*{1ex}}%
}
\def\noteabstract@attr@original{%
  \let\abstract\notedefaultabstract
  \let\endabstract\endnotedefaultabstract
}
\def\noteabstract@attr@small{%
  \g@addto@macro\noteabstracttextfont{\small}%
}
\def\noteabstract@attr@compact{%
  \renewcommand\noteabstractafterspacing{1ex}%
  \renewcommand\noteabstractbeforespacing{1ex}%
}
\def\noteabstract@attr@it{%
  \g@addto@macro\noteabstracttextfont{\itshape}%
}
%    \end{macrocode}
% 
% \begin{macro}{\phfnote@do@noteabstract}
%   This helper both defines the |abstract| environment, and also sets the abstract
%   attributes.  This macro will be called according to the package options.
%
%  |#1| = a comma-separated list of attributes.
%    \begin{macrocode}
\def\phfnote@do@noteabstract#1{
  \let\abstract\noteabstract
  \let\endabstract\endnoteabstract
  \phfnote@internal@execattribs{noteabstract@attr@}{abstract attribute}{#1}
}
%    \end{macrocode}
% \end{macro}
% 
%
%
% \subsection{Page Geometry Settings}
%
% For the page geometry settings, we just have a bunch of styles which we define as
% macros.  The macros just set up |\PassOptionsToPackage| for the \pkgname{geometry} package.
% Then the correct macro will be selected according to the current \pkgname{phfnote} package
% options.
%
% The description of these settings are given in \autoref{sec:pagegeomdefs}.
%
% \begin{macro}{\phfnote@pagegeomstyle@default}
%   Default setting.
%    \begin{macrocode}
\def\phfnote@pagegeomstyle@default{
  \if@twocolumn
    \PassOptionsToPackage{hmargin=1in,vmargin=0.75in,includeheadfoot}{geometry}%
  \else
    % fix the margins a bit to make text wider
    \ifcase\@ptsize% mods for 10 pt
      \PassOptionsToPackage{hmargin=1.5in,vmargin=1.25in}{geometry}%
    \or% mods for 11 pt
      \PassOptionsToPackage{hmargin=1.5in,vmargin=1.25in}{geometry}%
    \or% mods for 12 pt
      \PassOptionsToPackage{hmargin=1.25in,vmargin=1.25in}{geometry}%
    \fi%
  \fi
}
%    \end{macrocode}
% \end{macro}
% 
% \begin{macro}{\phfnote@pagegeomstyle@narrow}
%   Narrow style.
%    \begin{macrocode}
\def\phfnote@pagegeomstyle@narrow{
  \if@twocolumn
    \PassOptionsToPackage{hmargin=1.25in,vmargin=0.75in,includeheadfoot}{geometry}%
  \else
    % fix the margins a bit to make text wider
    \ifcase\@ptsize% mods for 10 pt
      \PassOptionsToPackage{hmargin=1.75in,vmargin=1.5in}{geometry}%
    \or% mods for 11 pt
      \PassOptionsToPackage{hmargin=1.75in,vmargin=1.5in}{geometry}%
    \or% mods for 12 pt
      \PassOptionsToPackage{hmargin=1.5in,vmargin=1.5in}{geometry}%
    \fi%
  \fi
}
%    \end{macrocode}
% \end{macro}
%
% \begin{macro}{\phfnote@pagegeomstyle@wide}
%   Wide style.
%    \begin{macrocode}
\def\phfnote@pagegeomstyle@wide{
  \if@twocolumn
    \PassOptionsToPackage{hmargin=0.75in,vmargin=0.75in,includeheadfoot}{geometry}%
  \else
    % fix the margins a bit to make text wider
    \ifcase\@ptsize% mods for 10 pt
      \PassOptionsToPackage{hmargin=1.25in,vmargin=1.25in}{geometry}%
    \or% mods for 11 pt
      \PassOptionsToPackage{hmargin=1.25in,vmargin=1.25in}{geometry}%
    \or% mods for 12 pt
      \PassOptionsToPackage{hmargin=1in,vmargin=1.25in}{geometry}%
    \fi%
  \fi
}
%    \end{macrocode}
% \end{macro}
% 
% \begin{macro}{\phfnote@pagegeomstyle@xwide}
%   Extra wide.
%    \begin{macrocode}
\def\phfnote@pagegeomstyle@xwide{
  \if@twocolumn
    \PassOptionsToPackage{hmargin=0.5in,vmargin=0.5in,includeheadfoot}{geometry}%
  \else
    % fix the margins a bit to make text wider
    \ifcase\@ptsize% mods for 10 pt
      \PassOptionsToPackage{hmargin=1in,vmargin=1.25in}{geometry}%
    \or% mods for 11 pt
      \PassOptionsToPackage{hmargin=1in,vmargin=1.25in}{geometry}%
    \or% mods for 12 pt
      \PassOptionsToPackage{hmargin=0.75in,vmargin=1.25in}{geometry}%
    \fi%
  \fi
}
%    \end{macrocode}
% \end{macro}
%
% \begin{macro}{\phfnote@pagegeomstyle@bigmargin}
%   |bigmargin| style.
%    \begin{macrocode}
\def\phfnote@pagegeomstyle@bigmargin{%
  \if@twocolumn
    \PassOptionsToPackage{hmargin=1.5in,vmargin=0.75in,includeheadfoot}{geometry}%
  \else
    % fix the margins a bit to make text wider
    \ifcase\@ptsize% mods for 10 pt
      \PassOptionsToPackage{hmargin={2.25in,1.75in},vmargin=1.25in}{geometry}%
    \or% mods for 11 pt
      \PassOptionsToPackage{hmargin={2.25in,1.75in},vmargin=1.25in}{geometry}%
    \or% mods for 12 pt
      \PassOptionsToPackage{hmargin={2in,1.5in},vmargin=1.25in}{geometry}%
    \fi%
  \fi
}
%    \end{macrocode}
% \end{macro}
% 
% 
% \begin{macro}{\phfnote@do@pagegeomdefs}
%   Finally, provide a helper to set the page geometry.  Just call the right macro.
%    \begin{macrocode}
\newcommand{\phfnote@do@pagegeomdefs}[1]{
  \ifcsname phfnote@pagegeomstyle@#1\endcsname
    \csname phfnote@pagegeomstyle@#1\endcsname
  \else
    \PackageWarning{phfnote}{Unknown page geometry style: `#1'!}
  \fi

  \RequirePackage{geometry}%
}
%    \end{macrocode}
% \end{macro}
%
%
% \subsection{Text, Paragraph and Line Spacing}
%
% \paragraph{Text \& Line Spacing}
%
% \begin{macro}{\phfnote@do@spacing}
%   Some cosmetic definitions to adjust line spacing.  The line spacing is slightly
%   adjusted according to font size to make the document more readable.  Depending on
%   whether the \pkgname{setspace} package is loaded, we use it or go low-level with a
%   redefinition of \LaTeX{}' |\baselinestretch|.  If the \pkgname{captions} package is loaded,
%   the figure captions' line spacing is also adjusted.
%   
%   Also set an |\emergencystretch| so that lines get spaced out for underfull boxes,
%   rather than overflowing far into the margin.
%    \begin{macrocode}
\def\phfnote@do@spacing{
  \@ifpackageloaded{setspace}{
    \def\phfnote@dostretch##1{%
      \setstretch{##1}\phfnote@docaptionstretch{##1}}
  }{
    \def\phfnote@dostretch##1{%
      \renewcommand\baselinestretch{##1}\phfnote@docaptionstretch{##1}}
  }
  \@ifpackageloaded{caption}{
    \def\phfnote@docaptionstretch##1{\captionsetup{font={stretch=##1}}}
  }{
    \def\phfnote@docaptionstretch##1{\PackageWarning{phfnote}{Can't
        set line spacing for captions, because the package `caption'
        is not loaded.  Please load it before `phfnote', or use an
        appropriate (e.g. `rich') pkgset which loads this package
        automatically .}}
  }
  \if@twocolumn
    \phfnote@dostretch{1.0} % leave default
    \emergencystretch=3em\relax
  \else
    \ifcase\@ptsize% 10pt
      \phfnote@dostretch{1.1}
    \or% 11pt
      \phfnote@dostretch{1.0} % 1.05? better 1.0...
    \or% 12pt
      \phfnote@dostretch{1.0} % 1.03? not really noticeable...
    \fi
    \emergencystretch=6em\relax
  \fi
}
%    \end{macrocode}
% \end{macro}
% 
%
% \paragraph{Paragraph Spacing Presets}
%
% Here again, we define several possibilities for paragraph settings as
% individual macros (see \autoref{sec:par-defs}).  Depending on the package
% option, we execute the corresponding macro.
%
%    \begin{macrocode}
\def\phfnote@par@original{%
}
\def\phfnote@par@indent{%
  \parindent=1.5em\relax
  \parskip=0pt\relax
}
\def\phfnote@par@indentminiskip{%
  \parindent=1.5em\relax
  \parskip=0.3em plus 0.1em\relax
}
\def\phfnote@par@skip{%
  \parindent=0pt\relax
  \parskip=0.8em plus 0.2em minus 0.1em\relax
}
%    \end{macrocode}
% 
% \begin{macro}{\phfnote@do@par}
%   Execute the given paragraph setting.  The argument |#1| is the setting, for example,
%   |skip|.
%    \begin{macrocode}
\def\phfnote@do@par#1{%
  \ifcsname phfnote@par@#1\endcsname
    \csname phfnote@par@#1\endcsname
  \else
    \PackageWarning{phfnote}{Bad paragraph setting: #1. Leaving original}
  \fi
}
%    \end{macrocode}
% \end{macro}
% 
%
%
%
% \subsection{Section Styling}
%
% Very limited support for styling section and paragraph headers
% (\autoref{sec:secfmt}).  If you want anything serious, use \pkgname{sectsty} or
% \pkgname{titlesec} directly.
%
% \begin{macro}{\notesectionallfont}
% \begin{macro}{\notesectionallfontfamily}
%   Define the |\notesectionallfont| and |\notesectionallfontfamily|, which
%   control the general font used in section headings.
%    \begin{macrocode}
\newcommand{\notesectionallfont}{%
  \fontfamily{\notesectionallfontfamily}\fontseries{bx}\selectfont}
\newcommand{\notesectionallfontfamily}{ppl}
%    \end{macrocode}
% \end{macro}
% \end{macro}
% 
%
% \begin{macro}{\notesectionfont}
% \begin{macro}{\notesubsectionfont}
% \begin{macro}{\notesubsubsectionfont}
% \begin{macro}{\noteparagraphfont}
% \begin{macro}{\notesubparagraphfont}
%   These macros are called for their respective sectioning command,
%   after |\notesectionallfont| has been invoked. (Again, only for
%   those sectioning commands which are styled by us.)
%
%    \begin{macrocode}
\newcommand{\notesectionfont}{\large}
\newcommand{\notesubsectionfont}{\normalsize}
\newcommand{\notesubsubsectionfont}{\small}
\newcommand{\noteparagraphfont}{\normalsize}
\newcommand{\notesubparagraphfont}{\normalsize}
%    \end{macrocode}
% \end{macro}
% \end{macro}
% \end{macro}
% \end{macro}
% \end{macro}
% 
%
% \begin{macro}{\notesectionsetfonts}
% \begin{macro}{\noteparagraphsetfonts}
%   Helpers to directly set the font commands for |\section|,
%   |\subsection| and |\subsubsection| (with |\notesectionsetfonts|),
%   and for |\paragraph| and |\subparagraph| (with |\noteparagraphsetfonts|).
%    \begin{macrocode}
\newcommand{\notesectionsetfonts}[3]{%
  \renewcommand{\notesectionfont}{#1}%
  \renewcommand{\notesubsectionfont}{#2}%
  \renewcommand{\notesubsubsectionfont}{#3}%
}
\newcommand{\noteparagraphsetfonts}[2]{%
  \renewcommand{\noteparagraphfont}{#1}%
  \renewcommand{\notesubparagraphfont}{#2}%
}
%    \end{macrocode}
% \end{macro}
% \end{macro}
% 
%
% Define the attributes which the user can set.  See
% \autoref{sec:secfmt}.
%
%    \begin{macrocode}
\def\phfnote@do@secfmt@section{
  \RequirePackage{sectsty}
  \sectionfont{\notesectionallfont\notesectionfont}
  \subsectionfont{\notesectionallfont\notesubsectionfont}
  \subsubsectionfont{\notesectionallfont\notesubsubsectionfont}
}
\def\phfnote@do@secfmt@paragraph{
  \RequirePackage{sectsty}
  \paragraphfont{\notesectionallfont\noteparagraphfont}
  \subparagraphfont{\notesectionallfont\notesubparagraphfont}
}
\def\phfnote@do@secfmt@compact{
  \notesectionsetfonts{\normalsize}{\small}{\small}
}
\def\phfnote@do@secfmt@larger{
  \notesectionsetfonts{\Large}{\large}{\normalsize}
}

\def\phfnote@do@secfmt@secsquares{
  \RequirePackage{amssymb}
  \let\phfnote@secsquares@old@seccntformat\@seccntformat
  \def\@seccntformat##1{%
    \expandafter\ifx\csname ##1\endcsname\section\relax%
    \unexpanded{\makebox[0pt][r]{\raisebox{0.15ex}{{%
            \notesmaller[0.6]\ensuremath{\blacksquare}}}%
        \hspace*{1.2ex}}}%
    \fi%
    \phfnote@secsquares@old@seccntformat{##1}}
}
\def\phfnote@do@secfmt@secnummargin{
  \let\phfnote@secnummargin@old@seccntformat\@seccntformat
  \def\@seccntformat##1{%
    \protect\makebox[0pt][r]{\phfnote@secnummargin@old@seccntformat{##1}}}
}

\def\phfnote@do@secfmt@rmfamily{
  \renewcommand\notesectionallfontfamily{\rmdefault}
}
\def\phfnote@do@secfmt@sffamily{
  \renewcommand\notesectionallfontfamily{\sfdefault}
}
\def\phfnote@do@secfmt@itpar{
  \def\noteparagraphfont{\normalfont\normalsize\itshape}
  \def\notesubparagraphfont{\normalfont\normalsize\itshape}
}
\def\phfnote@do@secfmt@blockpar{
  \let\phfnote@old@paragraph\paragraph
  \def\paragraph##1{%
    \phfnote@old@paragraph{##1}%
    \hspace*{0pt}\par\nopagebreak% ugly hack!!
  }
}
%    \end{macrocode}
% 
%
% \begin{macro}{\phfnote@do@secfmt}
%   Actually perform the required styling, according to the package
%   options given as argument.  The argument is a comma-separated list
%   of attributes specified by the user.
%    \begin{macrocode}
\def\phfnote@do@secfmt#1{%
  \phfnote@internal@execattribs{phfnote@do@secfmt@}{section formatting preset}{#1}
}
%    \end{macrocode}
% \end{macro}
% 
%
%
% \subsection{\LaTeX{} Package Sets}
%
% Define the package sets as macros.  Depending on the user-specified
% options we load the corresponding one(s) (several may be specified).
%
% See \autoref{sec:package-sets} for a description of what these
% package sets do.
%
% \begin{macro}{\phfnote@do@pkgset@none}
% \begin{macro}{\phfnote@do@pkgset@minimal}
% \begin{macro}{\phfnote@do@pkgset@rich}
% \begin{macro}{\phfnote@do@pkgset@extended}
%   Macros which implement the package sets.  Each macro invokes |\RequirePackage|
%   for the appropriate packages.
%    \begin{macrocode}
\def\phfnote@do@pkgset@none{
}

\def\phfnote@do@pkgset@minimal{

  \RequirePackage{amsmath}
  \RequirePackage{amsfonts}
  \RequirePackage{amssymb}
  \RequirePackage{amsthm}
  
  \RequirePackage{xcolor}

}

\def\phfnote@do@pkgset@rich{

  \phfnote@do@pkgset@minimal

  \RequirePackage{setspace}
  \RequirePackage{caption}

  \RequirePackage{microtype}

  \PassOptionsToPackage{shortlabels}{enumitem}
  \RequirePackage{enumitem}

  \RequirePackage{graphicx}

  \PassOptionsToPackage{T1}{fontenc}
  \RequirePackage{fontenc}

  \PassOptionsToPackage{utf8}{inputenc}
  \RequirePackage{inputenc}
}

\def\phfnote@do@pkgset@extended{

  \phfnote@do@pkgset@rich

  \RequirePackage{float}

  \RequirePackage{verbdef}

  \PassOptionsToPackage{autostyle,autopunct=true}{csquotes}
  \RequirePackage{csquotes}

  \RequirePackage{dsfont}
  \RequirePackage{bbm}
  \RequirePackage{mathtools}

}
%    \end{macrocode}
% \end{macro}
% \end{macro}
% \end{macro}
% \end{macro}
%
%
% \begin{macro}{\phfnote@do@pkgset}
%   Finally, define the helper which will load the required package sets.
%    \begin{macrocode}
\def\phfnote@do@pkgset#1{
  \phfnote@internal@execattribs{phfnote@do@pkgset@}{package set}{#1}
}
%    \end{macrocode}
% \end{macro}
% 
%
% \subsection{Hyperref Support and Hyperlinks}
%
% \begin{pkgnote}
%   The name `|docnotelinkcolor|' is historical and hard-coded in many
%   other files I've used, so I'm DEFINITELY NOT changing it.
% \end{pkgnote}
% 
% \needspace{3\baselineskip}
% \begin{macro}{\phfnote@do@pdfhyperrefdefs}
% \begin{macro}{\email}
% \begin{macro}{\url}
%   Load the \pkgname{hyperref} package and provide sensible defaults.
%    \begin{macrocode}
\newcommand{\phfnote@do@pdfhyperrefdefs}{%
%    \end{macrocode}
% Make sure a color-managing package is loaded, {color} or {xcolor}, and define our default color:
%    \begin{macrocode}
  \phfnote@requirecolorpackage%
  \definecolor{docnotelinkcolor}{rgb}{0,0,0.4}%
%    \end{macrocode}
% 
% Load URL package, and save a version of |\url| which is not patched
% by \pkgname{hyperref}:
%    \begin{macrocode}
  \RequirePackage{url}%
  \DeclareUrlCommand\phfnote@format@url{}%
%    \end{macrocode}
% 
% Set up \pkgname{hyperref} options:
%    \begin{macrocode}
  \PassOptionsToPackage{bookmarks=true,backref=false}{hyperref}%
  \RequirePackage{hyperref}%
  %
  \hypersetup{unicode=true,%
    bookmarksnumbered=false,bookmarksopen=false,bookmarksopenlevel=1,%
    breaklinks=true,pdfborder={0 0 0},colorlinks=true}%
  \hypersetup{%
    anchorcolor=docnotelinkcolor,citecolor=docnotelinkcolor,%
    filecolor=docnotelinkcolor,linkcolor=docnotelinkcolor,%
    menucolor=docnotelinkcolor,runcolor=docnotelinkcolor,%
    urlcolor=docnotelinkcolor}%
%    \end{macrocode}
% 
% Provide an |\email| command for specifying e-mails.  Note that the |\url| command is
% already provided by the packages \pkgname{url} and \pkgname{hyperref}.
%    \begin{macrocode}
  \let\email\phfnote@email%
%    \end{macrocode}
% 
% And finally set a nicer default |\url|/|\email| style:
%    \begin{macrocode}
  \urlstyle{notesf}%
}
%    \end{macrocode}
% \end{macro}
% \end{macro}
% \end{macro}
% 
% \begin{macro}{\phfnotePdfLinkColor}
%   Set links color.  Use as |\phfnotePdfLinkColor|\marg{color}.
%   Color may be any color name or specification recognized by the
%   \pkgname{xcolor} package.
%
%    \begin{macrocode}
\newcommand{\phfnotePdfLinkColor}[1]{%
  \@ifpackageloaded{xcolor}{%
    \colorlet{docnotelinkcolor}{#1}%
  }{% else:
    \PackageWarning{phfnote}{\protect\phfnotePdfLinkColor may only be
      used if the package xcolor is loaded.}%
  }%
}
%    \end{macrocode}
% \end{macro}
% 
%
% \needspace{3\baselineskip}
% \begin{macro}{\phfnote@sanitize@url}
% \begin{macro}{\phfnote@format@url}
% \begin{macro}{\phfnote@email}
%   Provide base macros to be able to build up |\email| command for emails and other
%   URL-like commands which should sanitize their arguments.
%
%   Also prepare the command |\phfnote@email| which will be renamed |\email| in our
%   \pkgname{hyperref} package setup (see above).
% 
%    \begin{macrocode}
\def\phfnote@sanitize@url{%
  \catcode`\$12%
  \catcode`\&12%
  \catcode`\#12%
  \catcode`\^12%
  \catcode`\_12%
  \catcode`\%12%
  % \catcode`\^^J10%  newline = space
  % \catcode`\^^M10%  newline = space
  \relax%
}%
\providecommand\phfnote@format@url{\texttt}
\def\phfnote@email{\begingroup\phfnote@sanitize@url\phfnote@impl@email@}%
\def\phfnote@impl@email@#1{\endgroup\href{mailto:#1}{\phfnote@format@url{#1}}}%
%    \end{macrocode}
% \end{macro}
% \end{macro}
% \end{macro}
%
%
% 
% \begin{macro}{\phfnote@requirecolorpackage}
%   And finally define an internal utility to make sure that a color package (either
%   \pkgname{color} or \pkgname{xcolor}) is loaded.  If none are loaded, the \pkgname{xcolor} package is loaded.
%    \begin{macrocode}
\def\phfnote@requirecolorpackage{%
  \@ifpackageloaded{color}{%
  }{%
    \@ifpackageloaded{xcolor}{%
    }{%
      \RequirePackage{xcolor}%
    }%
  }%
}
%    \end{macrocode}
% \end{macro}
% 
%
% \subsection{Cosmetic Font Definitions}
%
% \begin{macro}{\phfnote@do@fontdefs}
%   Minimalist cosmetic definition for fonts: load the |T1| font
%   encoding which is better.  Also, use Computer Modern Bright as
%   sans-serif font by default instead of Computer Modern Sans Serif.
%
%    \begin{macrocode}
\def\phfnote@do@fontdefs{
 
  \PassOptionsToPackage{T1}{fontenc}
  \RequirePackage{fontenc}

  \renewcommand\sfdefault{cmbr}
  
}
%    \end{macrocode}
% \end{macro}
% 
%
% \subsection{Bibliography Stuff}
%
%   Provide some fixes for the bibliography.
%
% \begin{macro}{\phfnote@bibstyle}
% \begin{macro}{\phfnote@bibfont}
%   Our default bibliography style is stored in |\phfnote@bibstyle|.
%   By default, it's our own hacked version of the |naturemag| style.
%   The font in which to typeset the bibliography is stored in
%   |\phfnote@bibfont|.  By default, it's a little smaller than the
%   main text.
%    \begin{macrocode}
\newcommand{\phfnote@bibstyle}{naturemagdoi}
\newcommand{\phfnote@bibfont}{\fontsize{9}{11}\selectfont}
%    \end{macrocode}
% \end{macro}
% \end{macro}
% 
%
% \begin{macro}{\phfnote@bibliography}
%   These are a tentative implementation for |\bibliography|.  The latter will be set to
%   this implementation according to the user's package options.
%    \begin{macrocode}
\let\phfnote@old@bibliography\bibliography
\let\phfnote@old@bibliographystyle\bibliographystyle
\newcommand{\phfnote@bibliography}[1]{%
  \begingroup%
    \phfnote@bibfont%
    \phfnote@old@bibliographystyle{\phfnote@bibstyle}%
%    \end{macrocode}
% 
% Our hack: make sure that the next instance of |\section*| will generate a TOC
% entry. (See |\phfnoteHackSectionStarWithTOC|.)
%    \begin{macrocode}
    \phfnoteHackSectionStarWithTOC%
%    \end{macrocode}
% 
% Some special chars may appear in output of some ill-advised bibliography
% managers. Mostly the |&| symbol, such as in |Taylor & Francis|.  We won't be needing a
% \LaTeX{} alignment operator here, so just make |&| a normal printable character
% (``other'' catcode).
%    \begin{macrocode}
    \catcode`\&=12\relax% normal char
%    \end{macrocode}
%
% Adjust the appearance of e-prints. We assume e-prints refer to the arXiv; here we
% generate a hyperlink and format them better.
%
%    \begin{macrocode}
    \providecommand\eprint[2][]{\href{http://arxiv.org/abs/##2}{arXiv:##2}}
%    \end{macrocode}
% 
% Relay the call to the ``old'' |\bibliography| command to actually implement the
% bibliography.
%    \begin{macrocode}
    \phfnote@old@bibliography{#1}%
  \endgroup%
}
%    \end{macrocode}
% \end{macro}
% \begin{macro}{\phfnote@bibliographystyle}
%   Tentative implementation of |\bibliographystyle|.  Just register the new style in an
%   internal variable, so that the style is actually loaded in |\phfnote@bibliography|.
%
%   This will be renamed to replace |\bibliographystyle| later, according to package
%   options.
%    \begin{macrocode}
\newcommand{\phfnote@bibliographystyle}[1]{%
  \renewcommand{\phfnote@bibstyle}{#1}%
}
%    \end{macrocode}
% \end{macro}
% 
%
% \begin{macro}{\phfnote@do@bibliographydefs}
%   Make our changes live.  Will be called later according to package options.
%    \begin{macrocode}
\def\phfnote@do@bibliographydefs{%
  \let\bibliographystyle\phfnote@bibliographystyle%
  \let\bibliography\phfnote@bibliography%
}
%    \end{macrocode}
% \end{macro}
% 
%
%
% \subsection{Better Footnote Style}
%
% \begin{macro}{\phfnote@do@footnotedefs}
%   Adjust the formatting of footnotes so they look better.  Again,
%   this is called later according to the package options.
%    \begin{macrocode}
\def\phfnote@do@footnotedefs{
  \let\phfnote@orig@makefnmark\@makefnmark
%%  \def\@makefnmark{\hbox{\@textsuperscript{%
%%      \normalfont\tiny\fontseries{sb}\selectfont\@thefnmark}}}
  \def\@makefnmark{\hbox{\@textsuperscript{%
        \normalfont\tiny\bfseries\@thefnmark}}}
%%  \def\@makefnmark{\hbox{\@textsuperscript{%
%%      \normalfont\scriptsize\bfseries\@thefnmark}}}% too large
}
%    \end{macrocode}
% \end{macro}
% 
% 
%
%
% \subsection{Other Stand-Alone Definitions and Helpers}
% \label{sec:impl-other-stand-alone-defs}
%
% \subsubsection{A \phfverb{\notesmaller} command}
%
%
% \begin{macro}{\notesmaller}
%   Relative font size command.  Makes the text a fraction smaller
%   than its surroundings.  The fraction is either given explicitly as
%   optional argument (1.0=same size) or is by default set by
%   |\notesmallerfrac|.
%
%   To impalement this, we exploit the fact that \LaTeX{} saves the
%   current font size in the macro |\f@size|.
%    \begin{macrocode}
\newcommand\notesmaller[1][\notesmallerfrac]{%
  \fontsize{#1\dimexpr\f@size pt\relax}{#1\dimexpr\f@baselineskip pt\relax}%
  \selectfont\ignorespaces%
}
%    \end{macrocode}
% \end{macro}
% \begin{macro}{\notesmallerfrac}
%   Default fraction by which |\notesmaller| acts.  Redefine to change defaults.
%    \begin{macrocode}
\def\notesmallerfrac{0.9}
%    \end{macrocode}
% \end{macro}
%
%
% \subsubsection{Customized, ``Inline,'' Table of Contents}
%
% \begin{macro}{\inlinetoc}
%   Just a customized table of contents.  Horizontal rules before and
%   after, and spacing is adjusted, and no ``Contents'' title.  The
%   table of contents looks just like at the \hyperref[sec:toc]{top of
%   this document}.  The command is described in
%   \autoref{sec:inline-toc}.
%
%   We call |\@starttoc| directly, bypassing the |\section*| included by
%   |\tableofcontents| (see definition |\tableofcontents| in latex sources).
%    \begin{macrocode}
\newcommand{\inlinetoc}{%
  \begingroup%
    \vspace*{2mm}%
    \hrule%
    \vspace*{2mm}%
    \parskip=1pt\relax%
    \@starttoc{toc}%
    \vspace*{4mm}%
    \hrule%
    \vspace*{6mm}%
  \endgroup%
}
%    \end{macrocode}
% 
% \end{macro}
%
% \subsubsection{URL Styles}
%
% \needspace{7\baselineskip}
% \begin{macro}{\url@notettstyle}
% \begin{macro}{\url@notesfstyle}
% \begin{macro}{\url@notesfssstyle}
% \begin{macro}{\url@noteitsfstyle}
% \begin{macro}{\url@notermstyle}
% \begin{macro}{\url@noteitstyle}
% \begin{macro}{\url@notesmlstyle}
%   We also provide some URL styles.  These can directly set with
%   |\urlstyle|\marg{style-name}.
%
%    \begin{macrocode}
\def\url@notettstyle{%
  \def\UrlFont{\ttfamily\notesmaller}%
  \phfnote@urlstyle@common%
}
\def\url@notesfstyle{%
  \def\UrlFont{\sffamily\notesmaller}%
  \phfnote@urlstyle@common%
}
\def\url@notesfssstyle{%
  \def\UrlFont{\fontfamily{cmss}\selectfont\notesmaller}%
  \phfnote@urlstyle@common%
}
\def\url@noteitsfstyle{%
  \def\UrlFont{\sffamily\itshape\notesmaller}%
  \phfnote@urlstyle@common%
}
\def\url@notermstyle{%
  \def\UrlFont{\rmfamily\notesmaller}%
  \phfnote@urlstyle@common%
}
\def\url@noteitstyle{%
  \def\UrlFont{\itshape\notesmaller}%
  \phfnote@urlstyle@common%
}
\def\url@notesmlstyle{%
  \def\UrlFont{\notesmaller}%
  \phfnote@urlstyle@common%
}
%    \end{macrocode}
% \end{macro}
% \end{macro}
% \end{macro}
% \end{macro}
% \end{macro}
% \end{macro}
% \end{macro}
%
% \begin{macro}{\phfnote@urlstyle@common}
% The following code is common to all our styles. We do an ugly hack
% in which the tilde character (`\textasciitilde') is fixed to the
% tilde char in the Adobe Times font (|ptm| code), so that it looks
% nicer and its alignment is correct.
%    \begin{macrocode}
\def\phfnote@url@tilde{\hbox{\fontfamily{ptm}\selectfont\textasciitilde}}
%%\def\phfnote@url@tilde{\raise-0.8ex\hbox{%
%%    \kern-0.2ex\fontfamily{cmbr}\selectfont\textasciitilde}}
\def\phfnote@urlstyle@common{%
  \def\UrlTildeSpecial{\do\~{\phfnote@url@tilde}}%
  \let\Url@force@Tilde\UrlTildeSpecial%
}
%    \end{macrocode}
% \end{macro}
% 
%
% \subsubsection{Utility to Add TOC Entry For Starred Section }
%
% Here we provide an ugly hack which introduces an entry in the table
% of contents for |\section*| commands.
%
% [Note: An existing way of adding the toc entry in these cases is to issue
% a |\addcontentsline| command before the relevant command
% (say |\bibliography|).  However this is unreliable, because on page
% boundaries the |\addcontentsline| will pick up the previous page.  This
% is why |\addcontentsline| should be issued right \emph{after} the
% |\section*| command.]
%
% \begin{pkgwarning}
%   This command is truly a hack, don't apply it globally!  It forces
%   (locally) the |\section| command to be followed by a `|*|' !  Do this
%   within a group, just before a command which you are sure is
%   invoking |\section*| (such as |\bibliography| in the \pkgname{article}
%   class).
% \end{pkgwarning}
%
% \begin{macro}{\phfnoteHackSectionStarWithTOC}
%   Locally force |\section| to be followed by |*| and introduce an
%   entry in the table of contents.
%    \begin{macrocode}
\def\phfnoteHackSectionStarWithTOC{%
    \let\phfnote@old@section\section%
    \def\section*##1{\phfnote@old@section*{##1}\addcontentsline{toc}{section}{##1}}%
}
%    \end{macrocode}
% \end{macro}
% 
% \begin{macro}{\phfnoteHackSectionStarWithTOCInCommand}
%   Patches the given command (|#1|), which is known to invoke
%   |\section*|, to locally first invoke
%   |\phfnoteHackSectionStarWithTOC| and thus generate a TOC entry.
%
%    \begin{macrocode}
\def\phfnoteHackSectionStarWithTOCInCommand#1{%
  \expandafter\let\csname phfnote@old@\string#1\endcsname#1%
  \gdef#1{%
    \begingroup%
    \phfnoteHackSectionStarWithTOC%
    \csname phfnote@old@\string#1\endcsname%
    \endgroup%
  }%
}
%    \end{macrocode}
% \end{macro}
%
%
% \subsubsection{Hack to save \& restore a set of commands}
%
% Exactly what it sounds like.  You can store a set of commands, specified by
% their name, by specifying an identifier.  The commands corresponding to a
% given identifier can then later be restored.
%
% \begin{macro}{\phfnoteSaveDefs}
%   The command |\phfnoteSaveDefs|\marg{identifier}\marg{list of macro names}
%   saves the current definitions of the given list of macro and associates them
%   to the given identifier.
%   The list of macros is specified as a comma-separated list of macro names.
%    \begin{macrocode}
\def\phfnoteSaveDefs#1#2{%
%    \end{macrocode}
% 
% The macro |\phfnote@restoredefs@<identifier>| will store the code necessary to
% restore the macros.
%    \begin{macrocode}
  \csgdef{phfnote@restoredefs@#1}{}%
%    \end{macrocode}
% 
% Iterate over the macros we are supposed to store.
%    \begin{macrocode}
  \def\@tmpa{#2}%
  \@for\next:=\@tmpa\do{%
%    \end{macrocode}
% 
% For each macro we are supposed to store (whose name is given in |\next|), we
% |\let| |\phfnote@restoredefs@<identifier>@<macro-name>| store the current value
% of the macro.
%    \begin{macrocode}
    \global\csletcs{phfnote@restoredefs@#1@\next}{\next}%
%    \end{macrocode}
% 
% Then, we append to |\phfnote@restoredefs@<identifier>| the code necessary to
% restore this macro.  That code is simply a |\cslet| instruction.
%
% Recall that |\xappto| expands its second argument (as |\xdef| does), allowing
% us to expand the value of |\next|.
%    \begin{macrocode}
    \expandafter\xappto\csname phfnote@restoredefs@#1\endcsname{%
      \noexpand\csletcs{\next}{phfnote@restoredefs@#1@\next}%
    }%
  }%
}
%    \end{macrocode}
% \end{macro}
% 
% \begin{macro}{\phfnoteRestoreDefs}
%   Restores the macro saved by |\phfnoteSaveDefs|.  We simply execute the macro
%   |\phfnote@restoredefs@<identifier>|, in which we duly stored the code
%   necessary to restore all the saved macros.
%    \begin{macrocode}
\def\phfnoteRestoreDefs#1{%
  \ifcsname phfnote@restoredefs@#1\endcsname%
    \csname phfnote@restoredefs@#1\endcsname%
  \else%
    \PackageError{phfnote}{\string\phfnoteRestoreDefs: no such
      definitions stored (#1)}
  \fi%
}
%    \end{macrocode}
% \end{macro}
%
%
% \subsubsection{A utility for verbatim stuff in arguments of other macros}
%
% FIXME: DOCUMENT ME!
%
% A utility for using verbatim stuff in arguments of other macros---exploit
% |\detokenize|
%
%    \begin{macrocode}
\def\phfverb#1{%
  \ifx\protect\relax%
    \phfverbfmt{\detokenize{#1}\unskip}%
  \else%
    \noexpand\phfverb{\unexpanded{#1}}%
  \fi%
}
\def\phfverbfmt#1{{\normalfont\texttt{#1}}}
%    \end{macrocode}
%
% \subsection{Handle Package Options}
%
% \subsubsection{Define and Parse Package Options}
%
% Initialization code for \pkgname{kvoptions} for our package options.  See \autoref{sec:package-options}.
%
%    \begin{macrocode}
\SetupKeyvalOptions{
  family=phfnote,
  prefix=phfnote@opt@
}
%    \end{macrocode}
%
%
% The title style to use. \marginpar{\raggedleft \phfverb{[title=...]}}
% See \autoref{sec:title-styles}.
%    \begin{macrocode}
\DeclareStringOption[default]{title}
%    \end{macrocode}
% 
%
% Option for abstract attributes \marginpar{\raggedleft
% \phfverb{[abstract=...]}}  (\autoref{sec:abstract-attributes}).
%    \begin{macrocode}
\DeclareStringOption[]{abstract}
%    \end{macrocode}
%
% Option for Package sets \marginpar{\raggedleft \phfverb{[pkgset=...]}}
% (\autoref{sec:package-sets})
%    \begin{macrocode}
\DeclareStringOption[rich]{pkgset}
%    \end{macrocode}
%
% Define the page geometry.  \marginpar{\raggedleft
% \phfverb{[pagegeomdefs=...]}} \marginpar{\raggedleft \phfverb{[pagegeom=...]}}
% See \autoref{sec:pagegeomdefs}.
%    \begin{macrocode}
\DeclareBoolOption[true]{pagegeomdefs}
\DeclareComplementaryOption{nopagegeomdefs}{pagegeomdefs}
\DeclareStringOption[default]{pagegeom}
%    \end{macrocode}
% 
%
%
% Styling of section headings.  \marginpar{\raggedleft \phfverb{[secfmt=...]}}
% See \autoref{sec:secfmt}.
%    \begin{macrocode}
\DeclareStringOption[section]{secfmt}
%    \end{macrocode}
%
% How to treat paragraphs.  \marginpar{\raggedleft \phfverb{[par=...]}} See
% \autoref{sec:par-defs}.
%    \begin{macrocode}
\DeclareStringOption[skip]{par}
%    \end{macrocode} 
% 
% Add definitions to adjust spacing of lines and words. \marginpar{\raggedleft
% \phfverb{[spacingdefs=...]}}  See \autoref{sec:spacingdefs}.
%    \begin{macrocode}
\DeclareBoolOption[true]{spacingdefs}
\DeclareComplementaryOption{nospacingdefs}{spacingdefs}
%    \end{macrocode}
%
% Do some adjustments to the fonts. \marginpar{\raggedleft
% \phfverb{[fontdefs=...]}} See \autoref{sec:fontdefs}.
%    \begin{macrocode}
\DeclareBoolOption[true]{fontdefs}
\DeclareComplementaryOption{nofontdefs}{fontdefs}
%    \end{macrocode}
%
% Adjustments for footnotes. \marginpar{\raggedleft
% \phfverb{[footnotedefs=...]}}  See \autoref{sec:footnotedefs}.
%    \begin{macrocode}
\DeclareBoolOption[true]{footnotedefs}
\DeclareComplementaryOption{nofootnotedefs}{footnotedefs}
%    \end{macrocode}
%
% Load hyperref and corresponding definitions. \marginpar{\raggedleft
% \phfverb{[hyperrefdefs=...]}} See \autoref{sec:hyperrefdefs}.
%    \begin{macrocode}
\DeclareBoolOption[true]{hyperrefdefs}
\DeclareComplementaryOption{nohyperrefdefs}{hyperrefdefs}
%    \end{macrocode}
%
% Adjustments for bibliography, including default style. \marginpar{\raggedleft
% \phfverb{[bibliographydefs=...]}} See \autoref{sec:bibliographydefs}.
%    \begin{macrocode}
\DeclareBoolOption[true]{bibliographydefs}
\DeclareComplementaryOption{nobibliographydefs}{bibliographydefs}
%    \end{macrocode}
% 
% 
% Preset option.  \marginpar{\raggedleft \phfverb{[preset=...]}} See
% \autoref{sec:presets}.
%    \begin{macrocode}
\define@key{phfnote}{preset}{%
  \ifcsname phfnote@preset@#1\endcsname%
    \csname phfnote@preset@#1\endcsname%
  \else%
    \PackageError{phfnote}{Unknown preset: `#1'!}{You specified the
      option 'preset=...' with an invalid value.  Please look up the
      package documentation corresponding to your version of phfnote
      for possible values.}
  \fi%
}
%    \end{macrocode}
% 
%
% Provide the standard error message for unknown options.
%    \begin{macrocode}
\DeclareDefaultOption{%
  \@unknownoptionerror
}
%    \end{macrocode}
%
% \subsubsection{Define Global Presets}
% \label{impl:presets}
%
% Define the global presets here.  See \autoref{sec:presets} for a description
% of what these presets do.
%
% \begin{macro}{\phfnote@hook@atendload}
%   A hook for presets to do stuff at the end of package load.
%    \begin{macrocode}
\def\phfnote@hook@atendload{}
%    \end{macrocode}
% \end{macro}
%
%
% \begin{macro}{\phfnote@preset@article}
%   Article preset.
%    \begin{macrocode}
\def\phfnote@preset@article{
  \def\phfnote@opt@title{article}
  \def\phfnote@opt@par{indent}
  \def\phfnote@opt@pagegeom{default}
}
%    \end{macrocode}
% \end{macro}
%
% \begin{macro}{\phfnote@presetcommon@xnote}
%   Specify some common definitions for all our |*note| preset styles.  The
%   optional argument is the URL style to set.
%    \begin{macrocode}
\newcommand\phfnote@presetcommon@xnote[1][noteitsf]{
  \def\phfnote@opt@title{default}
  \def\phfnote@opt@par{skip}
  \phfnote@opt@pagegeomdefstrue
  \def\phfnote@opt@pagegeom{wide}
  \setlength{\footnotesep}{5pt}
  \g@addto@macro\phfnote@hook@atendload{
    \ifdefined\urlstyle
      \urlstyle{#1}
    \fi
  }
}
%    \end{macrocode}
% \end{macro}
% 
% \needspace{5\baselineskip}
% \begin{macro}{\phfnote@preset@sfnote}
% \begin{macro}{\phfnote@preset@sfssnote}
% \begin{macro}{\phfnote@preset@opensansnote}
% \begin{macro}{\phfnote@preset@utopianote}
% \begin{macro}{\phfnote@preset@mnmynote}
%   Define the different |*note| styles.
%    \begin{macrocode}
\def\phfnote@preset@sfnote{
  \phfnote@presetcommon@xnote
  \phfnote@opt@footnotedefstrue
  \phfnote@opt@fontdefstrue
  \renewcommand\familydefault{\sfdefault}
  \renewcommand{\notesectionallfontfamily}{\sfdefault}
}
\def\phfnote@preset@sfssnote{
%    \end{macrocode}
% set up all the settings as for |sfnote| \ldots
%    \begin{macrocode}
  \phfnote@preset@sfnote
%    \end{macrocode}
% \ldots but override:
%    \begin{macrocode}
  \phfnote@opt@fontdefsfalse
  \PassOptionsToPackage{T1}{fontenc}
  \RequirePackage{fontenc}
  \renewcommand\sfdefault{cmss}
}
\def\phfnote@preset@opensansnote{
%    \end{macrocode}
% set up all the settings as for |sfnote| \ldots
%    \begin{macrocode}
  \phfnote@preset@sfnote
%    \end{macrocode}
% \ldots but override:
%    \begin{macrocode}
  \phfnote@opt@fontdefsfalse
  \PassOptionsToPackage{T1}{fontenc}
  \RequirePackage{fontenc}
  \PassOptionsToPackage{default,osfigures,scale=0.9}{opensans}
  \RequirePackage{opensans}
}
\def\phfnote@preset@utopianote{
  \phfnote@presetcommon@xnote[noteit]
  \phfnote@opt@fontdefsfalse
  \PassOptionsToPackage{T1}{fontenc}
  \RequirePackage{fontenc}
  \RequirePackage{fourier}
  \renewcommand{\notesectionallfontfamily}{put}
  \renewcommand{\notetitlefont}{\bfseries}
  \renewcommand{\sfdefault}{phv}
}
\def\phfnote@preset@mnmynote{
  \phfnote@presetcommon@xnote[noteit]
  \phfnote@opt@footnotedefsfalse
  \phfnote@opt@fontdefsfalse
  \PassOptionsToPackage{T1}{fontenc}
  \RequirePackage{fontenc}
  \renewcommand{\notesectionallfontfamily}{\sfdefault}
%    \end{macrocode}
%
% Require these packages AFTER the default package set, because some symbols may
% be defined in package sets, and I've had problems with re-definitions
% etc\ldots anyway this seems to work this way:
%    \begin{macrocode}
  \g@addto@macro\phfnote@hook@atendload{
    \RequirePackage{MnSymbol}
    \PassOptionsToPackage{medfamily,textosf,mathlf,minionint,footnotefigures}{MinionPro}
    \RequirePackage{MinionPro}
    \PassOptionsToPackage{medfamily}{MyriadPro}
    \RequirePackage{MyriadPro}
  }
}
%    \end{macrocode}
% \end{macro}
% \end{macro}
% \end{macro}
% \end{macro}
% \end{macro}
%
%
% \begin{macro}{\phfnote@preset@pkgdoc}
%   Preset for a package documentation.
%
%   Start by setting the same settings as for other |Xnote| presets.
%    \begin{macrocode}
\def\phfnote@preset@pkgdoc{
  \phfnote@presetcommon@xnote[noteit]
  \phfnote@opt@fontdefsfalse
%    \end{macrocode}
% 
% Then set up the font, which is done in a separate macro
% |\phfnote@pkgdoc@setupfont| in case individual documents would like more
% specific settings.  (For example, some packages may want a different math
% font.)
%    \begin{macrocode}
  \phfnote@pkgdoc@setupfont
%    \end{macrocode}
% 
% Finally, set up general appearance.
%    \begin{macrocode}
  \def\phfnote@opt@secfmt{section,paragraph,itpar,blockpar,larger,secsquares,secnummargin}
  \def\phfnote@opt@pagegeom{bigmargin}
  \def\phfnote@opt@abstract{noname}
}
%    \end{macrocode}
%
% Also provide a helper macro which is to load the font packages we
% want.  By default, we use Utopia fonts via the \pkgname{fourier}
% package, but some package documentations may want a different math
% font.  Override |\phfnote@pkgdoc@setupfont| to adjust the whole font
% set-up, or |\phfnote@pkgdoc@setupmainfont| to adjust only the main
% document font.
%    \begin{macrocode}
\providecommand\phfnote@pkgdoc@setupfont{
  \PassOptionsToPackage{T1}{fontenc}
  \RequirePackage{fontenc}
  \phfnote@pkgdoc@setupmainfont
  \renewcommand{\notesectionallfontfamily}{put}
  \renewcommand{\notetitlefont}{\bfseries}
  \IfFileExists{opensans.sty}{}{\PackageError{phfnote}{Font OpenSans is not
      available (need `opensans' package)}{Please install the opensans
      package, which provides the OpenSans font.}}
  \def\opensans@scale{s*[0.85]}
  \renewcommand{\sfdefault}{fosj}
}
\providecommand\phfnote@pkgdoc@setupmainfont{\RequirePackage{fourier}}
%    \end{macrocode}
% \end{macro}
% 
% \begin{macro}{\phfnote@preset@xpkgdoc}
%   Same as |preset=pkgdoc|, but also provide some handy hacks and
%   commands.
%
%    \begin{macrocode}
\def\phfnote@preset@xpkgdoc{
  \phfnote@preset@pkgdoc
%    \end{macrocode}
% 
% Include the \pkgname{verbdef} package, because it's always useful.
%    \begin{macrocode}
  \RequirePackage{verbdef}
%    \end{macrocode}
%
% \textbf{Some patching first:}
% Patch up |\PrintChanges| and |\PrintIndex|, if they are defined (for if we are
% using the \pkgname{ltxdoc} package for latex package documentation).  We want
% these to generate an entry in the table of contents.  Also provide the utility
% |\PrintChangesAndIndex|, which calls both |\PrintChanges| and |\PrintIndex| with
% some additional spacing.
%    \begin{macrocode}
  \ifdefined\PrintChanges
    \phfnoteHackSectionStarWithTOCInCommand\PrintChanges
  \fi
  \ifdefined\PrintIndex
    \phfnoteHackSectionStarWithTOCInCommand\PrintIndex
  \fi
  \def\PrintChangesAndIndexSpacing{\vspace{3cm plus 2cm minus 2cm}}
  \def\PrintChangesAndIndex{\PrintChangesAndIndexSpacing\PrintChanges
    \PrintChangesAndIndexSpacing\PrintIndex}
%    \end{macrocode}
%
% Set the index to TWO columns only (three is too tight).
%    \begin{macrocode}
  \ifdefined\c@IndexColumns
    \setcounter{IndexColumns}{2}
  \fi
%    \end{macrocode}
% 
% And set the glossary, that is, the list of changes history to single-column.  For this,
% renew the environment completely to remove the |multicols| environment.
%    \begin{macrocode}
  \let\phfnote@xpkgdoc@old@theglossary\theglossary
  \let\phfnote@xpkgdoc@old@endtheglossary\endtheglossary
  \renewenvironment{theglossary}{%
    \glossary@prologue%
    \GlossaryParms \let\item\@idxitem \ignorespaces}
  {}
%    \end{macrocode}
%
%
% \textbf{Hyperref:} No ``default'' \pkgname{hyperref} definitions, we'll use
% \pkgname{hyperdoc} instead.
%    \begin{macrocode}
  \phfnote@opt@hyperrefdefsfalse
  \g@addto@macro\phfnote@hook@atendload{
    \definecolor{docnotelinkcolor}{rgb}{0,0,0.4}%
    \RequirePackage{url}%
    \DeclareUrlCommand\phfnote@format@url{}%
    \RequirePackage{hypdoc}
    % 
    \hypersetup{bookmarks=true,backref=false,unicode=true,%
      bookmarksnumbered=false,bookmarksopen=false,bookmarksopenlevel=1,%
      breaklinks=true,pdfborder={0 0 0},colorlinks=true}%
    \hypersetup{%
      anchorcolor=docnotelinkcolor,citecolor=docnotelinkcolor,%
      filecolor=docnotelinkcolor,linkcolor=docnotelinkcolor,%
      menucolor=docnotelinkcolor,runcolor=docnotelinkcolor,%
      urlcolor=docnotelinkcolor}%
    \let\email\phfnote@email%
    \urlstyle{noteit}
  }
%    \end{macrocode}
%
% \textbf{Provide Macro:} |\pkgname|\marg{package name} to format a package
% name.  Also place it in the general index.  This command is robust and
% can be used in section titles etc.
%    \begin{macrocode}
  \def\pkgname##1{%
    \pkgnamefmt{##1}%
    \index{##1=\pkgnamefmt{##1}|hyperpage}%
    \index{packages:>##1=\pkgnamefmt{##1}|hyperpage}%
  }
  \robustify\pkgname
  \def\pkgnamefmt##1{\textsf{##1}}
  \robustify\pkgnamefmt
%    \end{macrocode}
% 
%
% \textbf{Provide Macros:} |\changed| and |\changedreftext|, with more
% advanced support for displaying changes in package functionality or API.
% 
% First, we need a counter for the x-ref system.
%    \begin{macrocode}
  \newcounter{phfnotechanged}
%    \end{macrocode}
% 
% Mark changes in the implementation section of the package documentation
% with the command |\changed|\oarg{label
% name}\marg{v1.0}\marg{2016/05/22}\marg{description}.  This command
% automatically adds the change to the package's change history list, and
% allows you to refer to this change anywhere else in the package doc with
% |\changedreftext|.
%    \begin{macrocode}
  \newcommand*\changed[4][]{%
%    \end{macrocode}
% 
% First, if no label is given as optional argument, then just display the
% change and add it to the package changes list.
%    \begin{macrocode}
    \if\relax\detokenize{##1}\relax%
      \changedtextfmt{##2}{##3}{##4}%
      \changes{##2}{##3}{##4}%
    \else%
%    \end{macrocode}
% 
% If a label name is provided as optional argument, then we need to write
% some stuff to the \texttt{.aux} file to make the change visible in the
% whole document.
%    \begin{macrocode}
      \protected@edef\phfnotechanged@tmpa{{##2}{##3}{##4}}%
      \immediate\write\@auxout{\string\phfnote@changed@set%
        {##1}{\expandonce\phfnotechanged@tmpa}}%
      \par\hspace*{0pt}\refstepcounter{phfnotechanged}\label{phfnotechanged:##1}%
      \begingroup\let\phfnote@changedreftext@par\relax
        \changedreftext{##1}%
      \endgroup
      \changes{##2}{##3}{\hyperref[phfnotechanged:##1]{##4}}%
    \fi
  }
  \def\phfnote@changed@set##1{%
    \expandafter\gdef\csname phfnote@changed@lbl@##1\endcsname%
  }
%    \end{macrocode}
% 
% When you document changes with the help of |\changed|, you may refer to
% any specific change from anywhere else in the package doc with the help
% of |\changedreftext|\marg{label name}.
%    \begin{macrocode}
  \def\phfnote@changedreftext@par{\par}
  \newcommand*\changedreftext[1]{%
    \phfnote@changedreftext@par%
    \ifcsname phfnote@changed@lbl@##1\endcsname
      \hyperref[phfnotechanged:##1]{%
        \expandafter\expandafter\expandafter\changedtextfmt%
            \csname phfnote@changed@lbl@##1\endcsname
      }
    \else
      \hyperref[phfnotechanged:##1]{%
        \changedtextfmt{???}{???}{[\textbf{missing ref}]}%
      }%
    \fi
    \par
  }
%    \end{macrocode}
% 
% The macro |\changedtextfmt|\marg{v1.0}\marg{2016/05/22}\marg{description}
% takes care of formatting the change on the spot.
%    \begin{macrocode}
  \newcommand*\changedtextfmt[3]{%
    \textit{Changed in {##1\kern 0.3ex\relax[##2]}:} ##3.
  }
%    \end{macrocode}
% 
% \textbf{Provide environment \phfverb{pkgoptions}:} Set up an elaborate
% environment (based on a |description| environment) to describe package
% options.
%    \begin{macrocode}
  \RequirePackage{enumitem}
  \newlist{pkgoptions}{description}{1}
  \setlist[pkgoptions]{font=\pkgoptionfmt[{\vspace*{5pt}}],style=nextline}
%    \end{macrocode}
% 
% But patch the |pkgoptions|' |\item| command, so that it puts an additional
% pair of braces around its argument.  In this way, the |font=| attribute for
% the list sees the full label as its next token, and can be used as a macro
% argument.  (This is not needed for newer versions of \pkgname{enumitem}.)
%    \begin{macrocode}
  \apptocmd\pkgoptions{\let\pkgoptions@old@item\item%
    \def\item{\@ifnextchar[\pkgoptions@item@\pkgoptions@item@@}%]
    \def\pkgoptions@item@[##1]{\pkgoptions@old@item[{{##1}}]}%
    \def\pkgoptions@item@@{\PackageWarning{phfnote}{{pkgoptions}: you must
        specify label to \string\item as \string\item[label].}%
      \pkgoptions@old@item}%
  }{}{\PackageWarning{phfnote}{preset xpkgdoc: Failed to patch command
      \string\pkgoptions}}
  \def\pkgoptionscombineitem{\leavevmode\vspace{\dimexpr-\baselineskip-\parskip-\itemsep\relax}}
%    \end{macrocode}
% 
% For convenience, also provide a |\meta|-like command for boolean arguments
% (|true| or |false|). `|\metatruefalsearg|' typesets as `\metatruefalsearg'.
%    \begin{macrocode}
  \def\metatruefalsearg{\meta{\phfverb{true} $\mid$ \phfverb{false}}}
%    \end{macrocode}
% 
% Include also a command to format a package option.  Puts the option in a box
% in typewriter text style, and indexes it.  The optional argument is meant to
% be internal---it adds commands after the displayed text (use it to add, e.g.\@
% spacing).
%
% When indexing the packages, make sure to remove the protective braces if any.
%    \begin{macrocode}
  \newcommand\pkgoptionfmt[2][]{%
    \begingroup\let\meta\pkgoptfmt@meta\fbox{\normalfont\ttfamily ##2}\endgroup%
    \expandafter\phfnote@pkgdoc@index\expandafter{\@firstofone ##2}%
    ##1}
  \let\pkgopt@save@meta\meta
  \def\pkgoptfmt@meta##1{\begingroup\normalfont\itshape\pkgopt@save@meta{##1}\endgroup}
%    \end{macrocode}
% 
% Whenever a package option is formatted with |\pkgoptionfmt|, it is placed in
% the index.  Because package options may be of the form |key=val|, we want to
% split keys from values and put them independently in the index.  This is done
% by entering a \TeX{} group, and using an |\lccode| trick: the code is prepared
% to iterate over a list of comma-separated stuff, but then the ``lowercase''
% version of that code is executed instead, where the |=|'s have been replaced
% by |,|'s.
%    \begin{macrocode}
  \def\phfnote@pkgdoc@index##1{%
    \begingroup\lccode`\= = `\,\relax%
      \def\x{\lowercase{\def\@tmpa{##1}}}%
      \x%
      \let\meta\@gobble%
      \let\marg\@gobble%
      \let\oarg\@gobble%
      \let\parg\@gobble%
      \let\pkgoptattrib\@firstofone%
      \let\pkgoptattribnodots\@firstofone%
      \let\pkgoptattribempty\@empty%
      \def\handleitemindex####1{%
        \edef\@tmpc{####1}%
        \if\relax\detokenize\expandafter{\@tmpc}\relax\else%
          \edef\@tmpb{{\expandonce\@tmpc=\string\verb!*+\expandonce\@tmpc+ (\pkgoptname)|hyperpage}}%
          \expandafter\index\@tmpb%
          \edef\@tmpb{{\packageoptionsname:>\expandonce\@tmpc=\string\verb!*+\expandonce\@tmpc+|hyperpage}}%
          \expandafter\index\@tmpb%
        \fi%
      }%
      \def\@tmpc{\forcsvlist{\handleitemindex}}%
      \expandafter\@tmpc\expandafter{\@tmpa}%
    \endgroup%
  }
  \def\pkgoptname{pkg. opt.}
  \def\packageoptionsname{package options}
%    \end{macrocode}
%
% \textbf{Provide environment \phfverb{cmdoptions}:} hijack the |pkgoptions|
% environment to do the same thing, except we place the items in the index under
% ``command options'' instead of ``package options.''
%    \begin{macrocode}
  \def\cmdoptions{\begingroup\setcmdnotpkgoptions
    \pkgoptions}
  \def\endcmdoptions{\endpkgoptions\endgroup}
  \newcommand\cmdoptionfmt[2][]{\begingroup\setcmdnotpkgoptions
    \pkgoptionfmt[{##1}]{##2}\endgroup}
  \def\cmdoptname{cmd. opt.}
  \def\commandoptionsname{command options}
  \def\setcmdnotpkgoptions{\let\pkgoptname\cmdoptname
    \let\packageoptionsname\commandoptionsname
    \let\fbox\cmdoptionsfbox}
  \def\cmdoptionsfbox##1{\ensuremath{\underline{{\text{##1}}}}}
%    \end{macrocode}
% 
% Provide the |\pkgoptattrib| command, which typesets its argument as
% |\{arg, ...\}|---useful to typeset attributes such as in
% \autoref{sec:abstract-attributes}.  The variant |\pkgoptattribondots{arg}|
% typesets |\{arg\}| while |\pkgoptattribempty| expands to |{}|.
%    \begin{macrocode}
  \def\pkgoptattrib##1{\{##1,...\}}
  \def\pkgoptattribnodots##1{\{##1\}}
  \def\pkgoptattribempty{\{\}}
%    \end{macrocode}
% 
% \textbf{Colorful boxes: environments \phfverb{pkgnote},
%   \phfverb{pkgwarning}, and \phfverb{pkgtip}.}  Now, load the
% \pkgname{tcolorbox} package to provide visual ``Note,'' ``Warning,'' and
% ``Tip'' boxes.  Because \pkgname{tcolorbox} includes the
% \pkgname{verbatim} package which messes up the |verbatim| environment in
% latex dtx files (for which source lines all start with a |%| which needs
% to be stripped), we save the |verbatim|-related commands, and restore
% them after the interfering packages have been loaded.
%    \begin{macrocode}
  \phfnoteSaveDefs{verbatimstuff}{%
    verbatim,@verbatim,@xverbatim,@sxverbatim,endverbatim}
  \usepackage{tcolorbox}
  \newtcolorbox{pkgnote}{
    colback=blue!5!white,
    colframe=blue!5!white,
    coltitle=blue!50!black,
    toptitle=1.5ex,
    fonttitle=\bfseries,
    title={NOTE}
  }
  \newtcolorbox{pkgwarning}{
    colback=red!5!white,
    colframe=red!5!white,
    coltitle=red!50!black,
    toptitle=1.5ex,
    fonttitle=\bfseries,
    title={WARNING}
  }
  \newtcolorbox{pkgtip}{
    colback=green!5!white,
    colframe=green!5!white,
    coltitle=green!50!black,
    toptitle=1.5ex,
    fonttitle=\bfseries,
    title={TIP}
  }
  \phfnoteRestoreDefs{verbatimstuff}
%    \end{macrocode}
% 
%    \begin{macrocode}
  \def\phfqitltxPkgTitle##1{The \pkgname{##1} package\thanks{\itshape
    This document corresponds to \pkgname{##1}~\fileversion, dated \filedate. It
    is part of the
    \href{https://github.com/phfaist/phfqitltx/}{\pkgname{phfqitltx}} package
    suite, see \url{https://github.com/phfaist/phfqitltx}.}}  }
%    \end{macrocode}
% \end{macro}
% 
%
% \begin{macro}{\phfnote@preset@reset}
%   Finally, the |reset| preset:
%    \begin{macrocode}
\def\phfnote@preset@reset{
  \def\phfnote@opt@pkgset{none}
  \def\phfnote@opt@title{}
  \phfnote@opt@pagegeomdefsfalse
  \phfnote@opt@spacingdefsfalse
  \def\phfnote@opt@par{original}
  \def\phfnote@opt@abstract{original}
  \phfnote@opt@hyperrefdefsfalse
  \phfnote@opt@fontdefsfalse
  \def\phfnote@opt@secfmt{}
  \phfnote@opt@bibliographydefsfalse
  \phfnote@opt@footnotedefsfalse
%    \end{macrocode}
% 
% \begin{pkgwarning}
%   SELF-NOTE: DO NOT FORGET TO ADD HERE RESET COMMANDS FOR ANY NEW OPTION THAT
%   WE PROVIDE IN THE FUTURE.
% \end{pkgwarning}
% 
%    \begin{macrocode}
}
%    \end{macrocode}
% \end{macro}
% 
%
%
%
% \subsubsection{Finally, Process and Execute the Package Options}
%
%
%
% Process the options:
%
%    \begin{macrocode}
\ProcessKeyvalOptions*
%    \end{macrocode}
%
%
% Take action according to the user options.
%
%    \begin{macrocode}
\phfnote@do@pkgset{\phfnote@opt@pkgset}

\phfnote@do@notetitle{\phfnote@opt@title}

\phfnote@do@noteabstract{\phfnote@opt@abstract}

\phfnote@do@secfmt{\phfnote@opt@secfmt}

\ifphfnote@opt@pagegeomdefs
  \phfnote@do@pagegeomdefs{\phfnote@opt@pagegeom}
\fi

\ifphfnote@opt@spacingdefs
  \phfnote@do@spacing
\fi

\phfnote@do@par{\phfnote@opt@par}

\ifphfnote@opt@hyperrefdefs
  \phfnote@do@pdfhyperrefdefs
\fi

\ifphfnote@opt@fontdefs
  \phfnote@do@fontdefs
\fi

\ifphfnote@opt@bibliographydefs
  \phfnote@do@bibliographydefs
\fi

\ifphfnote@opt@footnotedefs
  \phfnote@do@footnotedefs
\fi
%    \end{macrocode}
% 
% Finally, execute the hook we set up for definitions at the end of the package
% loading:
%    \begin{macrocode}
\phfnote@hook@atendload
%    \end{macrocode}
%
%\Finale
\endinput
