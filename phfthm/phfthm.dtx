% \iffalse meta-comment
%
% Copyright (C) 2016 by Philippe Faist, philippe.faist@bluewin.ch
% -------------------------------------------------------
% 
% This file may be distributed and/or modified under the
% conditions of the LaTeX Project Public License, either version 1.3
% of this license or (at your option) any later version.
% The latest version of this license is in:
%
%    http://www.latex-project.org/lppl.txt
%
% and version 1.3 or later is part of all distributions of LaTeX 
% version 2005/12/01 or later.
%
% \fi
%
% \iffalse
%<*driver>
\ProvidesFile{phfthm.dtx}
%</driver>
%<package>\NeedsTeXFormat{LaTeX2e}[2005/12/01]
%<package>\ProvidesPackage{phfthm}
%<*package>
    [2018/02/08 v1.0 phfthm package]
%</package>
%
%<*driver>
\documentclass{ltxdoc}
\usepackage{xcolor}
\usepackage{phfthm}
\usepackage[preset=xpkgdoc]{phfnote}
\usepackage{needspace}
\EnableCrossrefs         
\CodelineIndex
\RecordChanges
\begin{document}
  \DocInput{phfthm.dtx}
\end{document}
%</driver>
% \fi
%
% \CheckSum{0}
%
% \CharacterTable
%  {Upper-case    \A\B\C\D\E\F\G\H\I\J\K\L\M\N\O\P\Q\R\S\T\U\V\W\X\Y\Z
%   Lower-case    \a\b\c\d\e\f\g\h\i\j\k\l\m\n\o\p\q\r\s\t\u\v\w\x\y\z
%   Digits        \0\1\2\3\4\5\6\7\8\9
%   Exclamation   \!     Double quote  \"     Hash (number) \#
%   Dollar        \$     Percent       \%     Ampersand     \&
%   Acute accent  \'     Left paren    \(     Right paren   \)
%   Asterisk      \*     Plus          \+     Comma         \,
%   Minus         \-     Point         \.     Solidus       \/
%   Colon         \:     Semicolon     \;     Less than     \<
%   Equals        \=     Greater than  \>     Question mark \?
%   Commercial at \@     Left bracket  \[     Backslash     \\
%   Right bracket \]     Circumflex    \^     Underscore    \_
%   Grave accent  \`     Left brace    \{     Vertical bar  \|
%   Right brace   \}     Tilde         \~}
%
%
% \changes{v1.0}{2016/04/20}{Initial version}
%
% \GetFileInfo{phfthm.dtx}
%
% \DoNotIndex{\newcommand,\newenvironment,\def,\gdef,\edef,\xdef,\if,\else,\fi,\ifx}
% \DoNotIndex{\expandafter,\csname,\endcsname,\let}
% 
% \title{\phfqitltxPkgTitle{phfthm}}
% \author{Philippe Faist\quad\email{philippe.faist@bluewin.ch}}
% \date{\pkgfmtdate\filedate}
% \maketitle
%
% \begin{abstract}
%   \pkgname{phfthm}---Goodies for theorems and proofs.
% \end{abstract}
%
% \inlinetoc
%
% \section{Introduction}
%
% The \pkgname{phfthm} package provides enhanced theorem and proof environments,
% based on the \pkgname{amsthm} original versions.  It allows for hooks to be
% placed, adds some default goodies and is highly customizable.
%
% There are three generic types of environments provided: theorem environments,
% proof environments and ``thmheading'' environments.
%
% \subsection{Theorem environments}
%
% Theorem environments look like this:
% \begingroup\setlength{\fboxsep}{1ex}
% \par\noindent\fbox{\begin{minipage}{\dimexpr\textwidth-2\fboxsep-2\fboxrule\relax}
%   \begin{theorem}[Gauss]\noproofref
%     \label{thm:example-Gauss}
%     For a closed surface $S$ enclosing a volume $V$, we have
%     \begin{equation}
%       \oint_S\vec u\cdot d\vec S = \int_V(\vec\nabla\cdot\vec u)\,dV\ .
%     \end{equation}
%   \end{theorem}
% \end{minipage}}\endgroup
%
% \subsection{Proof environments}
%
% A proof environment might look like this:
% \begingroup\setlength{\fboxsep}{1ex}
% \par\noindent\fbox{\begin{minipage}{\dimexpr\textwidth-2\fboxsep-2\fboxrule\relax}
%   \begin{proof}[*thm:example-Gauss]
%     The proof of the theorem goes here. \par\ldots
%   \end{proof}
% \end{minipage}}\endgroup
%
% The enhanced theorem and proof environments provided by this package allow to
% pair theorems with proofs, automatically generating references from one to the
% other (see \autoref{sec:proof-ref-mechanism}).
%
%
% \subsection{Theorem-heading environments}
%
% Finally, theorem-heading environments are formatted like theorems, but the
% heading title is set as an argument to the environment.  These environments
% are a nice alternative for definitions, and look like this:
% \begingroup\setlength{\fboxsep}{1ex}
% \par\noindent\fbox{\begin{minipage}{\dimexpr\textwidth-2\fboxsep-2\fboxrule\relax}
%   \begin{thmheading}{Trace Distance}
%     The `trace distance' between $\rho$ and $\sigma$ is defined as
%     \begin{equation}
%       \delta(\rho,\sigma)=\frac12\,\left\Vert \rho - \sigma \right\Vert_1\ ,
%     \end{equation}
%     where $\lVert\cdot\rVert_1$ is the Schatten-1 norm.
%   \end{thmheading}
% \end{minipage}}\endgroup
% 
%
% 
% \section{Quick start and package options}
% \label{sec:quick-start}
% \label{sec:global-pkg-options}
%
% Example: Load the rich theorem set, with separate counters, and with proof-ref
% mechanism on and always displaying the proof reference in the margin:
% \begin{verbatim}
% \usepackage[thmset=rich,sepcounters=true,proofref={always,margin}]{phfthm}
% \end{verbatim}
%
% By default, some styles are tweaked a bit so that they appear nicely as
% documented below (for example, by using a filled square instead of a simple
% square for end-of-proof QED markers).  Use the package option
% \pkgoptionfmt{resetstyle} to instruct \pkgname{phfthm} not to proceed to these
% style adjustments; this allows you to enable features individually and
% selectively:
% \begin{verbatim}
% \usepackage[resetstyle,prooftitleitbf=true]{../phfthm}
% \end{verbatim}
% 
%
% \subsection{Predefined theorem environments}
% \label{sec:theorem-sets}
%
% You may load predefined theorem sets via the package option
% \pkgoptionfmt{thmset}.  Theorem sets group common environments used in
% mathematical works such as Theorem, Proposition, Definition, etc.
%
% Some package options control the way these environments are defined.  If you
% would like more refined control over the appearance of these environments, or
% over which environments are defined, you may consider calling |\phfLoadThmSet|
% manually or defining individual environments with |\phfMakeTheorem|.
%
% Possible theorem sets are:
% \begin{pkgoptions}
% \item[thmset=]\pkgoptionscombineitem
% \item[thmset=empty] Do not define any environment at package loading.  You may
%   of course invoke |\phfLoadThmSet| or |\phfMakeTheorem| manually at any
%   later point.
% \item[thmset=simple]
%   \DescribeEnv{theorem} \DescribeEnv{proposition}
%   \DescribeEnv{lemma} \DescribeEnv{corollary} \DescribeEnv{definition}
%   Define the environments |theorem|, |proposition|, |lemma|
%   and |corollary| as theorem-like environments, and |definition| as a
%   definition-like environment.
% \item[thmset=default] \DescribeEnv{conjecture} \DescribeEnv{remark} Define the
%   environments |theorem|, |proposition|, |lemma|, |corollary|, |conjecture|,
%   |remark| as theorem-like environments, and |definition| as a definition-like
%   environment.
% \item[thmset=shortnames] For if you like typing less: the same environments
%   are defined as the default set, but with shorter names.  Define the
%   environments |thm|, |prop|, |lem|, |cor|, |conj|, |rem| as theorem-like
%   environments, and |defn| as a definition-like environment.
% \item[thmset=rich] \DescribeEnv{idea} \DescribeEnv{question}
%   \DescribeEnv{claim} \DescribeEnv{problem} Provides the same environments as
%   the |default| theorem set, as well as in addition: |idea|, |question|,
%   |claim|, and |problem| as theorem-like environments.
% \end{pkgoptions}
%
% You may also load a theorem set at a later
% point after loading the \pkgname{phfthm} package by invoking the
% |\phfLoadThmSet| macro, see \autoref{sec:load-thm-set-manually}.
%
% Further package options modify the style of the theorem-like and
% definition-like environments defined via the \pkgoptionfmt{thmset} package
% option:
% 
% \begin{pkgoptions}
% \item[theoremstyle=\meta{theorem style name}] Use this package option to
%   specify which style to use for theorem-like environments when loading the
%   theorem set specified via the \pkgoptionfmt{thmset} package option.  The
%   theorem style name should be one of |plain|, |definition|, |remark|, or any
%   other |\newtheoremstyle|-defined theorem style (see documentation of
%   \pkgname{amsthm}).
% \item[definitionstyle=\meta{theorem style name}] Use this package option to
%   specify which style to use for definition-like environments when loading the
%   theorem set specified via the \pkgoptionfmt{thmset} package option.  The
%   theorem style name should be one of |plain|, |definition|, |remark|, or any
%   other |\newtheoremstyle|-defined theorem style (see documentation of
%   \pkgname{amsthm}).
% \end{pkgoptions}
%
% Further options control various aspects of the environments defined by
% \pkgoptionfmt{thmset}.
% \begin{pkgoptions}
% \item[sepcounters=\metatruefalsearg] Each theorem environment defined with the
%   \pkgoptionfmt{thmset} package option will use a separate counter if this
%   option is set; otherwise (the default), there is a single counter which is
%   shared by all those theorem environments.
% \end{pkgoptions}
%
% The \pkgoptionfmt{proofref} package option allows to specify a comma-separated
% list of attributes to apply to the proof reference (``proof on page XYZ'')
% displayed along with the theorem.  The following attributes may be specified:
% \begin{pkgoptions}
% \item[proofref=\pkgoptattribempty{}]\pkgoptionscombineitem
% \item[proofref=\pkgoptattribnodots{default}] Do not change the default
%   proof reference appearance.
%
% \item[proofref=false] Deactivate the proof-ref mechanism.
%
% \item[proofref=\pkgoptattrib{margin}] The proof reference is displayed in
%   the margin, instead of after the theorem.
%
% \item[proofref=\pkgoptattrib{longref}] The proof reference is displayed as
%   a full sentence (``The proof of this \meta{Theorem Name} can be found on
%   page \meta{XYZ}.'').
%
% \item[proofref=\pkgoptattrib{off}] Turn off the proof reference mechanism
%   completely for theorems defined with the \pkgoptionfmt{thmset} option.
%
% \item[proofref=\pkgoptattrib{always}] Always display the proof reference,
%   even if the proof is on the same page or on a nearby page.
%
%   Note: this option has a global effect.
%
% \item[proofref=\pkgoptattrib{onyifveryfar}] The proof reference is only
%   displayed if the proof is at least two pages back, or four pages ahead.
%
%   Note: this option has a global effect.
%
% \end{pkgoptions}
%
% \begin{pkgnote}
%   The two package options \pkgoptionfmt{proofref=\pkgoptattribnodots{always}}
%   and \pkgoptionfmt{proofref=\pkgoptattribnodots{onlyifveryfar}} apply to
%   \emph{all} theorem environments which use the proof-ref mechanism, whether
%   they have already been defined or not (see |\phfProofrefPageBackTolerance|
%   and |\phfProofrefPageAheadTolerance|).
%
%   All the other above options apply only to the theorem environments defined
%   via the \pkgoptionfmt{thmset} package option.
% \end{pkgnote}
% 
% \subsection{The proof environment}
% \label{sec:proof-env}
%
% \DescribeEnv{proof}
% By default, the \pkgname{phfthm} package overrides the |proof| environment
% with a the package's own enhanced version.  If you want to preserve the
% original \emph{AMS} environment, you should use the
% \pkgoptionfmt{proofenv=false} package option.
%
% \begin{pkgoptions}
% \item[proofenv=\metatruefalsearg] If set to |true|, then define an enhanced
%   |proof| environment when loading this package.  This will override any
%   previously existing |proof| environment such as \emph{AMS}'.
%
%   If set to |false|, no action is taken at package loading time.  You should
%   then directly use the |\phfMakeProofEnv| macro to define proof environments.
% \end{pkgoptions}
%
% If you want finer control over how the proof environment is defined, or if you
% want to customize its appearance, you should use the |\phfMakeProofEnv| macro
% directly (\autoref{sec:mk-proof-env}).
%
% If you set \pkgoptionfmt{proofenv=true}, there are a couple package options
% which alter the way the proof displays:
% \begin{pkgoptions}
% \item[smallproofs=\metatruefalsearg] If set to |true|, then proofs display in
%   a smaller font.
%
% \item[qedsymbolblacksquare=\metatruefalsearg] If set to |true|, the QED
%   end-of-proof symbol (usually ``$\square$'' with \pkgname{amsthm}) is
%   replaced by a filled square (``$\blacksquare$'').
%
% \item[prooftitleitbf=\metatruefalsearg] If set to |true|, then the proof title
%   (``Proof'' or ``Proof of Theorem 1'') is typeset in bold italic font.
% \end{pkgoptions}
%
%
% \subsection{The theorem-heading environment}
% \label{sec:thmheading-default}
% \DescribeEnv{thmheading} By default, the |thmheading| environment is provided
% by the \pkgname{phfthm} package:
% \begin{verbatim}
% \begin{thmheading}{Trace Distance}
%   The `trace distance' between $\rho$ and $\sigma$ is defined as
%   \begin{equation}
%     \delta(\rho,\sigma)=\frac12\,\left\Vert\rho-\sigma\right\Vert_1\ ,
%   \end{equation}
%   where $\lVert\cdot\rVert_1$ is the Schatten-1 norm.
% \end{thmheading}
% \end{verbatim}
% \begingroup\setlength{\fboxsep}{1ex}
% \par\noindent\fbox{\begin{minipage}{\dimexpr\textwidth-2\fboxsep-2\fboxrule\relax}
%   \begin{thmheading}{Trace Distance}
%     The `trace distance' between $\rho$ and $\sigma$ is defined as
%     \begin{equation}
%       \delta(\rho,\sigma)=\frac12\,\left\Vert \rho - \sigma \right\Vert_1\ ,
%     \end{equation}
%     where $\lVert\cdot\rVert_1$ is the Schatten-1 norm.
%   \end{thmheading}
% \end{minipage}}\endgroup
%
% You may also use |\label| and |\ref| as usual (|\ref| simply displays the
% given title).
%
% Some package options control the way this environment is defined.
%
% \begin{pkgoptions}
% \item[thmheading=\metatruefalsearg] Define the environment
%   |\begin{thmheading}...\end{thmheading}| when loading the \pkgname{phfthm}
%   package, with reasonable default settings.
% \item[thmheadingstyle=\meta{theorem style}] If \pkgoptionfmt{thmheading=true}
%   was specified, you may use this option to specify the theorem style to use
%   for the |thmheading| environment.  Possible values are \emph{AMS} theorem
%   style names (e.g.\@ the base styles |plain|, |definition| or |remark|), or
%   any other style defined with |\newtheoremstyle|.
% \end{pkgoptions}
%
% If you want to define theorem-heading environments manually, see
% \autoref{sec:thmheading-manually}.
%
%
%
% 
% \section{Theorem environments}
%
% A theorem environment is based on the environment furnished by
% \pkgname{amsthm}'s |\newtheorem| command, but with added goodies.
%
%
% \subsection{Define theorem environments manually}
%
% \DescribeMacro{\phfMakeTheorem} If you don't want to load a full theorem set
% (\autoref{sec:theorem-sets}), you can define theorem environments individually
% with |\phfMakeTheorem|:
%
% \noindent |\phfMakeTheorem|\hspace{0pt}\oarg{key-value
% options}\hspace{0pt}\marg{theorem environment name}\hspace{0pt}\marg{theorem
% name}
%
% This command defines a new environment (given as the first mandatory argument)
% which behaves as a theorem and is displayed as given by the second mandatory
% argument.  For example, we might call |\phfMakeTheorem{theorem}{Theorem}| to
% define the environment |\begin{theorem}...\end{theorem}| which displays
% ``\textit{Theorem N.} \ldots''
%
% The possible key-value options for the optional argument are:
% \begin{cmdoptions}
% \item[counter=\meta{\LaTeX{} counter $\mid$ (empty)}] The name of the \LaTeX{}
%   counter to use for the theorem environment.  If this is empty, then a new
%   counter will be created which is specific to this theorem environment (the
%   default).  If not empty, then the theorem environment uses the given counter
%   (or an alias thereof, see \cmdoptionfmt{aliascounter}).
%
%   If a counter is specified, the counter should already be defined with
%   \LaTeX's |\newcounter|.
%
% \item[aliascounter=\metatruefalsearg] In some cases
%   (e.g.\@ if you're using \pkgname{hyperref}'s |\autoref|), it is important to
%   have counters specific to each theorem environment (so you get ``Theorem 5''
%   or ``Proposition 5'' right).  However, you may want different theorem
%   environments to share a same logical counter (Say ``Definition 1'',
%   ``Definition 2'', ``Theorem 3'', ``Proposition 4'').  In this case, you
%   should specify \cmdoptionfmt{aliascounter=true}.
%
%   When this option is on, then first we define an alias counter of the one
%   given to the \cmdoptionfmt{counter} option, and then use the alias for the
%   theorem environment.  The alias is declared using the \pkgname{aliascnt}
%   package.  The alias counter is automatically set up correctly for using
%   |\autoref|.
%
%   Note that the \cmdoptionfmt{aliascounter} option only has an effect if the
%   \cmdoptionfmt{counter} option is set to some non-empty value.  If
%   \cmdoptionfmt{counter} is set to a non-empty value, then
%   \cmdoptionfmt{aliascounter} defaults to |true|.
%
% \item[thmstyle=\meta{theorem style $\mid$ (empty)}] The theorem style to use
%   to define this theorem environment.  The value of this option should be a
%   valid argument to \textit{AMS}'s |\theoremstyle|.  If you leave this empty
%   (the default), then the theorem style is not set explicitly and whatever
%   default style is used.
%
% \item[defnostar=\metatruefalsearg] Set this to
%   |true| if you want the corresponding non-starred theorem environment to be
%   defined, e.g.\@ |\begin{theorem}...\end{theorem}|.
%
%   Normal (non-starred) versions of the environments have an associated theorem
%   number, as you expect by default.
%
% \item[defstar=\metatruefalsearg] Set this to
%   |true| if you want the corresponding starred theorem environment to be
%   defined, e.g.\@ |\begin{theorem*}...\end{theorem*}|.
%
%   Starred versions of the environments do not have an associated theorem
%   number.
%
% \item[proofref=\metatruefalsearg] Enable or disable
%   the proof-ref mechanism for this theorem environment (enabled by default).
%
% \item[proofrefstyle=\meta{proof-ref style}] The style to use for the proof
%   references.  Here you may specify how the proof ref appears, for example (in
%   the margin, long sentence, ...).  Possible styles are
%   \cmdoptionfmt{proofrefstyle=default} (the default),
%   \cmdoptionfmt{proofrefstyle=margin} (display the proof ref in the margin of
%   the page) and \cmdoptionfmt{proofrefstyle=longref} (as by default but with a
%   full sentence).  See \autoref{sec:proof-ref-customize-appearance} for how to
%   further customize the appearance of the proof reference.
%
% \end{cmdoptions}
%
% For example, you may use the following command invocation to define a theorem
% environment named ``Remark'' implemented as |\begin{remark}...\end{remark}|,
% also with a starred verison |\begin{remark*}...\end{remark*}|, using the
% |plain| \emph{AMS} theorem style, and without the proof-ref mechanism:
% \begin{verbatim}
% \phfMakeTheorem[defstar=true,defnostar=true,thmstyle=plain,counter=,%
%     proofref=false]{remark}{Remark}
% \end{verbatim}
%
%
%
% \subsection{Loading theorem sets manually}
% \label{sec:load-thm-set-manually}
%
% \DescribeMacro{\phfLoadThmSet} You may load theorem sets at any time via the
% macro |\phfLoadThmSet|.  This may be useful, for example, to load theorem sets
% only after you have defined a custom theorem style.  The syntax of
% |\phfLoadThmSet| is:
%
% \noindent |\phfLoadThmSet|\hspace{0pt}\marg{options to
% \phfverb{\phfLoadThmSet} for theorem-like
% environments}\hspace{0pt}\marg{options to \phfverb{\phfLoadThmSet} for
% definition-like environments}\hspace{0pt}\marg{name of theorem set to load}
%
% The first and second argument to this macro are tokens to expand in front of
% |\phfMakeTheorem| for theorem-like or definition-like environments.  For
% example:
% \begin{verbatim}
% \newcounter{mythmcounter}
% \newtheoremstyle{mythmstyle}{...}
% \newtheoremstyle{mydefnstyle}{...}
% \phfLoadThmSet{[thmstyle=mythmstyle,counter=mythmcounter]}
%     {[thmstyle=mydefnstyle,counter=mythmcounter]}{rich}
% \end{verbatim}
% 
% \begin{pkgwarning}
%   The first and second arguments to |\phfLoadThmSet| must either be empty,
%   or be enclosed in square braces.
% \end{pkgwarning}
%
% \needspace{10\baselineskip}
% \DescribeMacro{\theoremname}
% \DescribeMacro{\propositionname}
% \DescribeMacro{\lemmaname}
% \DescribeMacro{\corollaryname}
% \DescribeMacro{\conjecturename}
% The title of the theorem environments defined in theorem sets use the same
% scheme as figures, tables, etc.\@ with regard to translations and
% \pkgname{babel}: they use |\theoremname|, |\propositionname|, etc.
% 
% \DescribeMacro{\remarkname}
% \DescribeMacro{\definitionname}
% \DescribeMacro{\ideaname}
% \DescribeMacro{\questionname}
% \DescribeMacro{\claimname}
% \DescribeMacro{\problemname}
%
% This package is language agnostic (with titles defined by default in English),
% and does not provide the titles for other languages.  In order to support
% language switching with \pkgname{babel} and |\selectlanguage|, you should add
% the relevant names to the corresponding |\captions|\meta{language name} macro,
% for example:
% \begin{verbatim}
% \usepackage[francais,...]{babel}
% ...
% \addto\captionsfrancais{%
%   \def\theoremname{Th\'eor\`eme}%
%   \def\propositionname{Proposition}%
%   \def\lemmaname{Lemme}%
%   \def\corollaryname{Corollaire}%
%   \def\conjecturename{Conjecture}%
%   \def\remarkname{Remarque}%
%   \def\definitionname{D\'efinition}%
%   \def\ideaname{Id\'ee}%
%   \def\questionname{Question}%
%   \def\claimname{Affirmation}%
%   \def\problemname{Probl\`eme}%
% }
% ... \selectlanguage{francais} ...
% \end{verbatim}
%
% 
% \subsection{Theorem hooks}
% \label{sec:theorem-hooks}
%
% Any theorem environment automatically calls some hooks.  There are hooks
% available per theorem environment as well as generic for all theorem
% environments.
%
% \DescribeMacro{\phfthm@hook@start@thmname} The hook
% |\phfthm@hook@start@|\meta{theorem environment name}\marg{theorem title} is
% called at the start of the environment.  More precisely, it is called inside
% the original \pkgname{amsthm} base environment; that is, after the heading was
% generated.  It takes one mandatory argument, the optional title provided to
% the theorem environment which may be empty.  By default, the hook defers to
% the global hook |\phfthm@hook@startcommonnostar|.
%
% \DescribeMacro{\phfthm@hook@start@thmname*} The hook
% |\phfthm@hook@start@|\meta{starred theorem environment name} is completely
% analogous, and is called for the starred environment.  The only difference is
% that by default, it defers its call to |\phfthm@hook@startcommonstar|.
%
% \DescribeMacro{\phfthm@hook@startcommonnostar} The hook
% |\phfthm@hook@startcommonnostar|\marg{theorem environment name}\marg{theorem
% optional given title} collects the default definitions for non-starred
% environments (none by default) and continues to defer to
% \DescribeMacro{\phfthm@hook@startcommon}
% |\phfthm@hook@startcommon|\marg{theorem environment name}\marg{theorem
% optional given title}.  \DescribeMacro{\phfthm@hook@startcommonstar}
% Analogously, the macro |\phfthm@hook@startcommonstar|\marg{theorem environment
% name}\marg{theorem optional given title} groups commands for starred
% environments (typically doesn't take care of |\label| stuff) and also defers
% to |\phfthm@hook@startcommon|.
%
% The end hooks work very much
% analogously. \DescribeMacro{\phfthm@hook@end@thmname}
% |\phfthm@hook@end@|\meta{theorem environment name} and
% \DescribeMacro{\phfthm@hook@end@thmname*} |\phfthm@hook@end@|\meta{starred
% theorem environment name} are called respectively for the non-starred and
% starred version of that theorem environment, and by default they defer to the
% common \DescribeMacro{\phfthm@hook@endcommonnostar}
% |\phfthm@hook@endcommonnostar|\marg{theorem environment name} or
% \DescribeMacro{\phfthm@hook@endcommonstar}
% |\phfthm@hook@endcommonstar|\marg{theorem environment name}.  Both these hooks
% defer their calls to |\phfthm@hook@endcommon|\marg{theorem environment name}.
%
% For theorems using the proof-reference mechanism, i.e.\@ for which
% \cmdoptionfmt{proofref=true} was specified to |\phfMakeTheorem| and which uses
% the |\label| hack (\autoref{sec:proof-ref-mechanism}), there is an additional
% hook.  \DescribeMacro{\phfthm@hook@afterlabel@thmname} The hook
% |\phfthm@hook@afterlabel@|\meta{theorem environment name} is called just after
% the |\label| command corresponding to the theorem is encountered (this should
% always be at the \emph{beginning} of the theorem, see
% \autoref{sec:proof-ref-mechanism}).  Depending on the proof-ref style, this
% hook may be used to generate the proof reference text (for example, with the
% |margin| proof-ref style).  The hook is called after the theorem label is set.
% The label itself can be recovered from the value of the macro
% |\phfthm@val@thmlabel|.  By default, that hook calls the common hook
% \DescribeMacro{\phfthm@hook@afterlabelcommon}
% |\phfthm@hook@afterlabelcommon|\marg{theorem environment name}. (After the
% first occurrence of the command |\label|, the latter's definition is
% restored.)
% 
%
%
% \section{Proof environments}
%
% Proof environments typeset mathematical proofs.  The proof environment(s)
% provided by \pkgname{phfthm} give some added functionality with respect to the
% \emph{AMS}-default |proof| environment, such as supporting the proof-reference
% mechanism described in \autoref{sec:proof-ref-mechanism}.
%
% A proof environment might look like the following:
% \begin{proof}[Theorem 5]
%   Let $\mathcal{T}_{X\to X'}$ be any trace nonincreasing completely positive
%   map such that $\mathcal{T}_{X\to X'}\left(\Gamma_X\right)$ lies within the
%   support of $\Gamma_{X'}$. Define the normalized state
%   $\gamma_X = \Gamma_X / \operatorname{tr}\Gamma_X$.
%
%   Now consider this and that \ldots
% \end{proof}
%
%
% The proof environments defined by this package wrap a given proof display
% environment (such as \emph{AMS}' (\pkgname{amsthm}'s) or \pkgname{IEEEtran}'s
% original |proof| environment) by adding functionality in the form of hooks.
% In the following, we refer to the ``underlying proof display environment'' as
% the original environment which is wrapped.  It may be any \LaTeX{} environment
% whose task is to format the proof nicely.
%
%
% \subsection{Manually define a proof environment}
% \label{sec:mk-proof-env}
%
% \DescribeMacro{\phfMakeProofEnv} You may use the macro |\phfMakeProofEnv| to
% declare a new proof environment. The syntax is:
%
% \noindent |\phfMakeProofEnv|\oarg{key-value options}\marg{proof environment name}
%
% This defines a new environment with the given name, which may be used to
% display proofs to theorems.  The options may be:
% \begin{cmdoptions}
% \item[displayenv=\meta{name of \LaTeX{} environment}] Set a
%   \LaTeX{} environment to use to actually format and display the proof.  (The
%   |\phfMakeProofEnv| command itself doesn't care about how the proof is
%   displayed or formatted; rather it adds a goodies infrastructure in which
%   stuff can be plugged in and provides options for such goodies.)
%
%   You may specify here the name of a \LaTeX{} environment, or give the special
%   value \cmdoptionfmt{displayenv=*} to indicate the default appearance
%   provided by \pkgname{phfthm}, or leave the value empty
%   \cmdoptionfmt{displayenv=} to signify that no underlying display environment
%   should be invoked.  (The latter may be useful if you are plugging a
%   |\phfMakeProofEnv|-generated environment into a larger environment which
%   already takes care of the display.)
%
% \item[defaultproofname=\meta{default proof title}] Specify here the title to
%   use (e.g.\@ ``Proof'') if no argument was given to the proof environment.
%   If you do not specify any |defaultproofname|, or pass an empty value, then
%   the value of |\proofname| is used.
%
% \item[parselabel=\metatruefalsearg] Specify whether
%   the environment should parse its argument for some special information.  If
%   set to |true|, then the proof argument is passed on to a command (specified
%   by the \cmdoptionfmt{parselabelcmd} option).
%
% \item[parselabelcmd=\meta{{\LaTeX} macro}] If \cmdoptionfmt{parselabel} is set
%   to |true|, then specify here a \LaTeX{} command which parses whatever it
%   wants from the proof environment's argument.  The macro should set the
%   |\phfthm@val@displayargs| macro to tokens which will be expanded just after
%   the invocation of the proof environment's display environment
%   (\cmdoptionfmt{displayenv}). It should also set |\phfthm@val@proofoflabel|
%   (if appropriate) to the label corresponding to the theorem for which this is
%   the proof of.
%
%   By default, the command |\phfthm@proof@parselabel| is used, which parses the
%   proof environment's argument for a reference to a theorem in the context of
%   a proof-ref mechanism (see \autoref{sec:proof-ref-mechanism}).  The label is
%   parsed to see if it is of the form |[*thm:reference]|, where |thm:reference|
%   is the label pinned to a theorem.
%
% \item[override=\metatruefalsearg] Whether to
%   override any existing environment with the same name as the new proof
%   environment. If |true| is specified here, then |\renewenvironment| is used
%   to define the proof environment, otherwise a simple |\newenvironment| is
%   used.
%
% \item[internalcounter=\meta{name of \LaTeX{} counter}] The name of the
%   internal counter the proof environment should use.  The count number is not
%   displayed (by default at least), but it is only used to pin down anchors for
%   PDF hyperlinks.
%
%   The counter should already be defined with |\newcounter|.
%
% \item[proofofname=\meta{\LaTeX{} macro}] Specify here a macro
%   which will be called with a single argument. The macro produces the text to
%   display when the proof environment is parsed as the proof of a specific
%   theorem or proposition (or other theorem environment).  The argument which
%   will be given to it is the title of what the proof is of (e.g.\@ ``Theorem
%   3''). Typically, the macro should produce something like ``Proof of Theorem
%   XYZ.''
%
%   By default, the global macro |\proofofname| is used.
% \end{cmdoptions}
%
% \DescribeMacro{\proofname} Text to use to display ``Proof.''  This should be
% already defined by the \LaTeX{} system, and \pkgname{babel} should already
% provide translations in different languages.
%
% \DescribeMacro{\proofofname} The globally defined macro |\proofofname|
% specifies the default way of displaying ``Proof of Theorem~5.''  It is
% originally defined as something like
% \begin{verbatim}
% \newcommand\proofofname[1]{\proofname{} of #1}
% \end{verbatim}
% You may override this to obtain something fancier, of you wish to display the
% document in a different language:
% \begin{verbatim}
% \def\proofofnamefrancais#1{\proofname{} (#1)}
% \addto\captionsfrancais{\let\proofofname\proofofnamefrancais}
% ...
% \selectlanguage{francais} ...
% \end{verbatim}
%
%
% 
% \subsection{Proof hooks}
% \label{sec:proof-hooks}
%
% The proof hooks are relatively straightforward.  All hooks presented here take
% no argument.
%
% Information about the argument of the proof, both the raw argument and the
% possibly parsed reference, are available as macros to some of the hooks (but
% don't change these values unless you know what you're doing).  The macro
% |\phfthm@val@proofarg| contains the raw argument to the proof environment, and
% is available to all hooks.  If you use the default proof environment argument
% parsing (which you must have enabled when calling |\phfMakeProofEnv|), then
% additionally the macros |\phfthm@val@prooftitle| and |\phfthm@val@proofofname|
% are available containing, respectively, the label of the theorem which is
% referenced, and the displayable reference to it (e.g. ``Theorem~5'').  The
% last two macros are available to all hooks except the first one (|..@start|).
%
% The hooks named |..@start...| are called within the call to
% |\beg||in{<proof environment>}|.
%
% \expandafter\DescribeMacro\expandafter{\csname phfthm@hookproof@...@start\endcsname}
% The hook named |\phfthm@hookproof@|\meta{environment name}|@start| is called
% at the very beginning of the proof environment.
%
% \expandafter\DescribeMacro\expandafter{\csname phfthm@hookproof@...@startafterdisplay\endcsname}
% 
% The hook named |\phfthm@hookproof@|\meta{environment name}|@startafterdisplay|
% is invoked immediately after the beginning of the underlying ``display''
% environment (the environment used to display the proof contents).
%
% \expandafter\DescribeMacro\expandafter{\csname phfthm@hookproof@...@startlast\endcsname}
% 
% The hook named |\phfthm@hookproof@|\meta{environment name}|@startlast| is
% called after we are sure that an anchor has been pinned down for the proof.
% This hook is called last within the commands in |\beg||in{<proof environment>}|.
%
% \DescribeMacro{\phfPinProofAnchor} By the way, the macro |\phfPinProofAnchor|
% may be used within the hooks to pin down an anchor for referring to the proof
% (especially via the proof-ref mechanism).  Just call it anywhere appropriate
% (a good idea is calling it after leaving v-mode before displaying the title,
% in order to avoid placing it just before a page break).  If you do not call
% this macro, it is automatically called for you just before the |...@startlast|
% hook.
%
%
% The two following hooks are called within the call to%
% |\en||d{<proof environment>}|.
%
% \expandafter\DescribeMacro\expandafter{\csname phfthm@hookproof@...@end\endcsname}
% The hook |\phfthm@hookproof@|\meta{environment name}|@end| is called before
% the proof display environment is closed.
%
% \expandafter\DescribeMacro\expandafter{\csname phfthm@hookproof@...@final\endcsname}
% The hook |\phfthm@hookproof@|\meta{environment name}|@final| is called after
% the proof environment display is finished, as the very last.
%
% All proof hooks call are defined by default to defer their call to a common
% hook.  The common hooks each take one argument (the proof environment name).
% They are named |\phfthm@hookproof@startcommon|\marg{environment name},
% |\phfthm@hookproof@startafterdisplaycommon|\marg{environment name},
% |\phfthm@hookproof@startlastcommon|\marg{environment name},
% |\phfthm@hookproof@endcommon|\marg{environment name}, and
% |\phfthm@hookproof@finalcommon|\marg{environment name}.
% They are all defined to be empty by default.
%
%
% \section{Pairing theorems to proofs and proof-reference mechanism}
% \label{sec:proof-ref-mechanism}
%
% One of the goodies provided by the \pkgname{phfthm} package is the proof-ref
% mechanism, where in a theorem environment, the text ``see proof on page
% \ldots'' is displayed to direct the reader to the location of the
% corresponding proof.  The mechanism is deactivated by default, but can be
% enabled with a simple package option.
%
% This only works if the proof is given the label of the corresponding theorem
% or proposition.  For example:
% \begin{verbatim}
% \begin{theorem}[Gauss]
%   \label{thm:Gauss}
%   For a closed surface $S$ enclosing a volume $V$, we have
%   \begin{equation}
%     \oint_S\vec u\cdot d\vec S = \int_V(\vec\nabla\cdot\vec u)\,dV\ .
%   \end{equation}
% \end{theorem}
%
% ...
%
% \begin{proof}[*thm:Gauss]
%   ...
% \end{proof}
% \end{verbatim}
%
% The above example might produce the following output:
% \begingroup\setlength{\fboxsep}{1em}
% \par\noindent\fbox{%
%   \begin{minipage}{\dimexpr\textwidth-2\fboxsep-2\fboxrule\relax}
%     \textbf{Theorem 17} (Gauss). For a closed surface $S$ enclosing a
%     volume $V$, we have
%     \setcounter{equation}{41}
%     \begin{equation}
%       \oint_S\vec u\cdot d\vec S = \int_V(\vec\nabla\cdot\vec u)\,dV\ .
%     \end{equation}
%     \hfill{\small\itshape (Proof on page XXX.)}\hfilneg
%     {\par\vspace{1ex}\relax
%     \ldots\par\vspace{1ex}}\relax
%     \par\textit{Proof of Theorem 17.}\hspace{2em}\ldots
%   \end{minipage}
% }\endgroup
%
% \subsection{On the theorem side}
%
% On the theorem side, the proof-ref mechanism works by hacking into the
% definition of \LaTeX's |\label|.  The |\label| command should be placed first
% within the theorem (see example above).  It is important, in theorems which
% use the proof-ref mechanism (on by default), to always have a corresponding
% label: Indeed, you may experience weird results if you don't have a theorem
% label, but then have labels for other objects in the theorem such as equations
% or itemize items.
%
% Once the corresponding proof is detected (a proof environment with an optional
% argument of the form |[*thm:the-label]| for the same label |thm:the-label| as
% specified to the theorem, see \autoref{sec:proof-ref-mechanism-proof-side}),
% then a text is generated (by default ``Proof on page \ldots'') and placed
% after the theorem. The appearance of this text is customizable
% (\autoref{sec:proof-ref-customize-appearance}).
%
% More precisely, the hack with the |\label| command works as follows: At the
% beginning of the theorem, the |\label| command is redefined so that at its
% fist occurrence, it stores its argument as the theorem's label to use for the
% proof reference, it then pins down a \LaTeX{} label as the original |\label|
% command would do, and finally it calls the |...@afterlabel| theorem hook (see
% \ref{sec:theorem-hooks}).  After the first occurrence of |\label|, the
% |\label| command is restored to its original \LaTeX{} meaning in case there
% are other objects within the theorem which are to be referred to.
%
% \begin{pkgtip}
%   The |\label| hack is only active within theorem environments where the
%   proof-ref mechanism has been enabled.  Outside these environments, the
%   |\label| macro retains its original \LaTeX{} definition.
% \end{pkgtip}
%
% \DescribeMacro{\noproofref} If, for any reason, you do not want to make sure
% you don't have any text ``Proof on page \ldots'' appearing (for example there
% is no corresponding proof because the theorem is obvious), then you should
% call |\noproofref| immediately inside the theorem:
% \begin{verbatim}
% \begin{theorem}
%   \noproofref
%   Theorem text ...
% \end{theorem}
% \end{verbatim}
%
% The command |\noproofref| temporarily disables the proof-ref mechanism (and
% restores |\label| to \LaTeX's original meaning) for the current theorem.
%
%
% \subsection{On the proof side}
% \label{sec:proof-ref-mechanism-proof-side}
%
% On the proof side, you just need to specify for which theorem this is the
% proof of.  For that (unless you override the defaults and plug in your own
% magic parsing; see \autoref{sec:mk-proof-env}), you should specify an optional
% argument to the proof which is of the following form:
% |\begin{proof}[*|\meta{label}|]|,
%   \iffalse meta-comment \end{proof} [-- emacs is confused] \fi
% where \meta{label} is the label name you have associated with the theorem in
% question (see example above).
%
% This has two effects: it sets the proof to display ``\textit{Proof of
% \ldots},'' and also does some background dark magic to display, at the
% location of the corresponding theorem, some text like ``\textit{Proof on page
% \ldots},'' where the page number corresponds to the page on which this proof
% is located.
%
%
% \subsection{Customizing appearance of the proof reference text}
% \label{sec:proof-ref-customize-appearance}
%
% Here we explain the workings of the |\phfthm@proofrefstyle@...| macros and how
% they are called.  It allows you to define new proof-ref styles, for example.
%
% When the option \cmdoptionfmt{proofref=true} is given to |\phfMakeTheorem| to
% define an environment (say |mytheoremenv|), then the hook
% |phfthm@hook@start@mytheoremenv| will automatically include the following
% calls:
% \begin{itemize}
% \item The macro |\phfthm@proofrefstyle@|\meta{proof-ref style}|@setup| is called
%   (for the proof-ref style given via the
%   \cmdoptionfmt{proofenvstyle=\meta{style name}} key-value option to
%   |\phfMakeTheorem|);
% \item The macro |\phfthm@def@label@thmlabel| is invoked, implementing the hack
%   on the |\label| macro;
% \item The macro |\phfthm@proofref@impl@start| is called.  This macro is
%   expected to be defined after calling |\phfthm@proofrefstyle@|\meta{proof-ref
%   style}|@setup|.
% \end{itemize}
% 
% Furthermore, the |\phfthm@hook@afterlabel@mytheoremenv| hook will include a
% call to |\phfthm@proofref@impl@afterlabel|\marg{label of the theorem}.  Again,
% the latter macro is expected to be defined after calling
% |\phfthm@proofrefstyle@|\meta{proof-ref style}|@setup|.
%
% Finally, the hook |\phfthm@hook@end@mytheoremenv| includes a call to
% |\phfthm@proofref@impl@end|\marg{label of the theorem}.  Again,
% the latter macro is expected to be defined after calling
% |\phfthm@proofrefstyle@|\meta{proof-ref style}|@setup|.
%
%
% Hence, to define a new proof-ref style, you simply need to define a macro
% called |\phfthm@proofrefstyle@<PROOF-REF-STYLE-NAME>@setup|.  This macro
% should include commands to locally define the macros
% |\phfthm@proofref@impl@start|, |\phfthm@proofref@impl@afterlabel|, and
% |\phfthm@proofref@impl@end|.
%
%
% Different proof-ref styles may work similarly and want to share most of the
% code.  A good idea is to build up on the |default| proof-ref style, which is
% highly modular and can be instantiated in different flavors.  For an example,
% check the |margin| proof-ref style which does precisely that.  For more
% documentation, check out the implementation of the |default| proof-ref style
% in \autoref{sec:impl-default-proof-ref-style}.
%
%
%
%
% \section{Theorem-heading definition-like environments}
% \label{sec:thmheading}
%
% A theorem-heading environment is an environment which displays in the same way
% as a theorem environment, but where the title may be any text (say, ``Trace
% Distance'' instead of, e.g., ``Theorem~5'').
%
% By default, the \pkgname{phfthm} package provides the |thmheading| environment
% (see \autoref{sec:thmheading-default}).
%
%
% \subsection{Define theorem-heading environments manually}
% \label{sec:thmheading-manually}
% 
% \DescribeMacro{\phfMakeThmheadingEnvironment} A new theorem-heading
% environment can be defined by calling |\phfMakeThmheadingEnvironment|.
% The syntax is:
%
% \noindent |\phfMakeThmheadingEnvironment|\oarg{key-value options}\marg{environment name}
%
% The key-value options may be any combination of the following:
% \begin{cmdoptions}
% \item[thmstyle=\meta{theorem style name}] The theorem style to use to display
%   the environment.  You may specify here any default \emph{AMS} style
%   (|plain|, |remark| or |definition|), or any other |\newtheoremstyle|-defined
%   style.
% \item[internalcounter=\meta{name of counter}] The name of a counter which will
%   internally track environment instances.  By default, a common internal
%   counter is used for all theorem-heading environments (named
%   |phfthmheadingcounter|).  The counter must be already defined (see \LaTeX's
%   |\newcounter|).
% \end{cmdoptions}
%
%
% You can also use |\label| and |\ref| (the latter simply displays the given
% title).
%
% \subsection{Available hooks for theorem-heading environments}
% 
% The hook |\phfthm@hook@thmheading@...@start|\marg{title} is invoked at start,
% within an internal environment created with |\newtheorem|.  This hook accepts
% one argument, the title of the theorem-heading.
%
% The hook |\phfthm@hook@thmheading@...@end| is called at the end, but still
% within the internal theorem environment.
%
% Replace the dots with the name of the theorem-heading environment (such as
% |thmheading|).
%
% By default, these hooks simply call the common hooks
% |\phfthm@hook@thmheading@start| and |\phfthm@hook@thmheading@end|.  These
% common hooks are empty by default.
%
%
%
%
%
% \StopEventually{\PrintChangesAndIndex}
%
%
%
% \section{Implementation}
% \label{sec:implementation}
%
% First, load some packages.  General toolboxes:
%    \begin{macrocode}
\RequirePackage{xkeyval}
\RequirePackage{etoolbox}
%    \end{macrocode}
% 
% To define alias counters for theorems, load \pkgname{aliascnt}:
%    \begin{macrocode}
\RequirePackage{aliascnt}
%    \end{macrocode}
% 
% And finally, load the \textit{AMS} math and theorem (\pkgname{amsmath},
% \pkgname{amsthm}) packages:
%    \begin{macrocode}
\RequirePackage{amsmath}
\RequirePackage{amsthm}
%    \end{macrocode}
% 
%
% \subsection{Generic Internal Stuff}
%
% \begin{macro}{\phfthm@internal@execattribs}
%
%   Internal command: execute all definitions given in list of attributes.  This was
%   copy-pasted from a similar definition in the \pkgname{phfnote} package.
%
%   |#1| = prefix to look for attributes
%   
%   |#2| = name of what |#1| represents, to use in message in case attribute is not found
%   
%   |#3| = list of attributes
%
%    \begin{macrocode}
\def\phfthm@internal@execattribs#1#2#3{%
  \@for\next:=#3\do{%
    \ifcsname #1\next\endcsname%
      \csname #1\next\endcsname%
    \else%
      \PackageWarning{phfthm}{Unknown #2: '\next'. Ignoring.}
    \fi
  }
}
%    \end{macrocode}
% \end{macro}
% 
%
% \subsection{Definitions for theorem environments}
%
% \subsubsection{\phfverb{\phfMakeTheorem}: definition of a new theorem environment}
%
% First, define some key-value syntax accepted by the |\phfMakeTheorem| command.
%    \begin{macrocode}
\define@cmdkey{phfmkthm}{counter}{}
\define@boolkey{phfmkthm}{aliascounter}[true]{}
\define@cmdkey{phfmkthm}{thmstyle}{}
\define@boolkey{phfmkthm}{defnostar}[true]{}
\define@boolkey{phfmkthm}{defstar}[true]{}
\define@boolkey{phfmkthm}{proofref}[true]{}
\define@cmdkey{phfmkthm}{proofrefstyle}{}
%    \end{macrocode}
% 
% \begin{macro}{\phfMakeTheorem}
%   Define a new theorem environment.  The syntax is
%   |\phfMakeTheorem|\hspace{0pt}\oarg{options}\hspace{0pt}\marg{theorem
%   environment name}\hspace{0pt}\marg{Theorem Display Name}.  For example:
%   |\phfMakeTheorem[counter=thmcounter]{prop}{Proposition}|
%    \begin{macrocode}
\newcommand\phfMakeTheorem[3][]{% }
%    \end{macrocode}
% Handle the [options].  First, ensure that the defaults are set, and then, parse the input.
%    \begin{macrocode}
  \KV@phfmkthm@aliascountertrue%
  \def\cmdKV@phfmkthm@counter{}%
  \def\cmdKV@phfmkthm@thmstyle{}%
  \KV@phfmkthm@defnostartrue%
  \KV@phfmkthm@defstartrue%
  \KV@phfmkthm@proofreftrue%
  \def\cmdKV@phfmkthm@proofrefstyle{default}%
  \setkeys{phfmkthm}{#1}%
%    \end{macrocode}
% 
% Now, react to whatever was given in the options.
%
% \verbdef\tmptheifconstruct|\if\relax\detokenize{...}\relax|
% Set the theorem style, if requested.\footnote{The construct
% \tmptheifconstruct\space tests whether \phfverb{...} is empty: see
% \url{http://tex.stackexchange.com/a/53091/32188}}
%    \begin{macrocode}
  \if\relax\detokenize\expandafter{\cmdKV@phfmkthm@thmstyle}\relax%
  \else%
    \theoremstyle{\cmdKV@phfmkthm@thmstyle}%
  \fi%
%    \end{macrocode}
% 
% If requested, define the default, unstarred version of the theorem.  Use
% |\newtheorem| for that, which we make sure to call appropriately depending on
% whether a separate counter is requested or not.  Make sure also to define
% |\...autorefname| for |\autoref|.  If an alias counter is requested, create it
% and pass that one to |\newtheorem|.
%
% At this point, we create a theorem named |phfthm@...| using |\newtheorem|
% (because we still want to add calls to hooks).
%    \begin{macrocode}
  \ifKV@phfmkthm@defnostar%
    \if\relax\detokenize\expandafter{\cmdKV@phfmkthm@counter}\relax%
%    \end{macrocode}
% ---in case we use a separate counter (if \cmdoptionfmt{counter=}):
%    \begin{macrocode}
      \newtheorem{phfthm@#2}{#3}%
      \csdef{phfthm@#2autorefname}{#3}%
    \else%
      \ifKV@phfmkthm@aliascounter%
%    \end{macrocode}
% ---in case we make a distinct alias counter, eg. for use with |\autoref|:
%    \begin{macrocode}
        \newaliascnt{#2}{\cmdKV@phfmkthm@counter}%
        \newtheorem{phfthm@#2}[#2]{#3}%
        \aliascntresetthe{#2}%
        \csdef{#2autorefname}{#3}%
      \else%
%    \end{macrocode}
% ---in case we directly instruct |\newtheorem| to use the other counter (does not work with |\autoref|):
%    \begin{macrocode}
        \newtheorem{phfthm@#2}[\cmdKV@phfmkthm@counter]{#3}%
      \fi%
    \fi%
%    \end{macrocode}
%
% And also define the actual theorem environment, adding calls to hooks.
%    \begin{macrocode}
    \newenvironment{#2}[1][]{%
      \begin{phfthm@#2}[{##1}]%
        \begingroup%
          \csname phfthm@hook@start@#2\endcsname{##1}%
      }{%
          \csname phfthm@hook@end@#2\endcsname%
        \endgroup%
      \end{phfthm@#2}%
    }%
%    \end{macrocode}
% Define hooks specific to this theorem with sensible defaults.  If proof-ref is
% on, call the appropriate callbacks. Then, call the common hooks (see
% |\phfthm@hook@startcommonnostar|, |\phfthm@hook@afterlabelcommon| and
% |\phfthm@hook@endcommonnostar|, detailed in \autoref{sec:theorem-hooks}).
%    \begin{macrocode}
    \csedef{phfthm@hook@start@#2}##1{%
      \ifKV@phfmkthm@proofref%
        \expandafter\noexpand%
          \csname phfthm@proofrefstyle@\cmdKV@phfmkthm@proofrefstyle @setup\endcsname%
        \noexpand\phfthm@def@label@thmlabel{#2}%
        \noexpand\phfthm@proofref@impl@start%
      \fi%
      \noexpand\phfthm@hook@startcommonnostar{#2}{##1}%
    }%
    \csedef{phfthm@hook@afterlabel@#2}{%
      \ifKV@phfmkthm@proofref%
        \noexpand\phfthm@proofref@expandthmlabeltoarg%
          \noexpand\phfthm@proofref@impl@afterlabel%
      \fi%
      \noexpand\phfthm@hook@afterlabelcommon{#2}%
    }%
    \csedef{phfthm@hook@end@#2}{%
      \ifKV@phfmkthm@proofref%
        \noexpand\phfthm@proofref@expandthmlabeltoarg%
          \noexpand\phfthm@proofref@impl@end%
      \fi%
      \noexpand\phfthm@hook@endcommonnostar{#2}%
    }%
  \fi%
%    \end{macrocode}
% 
%
% If requested, define the starred version of the theorem.  We call
% |\newtheorem*| to define the base theorem environment (which we call
% |phfthm@...|), after which as above we define the actual environment which
% also calls the relevant hooks.
%    \begin{macrocode}
  \ifKV@phfmkthm@defstar%
    \newtheorem*{phfthm@#2*}{#3}%
    \newenvironment{#2*}[1][]{%
      \begin{phfthm@#2*}[##1]%
        \begingroup%
          \csname phfthm@hook@start@#2*\endcsname{##1}%
      }{%
          \csname phfthm@hook@end@#2*\endcsname%
        \endgroup%
      \end{phfthm@#2*}%
    }%
  \fi%
%    \end{macrocode}
% 
% Finally, define the default hooks specific to the starred version of the
% theorem (see \autoref{sec:theorem-hooks}).
%    \begin{macrocode}
  \csdef{phfthm@hook@start@#2*}##1{\phfthm@hook@startcommonstar{#2}{##1}}%
  \csdef{phfthm@hook@end@#2*}{\phfthm@hook@endcommonstar{#2}}%
}
%    \end{macrocode}
% \end{macro}
% 
%
% \subsubsection{Default hooks for theorems}
%
% \needspace{3\baselineskip}
% \begin{macro}{\phfthm@hook@startcommonnostar}
% \begin{macro}{\phfthm@hook@startcommonstar}
% \begin{macro}{\phfthm@hook@startcommon}
% Common default hooks definitions for start of the theorems.
%
% For all three of these hooks, we have
% |#1| = theorem name, e.g. |proposition| and 
% |#2| = full (optional) title of proposition, if given, or empty.
%
% Make sure to invoke the |\label| re-definition hack only for non-starred
% theorems/propositions; indeed, if no theorem label is set we don't want to
% interfere with labels set to inner equations, itemizes etc.  Hence, call
% |\phfhtm@def@label@thmlabel| only in the ``nostar'' hook.
%    \begin{macrocode}
\def\phfthm@hook@startcommonnostar#1#2{%
  \phfthm@hook@startcommon{#1}{#2}%
}
\def\phfthm@hook@startcommonstar#1#2{%
  \phfthm@hook@startcommon{#1}{#2}%
}
\def\phfthm@hook@startcommon#1#2{%
%    \end{macrocode}
% Furthermore, in any case, set the |\postdisplaypenalty| to avoid an orphan
% line on a new page after display equation.
%    \begin{macrocode}
  \postdisplaypenalty=10000\relax%
}
%    \end{macrocode}
% \end{macro}
% \end{macro}
% \end{macro}
% 
% \begin{macro}{\phfthm@hook@afterlabelcommon}
% \begin{macro}{\phfthm@hook@endcommonnostar}
% \begin{macro}{\phfthm@hook@endcommonstar}
% \begin{macro}{\phfthm@hook@endcommon}
%   Further hooks, for after the theorem main |\label| command
%   (|\phfthm@hook@afterlabelcommon|) and for the end of the theorem.
%    \begin{macrocode}
\def\phfthm@hook@afterlabelcommon#1{}
\def\phfthm@hook@endcommonnostar#1{\phfthm@hook@endcommon{#1}}
\def\phfthm@hook@endcommonstar#1{\phfthm@hook@endcommon{#1}}
\def\phfthm@hook@endcommon#1{}
%    \end{macrocode}
% \end{macro}
% \end{macro}
% \end{macro}
% \end{macro}
%
% \subsubsection{Proof-ref mechanism (on the theorem side)}
%
% These macros enable the proof-ref mechanism
% (\autoref{sec:proof-ref-mechanism}).  The theorem's label is stored upon
% calling |\label|, because we (locally) hack into the definition of |\label|.
% (After the first usage of |\label| its meaning is restored.)
% \begin{macro}{\phfthm@def@label@thmlabel}
%   Main macro to invoke at the beginning of the theorem environment, so that
%   the theorem label is stored in a local macro once |\label| is invoked.  This
%   hacks the |\label| macro locally.  Here, |#1| = the theorem environment
%   name, e.g.\@ |proposition|.
%    \begin{macrocode}
\def\phfthm@def@label@thmlabel#1{%
  \ifdefined\phfthm@old@label
    \PackageWarning{phfthm}{Internal inconsistency: \string\phfthm@def@label@thmlabel
      called twice for the same theorem environment!}
  \else
    \let\phfthm@old@label\label%
    \edef\label{\noexpand\phfthm@thmlabel{#1}}%
  \fi
}
%    \end{macrocode}
% \end{macro}
% 
% \begin{macro}{\phfthm@thmlabel}
%   The first call to |\label| within the theorem redirects to the macro
%   |\phfthm@thmlabel|.  (Applies to theorem environments for which
%   |\phfthm@def@label@thmlabel| was called, which is the default).
%
%   Here |#1| = theorem environment name, e.g.\@ |proposition|; and |#2| = the
%   label value (argument to the |\label| macro).
%    \begin{macrocode}
\def\phfthm@thmlabel#1#2{%
%    \end{macrocode}
% First, store the label value into a macro called |\phfthm@val@thmlabel|.
%    \begin{macrocode}
  \def\phfthm@val@thmlabel{#2}%
%    \end{macrocode}
% Then, call the original |\label| macro to do what \LaTeX\space would normally
% do for a |\label{|\ldots|}| call.
%    \begin{macrocode}
  \phfthm@old@label{#2}%
%    \end{macrocode}
% Restore the old |\label| definition, in case there are other items in the
% theorem environment such as equations, itemizes etc.\@ which may themselves
% have |\label|'s.
%    \begin{macrocode}
  \let\label\phfthm@old@label%
%    \end{macrocode}
% Invoke the |\phfthm@hook@afterlabel@thmname| hook for this theorem
% environment.
%    \begin{macrocode}
  \csname phfthm@hook@afterlabel@#1\endcsname%
%    \end{macrocode}
% Finally, ignore any spaces following the |\label| command. (Maybe we should
% have done something with |\@bsphack| and |\@esphack| but oh well\ldots
%    \begin{macrocode}
  \ignorespaces%
}
%    \end{macrocode}
% \end{macro}
% 
%
% \subsection{Definitions for proof environments}
%
% Improved, smarter |proof| environments.
%
% \begin{macro}{\phfthm@old@proof}
% \begin{macro}{\endphfthm@old@proof}
%   Save old |proof| environment provided by \pkgname{amsthm}.
%    \begin{macrocode}
\let\phfthm@old@proof\proof
\let\endphfthm@old@proof\endproof
%    \end{macrocode}
% \end{macro}
% \end{macro}
%
% \begin{macro}{\proofname}
% And provide a default name for proofs (this should normally already be
% provided by \pkgname{amsthm}).
%    \begin{macrocode}
\providecommand\proofname{Proof}
%    \end{macrocode}
% \end{macro}
% 
% \begin{macro}{\proofofname}
%   Default text to display when we want to say e.g.\@ ``Proof of Theorem 3.''
%    \begin{macrocode}
\def\proofofname#1{\proofname\space of #1}
%    \end{macrocode}
% \end{macro}
%
% The default counter for proofs.  The value of this counter is typically not
% displayed, we just use it to pin down anchors for labels for cross references.
%    \begin{macrocode}
\newcounter{phfthmproofcnt}
%    \end{macrocode}
% 
% Utility: to see if an argument was specified (possibly empty) to the proof
% environment.
%    \begin{macrocode}
\def\phfthm@NOPROOFARG{}
\def\phfthm@test@NOPROOFARG{\phfthm@NOPROOFARG}
%    \end{macrocode}
% 
% \subsubsection{Define a proof environment: \phfverb{\phfMakeProofEnv}}
%
% Declare some key-value options accepted by |\phfMakeProofEnv|.  See
% \autoref{sec:mk-proof-env} for the documentation of these options.
%    \begin{macrocode}
\define@cmdkey{phfmkprf}{displayenv}{}
\define@cmdkey{phfmkprf}{defaultproofname}{}
\define@boolkey{phfmkprf}{override}[true]{}
\define@cmdkey{phfmkprf}{internalcounter}{}
\define@cmdkey{phfmkprf}{proofofname}{}
\define@boolkey{phfmkprf}{parselabel}[true]{}
\define@cmdkey{phfmkprf}{parselabelcmd}{}
%    \end{macrocode}
%
% \begin{macro}{\phfMakeProofEnv}
%   Make a proof environment. Syntax:
%   |\phfMakeProofEnv|\hspace{0pt}\oarg{options}\hspace{0pt}\marg{proof
%   environment name}.
%    \begin{macrocode}
\newcommand\phfMakeProofEnv[2][]{%}
%    \end{macrocode}
%
% Parse the key-value options.  First, make sure that all the defaults are set,
% then parse the options.
%    \begin{macrocode}
  \KV@phfmkprf@overridefalse%
  \def\cmdKV@phfmkprf@displayenv{*}%
  \def\cmdKV@phfmkprf@defaultproofname{\proofname}%
  \def\cmdKV@phfmkprf@internalcounter{phfthmproofcnt}%
  \def\cmdKV@phfmkprf@proofofname{\proofofname}%
  \KV@phfmkprf@parselabeltrue
  \def\cmdKV@phfmkprf@parselabelcmd{\phfthm@proof@parselabel}%
  \setkeys{phfmkprf}{#1}%
%    \end{macrocode}
% 
% The meaning of the options are detailed in \autoref{sec:mk-proof-env}.
% 
% The general idea here is first to pre-process all the options, and save all
% the useful information in macros named
% |\phfthm@prfenv@<proof-environment-name>@val@<something>|.  Then, we can define
% the begin/end environment macros which will recall the saved information.
%
% Take care of the display environment to use.  Recall that if |displayenv=*|,
% we use our own default; if |displayenv=| (empty), there is no display
% environment.  Here, we set
% |\phfthm@prfenv@<proof-environment-name>@val@displayenv| to the name of the
% environment to use (possibly empty), for later reference.
%    \begin{macrocode}
  \def\phfmkprf@tmp@star{*}%
  \ifx\cmdKV@phfmkprf@displayenv\phfmkprf@tmp@star\relax%
    \def\cmdKV@phfmkprf@displayenv{phfthm@proof@defaultdisplayenv}%
  \fi
  \cslet{phfthm@prfenv@#2@val@displayenv}\cmdKV@phfmkprf@displayenv%
%    \end{macrocode}
% 
% Process the default proof name.  If none is given, use |\proofname| and pass
% no option to the underlying display environment whenever the proof environment
% is called with no option.  Here, we set
% |\phfthm@prfenv@<proof-environment-name>@val@defaultproofnameargs| and
% |\phfthm@prfenv@<proof-environment-name>@val@setdefaultprooftitle|; the former
% is the tokens to put in front of the proof environment invocation in case no
% explicit proof title is given to the proof environment while the latter
% contains the command to set |\phfthm@val@prooftitle| to the default proof
% name.
%    \begin{macrocode}
  \if\relax\detokenize\expandafter{\cmdKV@phfmkprf@defaultproofname}\relax%
    \csdef{phfthm@prfenv@#2@val@defaultproofnameargs}{}%
    \csdef{phfthm@prfenv@#2@val@setdefaultprooftitle}{%
      \def\phfthm@val@prooftitle{\proofname}}%
  \else
    \csedef{phfthm@prfenv@#2@val@defaultproofnameargs}{%
      [\expandonce{\cmdKV@phfmkprf@defaultproofname}]}%
    \csedef{phfthm@prfenv@#2@val@setdefaultprooftitle}{%
      \noexpand\def\noexpand\phfthm@val@prooftitle{%
        \expandonce{\cmdKV@phfmkprf@defaultproofname}}}%
  \fi
  \csedef{phfthm@prfenv@#2@val@parselabelandmkdisplayargs}##1{%
    \ifKV@phfmkprf@parselabel
      \expandonce\cmdKV@phfmkprf@parselabelcmd{##1}%
    \else
      \noexpand\phfthm@proof@noparselabel{##1}%
    \fi
    \noexpand\def\noexpand\phfthm@val@displayargs{[{%
        \expandafter\noexpand\csname phfthm@prfenv@#2@val@proofofname\endcsname
        {\noexpand\phfthm@val@prooftitle}%
      }]}%
  }
%    \end{macrocode}
% 
% Store the macro which creates the ``Proof of \ldots'' text (|proofofname|
% option).
%    \begin{macrocode}
  \cslet{phfthm@prfenv@#2@val@proofofname}\cmdKV@phfmkprf@proofofname%
%    \end{macrocode}
% 
% Create the macro which will take care of pinning down the label for the
% proof-ref (see \autoref{sec:proof-ref-mechanism}). This macro first ref-steps
% the internal counter and then pins down a label, if appropriate.
%    \begin{macrocode}
  \csdef{phfthm@prfenv@#2@val@pinproofanchor}{%
    \csname phfthm@prfenv@#2@val@refstepinternalcounter\endcsname%
    \if\relax\detokenize\expandafter{\phfthm@val@proofoflabel}\relax\else%
      \edef\phfthm@tmp@larg{{proof:\phfthm@val@proofoflabel}}%
      \expandafter\label\phfthm@tmp@larg%
    \fi
  }%
  %
%    \end{macrocode}
% 
% The command to ref-step the internal proof counter.  Use the value of the
% \cmdoptionfmt{internalcounter} command option.
%    \begin{macrocode}
  \csedef{phfthm@prfenv@#2@val@refstepinternalcounter}{%
    \noexpand\refstepcounter{\cmdKV@phfmkprf@internalcounter}}%
%    \end{macrocode}
% 
% Make macros |\phfthm@prfenv@<proof-environment-name>@val@displayenvbegincmd|
% and |\phfthm@prfenv@<proof-environment-name>@val@displayenvendcmd|, which
% essentially expand to |\begin{<the-display-env>}| and
%   |\end{<the-display-env>}| for the display environment given in the option
% \cmdoptionfmt{displayenv}.
%    \begin{macrocode}
  \if\relax\detokenize\expandafter{\cmdKV@phfmkprf@displayenv}\relax%
    \csdef{phfthm@prfenv@#2@val@displayenvbegincmd}##1{}%
    \csdef{phfthm@prfenv@#2@val@displayenvendcmd}##1{}%
  \else
    \csedef{phfthm@prfenv@#2@val@displayenvbegincmd}##1{%
      \noexpand\begin{\csname phfthm@prfenv@#2@val@displayenv\endcsname}##1}%
    \csedef{phfthm@prfenv@#2@val@displayenvendcmd}##1{%
      \noexpand\end{\csname phfthm@prfenv@#2@val@displayenv\endcsname}##1}%
  \fi
%    \end{macrocode}
% 
% See if we need to call |\newenvironment| or |\renewenvironment|, depending on
% the value of the \cmdoptionfmt{override} option.
%    \begin{macrocode}
  \def\phfthm@tmp@defcmd{\newenvironment}%
  \ifKV@phfmkprf@override\def\phfthm@tmp@defcmd{\renewenvironment}\fi%
%    \end{macrocode}
% 
% Finally, (re-)define the environment.  The default value of the optional
% argument is the token |\phfthm@NOPROOFARG|, which indicates that no argument
% was provided.
%
% Start by storing the value of the argument into a macro, and then call the
% ``start'' hook (see proof hooks in \autoref{sec:proof-hooks}).
%    \begin{macrocode}
  \phfthm@tmp@defcmd{#2}[1][\phfthm@NOPROOFARG]{%
    \def\phfthm@val@proofarg{##1}%
    \csname phfthm@hookproof@#2@start\endcsname%
%    \end{macrocode}
% 
% First, parse the optional argument into proof label (maybe) and proof title.
% If no optional argument was given, don't give any argument to the underlying
% display environment.  If an empty argument was given, set some defaults;
% otherwise, use the necessary command to potentially parse the label and create
% the proper arguments for the underlying display environment.
%    \begin{macrocode}
    \ifx\phfthm@val@proofarg\phfthm@test@NOPROOFARG\relax%
      \def\phfthm@val@proofoflabel{}%
      \csname phfthm@prfenv@#2@val@setdefaultprooftitle\endcsname%
      \letcs\phfthm@val@displayargs{phfthm@prfenv@#2@val@defaultproofnameargs}%
    \else%
      \if\relax\detokenize{##1}\relax%
        \def\phfthm@val@proofoflabel{}%
        \csname phfthm@prfenv@#2@val@setdefaultprooftitle\endcsname%
        \def\phfthm@val@displayargs{[{%
            \csname phfthm@prfenv@#2@val@proofofname\endcsname
            {\phfthm@val@prooftitle}%
          }]}%
      \else
        \csname phfthm@prfenv@#2@val@parselabelandmkdisplayargs\endcsname{##1}%
      \fi
    \fi%
%    \end{macrocode}
% 
% Define the |\phfPinProofAnchor| command (locally) in case the
% display formatting environment takes care of where to place the anchor already.
%    \begin{macrocode}
    \def\phfPinProofAnchor{%
      \csname phfthm@prfenv@#2@val@pinproofanchor\endcsname%
      \global\let\phfPinProofAnchor\relax}%
%    \end{macrocode}
% [Also provide the obsolete |\phfthmPinProofAnchor| which I previously had in
% older versions of this package:]
%    \begin{macrocode}
    \def\phfthmPinProofAnchor{\phfPinProofAnchor}%
%    \end{macrocode}
% 
% Start the proof's display environment.  Don't be fooled here by the curly
% braces after |\x|, it only protects the argument to the
% |\phfthm@prfenv@#2@val@displayenvbegincmd| command itself: the
% |\phfthm@val@displayargs| are still just tokens which will be expanded in
% front of the |\beg||in{<proof-display-env>}| command.
%    \begin{macrocode}
    \def\x{\csname phfthm@prfenv@#2@val@displayenvbegincmd\endcsname}%
    \expandafter\x\expandafter{\phfthm@val@displayargs}%
%    \end{macrocode}
% And call the corresponding hook:
%    \begin{macrocode}
    \csname phfthm@hookproof@#2@startafterdisplay\endcsname%
%    \end{macrocode}
% If required, pin anchor after the proof-display-environment.
% (|\phfPinProofAnchor| auto-destructs after first use, so it's safe to
% potentially call it a second time here). Then, call the corresponding hook.
%    \begin{macrocode}
    \phfPinProofAnchor%
    \expandafter\noexpand\csname phfthm@hookproof@#2@startlast\endcsname%
  }%
%    \end{macrocode}
%
% Now, the definitions for the ``end'' part of the environment.  Just call the
% relevant hooks and close the display environment.
%    \begin{macrocode}
  {%
    \expandafter\noexpand\csname phfthm@hookproof@#2@end\endcsname%
    \csname phfthm@prfenv@#2@val@displayenvendcmd\endcsname
    \expandafter\noexpand\csname phfthm@hookproof@#2@final\endcsname%
  }%
%    \end{macrocode}
% 
% Finally, define the default values of the proof-environment-specific hooks.
% These just call the corresponding global hooks (see \autoref{sec:proof-hooks}).
%    \begin{macrocode}
  \csdef{phfthm@hookproof@#2@start}{\phfthm@hookproof@startcommon{#2}}%
  \csdef{phfthm@hookproof@#2@startafterdisplay}{%
    \phfthm@hookproof@startafterdisplaycommon{#2}}%
  \csdef{phfthm@hookproof@#2@startlast}{\phfthm@hookproof@startlastcommon{#2}}%
  \csdef{phfthm@hookproof@#2@end}{\phfthm@hookproof@endcommon{#2}}%
  \csdef{phfthm@hookproof@#2@final}{\phfthm@hookproof@finalcommon{#2}}%
}
%    \end{macrocode}
% \end{macro}
% 
%
% \subsubsection{Common hooks for proofs}
%
% The hooks are documented in \autoref{sec:proof-hooks}.
%    \begin{macrocode}
\def\phfthm@hookproof@startcommon#1{}
\def\phfthm@hookproof@startafterdisplaycommon#1{}
\def\phfthm@hookproof@startlastcommon#1{}
\def\phfthm@hookproof@endcommon#1{}
\def\phfthm@hookproof@finalcommon#1{}
%    \end{macrocode}
% 
%
% \subsubsection{Default display environment for proofs}
%
% \begin{environment}{phfthm@proof@defaultdisplayenv}
%   Provide an environment which displays a proof in a similar fashion as
%   \emph{AMS}', but with some small additional features.
%    \begin{macrocode}
\newenvironment{phfthm@proof@defaultdisplayenv}[1][\proofname]{%
  \par
  \pushQED{\qed}%
  \normalfont \topsep6\p@\@plus6\p@\relax
  \trivlist\item\relax
  \phfPinProofAnchor
  \phfthm@ProofTitleFmt{#1}%
  \phfthm@ProofTitleHspace
  \ignorespaces
}{%
  \popQED\endtrivlist\@endpefalse
}
%    \end{macrocode}
% \end{environment}
% 
% \begin{macro}{\phfthm@ProofTitleFmt}
% \begin{macro}{\phfthm@ProofTitleHspace}
%   These macros may be overridden to change the proof title appearance.
%    \begin{macrocode}
\def\phfthm@ProofTitleFmt#1{%
  {\itshape #1.}%
}
\def\phfthm@ProofTitleHspace{%
  \hspace{1.5ex plus 0.5ex minus 0.2ex}%
}
%    \end{macrocode}
% \end{macro}
% \end{macro}
%
% \subsubsection{Parsing the proof argument}
%
% These macros parse the argument of the proof environment to see if it is of
% the form |*<some-label>| (see \autoref{sec:proof-ref-mechanism}).
%
% \begin{macro}{\phfthm@proof@parselabel}
%   Call |\phfthm@proof@parselabel|\marg{proof environment argument} to parse
%   the argument string.  This macro will set |\phfthm@val@proofoflabel| and
%   |\phfthm@val@prooftitle| to appropriate values (respectively, the label name
%   of the corresponding theorem and a representative title such as ``Theorem
%   6'').
%    \begin{macrocode}
\def\phfthm@proof@parselabel#1{%
  \phfthm@proof@parselabel@maybelabel#1\phfthm@proof@parselabel@END%
}
\def\phfthm@proof@parselabel@maybelabel{%
  \@ifnextchar*\phfthm@proof@parselabel@label\phfthm@proof@parselabel@title%
}
\def\phfthm@proof@parselabel@label*#1\phfthm@proof@parselabel@END{%
%    \end{macrocode}
% The use of |\detokenize| here is a trick to make sure that all chars in the
% label text have a non-active category (e.g.\@ we would have problems, e.g., if
% in the label ``|thm:gauss|'' the ``|:|'' is an active char---such as in French):
%    \begin{macrocode}
  \edef\phfthm@val@proofoflabel{\detokenize{#1}}%
  \def\phfthm@val@prooftitle{\phfthm@autoref{#1}}%
}
\def\phfthm@proof@parselabel@title#1\phfthm@proof@parselabel@END{%
  \def\phfthm@val@proofoflabel{}%
  \def\phfthm@val@prooftitle{#1}%
}
%    \end{macrocode}
% \end{macro}
% \begin{macro}{\phfthm@proof@noparselabel}
%   Enjoys the same syntax as |\phfthm@proof@parselabel|, i.e., it is a drop-in
%   replacement for the latter, except that it invariably sets
%   |\phfhtm@val@proofoflabel| to an empty value and |\phfthm@val@prooftitle| to
%   the argument itself.  You could use this as a \cmdoptionfmt{parselabelcmd}
%   macro if you didn't want to parse the label.
%    \begin{macrocode}
\def\phfthm@proof@noparselabel#1{%
  \def\phfthm@val@proofoflabel{}%
  \def\phfthm@val@prooftitle{#1}%
}
%    \end{macrocode}
% \end{macro}
% 
%
% In order to look up what we are a proof of, we use |\autoref| provided by the
% \pkgname{hyperref} package.  If it is not available, fall back to the regular
% |\ref| command.
%    \begin{macrocode}
\def\phfthm@autoref{\ref}
\AtBeginDocument{%
  \@ifpackageloaded{hyperref}{\def\phfthm@autoref{\autoref}}{}
}
%    \end{macrocode}
% 
%
% \subsection{Implementation of the proof-ref machinery}
%
% \subsubsection{Small general stuff}
%
% \begin{macro}{\proofonname}
%   The macro |\proofonname| displays ``Proof on \ldots.''  Here, |#2| is the
%   full page reference and |#1| is the label name of the referenced theorem.
%    \begin{macrocode}
\providecommand\proofonname[2]{Proof on #2.}
%    \end{macrocode}
% \end{macro}
% 
% \begin{macro}{\proofrefsize}
%   Format the proof reference ``Proof on page \ldots''.  This macro is meant to
%   set the font size (or other font properties), but it may also be defined to
%   take one argument, the proof reference text.
%    \begin{macrocode}
\def\proofrefsize{\footnotesize}
%    \end{macrocode}
% \end{macro}
% 
%
% \begin{macro}{\noproofref}
%   Use |\noproofref| inside a theorem to signify that no proof reference should
%   be attempted.
%
%   The implementation just defines |\phfthm@val@noproofref|.  If this macro is
%   defined, then no proof ref should be generated for the current thmlabel.
%   Also, restore |\label| to its original definition in case it was overridden.
%    \begin{macrocode}
\def\noproofref{%
  \def\phfthm@val@noproofref{1}%
  \ifdefined\phfthm@old@label \let\label\phfthm@old@label \fi%
}
%    \end{macrocode}
% \end{macro}
%
% 
% \begin{macro}{\phfthm@proofref@warnnolabel}
%   Produce a warning that no label was provided in order to infer the proof
%   reference.
%    \begin{macrocode}
\def\phfthm@proofref@warnnolabel{%
  \PackageWarning{phfthm}{No label provided for proof reference!}%
}
%    \end{macrocode}
% \end{macro} 
%
% \begin{macro}{\phfthm@proofref@expandthmlabeltoarg}
%   Utility to expand the value of |\phfthm@val@thmlabel| as an argument to a
%   callback command.  |#1| = the macro to relay the call to.
%    \begin{macrocode}
\def\phfthm@proofref@expandthmlabeltoarg#1{%
%    \end{macrocode}
%
% First, check if the proof-ref mechanism was explicitly temporarily disabled,
% and do nothing if that is the case.
%    \begin{macrocode}
  \ifdefined\phfthm@val@noproofref\relax%
  \else%
%    \end{macrocode}
%
% Then make sure |\phfthm@val@thmlabel| is defined (maybe empty), and then
% either call the callback macro |#1| with the value of |\phfthm@val@thmlabel|
% as argument, or generate a warning if that value is empty.
%    \begin{macrocode}
    \providecommand\phfthm@val@thmlabel{}%
    \edef\phfthm@tmpa{{\phfthm@val@thmlabel}}%
    \expandafter\notblank\phfthm@tmpa{%
      \expandafter#1\phfthm@tmpa%
    }{%
      \phfthm@proofref@warnnolabel% no label provided
    }%
  \fi%
}
%    \end{macrocode}
% \end{macro}
%
%
% \subsubsection{Utilities for interacting with \phfverb{\autoref} labels}
%
% In this context, we also need some generic utilities for interacting with
% |\autoref| labels.
%
% \begin{macro}{\phfthm@autorefnameof}
%   The macro |\phfthm@autorefnameof| extracts the name of the counter which
%   generated this reference (e.g.\@ ``section'' or ``theorem'').
%    \begin{macrocode}
\def\phfthm@autorefnameof#1{%
%    \end{macrocode}
% 
% Extract the counter part of the reference |section.NN|, which is 4th element in the
% |\r@label| macro. (Code extracted from |hyperref.sty|.)
%    \begin{macrocode}
  \expandafter\ifx\csname r@#1\endcsname\relax%
    \textbf{??}%
  \else%
    \expandafter\expandafter\expandafter\phfthm@HyPsd@autorefname%
        \csname r@#1\endcsname{}{}{}{}\@nil%
  \fi%
}
\def\phfthm@HyPsd@autorefname#1#2#3#4#5\@nil{%
  \ifx\\#4\\%
  \else%
    \phfthm@HyPsd@@autorefname#4.\@nil%
  \fi%
}
\def\phfthm@HyPsd@@autorefname#1.#2\@nil{%
  \ltx@IfUndefined{#1autorefname}{%
    \ltx@IfUndefined{#1name}{%
    }{%
      \csname#1name\endcsname%
    }%
  }{%
    \csname#1autorefname\endcsname%
  }%
}
%    \end{macrocode}
% \end{macro}
% 
% \begin{macro}{\phfthm@min@pageref}
%   A minimal pageref macro, which just extracts the page number on which the
%   given label is located.
% 
%   The dark magic going on here is beyond me. The code was copied from
%   |hyperref.sty|, in ``|\def\HyPsd@@@pageref...|'' and seems to work.
% 
%    \begin{macrocode}
\def\phfthm@min@pageref#1{%
  \ifcsname r@#1\endcsname%
    \expandafter\expandafter\expandafter\expandafter
    \expandafter\expandafter\expandafter\@car
    \expandafter\expandafter\expandafter\@gobble
    \csname r@#1\endcsname{}\@nil
  \else%
    0%
  \fi%
}
%    \end{macrocode}
% \end{macro}
% 
%
% \subsubsection{Default proof-ref style, with basic machinery}
% \label{sec:impl-default-proof-ref-style}
%
% Now we define the relevant callbacks for the default style.  See documentation
% in \autoref{sec:proof-ref-customize-appearance}.  Recall a proof-ref style
% just needs to define |\phfthm@proofrefstyle@<stylename>@setup|, which in turn
% should just define the callbacks |\phfthm@proofref@impl@start|,
% |\phfthm@proofref@impl@afterlabel| and |\phfthm@proofref@impl@end|.  For our
% default style, these callbacks further call other callbacks of the form
% |\phfthm@proofref@impl@...|, such that these definitions can be re-used to
% create new styles.  The main proof-ref generation routine is
% |\phfthm@proofrefstyle@default@main|, which can be used for either the
% |...@afterlabel| or the |...@end| callback.
%
% \begin{macro}{\phfthm@proofrefstyle@default@fmt}
%   Format and display the proof reference. |#1| = the theorem's label
%   (e.g. |prop:1|); |#2| = the full reference (e.g. ``page XYZ'').
%
%   This macro is the default value of the callback |\phfthm@proofref@impl@fmt|,
%   which is called by the default style itself.
%
%   Use correct spacing for right-aligning the reference.\footnote{Thanks
%   \url{http://tex.stackexchange.com/a/43239/32188}!}  If there is room on the
%   current line, just right-align the proof-ref text; if not, add it on a
%   separate line.  [We can achieve this with the sequences |\hfil\null\hfil|:
%   if there is space, it all fits on the same line, if not, the line breaks at
%   the |\null| point.]
%    \begin{macrocode}
\def\phfthm@proofrefstyle@default@fmt#1#2{%
  {\parfillskip=0pt\relax%
    \hfil\null\hfil\null\hfil%
    \hbox{\proofrefsize{(\proofonname{#1}{#2})}}\par}%
}
%    \end{macrocode}
% \end{macro}
% 
% 
% \begin{macro}{\phfthm@proofrefstyle@default@fmtfarback}
% \begin{macro}{\phfthm@proofrefstyle@default@fmtfarahead}
% \begin{macro}{\phfthm@proofrefstyle@default@fmtcloseby}
%   These macros are the default values of the callbacks
%   |\phfthm@proofref@impl@fmtfarback|, |\phfthm@proofref@impl@fmtfarahead|, and
%   |\phfthm@proofref@impl@fmtcloseby|, which are called by the default style
%   itself.  These callbacks define how to format and (possibly not) display the
%   proof reference depending on whether the proof is ``far behind'' (several
%   pages back), ``far ahead'' (several pages ahead) or ``close by'' (neither
%   far back nor far ahead), as defined by |\phfProofrefPageBackTolerance|
%   and |\phfProofrefPageAheadTolerance|.
%    \begin{macrocode}
\def\phfthm@proofrefstyle@default@fmtfarback#1#2{%
  \phfthm@proofref@impl@fmt{#1}{#2}}
\def\phfthm@proofrefstyle@default@fmtfarahead#1#2{%
  \phfthm@proofref@impl@fmt{#1}{#2}}
\def\phfthm@proofrefstyle@default@fmtcloseby#1#2{}
%    \end{macrocode}
% \end{macro}
% \end{macro}
% \end{macro}
%
% \begin{macro}{\phfProofrefPageBackTolerance}
% \begin{macro}{\phfProofrefPageAheadTolerance}
%   The macros |\phfProofrefPageBackTolerance| and
%   |\phfProofrefPageAheadTolerance| define how many pages back or ahead the
%   proof should be in order to consider it ``far back'' or ``far ahead.''
%
%   Either value may be set to |-1| to force the proof to be considered ``far
%   back'' or ``far ahead.''
%    \begin{macrocode}
\newcommand\phfProofrefPageBackTolerance{1}
\newcommand\phfProofrefPageAheadTolerance{1}
%    \end{macrocode}
% \end{macro}
% \end{macro}
%
% Define the internal counter which allows to check on which page we are at the
% place of the proof reference.  This is used by
% |\phfthm@proofrefstyle@default@main|.
%    \begin{macrocode}
\newcounter{phfthmInternalProofrefCounter}
%    \end{macrocode}
% 
% \begin{macro}{\phfthm@proofrefstyle@default@main}
%   The main proof-ref generation routine.  The argument |#1| is the current
%   label of the theorem; the referenced label is |proof:#1|.
%    \begin{macrocode}
\def\phfthm@proofrefstyle@default@main#1{%
%    \end{macrocode}
% 
% Check to see if the proof is far away ahead or back (as defined by the
% tolerance macros above).  Depending on each case, call the corresponding
% callbacks.{\makeatletter\footnote{See \url{http://tex.stackexchange.com/a/2526} to test
% whether ref is on same page. Note that was problematic, probably due to
% hyperref. I needed to use my own \phfverb{\phfthm@min@pageref}
% without any hyper linking mechanism in place.}}
%
%    \begin{macrocode}
  \refstepcounter{phfthmInternalProofrefCounter}%
  \label{internalproofref\thephfthmInternalProofrefCounter}%
  \edef\phfthm@proofref@tmp@proofpage{\phfthm@min@pageref{proof:#1}}%
  \edef\phfthm@proofref@tmp@thispage{%
    \phfthm@min@pageref{internalproofref\thephfthmInternalProofrefCounter}}%
  \edef\phfthm@proofref@tmp@pagediff{%
    \the\numexpr\phfthm@proofref@tmp@proofpage-\phfthm@proofref@tmp@thispage\relax}%
%    \end{macrocode}
% 
% If the proof is ``far back,'' call the corresponding callback. 
%    \begin{macrocode}
  \ifnum\numexpr\phfthm@proofref@tmp@pagediff\relax%
      <\numexpr-\phfProofrefPageBackTolerance\relax%
    \phfthm@proofref@impl@fmtfarback{#1}{\autopageref{proof:#1}}%
  \else%
%    \end{macrocode}
% 
% If the proof is ``far ahead,'' call the corresponding callback. 
%    \begin{macrocode}
    \ifnum\numexpr\phfthm@proofref@tmp@pagediff\relax%
        >\numexpr\phfProofrefPageAheadTolerance\relax%
      \phfthm@proofref@impl@fmtfarahead{#1}{\autopageref{proof:#1}}%
%    \end{macrocode}
% 
% Otherwise, it is close by.
%    \begin{macrocode}
    \else%
      \phfthm@proofref@impl@fmtcloseby{#1}{\autopageref{proof:#1}}%
    \fi%
  \fi%
%%  [\number\numexpr\phfthm@proofref@tmp@proofpage\relax{} vs % DEBUG
%%  \number\numexpr\phfthm@proofref@tmp@thispage\relax or % DEBUG
%%  \number\numexpr1+\phfthm@proofref@tmp@thispage\relax] % DEBUG
}
%    \end{macrocode}
% \end{macro}
%
%
%
% \begin{macro}{\phfthm@proofrefstyle@default@setup}
%   The main set-up macro for the |default| proof-ref style.  It sets all the
%   call-backs to the default ones.
%    \begin{macrocode}
\def\phfthm@proofrefstyle@default@setup{%
  \let\phfthm@proofref@impl@start\relax
  \let\phfthm@proofref@impl@afterlabel\@gobble
  \let\phfthm@proofref@impl@end\phfthm@proofrefstyle@default@main
  \let\phfthm@proofref@impl@fmtfarback\phfthm@proofrefstyle@default@fmtfarback
  \let\phfthm@proofref@impl@fmtfarahead\phfthm@proofrefstyle@default@fmtfarahead
  \let\phfthm@proofref@impl@fmtcloseby\phfthm@proofrefstyle@default@fmtcloseby
  \let\phfthm@proofref@impl@fmt\phfthm@proofrefstyle@default@fmt
}
%    \end{macrocode}
% \end{macro}
% 
%
% \subsubsection{Other proof-ref styles: only \phfverb{margin} for now}
%
% These styles simply use the same mechanism as the default style, but plug in
% different sub-callbacks.
%
% \begin{macro}{\phfthm@proofrefstyle@margin@setup}
%   Set-up macro for the ``|margin|'' proof-ref style (displays the proof
%   reference in the margin of the page).
%    \begin{macrocode}
\def\phfthm@proofrefstyle@margin@setup{%
  \phfthm@proofrefstyle@default@setup
%    \end{macrocode}
% 
% The proof reference should be displayed directly at the top, not at the end of
% the theorem, so plug in |\phfthm@proofref@default@main| onto |...@afterlabel|
% and not onto |...@end|.  Don't forget that these macros accept one argument,
% the theorem label.
%    \begin{macrocode}
  \let\phfthm@proofref@impl@afterlabel\phfthm@proofrefstyle@default@main
  \let\phfthm@proofref@impl@end\@gobble
%    \end{macrocode}
% 
% Define the formatting callback to put the note in the margin of the page using
% a |\marginpar|.  We need |\leavevmode| to make sure it's aligned properly
% vertically with the paragraph.\footnote{See
% \url{http://tex.stackexchange.com/a/16161/32188}}
%    \begin{macrocode}
  \def\phfthm@proofref@impl@fmt##1##2{%
    \leavevmode\marginpar{\proofrefsize{\proofonname{##1}{##2}}}%
  }%
}
%    \end{macrocode}
% \end{macro}
% 
%
% \subsection{Thmheading definition-like environments}
%
%
% \subsubsection{Manually define a thmheading environment}
%
% Define the key-value options accepted by |\phfMakeThmheadingEnvironment|.
%    \begin{macrocode}
\define@cmdkey{phfthmmkthmheading}{thmstyle}{}
\define@cmdkey{phfthmmkthmheading}{internalcounter}{}
%    \end{macrocode}
% 
%    \begin{macrocode}
\newcounter{phfthmheadingcounter}%
%    \end{macrocode}
%
% \begin{macro}{\phfMakeThmheadingEnvironment}
%
%   Creates a new environment |\begin{thmheading}{Title}...\end{thmheading}| for
%   customizing the heading on-the-fly (see documentation in
%   \autoref{sec:thmheading}).  Useful for an alternative formatting of
%   definitions.  The syntax is:
%
%   \noindent|\phfMakeThmheadingEnvironment|\oarg{key-value options}\marg{environment name}
% 
%   You can also use |\label| and |\ref| (the latter simply displays the given
%   title).
%
% 
%    \begin{macrocode}
\newcommand\phfMakeThmheadingEnvironment[2][]{% }
%    \end{macrocode}
% 
% Parse the options. First set defaults, and then parse the input string.
%    \begin{macrocode}
  \def\cmdKV@phfthmmkthmheading@thmstyle{plain}%
  \def\cmdKV@phfthmmkthmheading@internalcounter{phfthmheadingcounter}%
  \setkeys{phfthmmkthmheading}{#1}%
%    \end{macrocode}
% 
% And now, produce the relevant definitions:
%    \begin{macrocode}
  \csdef{phfthm@thmheading@#2@val@title}{$\langle$No Title Given$\rangle$}%
  \theoremstyle{\cmdKV@phfthmmkthmheading@thmstyle}%
%    \end{macrocode}
% 
% We use |\newtheorem*| to create an unnumbered theorem. The fixed title is just
% a single token, the macro which will be set to the relevant title at the last
% moment.
%    \begin{macrocode}
  \newtheorem*{phfthm@internal@thmheading@#2}{%
    \csname phfthm@thmheading@#2@val@title\endcsname}%
%    \end{macrocode}
%
% Define the actual environment.
%    \begin{macrocode}
  \newenvironment{#2}[1]{%}
    \csdef{phfthm@thmheading@#2@val@title}{##1}%
    \letcs\thephfthmheadingcounter{phfthm@thmheading@#2@val@title}%
%    \end{macrocode}
% Relay call to the internal \textit{AMS}-defined ``theorem:''
%    \begin{macrocode}
    \csname phfthm@internal@thmheading@#2\endcsname%
%    \end{macrocode}
% Pin down an anchor.  The use of |\hspace*{0pt}| is explained at
% \url{http://tex.stackexchange.com/a/88493/32188} (see especially the first
% comment).
%    \begin{macrocode}
    \hspace*{0pt}\refstepcounter{\cmdKV@phfthmmkthmheading@internalcounter}%
    \csname phfthm@hook@thmheading@#2@start\endcsname{##1}%
%    \end{macrocode}
% Also, let's add some flexibility in the hspace:
%    \begin{macrocode}
    \hskip 0em plus 0.5em minus 0em%
    \ignorespaces%
  }%
%    \end{macrocode}
% 
% Now, the END part of the environment: just call the callback and close the
% internal AMS-defined theorem.
%    \begin{macrocode}
  {%
    \csname phfthm@hook@thmheading@#2@end\endcsname%
    \csname endphfthm@internal@thmheading@#2\endcsname%
  }%
%    \end{macrocode}
% 
% Also define the relevant callbacks, which just relay their calls to the
% default callbacks.
%    \begin{macrocode}
  \csdef{phfthm@hook@thmheading@#2@start}##1{%
    \phfthm@hook@thmheading@start{##1}}%
  \csdef{phfthm@hook@thmheading@#2@end}{\phfthm@hook@thmheading@end}%
}
%    \end{macrocode}
%
% Provide as well the obsolete command |\phfthmMakeThmheadingEnvironment| which
% was provided in earlier versions of this package:
%    \begin{macrocode}
\def\phfthmMakeThmheadingEnvironment{\phfMakeThmheadingEnvironment}
%    \end{macrocode}
% \end{macro}
%
%
% \begin{macro}{\phfthm@hook@thmheading@start}
% \begin{macro}{\phfthm@hook@thmheading@end}
%   Global callbacks which are called for all thmheading-type environments
%   defined with |\phfMakeThmheadingEnvironment| (unless their hooks have
%   been changed in order for them not to call these global hooks).
%    \begin{macrocode}
\def\phfthm@hook@thmheading@start#1{}
\def\phfthm@hook@thmheading@end{}
%    \end{macrocode}
% \end{macro}
% \end{macro}
% 
%
% \subsection{Theorem sets}
%
% Here, we define the theorem sets proposed by the package for quick loading.
%
% We first define the names.  These are defined in any case regardless of
% whether we are loading a theorem set or of which theorem set we are loading.
%    \begin{macrocode}
\def\theoremname{Theorem}
\def\propositionname{Proposition}
\def\lemmaname{Lemma}
\def\corollaryname{Corollary}
\def\conjecturename{Conjecture}
\def\remarkname{Remark}
\def\definitionname{Definition}
\def\ideaname{Idea}
\def\questionname{Question}
\def\claimname{Claim}
\def\problemname{Problem}
%    \end{macrocode}
% 
% As we define the theorem sets, remember the names in a comma-separated list
% which we can display in help text.  The |\phfthm@def@thmset| replaces the
% |\def| command and expects the definitions to follow immediately.
%    \begin{macrocode}
\def\phfthm@def@thmset@optlist{}
\def\phfthm@def@thmset#1{%
  \appto\phfthm@def@thmset@optlist{#1,}\csdef{phfthm@thmset@#1}}
%    \end{macrocode}
% 
%
% \begin{macro}{\phfthm@def@thmset@mktheorem}
% \begin{macro}{\phfthm@def@thmset@mkdefn}
%   In definitions of theorem sets, use these macros to define a new
%   theorem-like environment (theorem, proposition, corollary, etc.) or
%   definition-like environment (definition, remark).  The macros
%   |\phfthm@val@mkthmoptarg@theorem| and |\phfthm@val@mkthmoptarg@defn| are
%   defined by |\phfLoadThmSet|.
%    \begin{macrocode}
\def\phfthm@def@thmset@mktheorem{%
  \expandafter\phfMakeTheorem\phfthm@val@mkthmoptarg@theorem}
\def\phfthm@def@thmset@mkdefn{%
  \expandafter\phfMakeTheorem\phfthm@val@mkthmoptarg@defn}
%    \end{macrocode}
% \end{macro}
% \end{macro}
% 
% The default set (empty name, or name ``|empty|'') provides no theorem.  (The
% first line uses |\def| directly so that we don't include an empty item in the
% list of available choices.)
%    \begin{macrocode}
\def\phfthm@thmset@{}
\phfthm@def@thmset{empty}{}
%    \end{macrocode}
% 
% Theorem set |simple|:
%    \begin{macrocode}
\phfthm@def@thmset{simple}{
  \phfthm@def@thmset@mktheorem{theorem}{\theoremname}
  \phfthm@def@thmset@mktheorem{proposition}{\propositionname}
  \phfthm@def@thmset@mktheorem{lemma}{\lemmaname}
  \phfthm@def@thmset@mktheorem{corollary}{\corollaryname}
  \phfthm@def@thmset@mkdefn{definition}{\definitionname}
}
%    \end{macrocode}
% 
% Theorem set |default|:
%    \begin{macrocode}
\phfthm@def@thmset{default}{
  \phfthm@def@thmset@mktheorem{theorem}{\theoremname}
  \phfthm@def@thmset@mktheorem{proposition}{\propositionname}
  \phfthm@def@thmset@mktheorem{lemma}{\lemmaname}
  \phfthm@def@thmset@mktheorem{corollary}{\corollaryname}
  \phfthm@def@thmset@mktheorem{conjecture}{\conjecturename}
  \phfthm@def@thmset@mktheorem{remark}{\remarkname}
  \phfthm@def@thmset@mkdefn{definition}{\definitionname}
}
%    \end{macrocode}
% 
% Theorem set |shortnames|:
%    \begin{macrocode}
\phfthm@def@thmset{shortnames}{
  \phfthm@def@thmset@mktheorem{thm}{\theoremname}
  \phfthm@def@thmset@mktheorem{prop}{\propositionname}
  \phfthm@def@thmset@mktheorem{lem}{\lemmaname}
  \phfthm@def@thmset@mktheorem{cor}{\corollaryname}
  \phfthm@def@thmset@mktheorem{conj}{\conjecturename}
  \phfthm@def@thmset@mktheorem{rem}{\remarkname}
  \phfthm@def@thmset@mkdefn{defn}{\definitionname}
}
%    \end{macrocode}
% 
% Theorem set |rich|. Add definitions to the |default| set:
%    \begin{macrocode}
\phfthm@def@thmset{rich}{
  \phfthm@thmset@default
  \phfthm@def@thmset@mktheorem{idea}{\ideaname}
  \phfthm@def@thmset@mktheorem{question}{\questionname}
  \phfthm@def@thmset@mktheorem{claim}{\claimname}
  \phfthm@def@thmset@mktheorem{problem}{\problemname}
}
%    \end{macrocode}
% 
%
% \begin{macro}{\phfLoadThmSet}
%   The macro |\phfLoadThmSet| loads a theorem set.  See documentation at
%   \autoref{sec:load-thm-set-manually}.
%
%   |#1| = options to |\phfMakeTheorem| for theorem-like environments
%
%   |#2| = options to |\phfMakeTheorem| for definition-like environments
%
%   |#3| = name of the theorem set to load
%
%    \begin{macrocode}
\newcommand\phfLoadThmSet[3]{%
  \ifcsname phfthm@thmset@#3\endcsname%
    \edef\phfthm@val@mkthmoptarg@theorem{#1}%
    \edef\phfthm@val@mkthmoptarg@defn{#2}%
    \csname phfthm@thmset@#3\endcsname%
  \else%
    \PackageWarning{phfthm}{Unknown theorem set: `#3'!}%
  \fi%
}
%    \end{macrocode}
%
% For compatibility with my earlier versions of \pkgname{phfthm}, also provide
% the obsolete |\phfthmLoadThmSet|:
%    \begin{macrocode}
\def\phfthmLoadThmSet{\phfLoadThmSet}
%    \end{macrocode}
% \end{macro}
% 
%
% \subsection{Package option handling}
%
% The machinery is in place, now define and parse the package options.
%
% \subsubsection{Declaring the package options}
%
% The package options all use the |keyval| parsing mechanism using the
% \pkgname{xkeyval} package.
%
% Recall when using |\define@XXXkey| that the optional argument after the second
% mandatory argument is the value which is assumed if the key is given with no
% explicit value; it is not the initial default value.
%
% \paragraph{The \pkgoptionfmt{resetstyle} package option}
% An option to reset all options so that the package provides only stand-alone
% definitions and is not invasive (see \autoref{sec:global-pkg-options}).
%
% This option does not expect any argument (i.e., you should specify
% |\usepackage[resetstyle,|\meta{other options}|]{phfthm}|, and not
% |\usepackage[resetstyle=true,|\meta{other options}|]{phfthm}|).
%    \begin{macrocode}
\define@key{phfthmpkg}{resetstyle}[]{%
  \KV@phfthmpkg@smallproofsfalse%
  \KV@phfthmpkg@qedsymbolblacksquarefalse%
  \KV@phfthmpkg@prooftitleitbffalse%
  \KV@phfthmpkg@sepcountersfalse%
  \KV@phfthmpkg@proofreffalse%
  \if\relax\detokenize{#1}\relax\else%
    \PackageError{phfthm}{'resetstyle' does not take any argument.}{You
      specified the 'resetstyle' argument and provided a value to it
      ('resetstyle=...'). However the 'resetstyle' option does not accept
      any value argument.}
  \fi%
}
%    \end{macrocode}
%
%
% \paragraph{Options for loading theorem sets}
% Define the various package options for the loading of predefined theorem sets
% (\autoref{sec:theorem-sets}).
%
% The \pkgoptionfmt{sepcounters} option, off by default.
%    \begin{macrocode}
\define@boolkey{phfthmpkg}{sepcounters}[true]{}
\KV@phfthmpkg@sepcountersfalse
%    \end{macrocode}
% 
%
% The \pkgoptionfmt{proofref} option.  The proof-ref is off initially by
% default.
%    \begin{macrocode}
\newif\ifKV@phfthmpkg@proofref
\KV@phfthmpkg@proofreffalse
\def\cmdKV@phfthmpkg@proofref@style{}
%    \end{macrocode}
% Actually define the option itself.  Here we do some customized parsing of the
% value of the |proofref=...| option, to treat the cases |proofref=| (empty
% argument) and |proofref=false| separately.
%    \begin{macrocode}
\define@key{phfthmpkg}{proofref}[]{%
  \ifblank{#1}{%
%    \end{macrocode}
% If a blank argument provided, set some sensible defaults with proofref on:
%    \begin{macrocode}
    \KV@phfthmpkg@proofreftrue%
    \def\cmdKV@phfthmpkg@proofref@style{default}%
  }{%
%    \end{macrocode}
% Otherwise, check to see if the value is |false|, in which case deactivate the
% proof-ref mechanism, or else, activate it and set the given style value as
% documented in \autoref{sec:theorem-sets}.
%    \begin{macrocode}
    \ifstrequal{#1}{false}{%
      \KV@phfthmpkg@proofreffalse%
    }{%
      \KV@phfthmpkg@proofreftrue%
      \def\cmdKV@phfthmpkg@proofref@style{#1}%
    }%
  }%
}
%    \end{macrocode}
% 
% The \pkgoptionfmt{thmset} option.  We subtly construct the command
% |\define@choicekey{phfthmpkg}{thmset}[\val]{\phfthm@def@thmset@optlist}|, but
% with the last macro (option list) expanded.
%    \begin{macrocode}
\def\@tmpa{\define@choicekey{phfthmpkg}{thmset}[\val]}
\edef\@tmpb{{\phfthm@def@thmset@optlist}}
\expandafter\@tmpa\@tmpb{%
  \xdef\cmdKV@phfthmpkg@thmset{\val}%
}
%    \end{macrocode}
% By default we should load the |default| set.
%    \begin{macrocode}
\def\cmdKV@phfthmpkg@thmset{default}
%    \end{macrocode}
% 
%
% The options \pkgoptionfmt{theoremstyle} and \pkgoptionfmt{definitionstyle} set
% which theorem style to use for theorems and definitions, when loading the
% given thmset.
%    \begin{macrocode}
\define@cmdkey{phfthmpkg}{theoremstyle}{}
\def\cmdKV@phfthmpkg@theoremstyle{plain}
\define@cmdkey{phfthmpkg}{definitionstyle}{}
\def\cmdKV@phfthmpkg@definitionstyle{definition}
%    \end{macrocode}
%
%
% \paragraph{Proof environment options}
% Define the package options \pkgoptionfmt{proofenv},
% \pkgoptionfmt{smallproofs}, \pkgoptionfmt{qedsymbolblacksquare}, and
% \pkgoptionfmt{prooftitleitbf} (\autoref{sec:proof-env}).
%    \begin{macrocode}
\define@boolkey{phfthmpkg}{proofenv}[true]{}
\define@boolkey{phfthmpkg}{smallproofs}[true]{}
\define@boolkey{phfthmpkg}{qedsymbolblacksquare}[true]{}
\define@boolkey{phfthmpkg}{prooftitleitbf}[true]{}
%    \end{macrocode}
% 
% Set the initial default values for these options.
%    \begin{macrocode}
\KV@phfthmpkg@smallproofstrue
\KV@phfthmpkg@qedsymbolblacksquaretrue
\KV@phfthmpkg@proofenvtrue
\KV@phfthmpkg@prooftitleitbffalse
%    \end{macrocode}
%
% 
% \paragraph{Options for a theorem-like heading environment}
% Define the \pkgoptionfmt{thmheading} and \pkgoptionfmt{thmheadingstyle}
% package options, documented in \autoref{sec:thmheading}.
%    \begin{macrocode}
\define@boolkey{phfthmpkg}{thmheading}[true]{}
\define@cmdkey{phfthmpkg}{thmheadingstyle}{}
%    \end{macrocode}
% 
% The |thmheading| environment is provided by default; it's a stand-alone
% definition anyway.  The style defaults to the |plain| style.
%    \begin{macrocode}
\KV@phfthmpkg@thmheadingtrue
\def\cmdKV@phfthmpkg@thmheadingstyle{plain}
%    \end{macrocode}
%
%
% \subsubsection{Parsing the package options}
% The usual stuff (\pkgname{xkeyval}-flavored).
%    \begin{macrocode}
\DeclareOptionX*{%
  \PackageWarning{phfthm}{Invalid option: `\CurrentOption'}%
}
\ProcessOptionsX<phfthmpkg>
%    \end{macrocode}
% 
%
% \subsubsection{Execute package options-controlled actions}
%
% \paragraph{Loading a theorem set}
% First, we need to take into account the options which alter the way the
% theorem sets will be loaded (separate counters, proof-ref, etc.).
%
% Take care of the proof-ref stuff.  First, define the possible styles (note
% that these are not the same as the values to the |proofrefstyle| argument to
% the |\phfMakeTheorem| command).
%    \begin{macrocode}
\def\phfthm@val@mkthmopt@proofrefstyle{}
\ifKV@phfthmpkg@proofref
  \def\phfthm@proofref@style@default{}
  \def\phfthm@proofref@style@{}
%    \end{macrocode}
% Note that |proofref=always| and |proofref=onlyifveryfar| have a global effect,
% because they set |\phfProofrefPageBackTolerance| and
% |\phfProofrefPageAheadTolerance| (see documentation in
% \autoref{sec:theorem-sets}).
%    \begin{macrocode}
  \def\phfthm@proofref@style@always{
    \def\phfProofrefPageBackTolerance{-1}
    \def\phfProofrefPageAheadTolerance{-1}
  }
  \def\phfthm@proofref@style@onlyifveryfar{
    \def\phfProofrefPageBackTolerance{2}
    \def\phfProofrefPageAheadTolerance{4}
  }
  \def\phfthm@proofref@style@margin{
    \def\phfthm@val@mkthmopt@proofrefstyle{proofrefstyle=margin}
  }
  \def\phfthm@proofref@style@longref{
%    \end{macrocode}
% For |longref|: by setting |\proofonname| globally, this option can be combined
% with other styles.  But then we also change the default style formatting to
% avoid ugly line breaks.
%    \begin{macrocode}
    \def\proofonname##1##2{The proof of this \phfthm@autorefnameof{##1} can
      be found on ##2.}
    \def\phfthm@proofrefstyle@default@fmt##1##2{%
      \par{\raggedleft\proofrefsize{(\proofonname{##1}{##2})}\par}%
    }
  }
  \def\phfthm@proofref@style@off{
    \def\phfthm@val@mkthmopt@proofrefstyle{proofref=false}
  }
%    \end{macrocode}
% 
% Now execute the given styles.  Construct the command
% |\phfthm@internal@execattribs{phfthm@proofref@style@}|\hspace 
% {0pt}|{ProofRef Style}|\hspace{0pt}|{\cmdKV@phfthmpkg@proofref@style}|, but with
% the last macro expanded.
%    \begin{macrocode}
  \def\x{%
    \phfthm@internal@execattribs{phfthm@proofref@style@}{ProofRef Style}}
  \expandafter\x\expandafter{\cmdKV@phfthmpkg@proofref@style}
\fi
%    \end{macrocode}
% 
%
% Take care of counters.  In any case, define a common counter, in case we use a
% common counter for all theorem types.  (The counter is defined in any case, to
% avoid breaking other code which might use it if suddenly the user decides to
% use |sepcounters=true| for their document.)
%    \begin{macrocode}
\newcounter{phfthmcounter}
\setcounter{phfthmcounter}{0}
%    \end{macrocode}
% 
% Prepare an argument to |\phfMakeTheorem| according to the
% \pkgoptionfmt{sepcounters} option.
%    \begin{macrocode}
\ifKV@phfthmpkg@sepcounters
  \def\phfthm@val@mkthmopt@counteropts{}
\else
  \def\phfthm@val@mkthmopt@counteropts{counter=phfthmcounter}
\fi
%    \end{macrocode}
% 
%
% Finally, load the theorem set defined by the options.  The first argument
% regroups the options for theorem environments (Theorem, Proposition,
% Corollary, \ldots); the second argument regroups the options for definition
% environments (Definition); the third argument is the theorem set name itself.
%    \begin{macrocode}
\phfLoadThmSet%
{[\phfthm@val@mkthmopt@counteropts,\phfthm@val@mkthmopt@proofrefstyle,
  thmstyle=\cmdKV@phfthmpkg@theoremstyle]}%
{[\phfthm@val@mkthmopt@counteropts,proofref=false,
  thmstyle=\cmdKV@phfthmpkg@definitionstyle]}%
{\cmdKV@phfthmpkg@thmset}
%    \end{macrocode}
% (Note the absence of the proof-ref for definitions.)
%
% \paragraph{Define the proof environment}
% If requested, define the |proof| environment (\autoref{sec:proof-env}).
% First, make sure we take into account the options \pkgoptionfmt{smallproofs},
% \pkgoptionfmt{qedsymbolblacksquare} and \pkgoptionfmt{prooftitleitbf}.
%    \begin{macrocode}
\def\phfthm@pkgopterr@require@proofenv#1{%
  \ifKV@phfthmpkg@proofenv\else%
    \PackageError{phfthm}{Option `#1' depends on `proofenv=true'}%
  \fi
}
\ifKV@phfthmpkg@smallproofs
  \phfthm@pkgopterr@require@proofenv{smallproofs}
  \apptocmd\phfthm@hookproof@startcommon{%
    \def\baselinestretch{1.2}\footnotesize}{}{%
    Failed to change command \string\phfthm@hook@start@proof}
\fi
\ifKV@phfthmpkg@qedsymbolblacksquare
  \phfthm@pkgopterr@require@proofenv{qedsymbolblacksquare}
  \RequirePackage{amssymb}
  \providecommand\filledsquare{\ensuremath{\blacksquare}}
  \renewcommand\qedsymbol{\text{\tiny\ensuremath{\filledsquare}}}
\fi
\ifKV@phfthmpkg@prooftitleitbf
  \phfthm@pkgopterr@require@proofenv{prooftitleitbf}
  \def\phfthm@ProofTitleFmt#1{{\itshape\bfseries#1.}}
\fi
%    \end{macrocode}
% Go ahead and define the |proof| environment.  Because we have already loaded
% \pkgname{amsthm}, we need to override the existing |proof| environment.
%    \begin{macrocode}
\ifKV@phfthmpkg@proofenv
  \phfMakeProofEnv[override=true]{proof}
\fi
%    \end{macrocode}
% 
% \paragraph{Define the theorem-heading environment}
% Define the |thmheading| environment, if requested.
%    \begin{macrocode}
\ifKV@phfthmpkg@thmheading
  \phfMakeThmheadingEnvironment%
      [thmstyle=\cmdKV@phfthmpkg@thmheadingstyle]{thmheading}
\fi
%    \end{macrocode}
%
%\Finale
\endinput
