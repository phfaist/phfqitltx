% \iffalse meta-comment
%
% Copyright (C) 2016 by Philippe Faist <philippe.faist@bluewin.ch>
% -------------------------------------------------------
% 
% This file may be distributed and/or modified under the
% conditions of the LaTeX Project Public License, either version 1.3
% of this license or (at your option) any later version.
% The latest version of this license is in:
%
%    http://www.latex-project.org/lppl.txt
%
% and version 1.3 or later is part of all distributions of LaTeX 
% version 2005/12/01 or later.
%
% \fi
%
% \iffalse
%<*driver>
\ProvidesFile{phffullpagefigure.dtx}
%</driver>
%<package>\NeedsTeXFormat{LaTeX2e}[2005/12/01]
%<package>\ProvidesPackage{phffullpagefigure}
%<*package>
    [2016/05/02 v1.0 phffullpagefigure package]
%</package>
%
%<*driver>
\documentclass{ltxdoc}
\usepackage{xcolor}
\usepackage{phffullpagefigure}
\usepackage[preset=pkgdoc]{phfnote}
\phfnoteSaveDefs{verbatimstuff}{verbatim,@verbatim,endverbatim}
\usepackage[normalem]{ulem}
\usepackage{verbdef}
\usepackage{tcolorbox}
\newtcolorbox{pkgnote}{colback=blue!5!white,colframe=blue!5!white,coltitle=blue!50!black,fonttitle=\bfseries,title={NOTE}}
\newtcolorbox{pkgwarning}{colback=red!5!white,colframe=red!5!white,coltitle=red!50!black,fonttitle=\bfseries,title={WARNING}}
\EnableCrossrefs         
\CodelineIndex
\RecordChanges
\phfnoteRestoreDefs{verbatimstuff}
\begin{document}
  \DocInput{phffullpagefigure.dtx}
\end{document}
%</driver>
% \fi
%
% \CheckSum{0}
%
% \CharacterTable
%  {Upper-case    \A\B\C\D\E\F\G\H\I\J\K\L\M\N\O\P\Q\R\S\T\U\V\W\X\Y\Z
%   Lower-case    \a\b\c\d\e\f\g\h\i\j\k\l\m\n\o\p\q\r\s\t\u\v\w\x\y\z
%   Digits        \0\1\2\3\4\5\6\7\8\9
%   Exclamation   \!     Double quote  \"     Hash (number) \#
%   Dollar        \$     Percent       \%     Ampersand     \&
%   Acute accent  \'     Left paren    \(     Right paren   \)
%   Asterisk      \*     Plus          \+     Comma         \,
%   Minus         \-     Point         \.     Solidus       \/
%   Colon         \:     Semicolon     \;     Less than     \<
%   Equals        \=     Greater than  \>     Question mark \?
%   Commercial at \@     Left bracket  \[     Backslash     \\
%   Right bracket \]     Circumflex    \^     Underscore    \_
%   Grave accent  \`     Left brace    \{     Vertical bar  \|
%   Right brace   \}     Tilde         \~}
%
%
% \changes{v1.0}{2016/04/20}{Initial version}
%
% \GetFileInfo{phffullpagefigure.dtx}
%
% \DoNotIndex{\newcommand,\newenvironment,\def,\gdef,\edef,\xdef,\if,\else,\fi,\ifx}
% 
% \title{The \textsf{phffullpagefigure} package\thanks{\itshape
% This document corresponds to
% \textsf{phffullpagefigure}~\fileversion, dated \filedate. It is part of
% the Part of the
% \href{https://github.com/phfaist/phfqitltx/}{\textsf{phfqitltx}}
% package suite, see \url{https://github.com/phfaist/phfqitltx}.}}
% \author{Philippe Faist\quad\email{philippe.faist@bluewin.ch}}
%
% \maketitle
%
% \begin{abstract}
%   \textsf{phffullpagefigure}---Figures which fill up a full page of a document.
% \end{abstract}
%
% \inlinetoc
%
% \section{Introduction}
%
% The package \textsf{phffullpagefigure} provides an implementation for figures which are
% to be displayed to occupy a full page.
%
% A typical use case: suppose you have a figure in PDF format of the size of the document
% paper, for example, and wants to include it as a figure.
%
% This package takes care to display the caption of the figure on the preceeding page,
% with a caption of the form ``Figure X (on facing page): \meta{caption}.''
%
% For two-sided documents, you may specify on which (odd/even) page side
% you want the figure to appear.  If the document is not two-sided, the
% figure may appear on any page.
%
% A number of options allow you to set the exact figure contents (usually a
% PDF file, but it can be constructed from arbitrary \LaTeX{} commands),
% the figure caption placement (top, own page, bottom), the caption and
% label as usual, and the formatting of the caption if you want to replace
% the default ``(on facing page)'' or ``(on next page).''
%
%
% \section{The Full-Page-Figure Environment}
%
% \DescribeEnv{fullpagefigure} The |\begin{fullpagefigure}| |...|
%   |\end{fullpagefigure}| environment starts a full-page-figure.  Inside
% this environment, only the following commands may be used:
% \begin{itemize}
% \item one of the |\fig***| commands;
% \item the |\caption| command, to provide a figure caption as for regular
%   figures;
% \item the |\label| command to set a label for text references to this
%   figure, also as for regular figures.
% \end{itemize}
%
% A simple example to get started:
%
% \begin{verbatim}
% \begin{fullpagefigure}
%   \figpdf{fig/my-figure} % my PDF file
%   \caption{A colorful figure with letters and words.  The table design
%   may remind you of going to the optician.}
%   \label{fig:test}
% \end{fullpagefigure}
% \end{verbatim}
%
% \begin{fullpagefigure}%
%   \iffalse meta-comment
%     In this documentation code, we cheat and provide LaTeX code for the
%     figure instead of a PDF file.  This is really just so that we don't
%     have to ship an extra PDF file (and for no other reason).
%     
%     See a trick in http://tex.stackexchange.com/a/278101/32188.
%   \fi
%   \figcontents{\newpage\thispagestyle{empty}\relax
%     \newgeometry{margin=0.25in}\relax
%     \sffamily\fontsize{36pt}{42pt}\fontseries{bx}\selectfont
%     \hbox to 0.9\textwidth{\fcolorbox{red}{yellow}{\vbox to 0.99\textheight{\vfil\vfil\vfil\centering
%         \leavevmode\noindent\hbox to 0.8\textwidth{ \color{red!50!black} T H I S }\par\vfil
%         \vfil
%         \leavevmode\noindent\hbox to 0.8\textwidth{ \color{green!70!red} F I G U R E }\par\vfil
%         \vfil
%         \leavevmode\noindent\hbox to 0.8\textwidth{ \color{black!40!blue} F I L L S }\par\vfil
%         \vfil
%         \leavevmode\noindent\hbox to 0.8\textwidth{ \color{green!40!blue} U P }\par\vfil
%         \vfil
%         \leavevmode\noindent\hbox to 0.8\textwidth{ \color{red!50!gray} T H E }\par\vfil
%         \vfil
%         \leavevmode\noindent\hbox to 0.8\textwidth{ \color{blue} W H O L E }\par\vfil
%         \vfil
%         \leavevmode\noindent\hbox to 0.8\textwidth{ \color{red!50!yellow} P A G E }\par\vfil
%         \vfil\vfil\vfil}}}%
%       \clearpage\aftergroup\restoregeometry}%
%   \caption{A colorful figure with letters and words.  The table design
%   may remind you of going to the optician.}%
%   \label{fig:test}%
% \end{fullpagefigure}
%
% The |\figpdf| command sets the PDF file to be displayed in full page (see
% more details below).  You may contemplate the result of this code in
% \autoref{fig:test}.
%
% Some wizard once told me that old wise men had determined that the
% |fullpagefigure| environment should be placed at the beginning of a
% paragraph, or on its own paragraph.  (TODO: I'm not too sure why this is
% the case or if this is still relevant.)
%
%
% The contents and appearance of the figure can be adjusted by the
% following commands, which must be issued within the |fullpagefigure|
% environment.
%
% \DescribeMacro{\figcontents} Specify the contents of the figure by
% calling |\figcontents|\marg{\LaTeX{} commands}.  The contents can be any
% \LaTeX{} commands which will generate the figure content.  These commands
% will be called within an |afterpage| block.
%
% If you want a figure to occupy several pages, you may use for example
% |\figcontents{\includepdf{pdf-1}\includepdf{pdf-2}}|, and request that
% the figure start on an even-side page with |\figpageside{even}|.
%
% \DescribeMacro{\figpdf} As a shorthand, you may use
% |\figpdf|\oarg{options}\marg{pdf-file} as a shorthand for
% |\figcontents{\includepdf|\oarg{options}\marg{pdf-file}|}|.
% 
% You should explore the options provided by |\includepdf| (from the
% \textsf{pdfpages} package\footnote{See documentation at
% \url{https://www.ctan.org/pkg/pdfpages}}).  For example, the figure can
% be resized to fill the page, pages may be selected individually from a
% mult-page PDF, the image may be rotated, etc.
%
%
% \DescribeMacro{\caption} \DescribeMacro{\label} The |\caption|\oarg{short
% caption}\marg{caption} and |\label|\marg{identifier} macros may be used
% as for a normal figure.  Be warned, though, that some dark manipulations
% occur here, so it may be for example that the code passed as argument to
% these commands is expanded only later.
%
% \DescribeMacro{\figpageside} Use the commands |\figpageside{odd}|,
% |\figpageside{}|, or |\figpageside{even}| to specify on which side the
% figure should appear on if the document is two-sided.  This command has
% no effect if the document is not two-sided (|twoside| class option for
% the |article| or |book| classes, for example).  Calling |\figpageside{}|,
% i.e.\@ with an empty argument, instructs |fullpagefigure| to use either
% side, whichever is more convenient.
%
% \DescribeMacro{\figplacement} Specify the figure caption placement with
% |\figplacement||{b|$\mid$|t|$\mid$|p}|.  Each of |b| (bottom), |t| (top)
% and |p| (own page) work as for usual LaTeX floats.  If you specify |p|,
% do NOT combine it with any other option.  You may leave the argument
% empty (|\figplacement{}|) to use defaults.
%
% The figure placement can also be specified as an optional argument to the
% environment (e.g., |\begin{fullpagefigure}[p]| \ldots{} |\end{fullpagefigure}|).
%
% \DescribeMacro{\figcapmaxheight} Specify the maximum estimated height of
% the caption with |\figcapmaxheight|\marg{length}.  This is used to see
% whether the figure caption still fits on the current page.
%
% TODO: This is ugly, the height of the caption should be calculated
% automatically\ldots{} for next time.
%
% By default (if no |\figcapmaxheight| is present), the figure will never
% be assumed to fit in the remainder of the page.
%
% \DescribeMacro{\fullpagefigurecaptionfmt}
% The figure label in the caption may be changed (e.g. ``Figure X (on facing page): \ldots'') by
% redefining the command |\fullpagefigurecaptionfmt|.  See the default implementation for
% more info (\autoref{sec:impl}).
%
% \DescribeMacro{\FlushAllFullPageFigures} If you need to make sure that
% all full-page-figures have been placed up to a certain point, you may
% issue the command |\FlushAllFullPageFigures|.  (You may wish to do so
% before starting a new chapter.)  An optional argument
% |\FlushAllFullPageFigures[\clearpage]| or
% |\FlushAllFullPageFigures[\cleardoublepage]| specifies whether to
% continue on any page or on an odd-side page only.
% 
%
%
% \section{Package Options}
%
% Only a single package option is recognized:
%
% \begin{verbatim} \usepackage[nopdfpages]{phffullpagefigure} \end{verbatim}
%
% If this package option is given, then the |pdfpages| package is not
% loaded, and the command |\figpdf| is not made available.  You may use
% this package option if the |pdfpages| package conflicts with your setup.
%
%
%
%
%
%
%
%
% \StopEventually{\vskip 2cm plus 2cm minus 2cm\relax\PrintChanges
%     \vskip 2cm plus 2cm minus 2cm\relax\PrintIndex}
%
% \section{Implementation}
% \label{sec:impl}
%
% Include some general useful packages first.
%    \begin{macrocode}
\RequirePackage{etoolbox}
\RequirePackage{ifoddpage}
\RequirePackage{afterpage}
%    \end{macrocode}
% 
% The |placeins| package provies the |\FloatBarrier| command, which we use
% to ensure that no other float gets in the way.
%    \begin{macrocode}
\RequirePackage{placeins}
%    \end{macrocode}
% 
% \begin{macro}{phffpf@internal@pending}
%   Counter which stores how many full-page-figures still haven't been
%   placed.  Used for |\FlushAllFullPageFigures| as well as making sure
%   that the full-page-figures don't interfere with one another.
%    \begin{macrocode}
\newcounter{phffpf@internal@pending}
\setcounter{phffpf@internal@pending}{0}
%    \end{macrocode}
% \end{macro}
%
% \begin{macro}{\phffpfFloatBarrier}
%   Redefine this if you don't want to use a |\FloatBarrier|.  Be warned of
%   the following points:
%
%   \begin{itemize}
%   \item |\FloatBarrier| introduces automatically a new paragraph.
%     Nothing you can do about that a priori.
%
%   \item If you remove |\FloatBarrier|, you need to either be sure that
%     there are no floats which can mess up placement of the
%     fullpagefigure.  Alternatively, you need to provide your own
%     mechanism that ensures that.
%   \end{itemize}
%
%    \begin{macrocode}
\def\phffpfFloatBarrier{\FloatBarrier}
%    \end{macrocode}
% \end{macro}
% 
%
% \subsection{The Main Environment Definition}
%
% \begin{environment}{fullpagefigure}
%   The main |fullpagefigure| environment.
%    \begin{macrocode}
\newenvironment{fullpagefigure}[1][b]{% 
%    \end{macrocode}
% 
% Remember that we have a float pending to be placed:
%    \begin{macrocode}
  \addtocounter{phffpf@internal@pending}{1}%
%    \end{macrocode}
% 
% Don't allow any other floats to meddle with our calculations.
%    \begin{macrocode}
  \phffpfFloatBarrier%
  %[YYY]% -- debugging [where is a space being inserted?]
%    \end{macrocode}
% 
% The following variables will store the relevant values of options
% collected in the definition of the figure with e.g. |\figplacement|,
% |\figcontents|, etc.
% 
% NOTE TO SELF: If you add a |\phffpf@val@...| storage variable, don't
% forget to fix that value in |\phffpf@takecareofplacingfigure|.
% 
%    \begin{macrocode}
  \xdef\phffpf@val@pageside{\phffpf@side@}%
  \gdef\phffpf@val@captionopt{}%
  \gdef\phffpf@val@caption{}%
  \gdef\phffpf@val@label{}%
  \gdef\phffpf@val@placement{#1}%
  \gdef\phffpf@val@capmaxheight{\paperheight}%
  \gdef\phffpf@val@figcontents{}%
%    \end{macrocode}
%
% If this document is two-sided (facing odd/even pages), then by default
% place the float on an odd page.  Otherwise, we don't care.
% 
%    \begin{macrocode}
  \if@twoside%
    \xdef\phffpf@val@pageside{\phffpf@side@odd}%
  \else\fi%
%    \end{macrocode}
%
% Provide a set of commands within this figure block which allow to specify
% the figure contents and appearance:
%    \begin{macrocode}
  \begingroup%
  \let\figcontents\phffpf@impl@figcontents%
  \let\figpageside\phffpf@impl@figpageside%
  \let\caption\phffpf@impl@caption%
  \let\label\phffpf@impl@label%
  \let\figplacement\phffpf@impl@placement%
  \let\figcapmaxheight\phffpf@impl@capmaxheight%
%    \end{macrocode}
% 
% Provide |\figpdf| as a shorthand, but only if applicable (i.e., the
% |nopdfpages| package option was not specified and the |pdfpages| package
% was loaded):
%    \begin{macrocode}
  \phffpf@provide@figpdf%
%    \end{macrocode}
%
% Finally, ignore any spaces following this command, as well as after the
% |\endenvironment| command.
%    \begin{macrocode}
  \ignorespacesafterend%
  \ignorespaces%
}
%    \end{macrocode}
% 
% Now, the definitions for the end of the environment:
%    \begin{macrocode}
{%
%    \end{macrocode}
%
% Remove any spaces which might have been inserted.
%    \begin{macrocode}
  \ifhmode\unskip\fi%
%    \end{macrocode}
%
% Restore |\caption|, |\label|, etc.\@ to their original meaning:
%    \begin{macrocode}
  \endgroup%
%    \end{macrocode}
% 
% Finally we should actually take care of placing the figure.
%    \begin{macrocode}
  \phffpf@takecareofplacingfigure%
%    \end{macrocode}
% 
% Finally finally, ignore any spaces following this command.  Note that
% because the expansion of |\endfullpagefigure| is inside the definition of
% \LaTeX' ``|\end|'' and has internal commands after that, we can't just
% simply issue a |\ignorespaces|.
%
%    \begin{macrocode}
  \phfpf@useignorespacesandallpars%
}
%    \end{macrocode}
% \end{environment}
%
% \begin{macro}{\phffpf@useignorespacesandallpars}
%   Utility to ignore spaces and paragraphs after the
%   |\end{fullpagefigure}| command.\footnote{This solution was adapted from
% \url{http://tex.stackexchange.com/a/179034/32188} and
% \url{http://tex.stackexchange.com/a/23101/32188}.}
%    \begin{macrocode}
\def\phfpf@useignorespacesandallpars#1\ignorespaces\fi{%
  #1\fi\phffpf@ignorespacesandallpars}
\def\phffpf@ignorespacesandallpars{%
  \begingroup%
  \catcode`\^^M=10\relax%
  \catcode`\^^J=10\relax%
  \@ifnextchar\par%
    {\endgroup\expandafter\phffpf@ignorespacesandallpars\@gobble}%
    {\endgroup}%
}
%    \end{macrocode}
% \end{macro}
% 
%
%
% \begin{macro}{\fullpagefigurecaptionfmt}
% \begin{macro}{\fullpagefigurecaptionfmt@paren@O}
% \begin{macro}{\fullpagefigurecaptionfmt@paren@E}
% \begin{macro}{\fullpagefigurecaptionfmt@paren@x}
%   The macro |\fullpagefigurecaptionfmt| is called to generate the text
%   which is prepended to the figure caption.  It should essentially say
%   ``Figure X (on next page): ''.
%
%   The argument to |\fullpagefigurecaptionfmt| is |#1| = |O|, |E| or |x|
%   for if the figure is on an odd page, an even page, or an unspecified
%   page.
%    \begin{macrocode}
\def\fullpagefigurecaptionfmt#1{%
  \figurename\nobreakspace\thefigure\nobreakspace%
  (\csname fullpagefigurecaptionfmt@paren@#1\endcsname)%
}
\def\fullpagefigurecaptionfmt@paren@O{on facing page} % for odd page figures
\def\fullpagefigurecaptionfmt@paren@E{on next page}   % for even page figures
\def\fullpagefigurecaptionfmt@paren@x{on next page}   % for next-page figures
%    \end{macrocode}
% \end{macro}
% \end{macro}
% \end{macro}
% \end{macro}
%
%
%
% \subsection{Implementation of \phfverb{\fig}\phfverb{****} Commands}
%
% These macros really just store their values for later use.
%
% \begin{macro}{\figcontents}
%   This macro will become |\figcontents| inside the |fullpagefigure|
%   environment.
%    \begin{macrocode}
\newtoks\phffpf@tmp@toks
\long\def\phffpf@impl@figcontents#1{%
  \phffpf@tmp@toks={#1}%
  \xdef\phffpf@val@figcontents{\the\phffpf@tmp@toks}%
  \ignorespaces%
}
%    \end{macrocode}
% \end{macro}
% 
% \begin{macro}{\phffpf@side@odd}
% \begin{macro}{\phffpf@side@even}
% \begin{macro}{\phffpf@side@}
%   These hold one-character codes to signify ``odd side,'' ``even side,''
%   or ``no specification.''
%
%    \begin{macrocode}
\def\phffpf@side@odd{O}
\def\phffpf@side@even{E}
\def\phffpf@side@{x}
%    \end{macrocode}
% \end{macro}
% \end{macro}
% \end{macro}
% 
% \begin{macro}{\figpageside}
%   This will become |\figpageside| inside the |fullpagefigure|
%   environment.
%    \begin{macrocode}
\def\phffpf@impl@figpageside#1{%
  \ifcsname phffpf@side@#1\endcsname%
    \xdef\phffpf@val@pageside{\csname phffpf@side@#1\endcsname}%
  \else%
    \PacakgeError{phffullpagefigure}{Unknown page side designation:
      '#1'. Please use 'odd', 'even', or '' for no preference.}%
  \fi%
  \ignorespaces%
}
%    \end{macrocode}
% \end{macro}
% 
% \begin{macro}{\caption}
%   This will become |\caption| inside the |fullpagefigure| environment.
%    \begin{macrocode}
\def\phffpf@NOARG{}
\def\phffpf@test@NOARG{\phffpf@NOARG}
\newcommand\phffpf@impl@caption[2][\phffpf@NOARG]{%
  \gdef\phffpf@val@captionopt{#1}%
  \gdef\phffpf@val@caption{#2}%
  \ignorespaces%
}
%    \end{macrocode}
% \end{macro}
% 
% \begin{macro}{\label}
%   This will become |\label| inside the |fullpagefigure| environment.
%    \begin{macrocode}
\def\phffpf@impl@label#1{%
  \gdef\phffpf@val@label{#1}%
  \ignorespaces%
}
%    \end{macrocode}
% \end{macro}
% 
% \begin{macro}{\figplacement}
%   This will become |\figplacement| inside the |fullpagefigure| environment.
%    \begin{macrocode}
\def\phffpf@impl@placement#1{%
  \gdef\phffpf@val@placement{#1}%
  \ignorespaces%
}
%    \end{macrocode}
% \end{macro}
% 
% \begin{macro}{\figcapmaxheight}
%   This will become |\figcapmaxheight| inside the |fullpagefigure|
%   environment.
%    \begin{macrocode}
\def\phffpf@impl@capmaxheight#1{%
  \gdef\phffpf@val@capmaxheight{#1}%
  \ignorespaces%
}
%    \end{macrocode}
% \end{macro}
% 
%
%
% \subsection{Placing the figures}
%
% Here's the gory details of how the figures are placed.
%
%
% \begin{macro}{\phffpf@place@pending@figs@code}
%   This macro will store code to be executed after the next figure has
%   been placed.  This can be used to queue other figures to be placed
%   later.
%    \begin{macrocode}
\def\phffpf@place@pending@figs@code{\phffpf@place@pending@figs@code@start}
%    \end{macrocode}
% \end{macro}
%
% \begin{macro}{\phffpf@place@pending@figs@code@start}
%   When another figure is placed, and the
%   |\phffpf@place@pending@figs@code| is updated, then the macro
%   |\phffpf@place@pending@figs@code@start| contains the code which
%   reinitializes |\phffpf@place@pending@figs@code|.
%
%   This reinitialization code consists in precisely making sure that a
%   future execution of |\phffpf@place@pending@igs@code@start| will start
%   by reinitializing that macro.
%    \begin{macrocode}
\def\phffpf@place@pending@figs@code@start{%
  \gdef\phffpf@place@pending@figs@code{\phffpf@place@pending@figs@code@start}}
%    \end{macrocode}
% \end{macro}
%
%
% \begin{macro}{\phffpf@impl@figcode}
%   The code to be inserted to generate the figure.
%
%   The argument |#1| is the prefix for macro names where to look up the
%   contents of the figure and values of the figure settings.  The macro
%   names are determined as |\csname #1@|\meta{field-name}|\endcsname|.
%    \begin{macrocode}
\gdef\phffpf@impl@figcode#1{%
%    \end{macrocode}
% 
% Do we have a figure placement position request (|p|, |t|, |b|)?  If yes,
% then define a macro which we will expand in front of the |\begin{figure}|
%   command for the caption.  If no, then that macro should be left blank
%   (first case below):
%    \begin{macrocode}
  \expandafter\ifblank\expandafter{\csname #1@placement\endcsname}{%
    \edef\phffpf@tmp@figplacementarg{}%
  }{%
    \edef\phffpf@tmp@figplacementarg{[\csname #1@placement\endcsname]}%
  }
%    \end{macrocode}
% 
% Invoke the |figure| environment, which we use to typeset the caption.
% Use specified placement if applicable.  Set up some basic stuff in the
% figure: the contents, caption and label.
%    \begin{macrocode}
  \expandafter\figure\phffpf@tmp@figplacementarg%
  \centering%
  \begingroup%
  \def\fnum@figure{\fullpagefigurecaptionfmt{\csname #1@pageside\endcsname}}%
  \expandafter\afterpage\expandafter{\csname #1@figcontents\endcsname}%
  \expandafter\ifx\csname #1@captionopt\endcsname\phffpf@test@NOARG%
    \expandafter\caption\expandafter{\csname #1@caption\endcsname}%
  \else%
    \def\phffpf@tmp@captioncmdopt{%
      \expandafter\caption\expandafter[\csname #1@captionopt\endcsname]}%
    \expandafter\phffpf@tmp@captioncmdopt\expandafter{\csname #1@caption\endcsname}%
  \fi%
  \expandafter\notblank\expandafter{\csname #1@label\endcsname}{%
    \expandafter\label\expandafter{\csname #1@label\endcsname}%
  }{%
  }
  \endgroup%
  \endfigure%
%    \end{macrocode}
%
% Now we have placed the |figure|, so decrease our ``pending-to-be-placed''
% counter.
%    \begin{macrocode}
  \addtocounter{phffpf@internal@pending}{-1}%
%    \end{macrocode}
%
% \ldots{} and execute the code to place any other pending figures.  (We
% set |\ifphffpf@flag@forcenextmaybequeuetoplacefigure| to TRUE to force
% the next figure in queue to be placed now.)
%    \begin{macrocode}
  \afterpage{%
    \phffpf@flag@forcenextmaybequeuetoplacefiguretrue%
    \phffpf@place@pending@figs@code%
  }%
}
%    \end{macrocode}
% \end{macro}
%
%
% Now, all options have been set etc., the fullpagefigure environment has finished, so
% calculate the commands to place the figure appropriately.
%
%
% \begin{macro}{\phffpf@takecareofplacingfigure}
%   First, fix the values of the contents and settings (in case another
%   full-page-figure comes along and messes up the |\phffpf@val@...|
%   commands).
%
%   After the values have been fixed (in fact they are stored in the form
%   of ``restore code''), then we delegate to
%   |\phffpf@maybequeuefigurecode|, which checks whether we can place a
%   figure or if we should queue.
%    \begin{macrocode}
\def\phffpf@takecareofplacingfigure{%
%    \end{macrocode}
%
% A tricky part: make sure we save the values of |phffpf@val@|\meta{field}
% in a fixed way so that several figures won't overwrite each other's
% values.
%
% We build a bunch of tokens which are in fact restore code for the given
% variables, i.e., which is a list of commands of the form
% |\gdef\phffpf@val@|\meta{field}|{|\meta{first-level-expanded-value-of-this-field}|}|.
% This set of tokens have the values of these variables expanded to the
% first level, so that it is OK if the variables |\phffpf@val@|\meta{field}
% are overwritten.
%    \begin{macrocode}
  \edef\phffpf@tmp@fixallfieldvalues{%
    \noexpand\gdef\noexpand\phffpf@val@pageside{\expandonce\phffpf@val@pageside}%
    \noexpand\gdef\noexpand\phffpf@val@captionopt{\expandonce\phffpf@val@captionopt}%
    \noexpand\gdef\noexpand\phffpf@val@caption{\expandonce\phffpf@val@caption}%
    \noexpand\gdef\noexpand\phffpf@val@label{\expandonce\phffpf@val@label}%
    \noexpand\gdef\noexpand\phffpf@val@placement{\expandonce\phffpf@val@placement}%
    \noexpand\gdef\noexpand\phffpf@val@capmaxheight{\expandonce\phffpf@val@capmaxheight}%
    \noexpand\gdef\noexpand\phffpf@val@figcontents{\expandonce\phffpf@val@figcontents}%
  }%
%    \end{macrocode}
% 
% Finally, relay the call to
% |\phffpf@maybequeuefigurecode|\marg{restore-code-for-figure-settings}\hspace{0pt}\marg{full-figure-code}.
%    \begin{macrocode}
  \edef\phffpf@tmp@figcodetwoargs{%
    {\expandonce\phffpf@tmp@fixallfieldvalues}%
    {\noexpand\phffpf@impl@figcode{phffpf@val}}%
  }%
  \expandafter\phffpf@maybequeuefigurecode\phffpf@tmp@figcodetwoargs%
}
%    \end{macrocode}
% \end{macro}
% 
% \begin{macro}{\phffpf@maybequeuefigurecode}
%   USAGE: |\phffpf@maybequeuefigurecode|\marg{restore-code-for-figure-settings}\hspace{0pt}\marg{full-figure-code}.
%
%   Checks if we can place the figure; if yes then place it on the right
%   page, if no, then add it to the queue.
%
%   The arguments are: |#1| = code to restore correct |\phffpf@val@XYZ|
%   values; |#2| = figure code.  Make sure it's expanded.
%    \begin{macrocode}
\long\def\phffpf@maybequeuefigurecode#1#2{%
%    \end{macrocode}
% 
% Possibly we have been told to place the next figure now via the flag
% |\ifphffpf@flag@forcenextmaybequeuetoplacefigure|.  In this case, reset
% the flag and place the figure now (relay to |\phffpf@doplacefigure|).
%    \begin{macrocode}
  \ifphffpf@flag@forcenextmaybequeuetoplacefigure%
    \phffpf@flag@forcenextmaybequeuetoplacefigurefalse
    \phffpf@doplacefigure{#1}{#2}%
    %
  \else
%    \end{macrocode}
%
% See if there are other figures waiting to be placed first.  If so, add
% ours to the queue.
%    \begin{macrocode}
    \ifnum\value{phffpf@internal@pending}>1\relax%
      \xdef\phffpf@place@pending@figs@code{%
          \expandonce\phffpf@place@pending@figs@code%
          \unexpanded{\phffpf@maybequeuefigurecode{#1}{#2}}%
        }%
        %\show\phffpf@place@pending@figs@code
        %[figure queued:  \texttt{\detokenize{#1}}]%  -- DEBUGGING
    \else%
%    \end{macrocode}
%
% If not, deal with placing the figure now:
%    \begin{macrocode}
      \phffpf@doplacefigure{#1}{#2}%
      %[figure placed: \texttt{\detokenize{#1}}]  -- DEBUGGING
    \fi%
  \fi%
}
%    \end{macrocode}
% 
% Define also the flag which will force a next call to
% |\phffpf@maybequeuefigurecode| to place the next figure in the queue.
%    \begin{macrocode}
\newif\ifphffpf@flag@forcenextmaybequeuetoplacefigure
\phffpf@flag@forcenextmaybequeuetoplacefigurefalse
%    \end{macrocode}
% \end{macro}
% 
%
% \begin{macro}{\phffpf@doplacefigure}
%   Place the figure now.  Determine the correct number of |\afterpage|'s
%   to use so that the figure caption ends up on the correct page side.
%
%   The arguments to this macro are: |#1| = code to restore correct
%   |phffpf@val@XYZ| values, |#2| = the figure code. Make sure it's
%   expanded.
%    \begin{macrocode}
\long\def\phffpf@doplacefigure#1#2{%
%    \end{macrocode}
%
% Make sure the correct values of |phffpf@val@XYZ| are restored, because we
% need e.g. |\phffpf@val@pageside|.  They may be wrong because this might
% be called after a figure has been queued.
%    \begin{macrocode}
  #1%
%    \end{macrocode}
%
% Now, determine where exactly to place the figure code.  There are no other pending
% figures.  If there is no side preference, just place the figure pretty much now.
%    \begin{macrocode}
  \ifx\phffpf@val@pageside\phffpf@side@%
    \let\phffpf@tmp@doplace\@firstofone%
  \else%
%    \end{macrocode}
%
% If, however, we have a side preference, then check everything more
% carefully.  Use the helper macros
% |\phffpf@placecode@on|\meta{\phfverb{same}$\mid$\phfverb{other}}|parity|.
% (The latter essentially expand to the correct number of |\afterpage|'s.)
%    \begin{macrocode}
    \ifx\phffpf@val@pageside\phffpf@side@odd%
      %[CHECK DONE HERE/WANT ODD] % -- for debugging
      \checkoddpage\ifoddpage%
      %[IS ODD] % -- for debugging
        \let\phffpf@tmp@doplace\phffpf@placecode@onotherparity%
      \else%
        %[IS NOT ODD] % -- for debugging
        \let\phffpf@tmp@doplace\phffpf@placecode@onsameparity%
      \fi%
    \else%
      %[CHECK DONE HERE/WANT EVEN] % -- for debugging
      \checkoddpage\ifoddpage%
        %[IS ODD] % -- for debugging
        \let\phffpf@tmp@doplace\phffpf@placecode@onsameparity%
      \else%
        %[IS NOT ODD] % -- for debugging
        \let\phffpf@tmp@doplace\phffpf@placecode@onotherparity%
      \fi%
    \fi%
  \fi%
%    \end{macrocode}
% 
% I think an |\hbox{}| might help to place the anchor which determines
% which page side we are currently on.  Note that this starts a new
% paragraph and enters horizontal mode.
%    \begin{macrocode}
  \leavevmode\hbox{}%
%    \end{macrocode}
%
% Now, do place the figure somewhere.
%    \begin{macrocode}
  \phffpf@tmp@doplace{#1#2}%
}
%    \end{macrocode}
% \end{macro}
%
%
% \begin{macro}{\phffpf@placecode@onsameparity}
%   Place the figure code on the same parity (page side) as we are
%   currently.  If enough space remains on the current page, place the
%   figure immediately.  Otherwise, use two |\afterpage|'s so as the figure
%   caption to appear in two pages.
%
%    \begin{macrocode}
\newdimen\phffpf@tmp@spaceleft
\newdimen\phffpf@tmp@compareto
\long\def\phffpf@placecode@onsameparity#1{%
%    \end{macrocode}
%
% First, see if the caption itself requires to be on its own page (and thus
% no height calculations are necessary, and an additional |\clearpage| is
% required)
% 
%    \begin{macrocode}
  \def\@tmpa{p}%
  \ifx\phffpf@val@placement\@tmpa%
    \afterpage{\vspace*{0pt}\afterpage{#1\clearpage}}%
  \else%
%    \end{macrocode}
%
% Otherwise, the figure caption is there along with some text on the page.
% 
% See if there is enough place left on this page to place the figure
% caption; otherwise use two |\afterpage|'s.
%    \begin{macrocode}
    %[PLACING FIG CODE ON SAME PARITY]% -- debugging
    \phffpf@tmp@spaceleft=\textheight\relax%
    \phffpf@tmp@compareto=\phffpf@val@capmaxheight\relax%
    \advance\phffpf@tmp@spaceleft by -\pagetotal%
    %[DIM LEFT: \the\phffpf@tmp@spaceleft]%
    \ifdim\phffpf@tmp@spaceleft>\phffpf@tmp@compareto%
      %[ENOUGH DIM LEFT.] % -- debugging
      #1%\phffpf@tmp@figcode%
    \else%
      %[*NOT ENOUGH* DIM LEFT.] % -- debugging
      \afterpage{\vspace*{0pt}\afterpage{#1}}%
    \fi%
  \fi%
}
%    \end{macrocode}
% \end{macro}
% 
% \begin{macro}{\phffpf@placecode@onotherparity}
%   Place the figure caption on the opposite parity as the current page.
%   This just requires one |\afterpage| so as the figure code to appear on
%   the following page.
%
%    \begin{macrocode}
\def\phffpf@placecode@onotherparity#1{%
  %[PLACING FIG CODE ON OTHER PARITY]% -- debugging
%    \end{macrocode}
% 
% First, see if the caption requires to be on its own page (and thus no
% height calculations are necessary, and an additional |\clearpage| is
% required).
%    \begin{macrocode}
  \def\@tmpa{p}%
  \ifx\phffpf@val@placement\@tmpa%
    \afterpage{#1\clearpage}%
  \else%
%    \end{macrocode}
%
% The figure caption is not on its own page.  Just use a simple |\afterpage|.
%    \begin{macrocode}
    \afterpage{#1}%
  \fi%
}
%    \end{macrocode}
% \end{macro}
% 
%
% \subsection{Commands to Flush All Full-Page-Figures}
%
%
% Here are a set of commands which can be used to ensure that all full-page
% figure floats have been placed.
%
% \begin{macro}{\FlushAllFullPageFigures}
%   The name is pretty self-explanatory.  The command is documented in the user doc above.
%    \begin{macrocode}
\newcommand\FlushAllFullPageFigures[1][\phffpf@clearpage]{%
%    \end{macrocode}
% 
% As long as there are full-page-figures pending, clear pages until those
% figures have been placed.
%    \begin{macrocode}
  \ifnumcomp{\value{phffpf@internal@pending}}{>}{0}{%
    \clearpage%
    %[page cleared.]% DEBUG
    \FlushAllFullPageFigures[#1]% recurse again.
  }{%
%    \end{macrocode}
%
% All figures placed, all fine.  We still need to flush one last time
% because at this point the figure code (ie. caption) has been placed only,
% and we want the text that follows to come after the figure itself.  Here
% finally we use the clear command in |#1| to continue on any page or on an
% odd-side page only.
%    \begin{macrocode}
    #1%
  }%
}
\def\phffpf@clearpage{\if@twoside\cleardoublepage\else\clearpage\fi}
%    \end{macrocode}
% \end{macro}
% 
%
%
% \subsection{Package Option Parsing}
%
% Note the singular form of the word ``option.''
% 
%    \begin{macrocode}
\def\phffpf@provide@figpdf{}
\newcommand\phffpf@impl@figpdf[2][]{%
  \figcontents{\includepdf[#1]{#2}}%
}
\def\phffpf@do@pdfpages{%
  \RequirePackage{pdfpages}%
  \def\phffpf@provide@figpdf{\let\figpdf\phffpf@impl@figpdf}%
}
%
\DeclareOption{nopdfpages}{\def\phffpf@do@pdfpages{}}
\DeclareOption*{%
  \@unknownoptionerror%
}
\ProcessOptions\relax
%
\phffpf@do@pdfpages
%    \end{macrocode}
%
%\Finale
\endinput
