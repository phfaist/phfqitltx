% \iffalse meta-comment
%
% Copyright (C) 2016 by Philippe Faist <phfaist@gmail.com>
% -------------------------------------------------------
% 
% This file may be distributed and/or modified under the
% conditions of the LaTeX Project Public License, either version 1.3
% of this license or (at your option) any later version.
% The latest version of this license is in:
%
%    http://www.latex-project.org/lppl.txt
%
% and version 1.3 or later is part of all distributions of LaTeX 
% version 2005/12/01 or later.
%
% \fi
%
% \iffalse
%<*driver>
\ProvidesFile{phfquotetext.dtx}
%</driver>
%<package>\NeedsTeXFormat{LaTeX2e}[2005/12/01]
%<package>\ProvidesPackage{phfquotetext}
%<*package>
    [2016/08/15 v1.0 phfquotetext package]
%</package>
%
%<*driver>
\documentclass{ltxdoc}
\usepackage{xcolor}
\usepackage{phfquotetext}
\usepackage[preset=xpkgdoc]{phfnote}
\usepackage[normalem]{ulem}
\EnableCrossrefs         
\CodelineIndex
\RecordChanges
\begin{document}
  \DocInput{phfquotetext.dtx}
\end{document}
%</driver>
% \fi
%
% \CheckSum{0}
%
% \CharacterTable
%  {Upper-case    \A\B\C\D\E\F\G\H\I\J\K\L\M\N\O\P\Q\R\S\T\U\V\W\X\Y\Z
%   Lower-case    \a\b\c\d\e\f\g\h\i\j\k\l\m\n\o\p\q\r\s\t\u\v\w\x\y\z
%   Digits        \0\1\2\3\4\5\6\7\8\9
%   Exclamation   \!     Double quote  \"     Hash (number) \#
%   Dollar        \$     Percent       \%     Ampersand     \&
%   Acute accent  \'     Left paren    \(     Right paren   \)
%   Asterisk      \*     Plus          \+     Comma         \,
%   Minus         \-     Point         \.     Solidus       \/
%   Colon         \:     Semicolon     \;     Less than     \<
%   Equals        \=     Greater than  \>     Question mark \?
%   Commercial at \@     Left bracket  \[     Backslash     \\
%   Right bracket \]     Circumflex    \^     Underscore    \_
%   Grave accent  \`     Left brace    \{     Vertical bar  \|
%   Right brace   \}     Tilde         \~}
%
%
% \changes{v1.0}{2016/04/20}{Initial version}
%
% \GetFileInfo{phfquotetext.dtx}
%
% \DoNotIndex{\newcommand,\newenvironment,\def,\gdef,\edef,\xdef,\if,\else,\fi,\ifx}
% 
% \title{\phfqitltxPkgTitle{phfquotetext}}
% \author{Philippe Faist\quad\email{phfaist@gmail.com}}
% \date{\pkgfmtdate\filedate}
% \maketitle
%
% \begin{abstract}
%   \pkgname{phfquotetext}---Provides an environment for displaying
%   block text with special characters, such as verbatim quotes from a
%   referee report.
% \end{abstract}
%
% \inlinetoc
%
% \section{Introduction}
%
% This package provides a quoting environment useful to quote, e.g., email text.  It is
% somewhat like a verbatim environment (for instance, there is no macro expansion), but
% with also no white space formatting.
%
% For instance, you might find this useful for formatting replies to a
% referee, because any pseudo-\LaTeX{} commands are reproduced
% verbatim, but the bulk is still nicely formatted:
%
% \begin{verbatim}
% \begin{quotetext}
%   This is some text which was provided in plain text, for example,
%   by the referee.  The object \calE should probably be defined with
%   a factor 1/2 to help the \sum_k \ket{k} states not overwhelm rho
%   by this and that by more than 50%....
% \end{quotetext}
% \end{verbatim}
%
% \verbdef\tmpA|\quotetextfont|
% \verbdef\tmpB|\sffamily\color{blue}|
% The above code gives the following output:\footnote{
%   Here, we've assumed the \tmpA{} was redefined as
%   \tmpB{}.  Because it does look nicer.  See
%   \autoref{sec:customize-appearance}.}
%
% \renewcommand\quotetextfont{\sffamily\color{blue}}
% \begin{quotetext}   This is some text which was provided in plain text, for example, by a   referee.  The object \calE should probably be defined with a factor 1/2 to   help the \sum_k \ket{k} states not overwhelm rho by this and that by more than 50%.... \end{quotetext}
%
%
% 
%
%
% \section{Usage}
%
% \subsection{The main environment}
%
% The main environment is simply |\begin{quotetext} ... \end{quotetext}|.
%
% \DescribeEnv{quotetext}
% Delimit your email quote within this environment:
%
% \begin{verbatim}
% \begin{quotetext}
%   Stuff which you want to display verbatim, but with automatic white
%   space formatting and text re-flow.  \commands will not be
%   expanded, and will be displayed verbatim.  Percent signs also work
%   100%.
% \end{quotetext}
% \end{verbatim}
%
% The |{quotetext}| environment does not start a paragraph on its own, nor does it end it.
% You can thus in principle use this environment to quote text inline within a paragraph
% if you want to, unless you're enforcing new paragraphs in |\quotetextstart| and
% |\quotetextend|.
%
%
% \subsection{Customize appearance}
% \label{sec:customize-appearance}
%
% \DescribeMacro{\quotetextfont} This macro contains the formatting commands to set up the
% quote font.  By default, the contents is displayed in typewriter font.  You may redefine
% this command to achieve your favorite style:
% \begin{verbatim}
% \renewcommand\quotetextfont{\sffamily\color{blue}}
% \end{verbatim}
% 
% This macro may also be defined to accept one argument, which is the whole contents of
% the quoted text.  Make sure you define this with |\renewcommand| or |\long\def|, because
% the argument may contain several paragraphs.
%
% \DescribeMacro{\quotetextstart}
% This command gets called at the beginning of the quote environment.  It may be used
% e.g.\@ to add space, a horizontal rule, or add other formatting goodies.  By default, it
% is empty.
%
% \DescribeMacro{\quotetextend}
% This command gets called at the end of the quote environment.  It may be used e.g.\@ to
% add space, a horizontal rule, or add other formatting goodies.  By default, it is empty.
%
%
% The macros |\quotetextfont|, as well as |\quotetextstart| and
% |\quotetextend| are called within a \TeX{} group, so you don't need
% to worry that font overrides leak outside of the |quotetext|
% environment.
%
%
% You may try, for instance, the following to get started with customization:
% \begin{verbatim}
% \renewcommand\quotetextstart{%
%   \par\noindent{\color{blue}\hrule height 0.8pt}\vspace{2pt}\vspace{-\parskip}}
% \renewcommand\quotetextend{%
%   {\color{blue}\vspace{2pt}\hrule height 0.8pt\vspace{2pt}}}
% % \sout{} requires \usepackage[normalem]{ulem}:
% \renewcommand\quotetextfont[1]{\color{orange}\itshape\sout{\ignorespaces #1}}
% \end{verbatim}
% which gives:
% \renewcommand\quotetextstart{%
%   \par\noindent{\color{blue}\hrule height 0.8pt}\vspace{2pt}\vspace{-\parskip}}
% \renewcommand\quotetextend{%
%   {\color{blue}\vspace{2pt}\hrule height 0.8pt\vspace{2pt}}}
% \renewcommand\quotetextfont[1]{\color{orange}\itshape\sout{\ignorespaces #1}}
% \begin{quotetext} Stuff which you want to display verbatim, but with automatic white   space formatting and text re-flow.  \commands will not be expanded, and   will be displayed verbatim.  Percent signs also work 100%. \end{quotetext}
% 
%
% \subsection{Customize Catcodes---Advanced Usage}
%
% \DescribeMacro{\quotetextcatcodedefs}
% You may redefine this command to set catcodes before parsing the bulk of the
% |{quotetext}| environment.  This macro is called after \TeX's specials are already
% assigned catcode |12| (other).  By default, this macro is empty.
%
% \section{Known Caveats}
%
% \begin{itemize}
% \item The leading space in the environment content is not trimmed
%   from the argument passed on to |\quotetextfont|.  This just means
%   that if |\quotetextfont| is defined to accept an argument, it must
%   not forget to precede it with |\ignorespaces| as appropriate (as
%   in the example above in \autoref{sec:customize-appearance}).
%   I'm not sure how to remove trailing whitespace.
% \item By default, the |{quotetext}| environment does not start a new paragraph nor end
%   the paragraph, unless you enforce this explicitly in |\quotetextstart| and
%   |\quotetextend|.  This is purposely left as a feature to be aware of.
% \end{itemize}
%
%
%
%
% \StopEventually{\PrintChanges\PrintIndex}
%
% \section{Implementation}
%
% \subsection{Main Macros and Environment}
%
% \begin{macro}{\quotetextcatcodedefs}
%   Allows to set custom catcodes for parsing of quotetext contents.
%
%   By default, all char catcodes are set as "other" characters, but you may
%   redefine this macro to define you own catcodes.
%   
%   This macro is empty by default.
%    \begin{macrocode}
\newcommand\quotetextcatcodedefs{}
%    \end{macrocode}
% \end{macro}
%
% \begin{macro}{\quotetextstart}
% \begin{macro}{\quotetextend}
%   Definitions which are called at the beginning or end of every
%   quote environment.  They are empty by default.
%    \begin{macrocode}
\newcommand\quotetextstart{}
\newcommand\quotetextend{}
%    \end{macrocode}
% \end{macro}
% \end{macro}
%
% \begin{macro}{\quotetextfont}
% Set up the font and display for the quote text environment.
%
% This command may also be defined to take one single argument, which will be
% the whole contents of the quotetext environment.
%    \begin{macrocode}
\newcommand\quotetextfont{\ttfamily}
%    \end{macrocode}
% \end{macro}
%
%
% \begin{environment}{quotetext}
%   The main environment.  Starts group and relays the call to |\phfquotetext@warg| to
%   start reading the contents of the environment.
%
%   Start the environment definition.
%    \begin{macrocode}
\newenvironment{quotetext}{%
%    \end{macrocode}
% \iffalse meta-comment %} -- for emacs' paren matching
% \fi
%
% Do anything we should do before the |quotetext| environment.
%    \begin{macrocode}
  \quotetextstart%
%    \end{macrocode}
%
% Make sure to enclose the full quote text stuff into its own group to make definitions
% such as catcode overrides local.
%    \begin{macrocode}
  \begingroup%
%    \end{macrocode}
% 
% Finally, relay the call to |\phfquotetext@warg| to read the whole contents of the
% environment, and finish the environment definition.
%    \begin{macrocode}
  \phfquotetext@warg%
}
%    \end{macrocode}
% 
% Finally, in the closing part of the environment, do anything which
% needs to be done at the end.
%    \begin{macrocode}
{%
  \quotetextend%
}
%    \end{macrocode}
% \end{environment}
%
%
%
% \subsection{Internal Stuff}
%
% \begin{macro}{\phfquotetext@warg@}
%   Reads an argument delimited by ....|\end{quotetext}|, ends the current \TeX{} group
%   assumed open (in order to change the catcodes), and calls the start/show/end commands.
%
%   We need deep catcode dark magic to just even define this command.
%   \begin{macrocode}
\begingroup
  \catcode`|=0
  \catcode`\<=1
  \catcode`\>=2
  \catcode`\{=12
  \catcode`\}=12
  \catcode`\\=12
  |long|gdef|phfquotetext@warg@#1\end{quotetext}<|endgroup
    |phfquotetext@show<#1>>
|endgroup
%    \end{macrocode}
% \end{macro}
%
%
% \begin{macro}{\phfquotetext@show}
%   Shows the content given as argument with the proper formatting.  Includes an actual
%   call to |\end{quotetext}| to make \LaTeX's bookcounting mechanism for |\begin|/|\end|
%   happy, and which finally also calls |\quotetextend|.
%   \begin{macrocode}
\newtoks\qtt@mytoks
\long\def\phfquotetext@show#1{%
  \qtt@mytoks={#1}%
  \begingroup%
%    \end{macrocode}
% Crucially, |\the\toks| only expands once and prohibits further
% expansion.  This way, we may pass the argument directly to
% |\quotetextfont|, with tokens expanded exactly as they were given:
%    \begin{macrocode}
  \expandafter\quotetextfont\expandafter{\the\qtt@mytoks}%
  \endgroup%
  \end{quotetext}%
}
%    \end{macrocode}
% \end{macro}
%
%
% \begin{macro}{\phfquotetext@warg}
%   Prepares to read the contents of the environment.  A new \TeX{} group has already been
%   opened by |\begin{quotetext}|.
%    \begin{macrocode}
\def\phfquotetext@warg{%
  %
%    \end{macrocode}
% Set the catcode of all special chars to ``other'' (12).  We use |\dospecials| provided
% by \LaTeX{} itself.\footnote{see \url{http://tex.stackexchange.com/a/12890/32188}} We
% leave the space token to the ``space'' catcode, so that white space is properly managed
% by LaTeX.  Finally, include any user catcode overrides in |\quotetextcatcodedefs|.
%    \begin{macrocode}
  \let\do\@makeother%
  \dospecials%
  \catcode`\ =10\relax% keep space
  \quotetextcatcodedefs%
%    \end{macrocode}
%
% Finally, relay the call to |\phfquotetext@warg@| to read the contents of the group and
% close the \TeX{} group.
%
%    \begin{macrocode}
  \phfquotetext@warg@%
}
%    \end{macrocode}
% \end{macro}
%
%
%\Finale
\endinput
