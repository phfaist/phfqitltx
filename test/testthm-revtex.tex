\documentclass[%
  aps,%
  pra,%
  superscriptaddress,%
  reprint,%
  longbibliography,%
  nofootinbib,%
  notitlepage]{revtex4-2}


\usepackage[preset=reset,pkgset=extended,hyperrefdefs={defer,noemail}]{phfnote}
\usepackage{phfqit}
\usepackage{phfparen}
\usepackage[thmset=rich,smallproofs=false,proofref={marginbottom}]{phfthm}
%\renewcommand\proofonname[2]{\raggedright\emph{(Proof on #2.)}}
\renewcommand\proofonname[2]{\textit{(Proof on #2.)}}


%\usepackage[a5paper,vmargin=2in,marginparsep=4pt,marginparwidth=30pt]{geometry}
\usepackage[reset]{geometry}
\csname Gm@restore@org\endcsname


\usepackage{hyperref}
\usepackage[nameinlink,capitalize]{cleveref}


\makeatletter
\renewcommand\phfthmproofref@placephantommarginpar{%
  \begingroup\csdef{@captype}{figure}\marginpar{%
    \vspace{-\baselineskip}\rule{0pt}{0pt}}\endgroup%
}
\makeatother


\phfMakeTheorem[counter={},defstar=false,proofref=false]{algorithm}{Algorithm}

\def\proofrefsize{\tiny}


\begin{document}
\title{Test document}
\author{Me}
\author{You}
% \begin{abstract} Test document! Test document! Test document! Test document!
%   Test document! Test document! Test document! Test document! Test document!
%   Test document! Test document! Test document! Test document! Test document!
%   Test document! Test document!\end{abstract}
\maketitle


\section{Test section 1}

\begin{lemma}
\label{lem:1}
Another lemma here.
\end{lemma}
\begin{proof}[a different Lemma]
  this is another proof.
\end{proof}



\begin{proposition}
  \label{prop:1}
  Proposition here.
\end{proposition}
\begin{proof}[*prop:1]
  This is a proof.

  And more stuff.
\end{proof}

\begin{proof}[*x:backrefproof]
  This is a funny back-referenced proof.
\end{proof}


You might also like the defintion of the \ref{thmheading:trace-dist} on \autopageref{thmheading:trace-dist}.

\begin{proposition}
  \noproofref
  Another proposition here.  This proposition has no proof ref.
\end{proposition}

\begin{proof}
  This is yet another proof.  This proof has no associated theorem.
\end{proof}


\begin{theorem}
\label{thm:1}
And a theorem
\end{theorem}


\begin{theorem}
\noproofref
\label{thm:2}
I heard you like theorems?
\end{theorem}

\begin{definition*}
This is a definition.
\end{definition*}


\begin{proposition*}
Another proposition here, with no numbering.
\end{proposition*}
\begin{proof}
  this is yet another proof.
\end{proof}


\begin{figure}
\centering
\rule{4cm}{6cm}
\caption{Figure one}
\end{figure}

\begin{theorem}
  \label{thm:another}
  Yet another theorem here.
\end{theorem}


\begin{figure}
\centering
\rule{5cm}{5cm}
\caption{Figure two somehow messes up marginpar's in my other document?}
\end{figure}

See also \autoref{prop:1}, \autoref{thm:1} and \autoref{lem:1}

\clearpage
\newgeometry{hmargin=1.5in,vmargin=1in,marginparsep=12pt,marginparwidth=40pt}
\onecolumngrid

\appendix
\section{Appendix One}


\begin{proposition}[Solutions to the anticommutator equation]
  \label{thm:solutions-to-anticommutator-eqn}
  BVLah lbha kdnafklsfdn klsan fdklsanf kldsnaflk kasl BVLah lbha kdnafklsfdn
  klsan fdklsanf kldsnaflk kasl BVLah lbha kdnafklsfdn klsan fdklsanf kldsnaflk
  kasl BVLah lbha kdnafklsfdn klsan fdklsanf kldsnaflk kasl BVLah lbha
  kdnafklsfdn klsan fdklsanf kldsnaflk kasl BVLah lbha kdnafklsfdn klsan
  fdklsanf kldsnaflk kasl BVLah lbha kdnafklsfdn klsan fdklsanf kldsnaflk kasl
\end{proposition}

bla blah blah

\begin{proof}[**thm:solutions-to-anticommutator-eqn]
  .................. ....................... 
\end{proof}


\begin{proposition}[Locally optimal sensing]
  \label{thm:optimal-local-sensing-obs}
  Any operator $T$ that is optimal.................. ................
\end{proposition}


\begin{proof}[*thm:optimal-local-sensing-obs]
  Without loss of generality, we assume $t=0$ throughout this proof; this is
  achieved by shifting the parameter to center it at zero, implying the
\end{proof}


xxxx
\begin{proof}[*thm:1]
  ....... and this is yet yet another proof.  This proof is only on the next page of the
  theorem, so it should not necessary display the reference there.
\end{proof}

\begin{thmheading}{Trace Distance}
  \label{thmheading:trace-dist}
  The \emph{trace distance} is given by the following expression:
  \begin{align}
    \delta(\rho,\sigma) = \frac12 \left\lVert \rho - \sigma \right\rVert_1
  \end{align}
\end{thmheading}

\begin{theorem}
  \label{thm:long}
  Some theorem which has a little longer statement.
  Some theorem which has a little longer statement.
  Some theorem which has a little longer statement.
  Some theorem which has a little longer statement.

  Some theorem which has a little longer statement.
  Some theorem which has a little longer statement.
  Some theorem which has a little longer statement.
\end{theorem}

\cleardoublepage

\section{Test section 2}

\begin{proof}[*lem:1]
  this is another proof.
\end{proof}

\begin{algorithm}
  \label{algo:1}
  An algorithm here.
  % note that no proof-ref is generated.
\end{algorithm}

\begin{algorithm}
  \label{algo:2}
  Another algorithm here.
\end{algorithm}

\cleardoublepage

\begin{proof}[*thm:another]
  Proof proof proof proof. You're convinced?
\end{proof}

\begin{proof}[*algo:1]
  Proof of an algorithm (WTF?).
\end{proof}

\begin{proof}[*thm:long]
  Proof.
\end{proof}

\begin{claim}
  I claim this and that.
  % should generate warning of missing label
\end{claim}

\cleardoublepage

\begin{theorem}
  \label{thm:blablabla}
  A theorem with a proof that follows immediately.
\end{theorem}
\begin{proof}[**thm:blablabla]
  Proof of the immediately preceding theorem.
\end{proof}


\begin{theorem}[Final Theorem]
  \label{x:backrefproof}
  A very fine final theorem.
\end{theorem}

\end{document}


%%% Local Variables: 
%%% mode: latex
%%% TeX-master: t
%%% End: 
