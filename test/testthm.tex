\documentclass[12pt,a5paper]{article}

%\usepackage[thmset=default,sepcounters,proofref={margin,always}]{../phfthm}
\usepackage[thmset=default,sepcounters,proofref={longref,margin}]{../phfthm}

\def\proofrefsize{\tiny}

\usepackage{hyperref}  % for \autoref{}


\begin{document}


\begin{lemma}
\label{lem:1}
Another lemma here.
\end{lemma}
\begin{proof}[of a different Lemma]
  this is another proof.
\end{proof}


\begin{proposition}
  \label{prop:1}
  Proposition here.
\end{proposition}
\begin{proof}[*prop:1]
  This is a proof.

  And more stuff.
\end{proof}


\begin{proposition}
\noproofref
Another proposition here.
\end{proposition}
\begin{proof}
  this is yet another proof.
\end{proof}


\begin{theorem}
\label{thm:1}
And a theorem
\end{theorem}


\begin{theorem}
\noproofref
\label{thm:2}
I heard you like theorems?
\end{theorem}

\begin{definition*}
\label{def:1}
This is a definition.
\end{definition*}


\begin{proposition*}
Another proposition here, with no numbering.
\end{proposition*}
\begin{proof}
  this is yet another proof.
\end{proof}

\begin{theorem}
  \label{thm:another}
  Yet another theorem here.
\end{theorem}


See also \autoref{prop:1}, \autoref{thm:1} and \autoref{lem:1}

\cleardoublepage
x
\begin{proof}[*thm:1]
  ....... and this is yet yet another proof.  This proof is only on the next page of the
  theorem, so it should not necessary display the reference there.
\end{proof}

\cleardoublepage

\begin{proof}[*lem:1]
  this is another proof.
\end{proof}

\cleardoublepage

\begin{proof}[*thm:another]
  Proof proof proof proof. You're convinced?
\end{proof}



\end{document}


%%% Local Variables: 
%%% mode: latex
%%% TeX-master: t
%%% End: 
