\documentclass[12pt,a5paper]{article}

%\usepackage[thmset=rich,sepcounters,proofref={margin,always},prooftitleitbf]{phfthm}
%\usepackage[thmset=rich,sepcounters,countpersection,proofref={margin,always},prooftitleitbf]{phfthm}
\usepackage[thmset=rich,countpersection,proofref={marginbottom,always},prooftitleitbf]{phfthm}
%\usepackage[theoremstyle=definition,definitionstyle=remark,thmset=rich,proofref={margin,longref}]{phfthm}
%\usepackage[sepcounters,proofref,thmset=rich]{phfthm}
%\usepackage[proofref=off]{phfthm}
%\usepackage{phfthm}
%\usepackage[resetstyle,prooftitleitbf=true]{phfthm}

\phfMakeTheorem[counter={},defstar=false,proofref=false]{algorithm}{Algorithm}

\def\proofrefsize{\tiny}

\usepackage{hyperref}  % for \autoref{}


\begin{document}


\section{Test section 1}

\begin{lemma}
\label{lem:1}
Another lemma here.
\end{lemma}
\begin{proof}[a different Lemma]
  this is another proof.
\end{proof}


\begin{proposition}
  \label{prop:1}
  Proposition here.
\end{proposition}
\begin{proof}[*prop:1]
  This is a proof.

  And more stuff.
\end{proof}


You might also like the defintion of the \ref{thmheading:trace-dist} on \autopageref{thmheading:trace-dist}.

\begin{proposition}
  \noproofref
  Another proposition here.  This proposition has no proof ref.
\end{proposition}

\begin{proof}
  This is yet another proof.  This proof has no associated theorem.
\end{proof}


\begin{theorem}
\label{thm:1}
And a theorem
\end{theorem}


\begin{theorem}
\noproofref
\label{thm:2}
I heard you like theorems?
\end{theorem}

\begin{definition*}
This is a definition.
\end{definition*}


\begin{proposition*}
Another proposition here, with no numbering.
\end{proposition*}
\begin{proof}
  this is yet another proof.
\end{proof}

\begin{theorem}
  \label{thm:another}
  Yet another theorem here.
\end{theorem}


See also \autoref{prop:1}, \autoref{thm:1} and \autoref{lem:1}

\cleardoublepage
xxxx
\begin{proof}[*thm:1]
  ....... and this is yet yet another proof.  This proof is only on the next page of the
  theorem, so it should not necessary display the reference there.
\end{proof}

\begin{thmheading}{Trace Distance}
  \label{thmheading:trace-dist}
  The \emph{trace distance} is given by the following expression:
  \begin{align}
    \delta(\rho,\sigma) = \frac12 \left\lVert \rho - \sigma \right\rVert_1
  \end{align}
\end{thmheading}

\begin{theorem}
  \label{thm:long}
  Some theorem which has a little longer statement.
  Some theorem which has a little longer statement.
  Some theorem which has a little longer statement.
  Some theorem which has a little longer statement.

  Some theorem which has a little longer statement.
  Some theorem which has a little longer statement.
  Some theorem which has a little longer statement.
\end{theorem}

\cleardoublepage

\section{Test section 2}

\begin{proof}[*lem:1]
  this is another proof.
\end{proof}

\begin{algorithm}
  \label{algo:1}
  An algorithm here.
  % note that no proof-ref is generated.
\end{algorithm}

\begin{algorithm}
  \label{algo:2}
  Another algorithm here.
\end{algorithm}

\cleardoublepage

\begin{proof}[*thm:another]
  Proof proof proof proof. You're convinced?
\end{proof}

\begin{proof}[*algo:1]
  Proof of an algorithm (WTF?).
\end{proof}

\begin{proof}[*thm:long]
  Proof.
\end{proof}

\begin{claim}
  I claim this and that.
  % should generate warning of missing label
\end{claim}

\cleardoublepage

\begin{theorem}
  \label{thm:blablabla}
  A theorem with a proof that follows immediately.
\end{theorem}
\begin{proof}[**thm:blablabla]
  Proof of the immediately preceding theorem.
\end{proof}


\end{document}


%%% Local Variables: 
%%% mode: latex
%%% TeX-master: t
%%% End: 
