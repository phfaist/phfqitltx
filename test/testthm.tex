\documentclass[12pt,a5paper]{article}

\usepackage[thmset=default,sepcounters]{../phfthm}

\usepackage{hyperref}  % for \autoref{}

%\phfMakeTheorem[defstar=true,defnostar=false,thmstyle=definition,counter=phfthmcounter,aliascounter=true]{definition}{Definition}

\begin{document}

\begin{lemma}
\label{lem:1}
Another lemma here.
\end{lemma}
\begin{proof}[the Lemma]
  this is another proof.
\end{proof}

\bigskip

\begin{proposition}
  \label{prop:1}
  Proposition here.
\end{proposition}
\begin{proof}[*prop:1]
  This is a proof.

  And more stuff.
\end{proof}

\bigskip

\begin{proposition}
Another proposition here.
\end{proposition}
\begin{proof}
  this is yet another proof.
\end{proof}

\bigskip

\begin{proposition*}
Another proposition here, with no numbering.
\end{proposition*}
\begin{proof}
  this is yet another proof.
\end{proof}

\bigskip

\begin{theorem}
\label{thm:1}
And a theorem
\end{theorem}
\begin{proof}[*thm:1]
  ....... and  this is yet yet another proof.
\end{proof}

\bigskip

\begin{definition*}
\label{def:1}
This is a definition.
\end{definition*}

\bigskip

See also \autoref{prop:1}, \autoref{thm:1} and \autoref{lem:1}






\end{document}


%%% Local Variables: 
%%% mode: latex
%%% TeX-master: t
%%% End: 
