\documentclass[12pt,a5paper]{article}

\usepackage[french,english]{babel}
\selectlanguage{french}

\usepackage[thmset=default,sepcounters,proofref={margin,always},prooftitleitbf]{phfthm}
%\usepackage[theoremstyle=definition,definitionstyle=remark,thmset=rich,proofref={margin,longref}]{phfthm}
%\usepackage[sepcounters,proofref]{phfthm}
%\usepackage[proofref=off]{phfthm}
%\usepackage{phfthm}

\def\algorithmname{Algorithme}%
\phfMakeTheorem[counter={},defstar=false,proofref=false]{algorithm}{\algorithmname}

\let\proofofnameenglish\proofofname
\addto\captionsenglish{%
  \def\theoremname{Theorem}%
  \def\propositionname{Proposition}%
  \def\lemmaname{Lemma}%
  \def\corollaryname{Corollary}%
  \def\conjecturename{Conjecture}%
  \def\remarkname{Remark}%
  \def\definitionname{Definition}%
  \def\ideaname{Idea}%
  \def\questionname{Question}%
  \def\problemname{Problem}%
  \def\algorithmname{Algorithm}%
  \let\proofofname\proofofnameenglish
}
\def\proofofnamefrench#1{\proofname\space (#1)}
\addto\captionsfrench{%
  \def\theoremname{Th\'eor\`eme}%
  \def\propositionname{Proposition}%
  \def\lemmaname{Lemme}%
  \def\corollaryname{Corollaire}%
  \def\conjecturename{Conjecture}%
  \def\remarkname{Remarque}%
  \def\definitionname{D\'efinition}%
  \def\ideaname{Id\'ee}%
  \def\questionname{Question}%
  \def\problemname{Probl\`eme}%
  \def\algorithmname{Algorithme}%
  \let\proofofname\proofofnamefrench
}


\def\proofrefsize{\tiny}

\usepackage{hyperref}  % for \autoref{}


\begin{document}

\selectlanguage{english}

\begin{lemma}
\label{lem:1}
Another lemma here.
\end{lemma}
\begin{proof}[a different Lemma]
  this is another proof.
\end{proof}


\begin{proposition}
  \label{prop:1}
  Proposition here.
\end{proposition}
\begin{proof}[*prop:1]
  This is a proof.

  And more stuff.
\end{proof}


You might also like the defintion of the \ref{thmheading:trace-dist} on \autopageref{thmheading:trace-dist}.

\begin{proposition}
  \noproofref
  Another proposition here.
\end{proposition}
\begin{proof}
  this is yet another proof.
\end{proof}


\begin{theorem}
\label{thm:1}
And a theorem
\end{theorem}


\selectlanguage{french}

\begin{theorem}
\noproofref
\label{thm:2}
Vous aimez les th\'eor\`emes?
\end{theorem}

\begin{definition*}
Voil\`a une d\'efinition toute belle.
\end{definition*}


\begin{proposition*}
Voici une autre proposition, sans chiffre.
\end{proposition*}
\begin{proof}
  et voil\`a une preuve qui suit directement la proposition.
\end{proof}


\begin{proposition}
  \label{prop:dsakfdksanfl}
  Voici encore une autre proposition!
\end{proposition}
\begin{proof}[**prop:dsakfdksanfl]
  .... dont voil\`a la preuve.
\end{proof}

\begin{theorem}
  \label{thm:another}
  Encore un autre th\'eor\`eme.
\end{theorem}


Voir aussi \autoref{prop:1}, \autoref{thm:1} et \autoref{lem:1}.

\cleardoublepage

x
\begin{proof}[*thm:1]
  ........ encore une preuve.  Cette preuve est sur la page suivant imm\'ediatement le
  th\'eor\`eme.
\end{proof}

\selectlanguage{english}
Back to English:

\begin{thmheading}{Trace Distance}
  \label{thmheading:trace-dist}
  The \emph{trace distance} is given by the following expression:
  \begin{align}
    \delta(\rho,\sigma) = \frac12 \left\lVert \rho - \sigma \right\rVert_1
  \end{align}
\end{thmheading}

\begin{theorem}
  \label{thm:long}
  Some theorem which has a little longer statement.
  Some theorem which has a little longer statement.
  Some theorem which has a little longer statement.
  Some theorem which has a little longer statement.

  Some theorem which has a little longer statement.
  Some theorem which has a little longer statement.
  Some theorem which has a little longer statement.
\end{theorem}

\cleardoublepage

\begin{proof}[*lem:1]
  this is another proof.
\end{proof}

\begin{algorithm}
  \label{algo:1}
  An algorithm here.
  % note that no proof-ref is generated.
\end{algorithm}

\begin{algorithm}
  \label{algo:2}
  Another algorithm here.
\end{algorithm}

\cleardoublepage

\begin{proof}[*thm:another]
  Proof proof proof proof. You're convinced?
\end{proof}

\begin{proof}[*algo:1]
  Proof of an algorithm (WTF?).
\end{proof}

\begin{proof}[*thm:long]
  Proof.
\end{proof}


\end{document}


%%% Local Variables: 
%%% mode: latex
%%% TeX-master: t
%%% End: 
