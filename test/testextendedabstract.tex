\documentclass[
%    ptsize=11pt,
    papertype=a4paper,
%    sansstyle=false,
%    sectiondecorations=false,
%    compressverticalspacing=2,
%    pkgset=none,
%    loadtheorems=false,
]{phfextendedabstract}

\usepackage{lipsum}

%\geometry{margin=0.3in}

\parskip=2pt plus 1pt minus 0pt\relax

%\renewcommand\phfeaVerticalSpacingCompressionFactor{0.3} %2.5}
% \setstretch{1.2}



\def\phfeaDisplayVerticalSpacingFactorWeight{1.}

%\usepackage{amssymb}

\renewcommand\phfeaTitleStyle{\bfseries} % default \RevTeX title style

\renewcommand\phfeaParagraphAfterHSkip{3pt}

%\renewcommand\phfeaParagraphStyle{\itshape}

% \renewcommand\refname{References.}
% \let\bibsection\rtxapsbibsection

%\usepackage{mathptmx}

%\renewcommand\phfeaHeadingStyle{}

%\renewcommand\phfeaHeadingStyle{\ttfamily}



% \renewcommand{\phfeaSectionDecorationSymbol}{\Large$\Rightarrow$}
% \renewcommand{\phfeaParagraphDecorationSymbol}{\raisebox{1pt}{\tiny $>$}}

%\renewcommand\phfeaSectionFormatHeading[1]{\MakeUppercase{#1}}

\renewcommand\phfeaParagraphFormatHeading[1]{#1\,---}




\usepackage{hyperref}
%\usepackage{cleveref}

\begin{document}
\title{Test of my extended abstract document style}
\author{Philippe Faist}
%\affiliation{Affil. 1}
\author{Alter Ego}
% \affiliation[Affil. X}
\author{Someone Else}
% \affiliation{Affil. 1}
% \affiliation{Another Affil.}
%\email{someoneelse@example.com}
%\date{July 26, 2021}
\maketitle

%\begin{abstract}
%We're doing something really cool.
%\end{abstract}

\section{Introduction.}
\lipsum[1]

\section{Results.}
\lipsum[2]

\paragraph{First result}
\lipsum[3]
Therefore:

\begin{theorem}
  Let $a\in \mathcal{A}$.  Then for all $b\notin\mathcal{B}$ there exists
  $\eta>0$ with an $\epsilon$-conform $v$-differentiable manifold with the
  covariant property such that all $x$-symmetric $D$-forms are irreducible
  almost everywhere.
\end{theorem}

The proof is an elementary application of Gromov's $T$-property of
$(\epsilon,\eta)$-elevated maps.

\paragraph{Another part of the result}
\lipsum[4]
So we have:
\begin{align}
  a = b + \int_0^\infty f(x)\,dx\ .
\end{align}
(Additionally, \texttt{\the\abovedisplayskip, \the\belowdisplayskip,
  \the\abovedisplayshortskip, \the\belowdisplayshortskip}.)

\lipsum[5]
\marginpar{A margin note.}


\section{Hopefully more results.}
Nulla malesuada porttitor diam. Donec felis erat, congue non, volutpat at,
tincidunt tristique, libero. Vivamus viverra fermentum felis. Donec nonummy
pellentesque ante. Phasellus adipiscing semper elit. Proin fermentum massa ac
quam.   So here are our additional results.

\paragraph{Yes, more results}
\lipsum[6]

\paragraph{Wow, even more results}
\lipsum[7-9]

\section{Discussion.}
\lipsum[8]

And so that was some discussion\footnote{Let's check that footnotes work, too} to be done.\marginpar{What if we'd like a margin note?}

We can also check that:
\begin{itemize}
\item Itemization lists look okay;
\item Itemization lists can have a second point;
\item \ldots as well as a third point.
\end{itemize}

Finally, enumeration environments
\begin{enumerate}
\item Can have one;
\item two;
\item \ldots and three points.
\end{enumerate}

We can also have inline enumeration environments that have
\begin{enumerate*}[label=\textbf{(\roman*)},itemjoin*={{ \ldots\ and }}]
\item one;
\item two;
\item even three
\end{enumerate*}
points!

\section{Outlook.}
\lipsum[9]

\section!{Section with no decoration.}%
How does this look like?
\lipsum[1]

\paragraph!{Same for paragraphs.}%
Same here. Does this work?

\section!{} This is actually a new section, but with no decoration and an empty
title! Also, the spacing should be removed when the section has an empty title
argument.

\paragraph!{} Same here. Does this work?

\section{} The spacing should also be removed when the section has an empty
title argument even if we still have the section margin decoration symbol in
place (pretty fun, isn't it?).

\paragraph{} Same here. Does this work? work? work?

\section*{Starred} \ldots\ versions of the \verb+\section+ and \ldots

\paragraph*{B} the \verb+\paragraph+ commands should also work. Do they?


\section!{}\phfeaSectionDecoration{\guillemotright}%
You can use \verb+\section!{}+ to get the spacing for a section, without
producing anything.  For instance, you can use it to place your own decoration
in front of a line that should take the same type of space as a section would.

% typically you'd use \bibliography{bibtexfile}, but for the sake of the example
% I'll directly use \begin{thebibliography} here instead
\begin{thebibliography}{9}
\bibitem{Item1}
  A, B, and C, Journal of Future Results 3:99, 2033.
\bibitem{Item2}
  D E, F, and G, Another Journal of Cool Results 1:103, 2021.
\end{thebibliography}

\end{document}
