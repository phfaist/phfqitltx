\documentclass[
%    ptsize=11pt,
    papertype=a4paper,
%    sansstyle=false,
%    sectiondecorations=false,
%    noheadingdecorations,
%    compressverticalspacing=.5,
%    pkgset=none,
%    loadtheorems=false,
]{phfextendedabstract}

\usepackage{lipsum}



%\geometry{margin=0.3in}

\parskip=2pt plus 1pt minus 0pt\relax

%\renewcommand\phfeaVerticalSpacingCompressionFactor{0.3} %2.5}
\setstretch{1.1}

%\setcounter{secnumdepth}{2}


\def\phfeaDisplayVerticalSpacingFactorWeight{1.}

%\usepackage{amssymb}

\renewcommand\phfeaTitleStyle{\bfseries} % default \RevTeX title style

\renewcommand\phfeaParagraphAfterHSkip{3pt}

%\renewcommand\phfeaParagraphStyle{\itshape}

% \renewcommand\refname{References.}
% \let\bibsection\rtxapsbibsection

%\usepackage{mathptmx}

%\renewcommand\phfeaHeadingStyle{}

%\renewcommand\phfeaHeadingStyle{\ttfamily}

% \renewcommand{\phfeaSectionDecorationSymbol}{\Large$\Rightarrow$}
% \renewcommand{\phfeaParagraphDecorationSymbol}{\raisebox{1pt}{\tiny $>$}}

\renewcommand\phfeaSectionFormatHeading[1]{\MakeUppercase{#1}}
\renewcommand\phfeaSectionFormatRuninHeading[1]{\fbox{\MakeUppercase{#1}\ }}
%\renewcommand\phfeaSectionFormatHeading[1]{\fbox{#1}}

\renewcommand\phfeaParagraphFormatHeading[1]{#1\,---}


\renewcommand{\phfeaTitleAboveVSpace}{}
\renewcommand{\phfeaTitleInnerVSpace}{}
\renewcommand{\phfeaTitleBelowVSpace}{\baselineskip}



\makeatletter
% \def\frontmatter@preabstractspace{\z@}%
% \def\frontmatter@postabstractspace{\z@}%

% % \def\frontmatter@authorformat{%
% %  \skip@\@flushglue
% %  \@flushglue\z@ plus.3\hsize\relax
% %  \centering
% %  \advance\baselineskip\p@
% %  %\parskip11.5\p@\relax
% %  \@flushglue\skip@
% % }
% % \appto\frontmatter@authorformat{%
% %   \parskip=.5\baselineskip\relax
% % }%

% % \def\frontmatter@above@affiliation@script{%
% %  \skip@\@flushglue
% %  \@flushglue\z@ plus.3\hsize\relax
% %  \centering
% %  \@flushglue\skip@
% %  \addvspace{3.5\p@}%
% % }%


% %%% TESTS %%%
% \def\xxxhrule{\hrule width 400pt height .4pt}
% \def\titleblock@produce{%
%   \xxxhrule
%  \begingroup
%   \ltx@footnote@pop
%   \def\@mpfn{mpfootnote}%
%   \def\thempfn{\thempfootnote}%
%   \c@mpfootnote\z@
%   \let\@makefnmark\frontmatter@makefnmark
%   \frontmatter@setup
%   \thispagestyle{titlepage}\label{FirstPage}%
%   \frontmatter@title@produce
%   \xxxhrule
%   \groupauthors@sw{%
%    \frontmatter@author@produce@group
%   }{%
%    \frontmatter@author@produce@script
%    %\xxxhrule x\par
%   }%
%   \frontmatter@RRAPformat{%
%    \expandafter\produce@RRAP\expandafter{\@date}%
%    \expandafter\produce@RRAP\expandafter{\@received}%
%    \expandafter\produce@RRAP\expandafter{\@revised}%
%    \expandafter\produce@RRAP\expandafter{\@accepted}%
%    \expandafter\produce@RRAP\expandafter{\@published}%
%   }%
% \xxxhrule
%   \frontmatter@abstract@produce
% \xxxhrule
%   \@ifx@empty\@pacs{}{%
%    \@pacs@produce\@pacs
%   }%
%   \@ifx@empty\@keywords{}{%
%    \@keywords@produce\@keywords
%   }%
%   \par
%   \frontmatter@finalspace
% \xxxhrule
%  \endgroup
% \parskip=\z@
% }%
% %%% TESTS %%%
\makeatother





\usepackage{hyperref}
%\usepackage{cleveref}

\begin{document}
\title{Test of my extended abstract document style}
\author{Philippe Faist}
 % \affiliation{Affil. 1}
\author{Alter Ego}
 % \affiliation{Affil. X}
\author{Someone Else}
% \email{someoneelse@example.com}
%  \affiliation{Affil. 1}
%  \affiliation{Another Affil.}
% \date{July 26, 2021}
% \begin{abstract}
%   This is an abstract abstract.
%   This is an abstract abstract.
%   This is an abstract abstract.
%   This is an abstract abstract.
%   This is an abstract abstract.
%   This is an abstract abstract.
%   This is an abstract abstract.
%   This is an abstract abstract.
%   This is an abstract abstract.
% \end{abstract}
\maketitle
%XXXXXXXXXXXXXXXXXXXXXXXXXXXXXXXXXXXXXXXXXXXXXXXXXXXXXXXXXXXXXXXXXXXXXXXXXXXXXXXXXXXXXXXXXXXXXXXXXXXXXXXXXXXXXXXXX


\section{Introduction.}
\lipsum[1]

\section{Results.}
\lipsum[2]

\paragraph{First result}
\lipsum[3]
Therefore:

\begin{theorem}
  Let $a\in \mathcal{A}$.  Then for all $b\notin\mathcal{B}$ there exists
  $\eta>0$ with an $\epsilon$-conform $v$-differentiable manifold with the
  covariant property such that all $x$-symmetric $D$-forms are irreducible
  almost everywhere.
\end{theorem}

The proof is an elementary application of Gromov's $T$-property of
$(\epsilon,\eta)$-elevated maps.

\paragraph{Another part of the result}
\lipsum[4]
So we have:
\begin{align}
  a = b + \int_0^\infty f(x)\,dx\ .
\end{align}
(Additionally, \texttt{\the\abovedisplayskip, \the\belowdisplayskip,
  \the\abovedisplayshortskip, \the\belowdisplayshortskip}.)

\lipsum[5]
\marginpar{A margin note.}


\section{Hopefully more results.}
Nulla malesuada porttitor diam. Donec felis erat, congue non, volutpat at,
tincidunt tristique, libero. Vivamus viverra fermentum felis. Donec nonummy
pellentesque ante. Phasellus adipiscing semper elit. Proin fermentum massa ac
quam.   So here are our additional results.

\paragraph{Yes, more results}
\lipsum[6]

\paragraph{Wow, even more results}
\lipsum[7-9]

\section{Discussion.}
\lipsum[8]

And so that was some discussion\footnote{Let's check that footnotes work, too} to be done.\marginpar{What if we'd like a margin note?}

We can also check that:
\begin{itemize}
\item Itemization lists look okay;
\item Itemization lists can have a second point;
\item \ldots as well as a third point.
\end{itemize}

Finally, enumeration environments
\begin{enumerate}
\item Can have one;
\item two;
\item \ldots and three points.
\end{enumerate}

We can also have inline enumeration environments that have
\begin{enumerate*}[label=\textbf{(\roman*)},itemjoin*={{ \ldots\ and }}]
\item one;
\item two;
\item even three
\end{enumerate*}
points!

%\subsection{This should produce an error.} Does it?

\section{Outlook.}
\lipsum[9]

\section!{Section with no decoration.}%
How does this look like? (We used \verb+\section!{...}+ here.)
\lipsum[1]

\paragraph!{Same for paragraphs.}%
Same here. Does this work?

\section!{} This is actually a new section, but with no decoration and an empty
title! Also, the spacing should be removed when the section has an empty title
argument.

\paragraph!{} Same here. Does this work?

\section{} The spacing should also be removed when the section has an empty
title argument even if we still have the section margin decoration symbol in
place (pretty fun, isn't it?).

\paragraph{} Same here. Does this work? work? work?

\section*{Starred} \ldots\ versions of the \verb+\section+ command and \ldots

\paragraph*{of the \texttt{\textbackslash paragraph} command} should also work
without any noticeable difference. Do they?  They also suppress section numbers
in case you enable section numbering!


%\section*!{}\phfeaSectionDecoration{\guillemotright}%
\section*!{}%
You can use \verb+\section!{}+ (or \verb+\section*!{}+, in case you enabled
section numbering) to get the spacing for a section, without producing anything.
For instance, you can use it to place your own decoration in front of a line
that should take the same type of space as a section would.

\section<\guillemotright>{Here's a custom decoration}%
typeset using \verb+\section<...>{..}+. You can also

\paragraph<{\raisebox{1.5pt}{\tiny$\to$}}>{Do the same} for paragraphs with \verb+\paragraph<...>{..}+.

\section<\guilsinglright>{}   You can also use a custom decoration without any
section title.


\section={The section title} can also be typeset run-in, starting the
section's sentence.

\section!={The run-in title} can be combined with `\verb+!+' to remove the
decoration.  Be sure to place the `\verb+!+' before the `\verb+=+'.

\section<{$\blacktriangleleft$}>={The run-in title} can be combined with `\verb|<...>|' to
replace the decoration.


\paragraph={The paragraph title} can also be typeset run-in, starting the
paragraph's sentence.

\paragraph!={The run-in title} can be combined with `\verb+!+' to remove the
decoration.  Be sure to place the `\verb+!+' before the `\verb+=+'.

\paragraph<--->={The run-in title} can be combined with `\verb|<...>|' to
replace the decoration.




% We can also test if the table of contents works:
%
% \tableofcontents


% typically you'd use \bibliography{bibtexfile}, but for the sake of the example
% I'll directly use \begin{thebibliography} here instead
\begin{thebibliography}{9}
\bibitem{Item1}
  A, B, and C, Journal of Future Results 3:99, 2033.
\bibitem{Item2}
  D E, F, and G, Another Journal of Cool Results 1:103, 2021.
\end{thebibliography}

\end{document}
