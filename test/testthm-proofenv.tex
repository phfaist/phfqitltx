\documentclass[12pt,a5paper]{article}

\usepackage[thmset=default,sepcounters,proofref={margin,always},proofenv=false]{../phfthm}
%\usepackage[theoremstyle=definition,definitionstyle=remark,thmset=rich,proofref={longref,margin}]{../phfthm}
%\usepackage[sepcounters,proofref]{../phfthm}

\def\demof#1{D\'emonstration [#1]}
\makeatletter\def\myparse#1{\phfthm@proof@parselabel{#1}\def\phfthm@val@prooftitle{$\langle$#1$\rangle$}}\makeatother

\phfMakeProofEnv[proofofname=\demof,defaultproofname={D\'emonstration},parselabelcmd=\myparse]{myproof}
\phfMakeProofEnv[proofofname=\demof,displayenv=miniproofdisplayblock]{myotherproofstyle}

% can override default proof env's title formatting:
\makeatletter
\def\phfthm@ProofTitleFmt#1{%
  {\textsc{#1}:}%
}
\makeatother

\usepackage{tcolorbox}
\newenvironment{miniproofdisplayblock}[1][\proofname]{%
  \begin{tcolorbox}[colframe=blue,phantom={\phfthmPinProofAnchor},title={\tiny\scshape #1:}]%
}{%
  \end{tcolorbox}%
}

\def\proofrefsize{\tiny}

\usepackage{hyperref}  % for \autoref{}
  \hypersetup{unicode=true,%
    bookmarksnumbered=false,bookmarksopen=false,bookmarksopenlevel=1,%
    breaklinks=true,pdfborder={0 0 0},colorlinks=true}%
  \hypersetup{%
    anchorcolor=red,citecolor=red,%
    filecolor=red,linkcolor=red,%
    menucolor=red,runcolor=red,%
    urlcolor=red}%


\begin{document}


\begin{lemma}
\label{lem:1}
Another lemma here.
\end{lemma}
\begin{myproof}[a different Lemma]
  this is another proof.
\end{myproof}


\begin{proposition}
  \label{prop:1}
  Proposition here.
\end{proposition}
\begin{myotherproofstyle}[*prop:1]
  This is a proof.

  And more stuff.
\end{myotherproofstyle}


You might also like the defintion of the \ref{thmheading:trace-dist} on \autopageref{thmheading:trace-dist}.

\begin{proposition}
\noproofref
Another proposition here.
\end{proposition}
\begin{myproof}
  this is yet another proof.
\end{myproof}


\begin{theorem}
\label{thm:1}
And a theorem
\end{theorem}
\begin{myotherproofstyle}
  this is yet yet yet yet yet another another another proof.
\end{myotherproofstyle}


\begin{theorem}
\noproofref
\label{thm:2}
I heard you like theorems?
\end{theorem}

\begin{definition*}
\label{def:1}
This is a definition.
\end{definition*}


\begin{proposition*}
Another proposition here, with no numbering.
\end{proposition*}
\begin{myproof}
  this is yet another proof.
\end{myproof}

\begin{theorem}
  \label{thm:another}
  Yet another theorem here.
\end{theorem}


See also \autoref{prop:1}, \autoref{thm:1} and \autoref{lem:1}

\cleardoublepage
x
\begin{myproof}[*thm:1]
  ....... and this is yet yet another proof.  This proof is only on the next page of the
  theorem, so it should not necessary display the reference there.
\end{myproof}

\begin{thmheading}{Trace Distance}
  \label{thmheading:trace-dist}
  The \emph{trace distance} is given by the following expression:
  \begin{align}
    \delta(\rho,\sigma) = \frac12 \left\lVert \rho - \sigma \right\rVert_1
  \end{align}
\end{thmheading}

\cleardoublepage

\begin{myproof}[*lem:1]
  this is another proof.
\end{myproof}

\cleardoublepage

\begin{myproof}[*thm:another]
  Proof proof proof proof. You're convinced?
\end{myproof}



\end{document}


%%% Local Variables: 
%%% mode: latex
%%% TeX-master: t
%%% End: 
