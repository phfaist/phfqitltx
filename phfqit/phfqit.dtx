% \iffalse meta-comment
%
% Copyright (C) 2016 by Philippe Faist, philippe.faist@bluewin.ch
% -------------------------------------------------------
% 
% This file may be distributed and/or modified under the
% conditions of the LaTeX Project Public License, either version 1.3
% of this license or (at your option) any later version.
% The latest version of this license is in:
%
%    http://www.latex-project.org/lppl.txt
%
% and version 1.3 or later is part of all distributions of LaTeX 
% version 2005/12/01 or later.
%
% \fi
%
% \iffalse
%<*driver>
\ProvidesFile{phfqit.dtx}
%</driver>
%<package>\NeedsTeXFormat{LaTeX2e}[2005/12/01]
%<package>\ProvidesPackage{phfqit}
%<*package>
    [2021/10/08 v4.1 phfqit package]
%</package>
%
%<*driver>
\documentclass{ltxdoc}
\usepackage{xcolor}
\makeatletter
\providecommand\phfnote@pkgdoc@setupmainfont{
  \renewcommand{\rmdefault}{futs}% only rm font, not math
}\makeatother
\usepackage[preset=xpkgdoc]{phfnote}
\usepackage{phfqit}
\usepackage{needspace}
\EnableCrossrefs         
\CodelineIndex
\RecordChanges
\begin{document}
  \DocInput{phfqit.dtx}
\end{document}
%</driver>
% \fi
%
% \CheckSum{0}
%
% \CharacterTable
%  {Upper-case    \A\B\C\D\E\F\G\H\I\J\K\L\M\N\O\P\Q\R\S\T\U\V\W\X\Y\Z
%   Lower-case    \a\b\c\d\e\f\g\h\i\j\k\l\m\n\o\p\q\r\s\t\u\v\w\x\y\z
%   Digits        \0\1\2\3\4\5\6\7\8\9
%   Exclamation   \!     Double quote  \"     Hash (number) \#
%   Dollar        \$     Percent       \%     Ampersand     \&
%   Acute accent  \'     Left paren    \(     Right paren   \)
%   Asterisk      \*     Plus          \+     Comma         \,
%   Minus         \-     Point         \.     Solidus       \/
%   Colon         \:     Semicolon     \;     Less than     \<
%   Equals        \=     Greater than  \>     Question mark \?
%   Commercial at \@     Left bracket  \[     Backslash     \\
%   Right bracket \]     Circumflex    \^     Underscore    \_
%   Grave accent  \`     Left brace    \{     Vertical bar  \|
%   Right brace   \}     Tilde         \~}
%
%
% \changes{v1.0}{2016/04/20}{Initial version}
%
% \GetFileInfo{phfqit.dtx}
%
% \DoNotIndex{\newcommand,\newenvironment,\let,\def,\gdef,\edef,\xdef,\if,\else,\fi,\ifx,\cslet,\csdef,\begingroup,\endgroup,\expandafter,\csname,\endcsname,\appto,\hspace,\mathrm,\notblank,\the,\RequirePackage}
% 
% \title{\phfqitltxPkgTitle{phfqit}}
% \author{Philippe Faist\quad\email{philippe.faist@bluewin.ch}}
% \date{\pkgfmtdate\filedate}
% \maketitle
%
% \begin{abstract}
%   \pkgname{phfqit}---Utilities to typeset stuff in Quantum Information Theory,
%   in particular general mathematical symbols, operators, and shorthands for
%   entropy measures.
% \end{abstract}
%
% \inlinetoc
%
% \section{Introduction}
%
% This package provides some useful definitions, mainly for notation of
% mathematical expressions which are used in quantum information theory (at
% least by me).
%
% Are included utilities for:
% \begin{itemize}
% \item General symbols and mathematical expressions (identity operator,
%   trace, rank, diagonal, \ldots)  (\autoref{sec:symbols})
% \item Formatting of bits and bit strings (\autoref{sec:bits})
% \item Formatting of names of logical gates (\autoref{sec:gates})
% \item Typesetting the names of Lie groups and algebras, for example $\su(N)$
%   (\autoref{sec:Lie-groups-algebras})
% \item Bra-ket notation, and delimited expressions such as average, norm,
%   \ldots (\autoref{sec:delimited})
% \item Double bra-ket notation for operators (\autoref{sec:delimited})
% \item Typesetting entropy measures, including the Shannon/von Neumann entropy,
%   the smooth entropies, relative entropies, as well as my coherent relative
%   entropy
% \end{itemize}
%
%
% \section{Basic Usage}
%
% \label{sec:pkg-options}
%
% This package is straightforward to use:
% \begin{verbatim}
% \usepackage{phfqit}
% \end{verbatim}
%
% A single package option controls which entropy measures are defined for you.
%
% \begin{pkgoptions}
% \item[qitobjdef=\meta{\phfverb{stdset} $\mid$ \phfverb{none}}] Load
%   the predefined set of ``qit objects, '' i.e., entropy measures.  The entropy
%   measures documented below (and specified as such) will be loaded unless you
%   set \pkgoptionfmt{qitobjdef=none}.
% \item[newReIm=\metatruefalsearg] Do not override \LaTeX{}'s default
%   {\makeatletter $\phfqit@Re$ and $\phfqit@Im$} symbols by $\Re$ and $\Im$.
%   See \autoref{sec:description-newReIm}.
% \item[llanglefrommnsymbolfonts=\metatruefalsearg] In order to define the
%   operator-kets and operator-bra symbols, we need to have double-angle bracket
%   symbol delimiters loaded as |\llangle| and |\rrangle|.  Unlike |\langle| and
%   |\rangle|, they are not provided by default by \pkgname{latex},
%   \pkgname{amsmath} or \pkgname{amssymb}.  What we do is that we load the
%   |\llangle| and |\rrangle| symbols from the MnSymbol fonts in order to define
%   |\oket| and friends.  If you would like to provide your own definitions for
%   |\llangle| and |\rrangle|, or if you have problems loading the MnSymbol
%   fonts and don't need the operator-ket symbols, then you can specify
%   |llanglefrommnsymbolfonts=false| and we won't bother loading the MnSymbol
%   fonts.  (Note that if you provide your own definitions for |\llangle| and
%   |\rrangle|, they have to be valid delimiters, such that for example the
%   syntax |\left\llangle| is valid.)
% \end{pkgoptions}
% 
% \changed[v2.0-pkg-opt-qitobjdef]{v2.0}{2017/08/16}{Added the
% \phfverb{qitobjdef} package option}
% \changed[v2.0-pkg-opt-newReIm]{v2.0}{2017/08/16}{Added the \phfverb{newReIm}
% package option}
%
% \subsection{Semantic vs. Syntactic Notation}
%
% The macros in this package are meant to represent a \emph{mathematical
% quantity}, independently of its final \emph{notation}.  For example, |\Hmaxf|
% indicates corresponds to the ``new-style'' max-entropy defined with the
% fidelity,\footnote{see Marco Tomamichel, Ph. D., ETH Zurich (2012)
% \href{https://arxiv.org/abs/1203.2142}{arXiv:1203.2142}} independently of the
% notation.  Then, if the default notation ``$\Hmaxf{}$'' doesn't suit your
% taste, you may then simply redefine this command to display whatever you like
% (see for example instructions in \autoref{sec:entropy-measures}).  This allows
% to keep better distinction between different measures which may share the same
% notation in different works of literature.  It also allows to switch notation
% easily, even in documents which use several quantities whose notation may be
% potentially conflicting.
%
%
% \subsection{Size Specification}
% \label{topic:size-specification-backtick}
%
% Many of the macros in this package allow their delimiters to be sized
% according to your taste.  For example, if there is a large symbol in an
% entropy measure, say
% \begin{align}
%  \Hmin{\displaystyle\bigotimes_i A_i}[B]\ ,
% \end{align}
% then it may be necessary to tune the size of the parenthesis delimiters.
%
% This is done with the optional size specification \meta{size-spec}.  The
% \meta{size-spec}, whenever it is accepted, is always optional.
%
% The \meta{size-spec} starts with the backtick character ``|`|'', and is
% followed by a single token which may be a star |*| or a size modifier macro
% such as |\big|, |\Big|, |\bigg| and |\Bigg|.  If the star is specified, then
% the delimiters are sized with |\left| and |\right|; otherwise the
% corresponding size modifier is used.  When no size specification is present,
% then the normal character size is used.
%
% For example:
% \begin{center}
%   \begin{tabular}{ll}
%       |\Hmin{\bigotimes_i A_i}[B]|  & gives\quad  $\Hmin{\displaystyle\bigotimes_i A_i}[B]$, \\[1.5em]
%       |\Hmin`\Big{\bigotimes_i A_i}[B]|  & gives\quad  $\Hmin`\Big{\displaystyle\bigotimes_i A_i}[B]$,~~and \\[1.5em]
%       |\Hmin`*{\bigotimes_i A_i}[B]|  & gives\quad  $\Hmin`*{\displaystyle\bigotimes_i A_i}[B]$. \\
%   \end{tabular}
% \end{center}
%
%
%
% \section{General Symbols and Math Operators}
% \label{sec:symbols}
%
% \DescribeMacro{\Hs}
% Hilbert space = $\Hs$.
%
% \DescribeMacro{\Ident}
% Identity operator = $\Ident$.
%
% \DescribeMacro{\IdentProc}
% Identity process.  Possible usage syntax is:
% \begin{center}
% \begin{tabular}{lc}
%     |\IdentProc[A][A']{\rho}| &  $\IdentProc[A][A']{\rho}$ \\
%     |\IdentProc[A]{\rho}| &  $\IdentProc[A]{\rho}$ \\
%     |\IdentProc[A][A']{}| &  $\IdentProc[A][A']{}$ \\
%     |\IdentProc[A]{}| &  $\IdentProc[A]{}$ \\
%     |\IdentProc{}| &  $\IdentProc{}$ \\
%     |\IdentProc{\rho}| &  $\IdentProc{\rho}$ \\
%     |\IdentProc`\big[A]{\rho}| &  $\IdentProc`\big[A]{\rho}$ \\
% \end{tabular}
% \end{center}
% This macro accepts a size specification with the backtick (`|`|'), see
% \autoref{topic:size-specification-backtick}.
%
% \begingroup\catcode`\^=12\relax
% \DescribeMacro{\ee^X}\endgroup A macro for the exponential.  Type the \LaTeX{}
% code as if |\ee| were just the symbol, i.e.\@ as |\ee^{<ARGUMENT>}|.  The
% ideas is that this macro may be redefined to change the appearance of the $e$
% symbol, or even to change the notation to |\exp{<ARGUMENT>}| if needed for
% inline math.
%
% \DescribeMacro{\phfqitExpPowerExpression}
% To change the appearance of whatever you typed as |\ee^{XYZ}|, you can
% redefine |\phfqitExpPowerExpression|; for instance, to use an upright
% ``$\mathrm{e}$'' symbol, you could type:
% \begin{verbatim}
% \renewcommand\phfqitExpPowerExpression[1]{\mathrm{e}^{#1}}
% \end{verbatim}
%
%
% \subsection{Math/Linear Algebra Operators}
% \label{sec:math-operators}
% \label{sec:description-newReIm}
%
% \needspace{6\baselineskip}
% \DescribeMacro{\tr} \DescribeMacro{\supp} \DescribeMacro{\rank}
% \DescribeMacro{\linspan} \DescribeMacro{\spec} \DescribeMacro{\diag} Provide
% some common math operators.  The trace $\tr$, the support $\supp$, the rank
% $\rank$, the linear span $\linspan$, the spectrum $\spec$ and the diagonal
% matrix $\diag$.  (Note that |\span| is already defined by \LaTeX{}, so that we
% resort to |\linspan|.)  \vspace{1.5cm}
%
% \DescribeMacro{\Re} \DescribeMacro{\Im} Also, redefine |\Re| and |\Im| (real
% and imaginary parts of a complex number), to the more readable $\Re(z)$ and
% $\Im(z)$.  (The original symbols were {\makeatletter $\phfqit@Re(z)$ and
% $\phfqit@Im(z)$}.)  Keep the old definitions using the package option
% \pkgoptionfmt{newReIm=false}.
%
% \subsection{Poly symbol}
%
% \DescribeMacro{\poly} Can be typeset in $\poly(n)$ time.
% 
%
% \subsection{Bits and Bit Strings}
% \label{sec:bits}
%
% \DescribeMacro{\bit} Format a bit value, for example |\bit{0}| or |\bit0|
% gives $\bit0$ or $\bit1$.  This command works both in math mode and text mode.
%
% \DescribeMacro{\bitstring} Format a bit string.  For example
% |\bitstring{01100101}| is rendered as \bitstring{01100101}.  This command
% works both in math mode and text mode.
%
% \subsection{Logical Gates}
% \label{sec:gates}
%
% \DescribeMacro{\gate} Format a logical gate.  Essentially, this command
% typesets its argument in small-caps font.  For example, with |\gate{C-not}|
% you get \gate{C-not}.  (The default formatting ignores the given
% capitalization, but if you redefine this command you could exploit this,
% e.g.\@ by making the ``C'' in ``Cnot'' larger than the ``not''.)
%
% This command works both in math mode and in text mode.
%
% \needspace{5\baselineskip}
% \DescribeMacro{\AND} \DescribeMacro{\XOR} \DescribeMacro{\CNOT}
% \DescribeMacro{\NOT} \DescribeMacro{\NOOP} Some standard gates.  These typeset
% respectively as \AND, \XOR, \CNOT, \NOT, and \NOOP. \vspace{3\baselineskip}
%
%
% \section{Lie Groups and Algebras}
% \label{sec:Lie-groups-algebras}
%
% \needspace{7\baselineskip}
% \DescribeMacro{\uu(N)} \DescribeMacro{\UU(N)} \DescribeMacro{\su(N)}
% \DescribeMacro{\SU(N)} \DescribeMacro{\so(N)} \DescribeMacro{\SO(N)}
% \DescribeMacro{\slalg(N)} \DescribeMacro{\SL(N)}
% \DescribeMacro{\GL(N)} \DescribeMacro{\SN(N)}
% Format some common Lie groups and algebras.  Note the use of |\slalg| instead
% of |\sl| because of name conflict with \TeX's font command {\sl producing
% slanted text}.
%
% |\SN(N)| is the symmetric group of $N$ items, and formats by default as
% $\SN(N)$.
%
% \DescribeMacro{\phfqitLieGroup} \DescribeMacro{\phfqitLieAlgebra}
% \DescribeMacro{\phfqitDiscreteGroup}
% The macros |\phfqitLieGroup|, |\phfqitLieAlgebra|, and |\phfqitDiscreteGroup|
% format the name for a standard Lie group, Lie algebra or discrete group along
% with its argument.  Redefine these macros with |\renewcommand| to change the
% formatting font for Lie groups and algebras for instance.  For instance, to
% format standard Lie groups/algebras and the permutation group with simple
% italic letters, you can use:
% \begin{verbatim}
% \renewcommand\phfqitLieAlgebra[2]{\mathit{#1}({#2})}
% \renewcommand\phfqitLieGroup[2]{\mathit{#1}({#2})}
% \renewcommand\phfqitDiscreteGroup[2]{\mathit{#1}_{#2}}
% \end{verbatim}
% 
% \verbdef\GLNverbatim=\GL(N)=
% \changed[v2.0]{v2.0}{2018/02/28}{Added the macro \GLNverbatim}
% \verbdef\SLNverbatim=\slalg(n),\SL(N)=
% \changed[v3.0]{v3.0}{2020/07/31}{Added the macros \SLNverbatim, as well
%     as \phfverb{\phfqitLieAlgebra}, \phfverb{\phfqitLieGroup}, and
%     \phfverb{\phfqitDiscreteGroup}}
%
% \iffalse \vspace{3\baselineskip}  <-- space for all the \SU(xx) listings in margin \fi
%
% \section{Bra-Ket Notation and Delimited Expressions}
%
% \subsection{Bras and kets}
% \label{sec:bra-ket}
%
% All commands here work in math mode only.  They all accept a size modifier as
% described in \autoref{topic:size-specification-backtick}.  (The size may also
% be provided as an optional argument; the starred form of the command may also
% be used to enclose the delimiters with |\left...\right| and have the size
% determined automatically.)  Example usage is:
% \begin{center}
%   \begin{tabular}{lc}
%   |\ket{\psi}|  &  $\ket{\psi}$ \\[1em]
%   |\ket`\big{\psi}|~,~~|\ket[\big]{\psi}|  &  $\ket`\big{\psi}$ \\[1em]
%   |\ket`\Big{\psi}|~,~~|\ket[\Big]{\psi}|  &  $\ket`\Big{\psi}$ \\[1em]
%   |\ket`\bigg{\psi}|~,~~|\ket[\bigg]{\psi}|  &  $\ket`\bigg{\psi}$ \\[1em]
%   |\ket`\Bigg{\psi}|~,~~|\ket[\Bigg]{\psi}|  &  $\ket`\Bigg{\psi}$ \\[1em]
%   \begin{minipage}{7cm}\relax
%      \noindent|\ket`*{\displaystyle\sum_k \psi_k}|,\\
%      \hspace*{2em}|\ket*{\displaystyle\sum_k \psi_k}|
%   \end{minipage}
%                                &  $\ket*{\displaystyle\sum_k \psi_k}$
%   \end{tabular}
% \end{center}
%
% \DescribeMacro{\ket}
% Typeset a quantum mechanical ket.  |\ket{\psi}| gives $\ket{\psi}$.
%
% \DescribeMacro{\bra}
% Typeset a bra.  |\bra{\psi}| gives $\bra{\psi}$.
%
% \DescribeMacro{\braket}
% Typeset a bra-ket inner product.  |\braket{\phi}{\psi}| gives $\braket{\phi}{\psi}$.
%
% \DescribeMacro{\ketbra}
% Typeset a ket-bra outer product.  |\ketbra{\phi}{\psi}| gives $\ketbra{\phi}{\psi}$.
%
% \DescribeMacro{\proj}
% Typeset a rank-1 projector determined by a ket.  |\proj{\psi}| gives $\proj{\psi}$.
%
% \DescribeMacro{\matrixel} Typeset a matrix element.
% |\matrixel{\phi}{A}{\psi}| gives $\matrixel{\phi}{A}{\psi}$.
%
% \DescribeMacro{\dmatrixel} Typeset a diagonal matrix element of an operator.
% |\dmatrixel{\phi}{A}| gives $\dmatrixel{\phi}{A}$.
%
% \DescribeMacro{\innerprod} Typeset an inner product using the mathematicians' notation.
% |\innerprod{\phi}{\psi}| gives $\innerprod{\phi}{\psi}$.
%
%
% \DescribeMacro{\oket} \DescribeMacro{\obra} \DescribeMacro{\obraket}
% \DescribeMacro{\oketbra} \DescribeMacro{\oproj} \DescribeMacro{\omatrixel}
% \DescribeMacro{\odmatrixel} This package also provides associated
% double-bra-ket commands as is occasionally used to write ``vectors'' in
% Hilbert-Schmidt operator space, such as $\oket{A}$, $\obraket{A}{\Ident}$,
% $\oketbra{\Ident}{E_k}$, etc.  The commands are named |\oket|, |\obra|,
% |\obraket|, |\oketbra|, |\oproj|, |\omatrixel|, and |\odmatrixel|.  They offer
% the same syntax as their corresponding |\ket|, |\bra|, etc., for instance, you
% can type |\obra`\Big{\sum A_i}| to obtain $\obra`\Big{\sum A_i}$.
%
% \DescribeMacro{\llangle} \DescribeMacro{\rrangle} The \pkfname{phfqit} package
% defines the |\llangle| and |\rrangle| delimiters (taken from the MnSymbol fonts  The double-angle bracket symbol used in
% double-bra-ket type constructs
%
% \subsection{Delimited expressions: norms, absolute value, etc.}
% \label{sec:delimited}
%
% There are also some commonly used delimited expressions defined for
% convenience.
%
% \DescribeMacro{\abs} The absolute value of an expression.  |\abs{A}| gives
% $\abs{A}$.
%
% \DescribeMacro{\avg} The average of an expression. |\avg`\big{\sum_k A_k}|
% gives $\avg`\big{\sum_k A_k}$.
%
% \DescribeMacro{\norm} The norm of an expression.  |\norm{A_k}| gives
% $\norm{A_k}$.  (You can add subscripts, e.g.\@ |\norm{A_k}_\infty| is
% $\norm{A_k}_\infty$.)
%
% \DescribeMacro{\intervalc} A closed interval. |\intervalc{x}{y}| gives
% $\intervalc{x}{y}$.
%
% \DescribeMacro{\intervalo} An open interval. |\intervalo{x}{y}| gives
% $\intervalo{x}{y}$.
%
% \DescribeMacro{\intervalco} A semi-open interval, closed on the lower bound
% and open on the upper bound. |\intervalco{x}{y}| gives $\intervalco{x}{y}$.
%
% \DescribeMacro{\intervaloc} A semi-open interval, open on the lower bound
% and closed on the upper bound. |\intervaloc{x}{y}| gives $\intervaloc{x}{y}$.
%
%
% \DescribeMacro{\phfqitDefineNorm} The handy command |\phfqitDefineNorm| can be used
% to define custom norms (e.g. 1-norm, infinity-norm, p/q-norms, etc.).  The
% syntax is |\phfDefineNorm|\meta{command name}\meta{before}\meta{after}, for
% example:
% \begin{verbatim}
% \phfqitDefineNorm\onenorm{}{_1}
% \phfqitDefineNorm\opnorm{}{_\infty}
% \end{verbatim}
%
% \DescribeMacro{\phfqitDeclarePairedDelimiterXWithAltSizing}
% \DescribeMacro{\phfqitDeclarePairedDelimiterXPPWithAltSizing} For defining
% more advanced custom delimited expressions, you can use the
% |\phfqitDeclarePairedDelimiterXWithAltSizing| and
% |\phfqitDeclarePairedDelimiterXPPWithAltSizing| helpers.  These macros wrap
% the \pkgname{mathtools} package's |\DeclarePairedDelimiterX| and
% |\DeclarePairedDelimiterX| macros, by furthermore enabling the newly defined
% command to accept the size argument using the backtick syntax described in
% \autoref{topic:size-specification-backtick}.  These helpers are used
% internally to define the commands for kets, bras, norms, etc.
%
%
% \section{Entropy Measures and Other ``Qit Objects''}
%
% A ``Qit Object'' is any form of quantity which has several parameters and/or
% arguments which are put together in some notation.  The idea is to use
% \LaTeX{} macros to represent an actual quantity and not just some set of
% notational symbols.  For example, for the ``old'' max-entropy
% $H_\mathrm{max,old}(X)_\rho = \log\rank\rho$, you should use |\Hzero|
% independently of whether it should be denoted by $H_0$, $H_\mathrm{max}$ or
% $H_\mathrm{max,old}$.  This allows you to change the notation by redefining
% the command |\Hzero|, while making sure that the correct quantity is
% addressed.  (You might have both ``old''-style and ``new''-style max-entropy
% in the same paper.  Their meaning should never change, even if you change your
% mind on the notation.)  The macros |\Hmin|, |\Hzero|, |\Hmaxf| and |\HH| may
% be redefined to change the subscript by using the following code (change
% ``|\mathrm{max},0|'' to your favorite subscript text):
% \begin{verbatim}
% \renewcommand{\Hzero}{\Hbase{\HSym}{\mathrm{max},0}}
% \end{verbatim}
%
% The \pkgname{phfqit} package provides a basic infrastructure allowing to
% define such ``Qit Object'' implementations.  This package provides the
% following Qit Objects: entropy measures (|\Hbase|), an entropy function
% (|\Hfnbase|), relative entropy measures (|\Dbase|), as well as coherent
% relative entropy measures (|\DCohbase|).  The more specific commands |\Hmin|,
% |\Hzero|, etc.\@ are then defined based on these ``base commands.''
%
% You may also define your own Qit Object implementations. See
% \autoref{sec:QitObjectImpl} for documentation on that.
%
% The actual entropy measure definitions |\Hmin|, |\Hmaxf|, etc., can be
% disabled by specifying the package option \pkgoptionfmt{qitobjdef=none}.
%
%
% \subsection{Entropy, Conditional Entropy}
% \label{sec:entropy-measures}
%
% These entropy measures all share the same syntax.  This syntax is only
% described for the min-entropy |\Hmin|, but the other entropy measures enjoy
% the same features.
%
% These commands are robust, meaning they can be used for example in figure
% captions and section headings.
%
% \DescribeMacro{\Hmin} Min-entropy.  The general syntax is
% |\Hmin|\hspace{0pt}\meta{size-spec}\relax
% \hspace{0pt}\oarg{state}\hspace{0pt}\oarg{epsilon}\hspace{0pt}\relax
% \marg{target system}\hspace{0pt}\oarg{conditioning system}.  For example:
% \begin{center}
%   \begin{tabular}{lc}
%       |\Hmin{X}|   & $\Hmin{X}$ \\
%       |\Hmin[\rho]{X}|   & $\Hmin[\rho]{X}$ \\
%       |\Hmin[\rho][\epsilon]{X}[Y]|   & $\Hmin[\rho][\epsilon]{X}[Y]$ \\
%       \verb+\Hmin[\rho|\rho][\epsilon]{X}[Y]+
%                      & $\Hmin[\rho\mid\rho][\epsilon]{X}[Y]$ \\
%       |\Hmin[][\epsilon]{X}[Y]|   & $\Hmin[][\epsilon]{X}[Y]$ \\[1ex]
%       |\Hmin`\Big[\rho]{X}[Y]| & $\Hmin`\Big[\rho][\epsilon]{X}[Y]$ \\[0.5ex]
%       |\Hmin`*[\rho]{\bigoplus_i X_i}[Y]| &
%         $\displaystyle\Hmin`*[\rho][\epsilon]{\bigoplus_i X_i}[Y]$
%   \end{tabular}
% \end{center}
%
% \DescribeMacro{\HH} Shannon/von Neumann entropy.  This macro has the same
% arguments as for |\Hmin| (even though, of course, there is no real use in
% smoothing the Shannon/von Neumann entropy\ldots).  For example,
% |\HH[\rho]{X}[Y]| gives $\HH[\rho]{X}[Y]$.
%
% \DescribeMacro{\Hzero} R\'enyi-zero max-entropy.  This macro has the same
% arguments as for |\Hmin|.  For example, |\Hzero[][\epsilon]{X}[Y]| gives
% $\Hzero[][\epsilon]{X}[Y]$.
%
% \DescribeMacro{\Hmaxf} The max-entropy.  This macro has the same
% arguments as for |\Hmin|.  For example, |\Hmaxf[][\epsilon]{X}[Y]| gives
% $\Hmaxf[][\epsilon]{X}[Y]$.
%
% The commands |\Hmin|, |\HH|, |\Hzero|, and |\Hmaxf| are defined only if the
% package option \pkgoptionfmt{qitobjdef=stdset} is set (which is the default).
%
% \DescribeMacro{\HSym} You may redefine this macro if you want to change the
% ``$H$'' symbol of all entropy measures.
% \begingroup \def\HSym{\spadesuit} For example, with
% |\renewcommand\HSym{\spadesuit}|, |\Hmin{A}[B]| would give $\Hmin{A}[B]$.
% \endgroup
%
% \paragraph{Appearance and alternative notation.}
% You may change the notation of any of the above entropy measures by redefining
% the corresponding commands as follows:
% \begin{verbatim}
% \renewcommand{\Hzero}{\Hbase{\HSym}{\mathrm{max}}}
% \end{verbatim}
% \begingroup\renewcommand{\Hzero}{\Hbase{\HSym}{\mathrm{max}}}
% Then, |\Hzero[\rho]{A}[B]| would produce: $\Hzero[\rho]{A}[B]$.\endgroup
%
% \paragraph{Base entropy measure macro.}
% \DescribeMacro{\Hbase} Base macro entropy for an entropy measure.  The general
% syntax is:
% |\Hbase|\hspace{0pt}\marg{H-symbol}\hspace{0pt}\marg{subscript}\relax
% \hspace{0pt}\oarg{state}\hspace{0pt}\oarg{epsilon}\hspace{0pt}\relax
% \marg{target system}\hspace{0pt}\oarg{conditioning system}
%
% Using this macro it is easy to define custom special-purpose entropy measures,
% for instance:
% \begin{verbatim}
% \newcommand\Hxyz{\Hbase{\tilde\HSym}{\mathrm{xyz}}}
% \end{verbatim}
% \begingroup\newcommand\Hxyz{\Hbase{\tilde\HSym}{\mathrm{xyz}}}
% The above code defines the command |\Hxyz[\rho][\epsilon]{A}[B]| $\to$
% \fbox{$\Hxyz[\rho][\epsilon]{A}[B]$}. \endgroup
%
% See also the implementation documentation below for more specific information
% on how to customize parts of the rendering, for instance.
% 
% \subsection{Entropy Function}
% \label{sec:entropy-function}
%
% \DescribeMacro{\Hfn} The entropy, written as a mathematical function.  It is
% useful to write, e.g., $\Hfn(p_1\rho_1 + p_2\rho_2)$ as \relax
% |\Hfn(p_1\rho_1 + p_2\rho_2)|.  Sizing specifications also work, e.g.\@
% |\Hfn`\big(x)| or |\Hfn`*(x)|.
%
% Usage is: |\Hfn|\hspace{0pt}\meta{size-spec}\hspace{0pt}|(|\meta{argument}|)|
%
% This macro doesn't allow for any subscript, any epsilon-like superscript nor
% for any conditioning system.  Define your own macro on top of |\Hfnbase| if
% you need that.
%
% Note that the \meta{argument} may contain matching parentheses, e.g.,
% |\Hfn`\Big( g(x) + h(y) )| $\to$ \fbox{$\Hfn`\Big(g(x)+h(y))$}.
%
% \DescribeMacro{\Hfunc}
% The alias |\Hfunc| is provided for backwards compatibility; same as |\Hfn|.
%
% The commands |\Hfn| and |\Hfunc| are defined only if the package option
% \pkgoptionfmt{qitobjdef=stdset} is set (which is the default).
%
% \DescribeMacro{\Hfnbase} There is also a base macro for this kind of Qit
% Object, |\Hfnbase|.  It allows you to specify an arbitrary symbol to use for
% ``$H$'', as well as custom subscripts and superscripts.  The syntax is:
%
% |\Hfnbase|\marg{H-symbol}\hspace{0pt}\marg{sub}\hspace{0pt}\relax
% \marg{sup}\hspace{0pt}\relax
% \meta{size-spec}\hspace{0pt}|(|\meta{argument}|)|.
%
%
% \subsection{Relative Entropy}
% \label{sec:relative-entropies}
%
% Relative entropies also have a corresponding set of commands.
%
% The syntax varies from command to command, but all relative entropies accept
% the final arguments \meta{size-spec}\marg{state}\marg{relative-to-state}.  The
% size-spec is as always given using the backtick syntax described in
% \autoref{topic:size-specification-backtick}.
%
% \DescribeMacro{\DD}
% Generic relative entropy.  The syntax of this command is either of the following:
% \par
% |\DD|\hspace{0pt}\meta{size-spec}\hspace{0pt}\marg{state}\hspace{0pt}\marg{relative-to state},\\
% |\DD_|\marg{subscript}\hspace{0pt}\meta{size-spec}\hspace{0pt}\marg{state}\hspace{0pt}\marg{relative-to state},\\
% |\DD_|\marg{subscript}|^|\marg{superscript}\hspace{0pt}\meta{size-spec}\relax
% \hspace{0pt}\marg{state}\hspace{0pt}\marg{relative-to state},\\
% |\DD^|\marg{superscript}\hspace{0pt}\meta{size-spec}\hspace{0pt}\marg{state}\hspace{0pt}\marg{relative-to state}.
%
% In all cases, the argument is typeset as:
% $\bigl(\meta{state}\big\Vert\meta{relative-to state}\bigr)$.  The size of the
% delimiters can be set with a size specification using the standard backtick
% syntax as described in \autoref{topic:size-specification-backtick} (as for the
% other entropy measures).
% 
% Examples:
% \begin{center}
%   \begin{tabular}{lc}
%       |\DD{\rho}{\sigma}|  & $\DD{\rho}{\sigma}$ \\[1ex]
%       |\DD`*{M_1^\dagger M_1}{\sigma}|  & $\DD`*{M_1^\dagger M_1}{\sigma}$ \\[1ex]
%       |\DD`\Big{\rho}{\sigma}|  & $\DD`\Big{\rho}{\sigma}$ \\
%   \end{tabular}
% \end{center}
%
% You can also play around with subscripts and superscripts, but it is
% recommended to use the macros |\Dminf|, |\Dminz| and |\Dmax| directly.
% Specifying the subscripts and superscripts to |\DD| should only be done within
% new custom macros to define new relative entropy measures.
% \begin{center}
%   \begin{tabular}{lc}
%       |\DD_{\mathrm{Rob}}^{\epsilon}{\rho}{\sigma}|  & $\DD_{\mathrm{Rob}}^{\epsilon}{\rho}{\sigma}$ \\
%       |\DD^{sup}{\rho}{\sigma}|  & $\DD^{sup}{\rho}{\sigma}$ \\
%   \end{tabular}
% \end{center}
%
% \DescribeMacro{\Dmax} The max-relative entropy.  The syntax is
% |\Dmax|\hspace{0pt}\oarg{epsilon}\hspace{0pt}\meta{size-spec}\relax
% \hspace{0pt}\marg{state}\hspace{0pt}\marg{relative-to state}
%
% For example |\Dmax[\epsilon]{\rho}{\sigma}| gives
% $\Dmax[\epsilon]{\rho}{\sigma}$ and |\Dmax[\epsilon]`\big{\rho}{\sigma}| gives
% $\Dmax[\epsilon]`\big{\rho}{\sigma}$.
% 
% \DescribeMacro{\Dminz} The ``old'' min-relative entropy, based on the
% R\'enyi-zero relative entropy.  The syntax is the same as for
% |\Dmax|.
% 
% \DescribeMacro{\Dminf} The ``new'' min-relative entropy, defined using the
% fidelity.  The syntax is the same as for |\Dmax|.
%
% \DescribeMacro{\Dr} The Rob-relative entropy.  The syntax is the same as for
% |\Dmax|.
%
% \DescribeMacro{\DHyp} \DescribeMacro{\Dhyp} The hypothesis testing relative
% entropy (two possible variants).  The syntax is the same as for |\Dmax|,
% except that by default the optional argument is |\eta|.  The symbols
% `$\mathrm{H}$' and `$\mathrm{h}$' are used as subscripts, respectively, for
% |\DHyp| and |\Dhyp|.  Examples: The code |\DHyp{\rho}{\sigma}| gives
% $\DHyp{\rho}{\sigma}$ and |\Dhyp[\eta']{\rho}{\sigma}| gives
% $\Dhyp[\eta']{\rho}{\sigma}$.  (This is because this quantity is directly
% defined with a $\eta$ (or $\epsilon$) built in, and it is not a zero-error
% quantity which is smoothed with the purified distance.)
%
% The commands |\DD|, |\Dmax|, |\Dminz|, |\Dminf|, |\Dr|, |\DHyp| and |\Dhyp|
% are defined only if the package option \pkgoptionfmt{qitobjdef=stdset} is set
% (which is the default).
%
% \changedreftext{v3.1-added-Dhyp-qitobjdef-stdset}
%
% \DescribeMacro{\DSym} The symbol to use to denote a relative entropy.  You
% may redefine this command to change the symbol.  (This works like |\HSym|
% above.)
%
% \paragraph{Appearance and alternative notation}
% You may change the notation of any of the above relative entropy measures by
% redefining the corresponding commands as follows:
% \begin{verbatim}
% \renewcommand{\Dminz}[1][]{\Dbase{\DSym}_{\mathrm{MIN}}^{#1}}
% \end{verbatim}
% \begingroup\renewcommand{\Dminz}[1][]{\Dbase{\DSym}_{\mathrm{MIN}}^{#1}}
% The above command produces: |\Dminz[\epsilon]{\rho}{\sigma}| $\to$
% \fbox{$\Dminz[\epsilon]{\rho}{\sigma}$}.\endgroup
% 
%
% \paragraph{Base relative entropy command}
% As for the $H$-type entropy measures, there is a ``base relative entropy
% command'' |\Dbase|.  Its syntax is:
% \par |\Dbase|\marg{D-symbol}\hspace{0pt}\relax
% [|_|\marg{subscript}][|^|\marg{superscript}]\hspace{0pt}\meta{size-spec}\relax
% \hspace{0pt}\marg{state}\hspace{0pt}\marg{relative-to state}
%
% Example: |\Dbase{\hat\DSym}_{0}^{\eta'}`\Big{\rho}{\sigma}| $\to$
% \fbox{$\Dbase{\hat\DSym}_{0}^{\eta'}`\Big{\rho}{\sigma}$}
%
% The ``|_|\marg{subscript}'' and ``|^|\marg{superscript}'' parts are optional
% and may be specified in any order.
%
% See also the implementation documentation below for more specific information
% on how to customize parts of the rendering, for instance.
% 
%
% \subsection{Coherent Relative Entropy}
% \label{sec:coh-rel-entr}
%
% A macro for the coherent relative entropy is also available.
%
% \DescribeMacro{\DCohx} Typeset a coherent relative entropy using an
% alternative form for the reference system.  The syntax is:
% 
% |\DCohx|\hspace{0pt}\oarg{epsilon}\hspace{0pt}\relax
% \meta{size-spec}\hspace{0pt}\marg{rho}\hspace{0pt}\relax
% \marg{X}\hspace{0pt}\marg{X'}\hspace{0pt}\relax
% \marg{$\Gamma_X$}\hspace{0pt}\marg{$\Gamma_{X'}$}
%
% For example, |\DCohx[\epsilon]{\rho}{X}{X'}{\Gamma_X}{\Gamma_{X'}}| gives
% $\DCohx[\epsilon]{\rho}{X}{X'}{\Gamma_X}{\Gamma_{X'}}$.
%
% The subscript $X'R_X$ (or whatever the system names) is automatically added to
% the \meta{rho} argument.  The `$R$' symbol is used by default for designating
% the reference system; you may change that by redefining |\DCohxRefSystemName|
% (see below).  If no subscript should be added to the \meta{rho} argument, then
% begin the \meta{rho} argument with a star.  For example,
% |\DCoh{*\sigma_R\otimes\rho_{X'}}{X}{X'}{\Gamma_X}{\Gamma_{X'}}| gives
% $\DCoh{*\sigma_R\otimes\rho_{X'}}{X}{X'}{\Gamma_X}{\Gamma_{X'}}$.
% 
% The \meta{size-spec} is of course optional and follows the same syntax as
% everywhere else (\autoref{topic:size-specification-backtick}).
%
% The command |\DCohx| is defined only if the package option
% \pkgoptionfmt{qitobjdef=stdset} is set (which is the default).
%
% \DescribeMacro{\emptysystem} Use the |\emptysystem| macro to denote a trivial
% system.  For example, |\DCoh{\rho}{X}{\emptysystem}{\Gamma}{1}| gives
% $\DCoh{\rho}{X}{\emptysystem}{\Gamma}{1}$.
% 
% \DescribeMacro{\DCohxRefSystemName} When using |\DCohx|, the macro
% |\DCohxRefSystemName| is invoked to produce the reference system name
% corresponding to the input system name.  By default, this is a $R_\cdot$
% symbol with subscript the input system name.  You may redefine this macro if
% you prefer another reference system name:
% \begin{verbatim}
% \renewcommand\DCohxRefSystemName[1]{E_{#1}}
% \end{verbatim}
% \begin{flushleft}
% \begingroup\renewcommand\DCohxRefSystemName[1]{E_{#1}}
% Then: |\DCohx{\rho}{X}{X'}{\Gamma_X}{\Gamma_{X'}}| $\to$
% $\DCohx{\rho}{X}{X'}{\Gamma_X}{\Gamma_{X'}}$
% \endgroup
% \end{flushleft}
%
% \DescribeMacro{\DCSym} The symbol to use to denote a coherent relative
% entropy.  You may redefine this command to change the symbol.  (This works
% like |\HSym| and |\DSym| above.)
% 
% \DescribeMacro{\DCoh}
% Typeset a coherent relative entropy using the old notation.  The syntax is:
% 
% |\DCoh|\hspace{0pt}\oarg{epsilon}\hspace{0pt}\relax
% \meta{size-spec}\hspace{0pt}\marg{rho}\hspace{0pt}\relax
% \marg{R}\hspace{0pt}\marg{X'}\hspace{0pt}\relax
% \marg{$\Gamma_R$}\hspace{0pt}\marg{$\Gamma_{X'}$}
%
% For example, |\DCoh[\epsilon]{\rho}{R}{X'}{\Gamma_R}{\Gamma_{X'}}| gives
% $\DCoh[\epsilon]{\rho}{R}{X'}{\Gamma_R}{\Gamma_{X'}}$.
%
% The subscript $X'R$ (or whatever the system names) is automatically added to
% the \meta{rho} argument.  If this is not desired, then begin the \meta{rho}
% argument with a star.  For example,
% |\DCoh{*\sigma_R\otimes\rho_{X'}}{R}{X'}{\Gamma_R}{\Gamma_{X'}}| gives
% $\DCoh{*\sigma_R\otimes\rho_{X'}}{R}{X'}{\Gamma_R}{\Gamma_{X'}}$.
% 
% The \meta{size-spec} is of course optional and follows the same syntax as
% everywhere else (\autoref{topic:size-specification-backtick}).
%
% The command |\DCoh| is defined only if the package option
% \pkgoptionfmt{qitobjdef=stdset} is set (which is the default).
%
%
% \paragraph{Appearance and alternative notation}
% You may change the notation of any of the above relative entropy measures by
% redefining the corresponding commands as follows:
% \begin{verbatim}
% \renewcommand{\DCoh}{\DCohbase{\tilde\DSym}}
% \end{verbatim}
% \begingroup\renewcommand{\DCoh}{\DCohbase{\tilde\DSym}}
% Then: |\DCoh[\epsilon]{\rho}{R}{X'}{\Gamma_R}{\Gamma_{X'}}| $\to$
% \fbox{$\DCoh[\epsilon]{\rho}{R}{X'}{\Gamma_R}{\Gamma_{X'}}$}.\endgroup
% 
%
% \paragraph{Base relative entropy command}
% As for the other entropy measures, there is a ``base coherent relative entropy
% command'' |\DCohbase|.  Its syntax is:
% \par |\DCohbase|\marg{D-symbol}\hspace{0pt}\relax
% \hspace{0pt}\oarg{epsilon}\hspace{0pt}\relax
% \meta{size-spec}\hspace{0pt}\marg{rho}\hspace{0pt}\relax
% \marg{R}\hspace{0pt}\marg{X'}\hspace{0pt}\relax
% \marg{$\Gamma_R$}\hspace{0pt}\marg{$\Gamma_{X'}$}
%
% See also the implementation documentation below for more specific information
% on how to customize parts of the rendering, for instance.
% 
%
%
% \subsection{Custom Qit Objects}
% \label{sec:QitObjectImpl}
%
% \changedreftext{v2.0-qit-objects}
%
% You can create your own Qit Object Implementation as follows.  You need two
% components: a \emph{Parse} macro and a \emph{Render} macro.
%
% The \emph{Parse} macro is responsible for parsing input \LaTeX{} tokens as
% necessary, and building an argument list (which will be passed on to the
% \emph{Render} macro).
%
% \DescribeMacro{\qitobjAddArg} \DescribeMacro{\qitobjAddArgx} The \emph{Parse}
% macro (or any helper macro it calls) should call |\qitobjAddArg| to add
% arguments for the eventual call to \emph{Render}.  The |\qitobjAddArg| macro
% does not expand its argument.  The |\qitobjAddArgx| works like
% |\qitobjAddArg|, but it accepts a single \LaTeX{} command as its only
% argument, expands it, and adds the contents as a single new argument for the
% renderer.
%
% \DescribeMacro{\qitobjParseDone}
% Once the parser is finished, it must call |\qitobjParseDone|.
%
% The \emph{Render} macro is responsible for displaying the final Qit Object.
% It should accept mandatory arguments in the exact number as there were calls
% to |\qitobjAddArg|/|\qitobjAddArgx|.
%
% \DescribeMacro{\qitobjDone} The \emph{Render} macro must call |\qitobjDone|
% after it is finished, to do some cleaning up and to close the local \LaTeX{}
% group generated by the Qit Ojbect infrastructure.
%
% \DescribeMacro{\DefineQitObject} Declare your new Qit Object using the
% |\DefineQitObject| macro, using the syntax
% |\DefineQitObject|\marg{name}\marg{ParseCommand}\marg{RenderCommand}.
% This declares the command |\|\meta{name} as your Qit Object.
%
% You may define different Qit Objects (using different names) recycling the
% same parsers/renderers if needed.  For instance, |\Hfnbase| uses the same
% renderer as |\Hbase|.
%
% \DescribeMacro{\DefineTunedQitObject} The |\DefineTunedQitObject| macro is a
% bit more powerful.  It allows you to specify some fixed initial arguments to
% the parser, as well as to provide some local definitions which are in effect
% only during parsing and rendering of the Qit Object.  This is useful if you
% would like to declare an alternative type of Qit Object to an existing one,
% where you just change some aspect of the behavior of the original Qit Object.
%
% Usage: |\DefineTunedQitObject|\hspace{0pt}\marg{name}\relax
% \marg{parse command}\hspace{0pt}\marg{render command}\hspace{0pt}\relax
% \marg{fixed first argument(s)}\hspace{0pt}\marg{custom definitions}\relax
%
% The \marg{first fixed argument(s)} must be a single argument, i.e., a single
% \LaTeX{} group, which may contain several arguments, for instance: |{{A}{B}}|.
%
% For instance, |\DCohx| is defined, using the same parser and renderer as for
% |\DCoh|, as follows:
% \begin{verbatim}
%\def\DCohxRefSystemName#1{R_{#1}}
%\def\DCohxStateSubscripts#1#2{#2\DCohxRefSystemName{#1}}
%\DefineTunedQitObject{DCohx}{\DCohbaseParse}{\DCohbaseRender}%
%{{\DCSym}}% initial args
%{\let\DCohbaseStateSubscripts\DCohxStateSubscripts}% local defs
% \end{verbatim}
% 
%
% \paragraph{Useful helpers}
%
% There are some useful helpers for both the \emph{Parse} and \emph{Render}
% macros.  More extensive documentation is available in the ``Implementation''
% section below.
%
% \DescribeMacro{\phfqit@parse@sizesarg} Parse a \meta{size-spec} optional
% argument.
%
% \needspace{3\baselineskip}
% \DescribeMacro{\phfqitParen} \DescribeMacro{\phfqitSquareBrackets}
% \DescribeMacro{\phfqitCurlyBrackets} Produce a parenthetic expression (or
% square or curly brackets) with the appropriate size and with the given
% contents.
%
% \paragraph{Example}
% Here is a simple example: let's build a work cost of transition Qit Object to
% display something like ``$W(\sigma\to\rho)$.''
%
% The arguments to be given are: they are \meta{$\sigma$} and \meta{$\rho$}.  We
% would also like to accept an optional size specification \meta{size-spec}. We
% should decide on a convenient syntax to specify them.  Here, we'll settle for
% simply |\WorkCostTransition`\Big{\rho}{\sigma}|.
%
% We can now write the \emph{Parse} macro.  We use the |\phfqit@parsesizearg|
% helper, which stores the optional \meta{size-spec} into the
% |\phfqit@val@sizearg| macro before deferring our second helper macro.  We then
% add arguments (for an eventual call to the \emph{Render} macro) using
% |\qitobjAddArg| (or |\qitobjAddArgx|).
% \begin{verbatim}
% \makeatletter
% \newcommand\WorkCostTransitionParse{%
%   \phfqit@parsesizearg\WorkCostTransitionParse@%
% }
% % Helper to parse further input arguments:
% \newcommand\WorkCostTransitionParse@[2]{% {\rho}{\sigma}
%   \qitobjAddArgx\phfqit@val@sizearg% size arg
%   \qitobjAddArg{#1}% rho
%   \qitobjAddArg{#2}% sigma
%   \qitobjParseDone%
% }
% \makeatother
% \end{verbatim}
%
% The render macro should simply display the quantity, with the arguments given
% as usual mandatory arguments.  We invoke the |\phfqitParens| helper, which
% produces the parenthesis at the correct size given the size spec tokens.
% \begin{verbatim}
% \newcommand\WorkCostTransitionRender[3]{% {size-spec-tokens}{\rho}{\sigma}
%   W\phfqitParens#1{#2 \to #3}%
%    \qitobjDone
% }
% \end{verbatim}
%
% Now declare the Qit Object:
% \begin{verbatim}
% \DefineQitObject{WorkCostTransition}{\WorkCostTransitionParse}{\WorkCostTransitionRender}
% \end{verbatim}
% \begingroup\makeatletter
% \newcommand\WorkCostTransitionParse{\relax
%   \phfqit@parsesizearg\WorkCostTransitionParse@}
% \newcommand\WorkCostTransitionParse@[2]{\relax
%   \qitobjAddArgx\phfqit@val@sizearg\relax
%   \qitobjAddArg{#1}\relax
%   \qitobjAddArg{#2}\relax
%   \qitobjParseDone}
% \newcommand\WorkCostTransitionRender[3]{W\phfqitParens#1{#2 \to #3}\qitobjDone}
% \DefineQitObject{WorkCostTransition}{\WorkCostTransitionParse}{\WorkCostTransitionRender}
% Then: |\WorkCostTransition`\Big{\rho}{\sigma}| $\to$
% \fbox{$\WorkCostTransition`\Big{\rho}{\sigma}$}
% \endgroup
%
% You might want to check out the implementations of |\HbaseParse| and
% |\HbaseRender|, or |\DbaseParse| and |\DbaseRender| if you'd like to see some
% more involved examples.
%
%
%
%
%
% \StopEventually{\clearpage\PrintChanges
%     \vspace{2cm plus 2cm minus 2cm}\PrintIndex}
%
% \section{Implementation}
%
% First, load dependent packages.  Toolboxes, fonts and so on.
%    \begin{macrocode}
\RequirePackage{calc}
\RequirePackage{etoolbox}
\RequirePackage{amsmath}
\RequirePackage{amssymb}
\RequirePackage{dsfont}
\RequirePackage{mathrsfs}
\RequirePackage{mathtools}
%    \end{macrocode}
% 
% Package \pkgname{xparse} is needed in order to get paren matching right for
% |\Hfn|.
%    \begin{macrocode}
\RequirePackage{xparse}
%    \end{macrocode}
%
% Package options are handled via \pkgname{xkeyval} \& \pkgname{kvoptions} (see
% implementation doc for \pkgname{phfnote}).
%    \begin{macrocode}
\RequirePackage{xkeyval}
\RequirePackage{kvoptions}
%    \end{macrocode}
% 
% \subsection{Simple Symbols and Shorthands}
% 
% 
% \subsubsection{General Symbols}
%
% These symbols are documented in \autoref{sec:symbols}.
%
% \begin{macro}{\Hs}
%   Hilbert space.
%    \begin{macrocode}
\newcommand{\Hs}{\mathscr{H}}
%    \end{macrocode}
% \end{macro}
%
% \begin{macro}{\Ident}
%   Identity operator, $\Ident$.
%    \begin{macrocode}
\newcommand{\Ident}{\mathds{1}}
%    \end{macrocode}
% \end{macro}
% 
%
% \begin{macro}{\IdentProc}
%   Identity process.
%
%   TODO: this could be implemented as a Qit Object.
%    \begin{macrocode}
\def\IdentProc{%
  \phfqit@parsesizearg\phfqit@IdentProc@maybeA%
}
\newcommand\phfqit@IdentProc@maybeA[1][]{%
  \def\phfqit@IdentProc@val@A{#1}%
  \phfqit@IdentProc@maybeB%
}
\newcommand\phfqit@IdentProc@maybeB[1][]{%
  \def\phfqit@IdentProc@val@B{#1}%
  \phfqit@IdentProc@arg%
}
\def\phfqit@IdentProc@arg#1{%
  \def\phfqit@IdentProc@val@arg{#1}%
%    \end{macrocode}
% 
% At this point, prepare the three arguments, each expanded exactly as they were when
% given to these macros, and delegate the formatting to |\phfqit@IdentProc@do|.
%    \begin{macrocode}
  \edef\@tmp@args{%
    {\expandonce{\phfqit@IdentProc@val@A}}%
    {\expandonce{\phfqit@IdentProc@val@B}}%
    {\expandonce{\phfqit@IdentProc@val@arg}}%
  }%
  \expandafter\phfqit@IdentProc@do\@tmp@args%
}
\def\phfqit@IdentProc@do#1#2#3{%
  \operatorname{id}_{#1\notblank{#2}{\to #2}{}}%
  \notblank{#3}{\expandafter\phfqitParens\phfqit@val@sizearg{#3}}{}%
}
%    \end{macrocode}
% \end{macro}
% 
%
%
% \begingroup\catcode`\^=12\relax
% \begin{macro}{\ee^...}
%   Macro for the exponential.
%
%   Because the character |^| might have different catcodes (e.g. with the
%   \pkgname{breqn} package), we use a hack with |\detokenize|.  Basically this
%   boils down to |\def\ee^#1{\phfqitExpPowerExpression{#1}}| and
%   |\def\phfqitExpPowerExpression#1{e^{#1}}|, up to making sure that all the
%   |^| symbols are compared without catcodes.
%
%   \changed[v4.0-fixed-ee-for-catcodes]{v4.0}{2021/10/07}{Fixed the definition
%   of \phfverb{\ee} in order to support the case where the catcode of
%   \phfverb{^} changes}
%    \begin{macrocode}
\edef\phfqit@def@hat{\detokenize{^}}
\expandafter\def\expandafter\phfqit@ee@gobblehat\phfqit@def@hat{%
  \phfqitExpPowerExpression}
\def\phfqitExpPowerExpression#1{e^{#1}}
\def\ee#1{\expandafter\phfqit@ee@gobblehat\detokenize{#1}}
\robustify\phfqitExpPowerExpression
\robustify\ee
%    \end{macrocode}
% \end{macro}
% \endgroup
%
% \subsubsection{Math Operators}
%
% See user documentation in \autoref{sec:math-operators}.
%
% \needspace{6\baselineskip}
% \begin{macro}{\tr}
% \begin{macro}{\supp}
% \begin{macro}{\rank}
% \begin{macro}{\linspan}
% \begin{macro}{\spec}
% \begin{macro}{\diag}
%   Some common math operators.  Note that |\span| is already defined by \LaTeX{}, so we
%   resort to |\linspan| for the linear span of a set of vectors.
%    \begin{macrocode}
\DeclareMathOperator{\tr}{tr}
\DeclareMathOperator{\supp}{supp}
\DeclareMathOperator{\rank}{rank}
\DeclareMathOperator{\linspan}{span}
\DeclareMathOperator{\spec}{spec}
\DeclareMathOperator{\diag}{diag}
%    \end{macrocode}
% \end{macro}
% \end{macro}
% \end{macro}
% \end{macro}
% \end{macro}
% \end{macro}
% 
% \begin{macro}{\phfqit@Realpart}
% \begin{macro}{\phfqit@Imagpart}
%   Provide math operators for $\Re$ and $\Im$.  The aliasing to the actual
%   commands |\Re| and |\Im| is done later, when we process the package options.
%    \begin{macrocode}
\let\phfqit@Re\Re
\DeclareMathOperator{\phfqit@Realpart}{Re}%
\let\phfqit@Im\Im
\DeclareMathOperator{\phfqit@Imagpart}{Im}%
%    \end{macrocode}
% \end{macro}
% \end{macro}
%
%
% \subsubsection{Poly}
%
% \begin{macro}{\poly}
%   Poly symbol.
%   \iffalse meta-comment
%         \changed[v1.0-added-poly-command]{v1.0}{2015/05/22}{Added \phfverb\poly\space command}
%   \fi
%    \begin{macrocode}
\DeclareMathOperator{\poly}{poly}
%    \end{macrocode}
% \end{macro}
%
% \subsubsection{Bits and Bit Strings}
% 
% See documentation in \autoref{sec:bits}
%
% \begin{macro}{\bit}
% \begin{macro}{\bitstring}
%   Bits and bit strings.
%    \begin{macrocode}
\newcommand\bit[1]{\texttt{#1}}
\newcommand\bitstring[1]{\phfqit@bitstring{#1}}
%    \end{macrocode}
% 
% The implementation of |\bitstring| needs some auxiliary internal macros.
%    \begin{macrocode}
\def\phfqit@bitstring#1{%
  \begingroup%
  \setlength{\phfqit@len@bit}{\maxof{\widthof{\bit{0}}}{\widthof{\bit{1}}}}%
  \phfqitBitstringFormat{\phfqit@bitstring@#1\phfqit@END}%
  \endgroup%
}
%    \end{macrocode}
% 
% The internal |\phfqit@bitstring@| macro picks up the next bit, and puts it
% into a \LaTeX{} |\makebox| on its own with a fixed width.
%    \begin{macrocode}
\def\phfqit@bitstring@#1#2\phfqit@END{%
  \makebox[\phfqit@len@bit][c]{\phfqitBitstringFormatBit{#1}}%
  \if\relax\detokenize\expandafter{#2}\relax%
  \else%
%    \end{macrocode}
%
% If there are bits left, then recurse for the rest of the bitstring:
%    \begin{macrocode}
    \phfqitBitstringSep\phfqit@bitstring@#2\phfqit@END%
  \fi%
}
\newlength\phfqit@len@bit
%    \end{macrocode}
% \end{macro}
% \end{macro}
% 
% \begin{macro}{\phfqitBitstringSep}
% \begin{macro}{\phfqitBitstringFormat}
%   Redefine these to customize the bit string appearance.
%    \begin{macrocode}
\newcommand\phfqitBitstringSep{\hspace{0.3ex}}
\newcommand\phfqitBitstringFormat[1]{\ensuremath{\underline{\overline{#1}}}}
\def\phfqitBitstringFormatBit{\bit}
%    \end{macrocode}
% \end{macro}
% \end{macro}
%
%
% \subsubsection{Logical Gates}
%
% See user documentation in \autoref{sec:gates}.
%
% \begin{macro}{\gate}
%   Generic macro to format a gate name.
%    \begin{macrocode}
\DeclareRobustCommand\gate[1]{\ifmmode\textsc{\lowercase{#1}}%
  \else{\rmfamily\textsc{\lowercase{#1}}}\fi}
%    \end{macrocode}
% \end{macro}
% 
% \needspace{5\baselineskip}
% \begin{macro}{\AND}
% \begin{macro}{\XOR}
% \begin{macro}{\CNOT}
% \begin{macro}{\NOT}
% \begin{macro}{\NOOP}
%   Some common gates.
%    \begin{macrocode}
\newcommand{\AND}{\gate{And}}
\newcommand{\XOR}{\gate{Xor}}
\newcommand{\CNOT}{\gate{C-Not}}
\newcommand{\NOT}{\gate{Not}}
\newcommand{\NOOP}{\gate{No-Op}}
%    \end{macrocode}
% \end{macro}
% \end{macro}
% \end{macro}
% \end{macro}
% \end{macro}
% 
%
% \subsubsection{Lie Groups \& Algebras}
%
% \needspace{7\baselineskip}
% \begin{macro}{\uu(N)}
% \begin{macro}{\UU(N)}
% \begin{macro}{\su(N)}
% \begin{macro}{\SU(N)}
% \begin{macro}{\so(N)}
% \begin{macro}{\SO(N)}
% \begin{macro}{\slalg(N)}
% \begin{macro}{\SL(N)}
% \begin{macro}{\GL(N)}
% \begin{macro}{\SN(N)}
%   Some Lie Groups \& Algebras.  See \autoref{sec:Lie-groups-algebras}
%    \begin{macrocode}
\def\uu(#1){\phfqitLieAlgebra{u}{#1}}
\def\UU(#1){\phfqitLieGroup{U}{#1}}
\def\su(#1){\phfqitLieAlgebra{su}{#1}}
\def\SU(#1){\phfqitLieGroup{SU}{#1}}
\def\so(#1){\phfqitLieAlgebra{so}{#1}}
\def\SO(#1){\phfqitLieGroup{SO}{#1}}
\def\slalg(#1){\phfqitLieAlgebra{sl}{#1}} % \sl is "slanted font" in TeX
\def\SL(#1){\phfqitLieGroup{SL}{#1}}
\def\GL(#1){\phfqitLieGroup{GL}{#1}}
\def\SN(#1){\phfqitDiscreteGroup{S}{#1}}
%    \end{macrocode}
% \end{macro}
% \end{macro}
% \end{macro}
% \end{macro}
% \end{macro}
% \end{macro}
% \end{macro}
% \end{macro}
% \end{macro}
% \end{macro}
% 
% \begin{macro}{\phfqitLieAlgebra}
% \begin{macro}{\phfqitLieGroup}
% \begin{macro}{\phfqitDiscreteGroup}
%   Override these to change the appearance of the group names or algebra names.
%   The first argument is the name of the group or algebra (e.g. |su| or |SU|),
%   and the second argument is the parenthesized argument e.g. ``N''.
%    \begin{macrocode}
\newcommand\phfqitLieAlgebra[2]{\mathfrak{#1}({#2})}
\newcommand\phfqitLieGroup[2]{\mathrm{#1}({#2})}
\newcommand\phfqitDiscreteGroup[2]{\mathrm{#1}_{#2}}
%    \end{macrocode}
% \end{macro}
% \end{macro}
% \end{macro}
%
%
%
%
%
% \subsection{Helper for parsing a size argument}
%
% \begin{macro}{\phfqit@parsesizearg}
%   Utility to parse size argument with the backtick specification
%   (\autoref{topic:size-specification-backtick}).
%
%   Parses a size argument, if any, and stores it into |\phfqit@val@sizearg|.
%   The value stored can directly be expanded as an optional argument to a
%   |\DeclarePairedDelimiter|-compatible command (see \pkgname{mathtools} package).
%
%   |#1| should be a command token.  It is the next action to take, after
%   argument has been parsed.
%    \begin{macrocode}
\def\phfqit@parsesizearg#1{%
  \begingroup%
  \mathcode`\`="0060\relax%
  \gdef\phfqit@val@sizearg{}%
  \def\phfqit@tmp@contwithsize{\phfqit@parsesizearg@withsize{#1}}%
  \@ifnextchar`{\phfqit@tmp@contwithsize}{\endgroup#1}%
}
\def\phfqit@parsesizearg@withsize#1`#2{%
  \def\phfqit@tmp@x{#2}%
  \def\phfqit@tmp@star{*}%
  \ifx\phfqit@tmp@x\phfqit@tmp@star%
    \gdef\phfqit@val@sizearg{*}%
    \def\phfqit@tmp@cont{\endgroup#1}%
    \expandafter\phfqit@tmp@cont%
  \else%
    \gdef\phfqit@val@sizearg{[#2]}%
    \def\phfqit@tmp@cont{\endgroup#1}%
    \expandafter\phfqit@tmp@cont%
  \fi%
}
%    \end{macrocode}
% \end{macro}
%
% \begin{macro}{\phfqitDeclarePairedDelimiterXWithAltSizing}
% \begin{macro}{\phfqitDeclarePairedDelimiterXPPWithAltSizing}
%   Macros that behave exactly like \pkgname{mathtools}'
%   |\DeclarePairedDelimiterX| and |\DeclarePairedDelimiterXPP|, except that
%   their size argument can also be specified by a backtick as for entropy
%   measures or other objects in this package (e.g., to define |\abs| such that
%   |$\abs`\Big{x - y}$| gives $\abs`\Big{x - y}$).
%   \changed[v4.0-new-phfqitDeclarePairedDelimiterxxx]{v4.0}{2021/10/07}{Added
%   \phfverb{\phfqitDeclarePairedDelimiterXWithAltSizing} and
%   \phfverb{\phfqitDeclarePairedDelimiterXPPWithAltSizing}}
%    \begin{macrocode}
\def\phfqitDeclarePairedDelimiterXWithAltSizing{%
  \phfqitDeclareMathtoolsPairedDelimiterCmdWithAltSizing\DeclarePairedDelimiterX
}
\def\phfqitDeclarePairedDelimiterXPPWithAltSizing{%
  \phfqitDeclareMathtoolsPairedDelimiterCmdWithAltSizing\DeclarePairedDelimiterXPP
}
\def\phfqitDeclareMathtoolsPairedDelimiterCmdWithAltSizing#1#2{%
  \begingroup
    \escapechar=-1\relax
    \xdef\phfqit@tmp@thecmd{%
      \expandafter\noexpand\csname phfqit@paireddelim@def@\string#2\endcsname}%
  \endgroup
  \edef\x{%
    \noexpand\phfqit@paireddelim@parsesizearg{\expandonce\phfqit@tmp@thecmd}%
  }%
  \expandafter\DeclareRobustCommand\expandafter#2\expandafter{\x}%
  \expandafter#1\phfqit@tmp@thecmd
}
\def\phfqit@paireddelim@parsesizearg#1{%
  \phfqit@parsesizearg{\expandafter#1\phfqit@val@sizearg}%
}
%    \end{macrocode}
% \end{macro}
% \end{macro}
%
%
% \subsection{Bra-Ket Notation}
%
%
% \begin{macro}{\phfqitKetsBarSpace}
% \begin{macro}{\phfqitKetsRLAngleSpace}
%   These macros can be redefined to fine-tune the space that is inserted in
%   some of the ket/bra constructs.
%
%   |\phfqitKetsBarSpace| is the space around the vertical bars, e.g., in a
%   bra-ket, or in matrix elements.  (Previously, this space was hard-coded to
%   |\hspace*{0.2ex}|.  Now, the spacing can be specified in this macro.  We
%   furthermore use ``math units'' (|mu|), which are more suitable for
%   specifying space in math mode; recall that $1\,\mathrm{mu}$ is
%   $1/18\,\mathrm{em}$ in the math font.)
%    \begin{macrocode}
\def\phfqitKetsBarSpace{\mkern 1.5mu\relax}
%    \end{macrocode}
% The macro |\phfqitKetsRLAngleSpace| specifies the space between the right
% angle bracket and the left angle bracket in ket-bra type constructs
% ($\ketbra\phi\psi$).  By default, it expands to negative space to bring the
% angle brackets closer together.  (Previously, this space was hard-coded to
% |\hspace*{-0.25ex}|.)
%    \begin{macrocode}
\def\phfqitKetsRLAngleSpace{\mkern -1.8mu\relax}
%    \end{macrocode}
% \end{macro}
% \end{macro}
%
% \needspace{8\baselineskip}
% \begin{macro}{\ket}
% \begin{macro}{\bra}
% \begin{macro}{\braket}
% \begin{macro}{\ketbra}
% \begin{macro}{\proj}
% \begin{macro}{\matrixel}
% \begin{macro}{\dmatrixel}
% \begin{macro}{\innerprod}
%   Bras, kets, norms, some delimiter stuff.  User documentation in
%   \autoref{sec:bra-ket}.  
%    \begin{macrocode}
\phfqitDeclarePairedDelimiterXWithAltSizing\ket[1]{\lvert}{\rangle}{{#1}}
\phfqitDeclarePairedDelimiterXWithAltSizing\bra[1]{\langle}{\rvert}{{#1}}
\phfqitDeclarePairedDelimiterXWithAltSizing\braket[2]{\langle}{\rangle}{%
  {#1}\phfqitKetsBarSpace\delimsize\vert\phfqitKetsBarSpace{#2}%
}
\phfqitDeclarePairedDelimiterXWithAltSizing\ketbra[2]{\lvert}{\rvert}{%
  {#1}\delimsize\rangle\phfqitKetsRLAngleSpace\delimsize\langle{#2}%
}
\phfqitDeclarePairedDelimiterXWithAltSizing\proj[1]{\lvert}{\rvert}{%
  {#1}\delimsize\rangle\phfqitKetsRLAngleSpace\delimsize\langle{#1}%
}
\phfqitDeclarePairedDelimiterXWithAltSizing\matrixel[3]{\langle}{\rangle}{%
  {#1}\phfqitKetsBarSpace\delimsize\vert\phfqitKetsBarSpace{#2}%
  \phfqitKetsBarSpace\delimsize\vert\phfqitKetsBarSpace{#3}%
}
\phfqitDeclarePairedDelimiterXWithAltSizing\dmatrixel[2]{\langle}{\rangle}{%
  {#1}\phfqitKetsBarSpace\delimsize\vert\phfqitKetsBarSpace{#2}%
  \phfqitKetsBarSpace\delimsize\vert\phfqitKetsBarSpace{#1}%
}
\def\phfqitKetsBeforeCommaSpace{}
\def\phfqitKetsAfterCommaSpace{\mkern 1.5mu\relax}
\phfqitDeclarePairedDelimiterXWithAltSizing\innerprod[2]{\langle}{\rangle}{%
  {#1}\phfqitKetsBeforeCommaSpace,\phfqitKetsAfterCommaSpace{#2}%
}
%    \end{macrocode}
% \end{macro}
% \end{macro}
% \end{macro}
% \end{macro}
% \end{macro}
% \end{macro}
% \end{macro}
% \end{macro}
%
%
% \begin{macro}{\phfqitOKetsBarSpace}
% \begin{macro}{\phfqitOKetsRLAngleSpace}
%   These macros are the same as |\phfqitKetsBarSpace| and
%   |\phfqitKetsRLAngleSpace|, except that they specify the corresponding
%   spacing for the |\oket| family of bra/ket-type constructs.
%    \begin{macrocode}
\def\phfqitOKetsBarSpace{\phfqitKetsBarSpace}
\def\phfqitOKetsRLAngleSpace{\phfqitKetsRLAngleSpace}
%    \end{macrocode}
% \end{macro}
% \end{macro}
%
%
% \needspace{8\baselineskip}
% \begin{macro}{\oket}
% \begin{macro}{\obra}
% \begin{macro}{\obraket}
% \begin{macro}{\oketbra}
% \begin{macro}{\oproj}
% \begin{macro}{\omatrixel}
% \begin{macro}{\odmatrixel}
%   Again Bras, kets, but for operator space this time.  User documentation in
%   \autoref{sec:bra-ket}.  These definitions depend on |\llangle| and
%   |\rrangle| being available and expanding to valid delimiter symbols.  See
%   also the |llanglefrommnsymbolfonts| option.
%
%   \changed[v4.1-okets-obras]{v4.1}{2021/10/08}{Added support for
%   \phfverb{\oket} and friends}
%    \begin{macrocode}
\phfqitDeclarePairedDelimiterXWithAltSizing\oket[1]{\lvert}{\rrangle}{{#1}}
\phfqitDeclarePairedDelimiterXWithAltSizing\obra[1]{\llangle}{\rvert}{{#1}}
\phfqitDeclarePairedDelimiterXWithAltSizing\obraket[2]{\llangle}{\rrangle}{%
  {#1}\phfqitOKetsBarSpace\delimsize\vert\phfqitOKetsBarSpace{#2}%
}
\phfqitDeclarePairedDelimiterXWithAltSizing\oketbra[2]{\lvert}{\rvert}{%
  {#1}\delimsize\rrangle\phfqitOKetsRLAngleSpace\delimsize\llangle{#2}%
}
\phfqitDeclarePairedDelimiterXWithAltSizing\oproj[1]{\lvert}{\rvert}{%
  {#1}\delimsize\rrangle\phfqitOKetsRLAngleSpace\delimsize\llangle{#1}%
}
\phfqitDeclarePairedDelimiterXWithAltSizing\omatrixel[3]{\llangle}{\rrangle}{%
  {#1}\phfqitOKetsBarSpace\delimsize\vert\phfqitOKetsBarSpace{#2}%
  \phfqitOKetsBarSpace\delimsize\vert\phfqitOKetsBarSpace{#3}%
}
\phfqitDeclarePairedDelimiterXWithAltSizing\odmatrixel[2]{\llangle}{\rrangle}{%
  {#1}\phfqitOKetsBarSpace\delimsize\vert\phfqitOKetsBarSpace{#2}%
  \phfqitOKetsBarSpace\delimsize\vert\phfqitOKetsBarSpace{#1}%
}
%    \end{macrocode}
% \end{macro}
% \end{macro}
% \end{macro}
% \end{macro}
% \end{macro}
% \end{macro}
% \end{macro}
%
%
% \subsection{Delimited Expressions}
% Delimited expressions are documented in \autoref{sec:delimited}.
% 
% \begin{macro}{\abs}
% \begin{macro}{\avg}
% \begin{macro}{\norm}
%   Other delimited expressions.
%    \begin{macrocode}
\phfqitDeclarePairedDelimiterXWithAltSizing\abs[1]{\lvert}{\rvert}{{#1}}
\phfqitDeclarePairedDelimiterXWithAltSizing\avg[1]{\langle}{\rangle}{{#1}}
\phfqitDeclarePairedDelimiterXWithAltSizing\norm[1]{\lVert}{\rVert}{{#1}}
%    \end{macrocode}
% \end{macro}
% \end{macro}
% \end{macro}
% 
% \begin{macro}{\phfqitDefineNorm}
%   Use |\phfqitDefineNorm\opnorm{}{_\infty}| to define specific norms, with the
%   syntax |\phfqitDefineNorm|\meta{new command name}\meta{tokens
%   before}\meta{tokens after}.  If you need arguments in the before/after
%   tokens, then your norm starts to be more complicated than what
%   |\phfqitDefineNorm| can handle, perhaps it's best you use
%   |\phfqitDeclarePairedDelimiterXPPWithAltSizing| directly.
%
%   \changed[v4.0-new-phfqitDefineNorm]{v4.0}{2021/10/07}{Added
%   \phfverb{\phfqitDefineNorm}}
%    \begin{macrocode}
\def\phfqitDefineNorm#1#2#3{%
  \phfqitDeclarePairedDelimiterXPPWithAltSizing#1[1]{#2}{\lVert}{\rVert}{#3}{{##1}}%
}
%    \end{macrocode}
% \end{macro}
% 
% \begin{macro}{\phfqit@insideinterval}
%   Format the contents of an interval.  Utility for defining |\intervalc| and
%   friends.
%    \begin{macrocode}
\def\phfqit@insideinterval#1#2{{#1\mathclose{},\mathopen{}#2}}
%    \end{macrocode}
% \end{macro}
%
% \needspace{4\baselineskip}
% \begin{macro}{\intervalc}
% \begin{macro}{\intervalo}
% \begin{macro}{\intervalco}
% \begin{macro}{\intervaloc}
% Open/Closed/Semi-Open Intervals
%    \begin{macrocode}
\phfqitDeclarePairedDelimiterXWithAltSizing\intervalc[2]{[}{]}{%
  \phfqit@insideinterval{#1}{#2}}
\phfqitDeclarePairedDelimiterXWithAltSizing\intervalo[2]{]}{[}{%
  \phfqit@insideinterval{#1}{#2}}
\phfqitDeclarePairedDelimiterXWithAltSizing\intervalco[2]{[}{[}{%
  \phfqit@insideinterval{#1}{#2}}
\phfqitDeclarePairedDelimiterXWithAltSizing\intervaloc[2]{]}{]}{%
  \phfqit@insideinterval{#1}{#2}}
%    \end{macrocode}
% \end{macro}
% \end{macro}
% \end{macro}
% \end{macro}
%
%
%
%
% \subsection{Entropy Measures and Other Qit Objects}
%
% \changed[v2.0-qit-objects]{v2.0}{2017/06/17}{Introduced the Qit Objects infrastructure}
%
%
% \subsubsection{Some Internal Utilities}
%
% \begin{macro}{\phfqitParens}
%   Simple parenthesis-delimited expression, with
%   |\DeclarePairedDelimiterX|-compatible syntax.  For example,
%   \par |\phfqitParens|\marg{content} \quad$\to$\quad
%   \fbox{\phfverb( \meta{content} \phfverb)}
%   \par |\phfqitParens*|\marg{content} \quad$\to$\quad
%   \fbox{\phfverb\left\phfverb( \meta{content} \phfverb\right\phfverb)}
%   \par |\phfqitParens[\big]|\marg{content} \quad$\to$\quad
%   \fbox{\phfverb\bigl\phfverb( \meta{content} \phfverb\bigr\phfverb)}
%
%    \begin{macrocode}
\DeclarePairedDelimiterX\phfqitParens[1]{(}{)}{#1}
%    \end{macrocode}
% \end{macro}
%
% \begin{macro}{\phfqitSquareBrackets}
%   Simple bracket-delimited expression, with
%   |\DeclarePairedDelimiterX|-compatible syntax.  For example,
%   \par |\phfqitSquareBrackets|\marg{content} \quad$\to$\quad
%   \fbox{\phfverb[ \meta{content} \phfverb]}
%   \par |\phfqitSquareBrackets*|\marg{content} \quad$\to$\quad
%   \fbox{\phfverb\left\phfverb[ \meta{content} \phfverb\right\phfverb]}
%   \par |\phfqitSquareBrackets[\big]|\marg{content} \quad$\to$\quad
%   \fbox{\phfverb\bigl\phfverb[ \meta{content} \phfverb\bigr\phfverb]}
%
%    \begin{macrocode}
\DeclarePairedDelimiterX\phfqitSquareBrackets[1]{[}{]}{#1}
%    \end{macrocode}
% \end{macro}
%
% \begin{macro}{\phfqitCurlyBrackets}
%   Simple bracket-delimited expression, with
%   |\DeclarePairedDelimiterX|-compatible syntax.  For example,
%   \par |\phfqitSquareBrackets|\marg{content} \quad$\to$\quad
%   \fbox{\phfverb\{ \meta{content} \phfverb\}}
%   \par |\phfqitSquareBrackets*|\marg{content} \quad$\to$\quad
%   \fbox{\phfverb\left\phfverb\{ \meta{content} \phfverb\right\phfverb\}}
%   \par |\phfqitSquareBrackets[\big]|\marg{content} \quad$\to$\quad
%   \fbox{\phfverb\bigl\phfverb\{ \meta{content} \phfverb\bigr\phfverb\}}
%
%    \begin{macrocode}
\DeclarePairedDelimiterX\phfqitCurlyBrackets[1]{\{}{\}}{#1}
%    \end{macrocode}
% \end{macro}
%
%
%
% \subsubsection{Machinery for Qit Objects}
%
% See also user documentation in \autoref{sec:QitObjectImpl}.
%
% \begin{macro}{\QitObject}
%   The argument is the entropic quantity type or object kind (or ``entropic
%   quantity driver''): one of |Hbase|, |Hfnbase|, |Dbase|, |DCbase|, or any
%   other custom object.
%    \begin{macrocode}
\newcommand\QitObject[1]{%
  \begingroup%
    \preto\QitObjectDone{\endgroup}%
    \QitObjectInit%
    \csname QitObj@reg@#1@initdefs\endcsname%
%%\message{DEBUG: \detokenize{\QitObject{#1}}}%
    \def\QitObj@args{}%
    \def\qitobjParseDone{\QitObj@proceedToRender{#1}}%
    \def\qitobjDone{\QitObjectDone}%
    \csname QitObj@reg@#1@parse\endcsname%
}
%    \end{macrocode}
% \end{macro}
%
%
% \begin{macro}{\DefineQitObject}
% \begin{macro}{\DefineTunedQitObject}
%   Define a new Qit Object implementation with this macro.  A Qit Object
%   implementation is specified in its simplest form by a \emph{name}, a
%   \emph{Parser} and a \emph{Renderer} (a single \LaTeX{} macro each).  The
%   more advanced |\DefineTunedQitObject| allows you in addition to specify
%   local definitions to override defaults, as well as some initial arguments to
%   the parser.
%    \begin{macrocode}
\def\DefineQitObject#1#2#3{%
  \DefineTunedQitObject{#1}{#2}{#3}{}{}%
}%
\def\DefineTunedQitObject#1#2#3#4#5{%
  \csdef{#1}{\QitObject{#1}#4}%
  \expandafter\robustify\csname #1\endcsname%
  \cslet{QitObj@reg@#1@parse}#2%
  \cslet{QitObj@reg@#1@render}#3%
  \csdef{QitObj@reg@#1@initdefs}{#5}%
}
%    \end{macrocode}
% \end{macro}
% \end{macro}
%
%
% Here are some callbacks meant for Qit Object implementations
% (``types''/``drivers'').
% 
% \begin{macro}{\qitobjAddArg}
% \begin{macro}{\qitobjAddArgx}
%   These macros should only be called from within a \emph{Parse} macro of a qit
%   object type.  Append an argument in preparation for an eventual call to the
%   corresponding \emph{Render} macro.  |\qitobjAddArg| does not expand its
%   contents. |\qitobjAddArgx| expects a single command name as argument; it
%   expands the command once and stores those tokens as a single new argument.
%    \begin{macrocode}
\def\qitobjAddArg#1{%
  \appto\QitObj@args{{#1}}%
}
\def\qitobjAddArgx#1{%
  \expandafter\qitobjAddArg\expandafter{#1}%
}
%    \end{macrocode}
% \end{macro}
% \end{macro}
% 
% \begin{macro}{\qitobjParseDone}
% \begin{macro}{\qitobjDone}
%   These macros MUST be called at the end of the respective \emph{Parse}
%   (|\qitobjParseDone|) and \emph{Render} (|\qitobjDone|) implementations
%   (otherwise processing doesn't proceed, \LaTeX{} groups won't be closed, and
%   it will be a mess).
%
%   These macros are correctly defined in |\QitObject| actually. Here we provide
%   empty definitions so that the \emph{Render} and \emph{Parse} user
%   implementation macros can be called stand-alone, too.
%    \begin{macrocode}
\def\qitobjParseDone{}
\def\qitobjDone{}
%    \end{macrocode}
% \end{macro}
% \end{macro}
% 
% \begin{macro}{\QitObjectDone}
%   A hook which gets called after a Qit Object is displayed.  This should
%   really stay empty on the global scope.  However you can locally append or
%   prepend to it in tuned definitions for |\DeclareTunedQitObject| to perform
%   additional actions at the end of the Qit Object, for instance to close an
%   additional \LaTeX{} group.
%    \begin{macrocode}
\def\QitObjectDone{}
%    \end{macrocode}
% \end{macro}
%
% \begin{macro}{\QitObjectInit}
%   A hook which gets called before the parsing phase of a Qit Object.  This
%   should really stay empty on the global scope.  However you can locally
%   append or prepend to it in tuned definitions for |\DeclareTunedQitObject| to
%   perform additional actions before parsing the Qit Object (but which have to
%   be made within the \LaTeX{} group of the Qit Object).  You can use this to
%   prepend code to |\QitObjectDone| so that you code gets called \emph{before}
%   the inner \LaTeX{} group is closed.
%    \begin{macrocode}
\def\QitObjectInit{}
%    \end{macrocode}
% \end{macro}
% 
% An internal helper; it's useful to keep it separate for readability and for
% debugging.
%    \begin{macrocode}
\def\QitObj@proceedToRender#1{%
%%\message{DEBUG: Rendering #1|\detokenize\expandafter{\QitObj@args}|}%
  \expandafter\def\expandafter\x\expandafter{%
    \csname QitObj@reg@#1@render\endcsname}%
  \expandafter\x\QitObj@args%
}
%    \end{macrocode}
% 
%
% \subsubsection{Qit Object Implementation: Entropy, Conditional Entropy}
%
% See also the user doc in \autoref{sec:entropy-measures}.
%
% \begin{macro}{\HbaseParse}
%   Base parser macro for usual entropy measures; possibly conditional and/or
%   smooth.
%
%   USAGE:
%   |\Hbase|\marg{H-symbol}\hspace{0pt}\relax
%   \marg{subscript}\hspace{0pt}\relax
%   \meta{size-spec}\hspace{0pt}\oarg{state}\relax
%   \hspace{0pt}\oarg{epsilon}\hspace{0pt}\marg{target system}\hspace{0pt}\relax
%   \oarg{conditioning system}
%
%   The argument \meta{size-spec} is optional, and is documented in
%   \autoref{topic:size-specification-backtick}.  For example \meta{size-spec} =
%   |`*| or |`\Big|.
%
%   Examples: 
%   \par |\Hbase{\hat{H}}{\mathrm{max}}[\rho][\epsilon]{E}[X']|
%   \quad$\to$\quad
%   \fbox{$\Hbase{\hat{H}}{\mathrm{max}}[\rho][\epsilon]{E}[X']$}
%   \par |\Hbase{\hat{H}}{\mathrm{max}}`*[\rho][\epsilon]{\bigotimes_i E}[X']|
%   \quad$\to$\quad
%   \fbox{$\Hbase{\hat{H}}{\mathrm{max}}`*[\rho][\epsilon]{\displaystyle\bigotimes_i E}[X']$}
% 
%   The |\HbaseParse| macro is responsible for parsing the arguments to
%   |\Hbase|.  We should parse the arguments using helper macros as needed,
%   adding rendering arguments with |\qitobjAddArg| or |\qitobjAddArgx|, and
%   then calling |\qitobjParseDone|.  The arguments are then automatically
%   provided as arguments to the |\HbaseRender| function.  We just have to make
%   sure we add the correct number of arguments in the correct order.
%   
%    \begin{macrocode}
\def\HbaseParse#1#2{%
%    \end{macrocode}
%
% The first arguments are the mandatory arguments
% \marg{H-symbol}\hspace{0pt}\marg{subscript}.  Then defer to helper macros for
% the rest of the parsing.
%    \begin{macrocode}
  \qitobjAddArg{#1}%
  \qitobjAddArg{#2}%
  \phfqit@parsesizearg\HbaseParse@%
}
%    \end{macrocode}
% 
% Store the delimiter size argument which |\phfqit@parsesizearg| has stored into
% |\phfqit@val@sizearg|, then parse an optional \oarg{state} argument.
%    \begin{macrocode}
\newcommand\HbaseParse@[1][]{%
  \qitobjAddArgx{\phfqit@val@sizearg}%
  \qitobjAddArg{#1}%
  \HbaseParse@@%
}
%    \end{macrocode}
% Then parse an optional \oarg{epsilon} argument, as well as a mandatory
% \marg{target system} argument.
%    \begin{macrocode}
\newcommand\HbaseParse@@[2][]{%
  \qitobjAddArg{#1}%
  \qitobjAddArg{#2}%
  \HbaseParse@@@%
}
%    \end{macrocode}
% Finally, parse an optional \oarg{conditioning system}.
%    \begin{macrocode}
\newcommand\HbaseParse@@@[1][]{%
  \qitobjAddArg{#1}%
  \qitobjParseDone%
}
%    \end{macrocode}
% \end{macro}
% 
%
% \begin{macro}{\HbaseRender}
%   Render the entropy measure.
%   \par |#1| = ``$H$'' symbol to use (e.g. |H|)
%   \par |#2| = subscript (type of entropy, e.g. |\marthrm{min},0|)
%   \par |#3| = possible size argument to expand in front of parens command (one
%   of \emph{(empty)}, |*|, or |[\big]| using a standard sizing command)
%   \par |#4| = the state (e.g. |\rho|), may be left empty
%   \par |#5| = epsilon argument (superscript to entropy measure), if any, or
%   leave argument empty
%   \par |#6| = system to measure entropy of
%   \par |#7| = conditioning system, if any, or else leave the argument empty
%    \begin{macrocode}
\def\HbaseRender#1#2#3#4#5#6#7{%
%%\message{DEBUG: HbaseRender\detokenize{{#1}{#2}{#3}{#4}{#5}{#6}{#7}}}%
%    \end{macrocode}
%
%   Start with the entropy symbol (`H'), the subscript, and the superscript:
%    \begin{macrocode}
  \HbaseRenderSym{#1}_{\HbaseRenderSub{#2}}^{\HbaseRenderSup{#5}}
%    \end{macrocode}
% Render the contents of the entropy (parenthetic expression with system \&
% conditioning system), only if the system or conditioning system or state are
% not empty:
%    \begin{macrocode}
  \notblank{#4#6#7}{%
    \HbaseRenderContents{#3}{#6}{#7}%
%    \end{macrocode}
% Finally, add the state as subscript, if any:
%    \begin{macrocode}
    \HbaseRenderTail{#4}%
  }{}%
%    \end{macrocode}
% We're done.
%    \begin{macrocode}
  \qitobjDone%
}
%    \end{macrocode}
% \end{macro}
%
% \needspace{5\baselineskip}
% \begin{macro}{\HbaseRenderSym}
% \begin{macro}{\HbaseRenderSub}
% \begin{macro}{\HbaseRenderSup}
% \begin{macro}{\HbaseRenderTail}
%   Macros to render different parts of the entropy measure.  By default, don't
%   do anything special to them (but this might be locally overridden in a tuned
%   Qit Object, for instance).
%    \begin{macrocode}
\def\HbaseRenderSym#1{#1}%
\def\HbaseRenderSub#1{#1}%
\def\HbaseRenderSup#1{#1}%
\def\HbaseRenderTail#1{_{#1}}%
%    \end{macrocode}
% \end{macro}
% \end{macro}
% \end{macro}
% \end{macro}
%
% \begin{macro}{\HbaseRenderContents}
% For the main contents rendering macro, we need to do a little more work.
% First, declare a token register in which we will prepare the contents of the
% parenthetic expression.
%    \begin{macrocode}
\newtoks\Hbase@tmp@toks
\def\Hbase@addtoks#1\@Hbase@END@ADD@TOKS{%
  \Hbase@tmp@toks=\expandafter{\the\Hbase@tmp@toks#1}}%
%    \end{macrocode}
% Now we need to define the macro which formats the contents of the entropy.
% The arguments are |#1| = possible sizing argument, |#2| = system name, |#3| =
% conditioning system if any.
%    \begin{macrocode}
\def\HbaseRenderContents#1#2#3{%
%    \end{macrocode}
% We need to construct the parenthetic argument to the entropy, which we will
% store in the token register |\Hbase@tmp@toks|.  Start with system name:
%    \begin{macrocode}
    \Hbase@tmp@toks={#2}%
%    \end{macrocode}
% \ldots{} add conditional system, if specified:
%    \begin{macrocode}
    \notblank{#3}{%
      \Hbase@addtoks\mathclose{}\,\delimsize\vert\,\mathopen{}%
          #3%
          \@Hbase@END@ADD@TOKS%
    }{}%
%    \end{macrocode}
% The tokens are ready now. Prepare the argument to the command
% |\HbaseRenderContentsInnerParens| (normally just |\phfqitParens|), and go:
%    \begin{macrocode}
    \edef\tmp@args{\unexpanded{#1}{\the\Hbase@tmp@toks}}%
    \expandafter\HbaseRenderContentsInnerParens\tmp@args%
}
%    \end{macrocode}
% \end{macro}
%
% \begin{macro}{\X}
%   Macro which expands to the parenthetic expression type macro we would like
%   to use. By default, this is |\phfqitParens|.
%    \begin{macrocode}
\def\HbaseRenderContentsInnerParens{\phfqitParens}
%    \end{macrocode}
% \end{macro}
%
%
% \begin{macro}{\Hbase}
%   Finally, we declare our base entropic quantity type:
%    \begin{macrocode}
\DefineQitObject{Hbase}{\HbaseParse}{\HbaseRender}
%    \end{macrocode}
% \end{macro}
%
%
% \subsubsection{Qit Object Implementation: Entropy Function}
%
% See also the user doc in \autoref{sec:entropy-function}.
%
% \begin{macro}{\Hfnbase}
%   Base implementation of an entropy function.
%
%   Usage: |\Hfnbase{H}{1}{2}(x)| $\to$ $\Hfnbase{H}{1}{2}(x)$,
%   |\Hfnbase{H}{1}{2}`*(x)| $\to$ $\Hfnbase{H}{1}{2}`*(x)$,
%   |\Hfnbase{H}{1}{2}`\big(x)| $\to$ $\Hfnbase{H}{1}{2}`\big(x)$.
%
%   We can use the same renderer as |\Hbase|, we just need a different parser.
%   The parser first accepts the mandatory arguments
%   \marg{H-symbol}\hspace{0pt}\marg{subscript}\hspace{0pt}\marg{superscript}.
%    \begin{macrocode}
\def\HfnbaseParse#1#2#3{%
  \qitobjAddArg{#1}% H-sym
  \qitobjAddArg{#2}% sub
  \phfqit@parsesizearg{\HfnbaseParse@{#3}}%
}
%    \end{macrocode}
% 
% Continue to parse a the argument given in parentheses.  The first mandatory
% argument is simply the subscript passed on from the previous macro.  It might
% be tempting to do simply |\def\HfnbaseParse@#1(#2){...}|, but this does not
% allow for recursive use of parenthesis within the entropy argument, for
% instance |\Hfn(g(x)+h(y))|.  Because of this, we use \pkgname{xparse}'s
% |\NewDocumentCommand| which can handle this.
%    \begin{macrocode}
\NewDocumentCommand{\HfnbaseParse@}{mr()}{%
  \qitobjAddArgx{\phfqit@val@sizearg}% size-arg
  \qitobjAddArg{}% state
  \qitobjAddArg{#1}% epsilon
  \qitobjAddArg{#2}% system--main arg
  \qitobjAddArg{}% cond system
%%\message{DEBUG: Hfnbase args are |\detokenize\expandafter{\QitObj@args}|}%
  \qitobjParseDone%
}
\DefineQitObject{Hfnbase}{\HfnbaseParse}{\HbaseRender}
%    \end{macrocode}
% \end{macro}
% 
%
% \subsubsection{Qit Object Implementation: Relative Entropy}
%
% User documentation in \autoref{sec:relative-entropies}.
%
%
% \begin{macro}{\DbaseParse}
%   Base macro for relative entropy macros.
%
%   USAGE:
%   |\Dbase|\marg{D-symbol}\hspace{0pt}\relax
%   [|_|\meta{subscript}]\hspace{0pt}\relax
%   [|^|\meta{superscript}]\hspace{0pt}\relax
%   \meta{size-spec}\hspace{0pt}\marg{state}\hspace{0pt}\marg{relative to state}
%
%   The subscript and superscripts are optional and don't have to be specified.
%   They may be specified in any order.  Repetitions are allowed and
%   concatenates the arguments, e.g., |^{a}_{x}_{y}^{z}_{w}| is the same as
%   |_{xyw}^{az}|.
%   
%   The \meta{size-spec} is a backtick-style specification as always.
%
%    \begin{macrocode}
\def\DbaseParse#1{%
  \qitobjAddArg{#1}% D-sym
  \def\DbaseParse@val@sub{}%
  \def\DbaseParse@val@sup{}%
  \DbaseParse@%
}
\def\DbaseParse@{%
  \@ifnextchar_{\DbaseParse@parsesub}{\DbaseParse@@}%
}
\def\DbaseParse@@{%
  \@ifnextchar^{\DbaseParse@parsesup}{\DbaseParse@@@}%
}
\def\DbaseParse@parsesub_#1{%
  \appto\DbaseParse@val@sub{#1}%
  \DbaseParse@% return to maybe parsing other sub/superscripts
}
\def\DbaseParse@parsesup^#1{%
  \appto\DbaseParse@val@sup{#1}%
  \DbaseParse@% return to maybe parsing other sub/superscripts
}
\def\DbaseParse@@@{%
  \qitobjAddArgx\DbaseParse@val@sub%
  \qitobjAddArgx\DbaseParse@val@sup%
  \phfqit@parsesizearg\DbaseParse@rest%
}
\def\DbaseParse@rest#1#2{%
  \qitobjAddArgx\phfqit@val@sizearg%
  \qitobjAddArg{#1}% rho
  \qitobjAddArg{#2}% Gamma
  \qitobjParseDone%
}
%    \end{macrocode}
% \end{macro}
%
%
%
%
%
% \begin{macro}{\DbaseRender}
%   Macro which formats a relative entropy of the form
%   $D_\mathrm{sub}^\mathrm{sup}(A\Vert B)$:
%   \par |\DbaseRender{D}{\mathrm{min}}{\epsilon}{[\big]}{\rho}{\Gamma}|
%   \quad$\to$\quad
%   \fbox{$\DbaseRender{D}{\mathrm{min}}{\epsilon}{[\big]}{\rho}{\Gamma}$}
%
%    \begin{macrocode}
\def\DbaseRender#1#2#3#4#5#6{%
%%\message{DEBUG: DbaseRender\detokenize{{#1}{#2}{#3}{#4}{#5}{#6}}}%
%    \end{macrocode}
%
%   Start with the entropy symbol (`H'), the subscript, and the superscript:
%    \begin{macrocode}
  \DbaseRenderSym{#1}_{\DbaseRenderSub{#2}}^{\DbaseRenderSup{#3}}
%    \end{macrocode}
% Render the contents of the entropy (parenthetic expression with the (one or)
% two states), only if the arguments are non-empty:
%    \begin{macrocode}
  \notblank{#5#6}{%
    \DbaseRenderContents{#4}{#5}{#6}%
  }{}%
%    \end{macrocode}
% We're done.
%    \begin{macrocode}
  \qitobjDone%
}
%    \end{macrocode}
% \end{macro}
%
% \needspace{5\baselineskip}
% \begin{macro}{\DbaseRenderSym}
% \begin{macro}{\DbaseRenderSub}
% \begin{macro}{\DbaseRenderSup}
%   Macros to render different parts of the entropy measure.  By default, don't
%   do anything special to them (but this might be locally overridden in a
%   tuned Qit Object).
%    \begin{macrocode}
\def\DbaseRenderSym#1{#1}%
\def\DbaseRenderSub#1{#1}%
\def\DbaseRenderSup#1{#1}%
%    \end{macrocode}
% \end{macro}
% \end{macro}
% \end{macro}
% 
% \begin{macro}{\DbaseRenderContents}
% Now we need to define the macro which formats the contents of the entropy.
% First, define a useful token register.
%    \begin{macrocode}
\newtoks\Dbase@tmp@toks
\def\Dbase@addtoks#1\@Dbase@END@ADD@TOKS{%
  \Dbase@tmp@toks=\expandafter{\the\Dbase@tmp@toks#1}}%
%    \end{macrocode}
%
% The arguments are |#1| = possible sizing argument, |#2| = first state, |#3| =
% second state (or operator), if any.
%    \begin{macrocode}
\def\DbaseRenderContents#1#2#3{%
%    \end{macrocode}
% We need to construct the parenthetic argument to the relative entropy, which
% we will store in the token register |\Dbase@tmp@toks|.  Start with system
% name:
%    \begin{macrocode}
    \Dbase@tmp@toks={#2}%
%    \end{macrocode}
% \ldots{} add conditional system, if specified:
%    \begin{macrocode}
    \notblank{#3}{%
      \Dbase@addtoks\mathclose{}\,\delimsize\Vert\,\mathopen{}%
          #3%
          \@Dbase@END@ADD@TOKS%
    }{}%
%    \end{macrocode}
% The tokens are ready now. Prepare the argument to the command
% |\DbaseRenderContentsInnerParens| (by default just |\phfqitParens|), and go:
%    \begin{macrocode}
    \edef\tmp@args{\unexpanded{#1}{\the\Dbase@tmp@toks}}%
    \expandafter\DbaseRenderContentsInnerParens\tmp@args%
}
%    \end{macrocode}
% \end{macro}
%
% \begin{macro}{\DbaseRenderContentsInnerParens}
%   Macro which expands to the parenthetic expression type macro we would like
%   to use. By default, this is |\phfqitParens|.
%    \begin{macrocode}
\def\DbaseRenderContentsInnerParens{\phfqitParens}
%    \end{macrocode}
% \end{macro}
%
% \begin{macro}{\Dbase}
%   Finally, define the |\Dbase| macro by declaring a new qit object.
%    \begin{macrocode}
\DefineQitObject{Dbase}{\DbaseParse}{\DbaseRender}
%    \end{macrocode}
% \end{macro}
% 
%
% \subsubsection{Qit Object Type: Coherent Relative Entropy}
% 
% See also user documentation in \autoref{sec:coh-rel-entr}.
%
% \begin{macro}{\DCohbaseParse}
%   Base macros for coherent relative entropy-type quantities of the form
%   ${\bar D}_{X\to X'}^{\epsilon}(\rho_{X'R}\Vert\Gamma_X,\Gamma_{X'})$.
%
%   USAGE:
%   |\DCohbase|\marg{D symbol}\hspace{0pt}\relax
%   \oarg{epsilon}\hspace{0pt}\relax
%   \marg{state or \texttt{\textup{*}}fully-decorated-state}\hspace{0pt}\relax
%   \marg{System In}\hspace{0pt}\relax
%   \marg{System Out}\hspace{0pt}\relax
%   \marg{Gamma In}\hspace{0pt}\relax
%   \marg{Gamma Out}
%
%    \begin{macrocode}
\def\DCohbaseParse#1{%
  \qitobjAddArg{#1}% D-sym
  \DCohbaseParse@%
}
\newcommand\DCohbaseParse@[1][]{%
  \qitobjAddArg{#1}% epsilon
  \phfqit@parsesizearg\DCohbaseParse@rest%
}
\def\DCohbaseParse@rest#1#2#3#4#5{%
  % rho, X, X', \Gamma_X, \Gamma_{X'}
  \qitobjAddArgx\phfqit@val@sizearg%
  \DCohbaseParse@parserhosub#1\DCohbaseParse@ENDSTATE{#2}{#3}%
  \qitobjAddArg{#2}%
  \qitobjAddArg{#3}%
  \qitobjAddArg{#4}%
  \qitobjAddArg{#5}%
  \qitobjParseDone%
}
\def\DCohbaseParse@parserhosub{%
  \@ifnextchar*\DCohbaseParse@parserhosub@nosub%
  \DCohbaseParse@parserhosub@wsub%
}
\def\DCohbaseParse@parserhosub@nosub*#1\DCohbaseParse@ENDSTATE#2#3{%
  \qitobjAddArg{#1}% rho
}
\def\DCohbaseParse@parserhosub@wsub#1\DCohbaseParse@ENDSTATE#2#3{%
  \qitobjAddArg{#1_{\begingroup\let\emptysystem\relax%
      \DCohbaseStateSubscripts{#2}{#3}\endgroup}}% all this for "rho" arg
}
%    \end{macrocode}
% \end{macro}
%
% \begin{macro}{\DCohbaseStateSubscripts}
%   Macro which produces the relevant subscript for the state.  By default,
%   simply produce ``$X'R$'' (but don't produce an ``empty system''
%   symbol). This macro may be overridden e.g. locally.
%    \begin{macrocode}
\def\DCohbaseStateSubscripts#1#2{%
  #2#1%
}
%    \end{macrocode}
% \end{macro}
% 
% \begin{macro}{\DCohbaseRender}
%   Render the coherent relative entropy.
%   \par |#1| = ``$D$'' symbol
%   \par |#2| = superscript (epsilon)
%   \par |#3| = possible size argument tokens (i.e., |[\big]|)
%   \par |#4| = fully decorated state (i.e., with necessary subscripts as required)
%   \par |#5| = input system name
%   \par |#6| = output system name
%   \par |#7| = Gamma-in
%   \par |#8| = Gamma-out
%    \begin{macrocode}
\def\DCohbaseRender#1#2#3#4#5#6#7#8{%
  % 
%%\message{DEBUG: DCohbaseRender here, args are |\detokenize{{#1}{#2}{#3}{#4}{#5}{#6}{#7}{#8}|.}}
  %
  \DCohbaseRenderSym{#1}%
  _{\DCohbaseRenderSystems{#5}{#6}}%
  ^{\DCohbaseRenderSup{#2}}%
  \notblank{#4#7#8}{%
    \DCohbaseRenderContents{#3}{#4}{#7}{#8}%
  }{}%
%    \end{macrocode}
% We're done.
%    \begin{macrocode}
  \qitobjDone%
}
%    \end{macrocode}
% \end{macro}
%
% \needspace{5\baselineskip}
% \begin{macro}{\DCohbaseRenderSym}
% \begin{macro}{\DCohbaseRenderSystems}
% \begin{macro}{\DCohbaseRenderSup}
%   Macros to render different parts of the entropy measure.  By default, don't
%   do anything special to them (but this might be locally overridden in a
%   tuned Qit Object)
%    \begin{macrocode}
\def\DCohbaseRenderSym#1{#1}%
\def\DCohbaseRenderSystems#1#2{#1\to #2}%
\def\DCohbaseRenderSup#1{#1}%
%    \end{macrocode}
% \end{macro}
% \end{macro}
% \end{macro}
%
% \begin{macro}{\DCohbaseRenderContents}
%   Now we define the macro which formats the contents of the entropy.
%
%   Define first a useful token register for rendering the contents.
%    \begin{macrocode}
\newtoks\DCohbase@tmp@toks
\def\DCohbase@addtoks#1\@DCohbase@END@ADD@TOKS{%
  \DCohbase@tmp@toks=\expandafter{\the\DCohbase@tmp@toks#1}}%
%    \end{macrocode}
%
% The arguments are |#1| = possible sizing argument tokens, |#2| = decorated
% state, |#3| = Gamma-X, |#4| = Gamma-{X'}.
%    \begin{macrocode}
\def\DCohbaseRenderContents#1#2#3#4{%
%    \end{macrocode}
% We need to construct the parenthetic argument to the coherent relative
% entropy, which we will prepare in the token register |\DCohbase@tmp@toks|.
% Start with the state:
%    \begin{macrocode}
    \DCohbase@tmp@toks={#2}%
%    \end{macrocode}
% \ldots{} add conditional system, if specified:
%    \begin{macrocode}
    \notblank{#3}{%
      \DCohbase@addtoks\mathclose{}\,\delimsize\Vert\,\mathopen{}%
          #3\@DCohbase@END@ADD@TOKS%
        \notblank{#4}{%
          \DCohbase@addtoks\mathclose{},\mathopen{}%
              #4\@DCohbase@END@ADD@TOKS%
        }{}%
    }{%
        \notblank{#4}{%
          \PackageWarning{phfqit}{Value `#4' ignored because previous parameter
            was blank}%
        }{}%
      }
%    \end{macrocode}
% The tokens are ready now. Prepare the argument to the command
% |\DCohbaseRenderContentsInnerParens| (by default just |\phfqitParens|), and go:
%    \begin{macrocode}
    \edef\tmp@args{\unexpanded{#1}{\the\DCohbase@tmp@toks}}%
    \expandafter\DCohbaseRenderContentsInnerParens\tmp@args%
}
%    \end{macrocode}
% \end{macro}
% 
% \begin{macro}{\DCohbaseRenderContentsInnerParens}
%   Macro which expands to the parenthetic expression type macro we would like
%   to use.  By default, this is |\phfqitParens|.
%    \begin{macrocode}
\def\DCohbaseRenderContentsInnerParens{\phfqitParens}
%    \end{macrocode}
% \end{macro}
%
% \begin{macro}{\DCohbase}
%   Finally, define the |\DCohbase| macro by declaring a new qit object.
%    \begin{macrocode}
\DefineQitObject{DCohbase}{\DCohbaseParse}{\DCohbaseRender}
%    \end{macrocode}
% \end{macro}
% 
%
% \subsection{Additional helpers for entropy measures}
%
% \begin{macro}{\HSym}
%   Symbol to use to denote an entropy measure.
%    \begin{macrocode}
\def\HSym{H}
%    \end{macrocode}
% \end{macro}
%
% \begin{macro}{\DSym}
%   Symbol to use to denote a relative entropy measure.
%    \begin{macrocode}
\newcommand\DSym{D}
%    \end{macrocode}
% \end{macro}
%
% \begin{macro}{\DCSym}
%   Symbol to use for the coherent relative entropy measure.
%    \begin{macrocode}
\newcommand\DCSym{\bar\DSym}
%    \end{macrocode}
% \end{macro}
%
% \begin{macro}{\emptysystem}
%   Designates the trivial system (uses symbol for empty set).  It is important
%   to this, because of the automatic indexes set on the ``rho'' argument.
%    \begin{macrocode}
\def\emptysystem{\ensuremath{\emptyset}}
%    \end{macrocode}
% \end{macro}
% 
% \begin{macro}{\DCohxRefSystemName}
% \begin{macro}{\DCohxStateSubscripts}
%   Macros helpful for defining |\DCohx|.
%    \begin{macrocode}
\def\DCohxRefSystemName#1{R_{#1}}
\def\DCohxStateSubscripts#1#2{#2\DCohxRefSystemName{#1}}
%    \end{macrocode}
% \end{macro}
% \end{macro}
%
%
% Finally, some macros provided for backwards compatibility:
%    \begin{macrocode}
\let\@HHbase\Hbase
\let\@DDbase\Dbase
\let\HHSym\HSym
\let\DDSym\DSym
%    \end{macrocode}
%
%
% \subsection{Handle package options}
%
% \changedreftext{v2.0-pkg-opt-qitobjdef}
% \changedreftext{v2.0-pkg-opt-newReIm}
%
% Initialization code for \pkgname{kvoptions} for our package options.  See
% \autoref{sec:pkg-options}.
%    \begin{macrocode}
\SetupKeyvalOptions{
   family=phfqit,
   prefix=phfqit@opt@
}
%    \end{macrocode}
% 
% Set of predefined qit objects to load. Either |stdset| (standard set, the
% default) or |none| (none).
%    \begin{macrocode}
\DeclareStringOption[stdset]{qitobjdef}
%    \end{macrocode}
%
% Whether we should load the |\llangle| and |\rrangle| delimiters from the
% MnSymbol fonts.
%    \begin{macrocode}
\DeclareBoolOption[true]{llanglefrommnsymbolfonts}
%    \end{macrocode}
%
% Whether to override \LaTeX{}'s default {\makeatletter $\phfqit@Re$ and
% $\phfqit@Im$} symbols by our more readable $\Re$ and $\Im$.
%    \begin{macrocode}
\DeclareBoolOption[true]{newReIm}
%    \end{macrocode}
%
% 
% Process package options.
%    \begin{macrocode}
\ProcessKeyvalOptions*
%    \end{macrocode}
%
%
% \subsubsection{Re/Im symbols}
%
% \begin{macro}{\Re}
% \begin{macro}{\Im}
%   Provide |\Re| and |\Im| commands to override \LaTeX{}'s default if the
%   corresponding package option is set (which is the default).
%    \begin{macrocode}
\ifphfqit@opt@newReIm
  \renewcommand{\Re}{\phfqit@Realpart}
  \renewcommand{\Im}{\phfqit@Imagpart}
\fi
%    \end{macrocode}
% \end{macro}
% \end{macro}
%
% \subsubsection{Load \phfverb{\llangle} and \phfverb{\rrangle} from the
% \emph{MnSymbol} fonts}
%
% We need to import |\llangle| and |\rrangle| from the |MnSymbol|
% fonts.\footnote{see e.g. \url{https://tex.stackexchange.com/a/79701/32188}}
%
%    \begin{macrocode}
\ifphfqit@opt@llanglefrommnsymbolfonts
  \DeclareFontFamily{OMX}{MnSymbolE}{}
  \DeclareSymbolFont{phfqit@MnLargeSymbols}{OMX}{MnSymbolE}{m}{n}
  \SetSymbolFont{phfqit@MnLargeSymbols}{bold}{OMX}{MnSymbolE}{b}{n}
  \DeclareFontShape{OMX}{MnSymbolE}{m}{n}{
      <-6>  MnSymbolE5
     <6-7>  MnSymbolE6
     <7-8>  MnSymbolE7
     <8-9>  MnSymbolE8
     <9-10> MnSymbolE9
    <10-12> MnSymbolE10
    <12->   MnSymbolE12
  }{}
  \DeclareFontShape{OMX}{MnSymbolE}{b}{n}{
      <-6>  MnSymbolE-Bold5
     <6-7>  MnSymbolE-Bold6
     <7-8>  MnSymbolE-Bold7
     <8-9>  MnSymbolE-Bold8
     <9-10> MnSymbolE-Bold9
    <10-12> MnSymbolE-Bold10
    <12->   MnSymbolE-Bold12
  }{}
  \let\llangle\@undefined
  \let\rrangle\@undefined
  \DeclareMathDelimiter{\llangle}{\mathopen}%
                       {phfqit@MnLargeSymbols}{'164}{phfqit@MnLargeSymbols}{'164}
  \DeclareMathDelimiter{\rrangle}{\mathclose}%
                       {phfqit@MnLargeSymbols}{'171}{phfqit@MnLargeSymbols}{'171}
\fi
%    \end{macrocode}
%
% \subsubsection{Standard entropy measures}
%
% Load the requested set of qit objects. 
%    \begin{macrocode}
\def\phfqit@tmp@str@none{none}
\def\phfqit@tmp@str@stdset{stdset}
\ifx\phfqit@opt@qitobjdef\phfqit@tmp@str@none%
%    \end{macrocode}
% In this case, do not load any definitions.
%    \begin{macrocode}
\else\ifx\phfqit@opt@qitobjdef\phfqit@tmp@str@stdset%
%    \end{macrocode}
% In this case, provide our standard set of ``qit objects'' (i.e., entropy
% measures).
% 
% \needspace{4\baselineskip}
% \begin{macro}{\HH}
% \begin{macro}{\Hzero}
% \begin{macro}{\Hmin}
% \begin{macro}{\Hmaxf}
%   The definition of individual entropy macros just delegates to |\Hbase|
%   with the relevant subscript.
%    \begin{macrocode}
\def\HH{\Hbase{\HSym}{}}
\def\Hzero{\Hbase{\HSym}{\mathrm{max},0}}
\def\Hmin{\Hbase{\HSym}{\mathrm{min}}}
\def\Hmaxf{\Hbase{\HSym}{\mathrm{max}}}
\def\Hfn{\Hfnbase{\HSym}{}{}}
\let\Hfunc\Hfn% backwards compatibility
%    \end{macrocode}
% \end{macro}
% \end{macro}
% \end{macro}
% \end{macro}
%
% 
% \begin{macro}{\DD}
%   (Usual) quantum relative entropy.  (Actually this is more versatile, because
%   you can also specify subscript and superscript, so you can make on-the-fly
%   custom relative entropy measures.)
%    \begin{macrocode}
\def\DD{\Dbase{\DSym}}
%    \end{macrocode}
% \end{macro}
% 
% \begin{macro}{\Dminz}
%   ``Old'' min-relative entropy, based on the R\'enyi-zero relative entropy.
%    \begin{macrocode}
\newcommand\Dminz[1][]{\Dbase{\DSym}_{\mathrm{min,0}}^{#1}}
%    \end{macrocode}
% \end{macro}
% 
% \begin{macro}{\Dminf}
%   Min-relative entropy (``new'' version).
%    \begin{macrocode}
\newcommand\Dminf[1][]{\Dbase{\DSym}_{\mathrm{min}}^{#1}}
%    \end{macrocode}
% \end{macro}
%
% \begin{macro}{\Dmax}
%   Max-relative entropy.
%    \begin{macrocode}
\newcommand\Dmax[1][]{\Dbase{\DSym}_{\mathrm{max}}^{#1}}
%    \end{macrocode}
% \end{macro}
%
% \begin{macro}{\Dr}
%   Rob-relative entropy.
%    \begin{macrocode}
\newcommand\Dr[1][]{\Dbase{\DSym}_{\mathrm{r}}^{#1}}
%    \end{macrocode}
% \end{macro}
%
%
% \begin{macro}{\DHyp}
%   Hypothesis testing relative entropy.
%    \begin{macrocode}
\newcommand\DHyp[1][\eta]{\Dbase{\DSym}_{\mathrm{H}}^{#1}}
%    \end{macrocode}
% \end{macro}
%
% \begin{macro}{\Dhyp}
%   Hypothesis testing relative entropy (alternative definition).
%   \changed[v3.1-added-Dhyp-qitobjdef-stdset]{v3.1}{2021/07/27}{Added the
%   \phfverb{\Dhyp} variant of the hypothesis testing relative entropy}
%    \begin{macrocode}
\newcommand\Dhyp[1][\eta]{\Dbase{\DSym}_{\mathrm{h}}^{#1}}
%    \end{macrocode}
% \end{macro}
%
% \begin{macro}{\DCoh}
%   Coherent relative entropy (old style).
%    \begin{macrocode}
\DefineTunedQitObject{DCoh}{\DCohbaseParse}{\DCohbaseRender}{{\DCSym}}{}
%    \end{macrocode}   
% \end{macro}
%
% \begin{macro}{\DCohx}
%   Coherent relative entropy (new style).
%    \begin{macrocode}
\DefineTunedQitObject{DCohx}{\DCohbaseParse}{\DCohbaseRender}%
{{\DCSym}}{%
  \let\DCohbaseStateSubscripts\DCohxStateSubscripts%
}
%    \end{macrocode}
% \end{macro}
%
%
% End case |qitobjdef=stdset|.  Last case is the final |\else| branch which is an
% error, as we have an unknown set of standard definitions to load.
%    \begin{macrocode}
\else
\PackageError{phfqit}{Invalid value `\phfqit@opt@qitobjdef' specified for
  package option `qitobjdef'.  Please specify one of `stdset' (the default) or
  `none'}{You specified an invalid value to the `qitobjdef' package option of
  the `phfqit' package.}
\fi
\fi
%    \end{macrocode}
%
% 
% \Finale
\endinput
