% \iffalse meta-comment
%
% Copyright (C) 2021 by Philippe Faist <philippe.faist@bluewin.ch>
% -------------------------------------------------------
% 
% This file may be distributed and/or modified under the
% conditions of the LaTeX Project Public License, either version 1.3
% of this license or (at your option) any later version.
% The latest version of this license is in:
%
%    http://www.latex-project.org/lppl.txt
%
% and version 1.3 or later is part of all distributions of LaTeX 
% version 2005/12/01 or later.
%
% \fi
%
% \iffalse
%<*driver>
\ProvidesFile{phfextendedabstract.dtx}
%</driver>
%<package>\NeedsTeXFormat{LaTeX2e}[2005/12/01]
%<package>\ProvidesClass{phfextendedabstract}
%<*package>
    [2021/07/26 v0.1 phfextendedabstract class]
%</package>
%
%<*driver>
\documentclass{ltxdoc}
\usepackage{xcolor}
\usepackage[preset=xpkgdoc]{phfnote}
\usepackage{needspace}
\makeatletter
\newsavebox\phfeaDocVirtualPage@contents
\newenvironment{phfeaDocVirtualPage}{%
  \par%
  \begingroup%
  \makeatletter%
  \begin{lrbox}{\phfeaDocVirtualPage@contents}%
  \begin{minipage}{12cm}\relax%
    \def\rmdefault{cmr}\def\sfdefault{cmbr}\normalfont%
    \def\shortlipsum{Lorem ipsum dolor sit amet, consectetuer
      adipiscing elit. Ut purus elit, vestibulum ut, placerat ac,
      adipiscing vitae, felis. Curabitur dictum gravida
      mauris. Nam arcu libero, nonummy eget, consectetuer id,
      vulputate a, magna.}%
    \let\RequirePackage\@gobble%
    \def\morepagecontents{\par\vspace{1em}\centering\ldots}%
    \ignorespaces%
  }%
  {%
  \end{minipage}%
  \end{lrbox}%
  %%\centering%
  \begin{tcolorbox}[text width=6cm,sharp corners,%
    before={\par\vspace{5pt}\centering\nopagebreak\parindent=0pt},after={\par\vspace{5pt}},%
    leftrule=0.4pt,toprule=0.4pt,rightrule=0.6pt,bottomrule=0.6pt,%
    colframe=black,colback=white]%
    \scalebox{0.5}{\usebox{\phfeaDocVirtualPage@contents}}%
  \end{tcolorbox}%
  %%\par%
  \endgroup%
}
\def\phfeaSectionDecoration{%
  \raisebox{0.2ex}{{%
      \notesmaller[0.4]\phfeaSectionDecorationSymbol}}%
  \hspace*{1.5ex}%
}
\def\phfeaSectionDecorationSymbol{\ensuremath{\blacksquare}}
\def\phfeaParagraphDecoration{%
  \raisebox{0.2ex}{{%
      \notesmaller[0.6]\phfeaParagraphDecorationSymbol}}%
  \hspace*{1ex}%
}
\def\phfeaParagraphDecorationSymbol{\ensuremath{\triangleright}}

\def\eqsign@{=}
\def\eqsign{\protect\eqsign@}
\robustify\eqsign
\makeatother

\def\RevTeX{{\small R\raise-0.2ex\hbox{\textsc{ev}}}\TeX}

\EnableCrossrefs         
\CodelineIndex
\RecordChanges
\begin{document}
  \DocInput{phfextendedabstract.dtx}
\end{document}
%</driver>
% \fi
%
% \CheckSum{0}
%
% \CharacterTable
%  {Upper-case    \A\B\C\D\E\F\G\H\I\J\K\L\M\N\O\P\Q\R\S\T\U\V\W\X\Y\Z
%   Lower-case    \a\b\c\d\e\f\g\h\i\j\k\l\m\n\o\p\q\r\s\t\u\v\w\x\y\z
%   Digits        \0\1\2\3\4\5\6\7\8\9
%   Exclamation   \!     Double quote  \"     Hash (number) \#
%   Dollar        \$     Percent       \%     Ampersand     \&
%   Acute accent  \'     Left paren    \(     Right paren   \)
%   Asterisk      \*     Plus          \+     Comma         \,
%   Minus         \-     Point         \.     Solidus       \/
%   Colon         \:     Semicolon     \;     Less than     \<
%   Equals        \=     Greater than  \>     Question mark \?
%   Commercial at \@     Left bracket  \[     Backslash     \\
%   Right bracket \]     Circumflex    \^     Underscore    \_
%   Grave accent  \`     Left brace    \{     Vertical bar  \|
%   Right brace   \}     Tilde         \~}
%
%
% \changes{v1.0}{2016/04/20}{Initial version}
%
% \GetFileInfo{phfextendedabstract.dtx}
%
% \DoNotIndex{\newcommand,\newenvironment,\def,\gdef,\edef,\xdef,\if,\else,\fi,\ifx}
% 
% \title{\phfqitltxPkgTitle{phfextendedabstract}}
% \author{Philippe Faist\quad\email{philippe.faist@bluewin.ch}}
%
% \maketitle
%
% \begin{abstract}
%   \pkgname{phfextendedabstract}---Typeset extended abstracts for conferences,
%   such as often encountered in quantum information theory.
% \end{abstract}
%
% \inlinetoc
%
% \section{Introduction}
%
% Several conferences in Quantum Information Theory (and presumably in other
% fields, too) require the submission of \emph{extended abstracts}.  An extended
% abstract is a summary of a scientific result, presented at a high level, and
% consisting of a handful of pages.
%
% I found myself repeatedly copying my \LaTeX\ preamble from one submission to
% another with definitions for a format that I elaborated and liked.  So at some
% point I collected the main definitions into this class file.
% 
% The \pkgname{phfextendedabstract} class sets up the document for an extended
% abstract submission as a single-column document, typeset by default at 11
% point size, with at most two sectioning levels (|\section|\,s and
% |\paragraph|\,s).
%
% The extended abstract style looks approximately like this:
% \nopagebreak
% \iffalse COMMENT: Spacings are copied/emulated ! They might appear wrong!! \fi
% \begin{phfeaDocVirtualPage}
%   {\parskip=8pt\relax
%     {\sffamily\Large\centering Title of my extended abstract\par}
%     {\centering Author One,\ \ Author Two,\ \  and\ \ Author Three\par}
%   }\vspace{0.2cm}
%   \parskip=3pt\relax
%   
%   \noindent{\normalfont\normalsize\sffamily\fontseries{sb}\selectfont
%     \makebox[0pt][r]{\phfeaSectionDecoration}Introduction.}
%   \shortlipsum
%   
%   \noindent{\normalfont\normalsize\sffamily\fontseries{sb}\selectfont
%     \makebox[0pt][r]{\phfeaSectionDecoration}Results.}
%   \shortlipsum
%   
%   \noindent{\normalfont\normalsize\sffamily\small
%     \makebox[0pt][r]{\phfeaParagraphDecoration}First result.}
%   \ldots
%
%   \noindent{\normalfont\normalsize\sffamily\small
%     \makebox[0pt][r]{\phfeaParagraphDecoration}Second result.}
%   \ldots
%
%   \noindent{\normalfont\normalsize\sffamily\fontseries{sb}\selectfont
%     \makebox[0pt][r]{\phfeaSectionDecoration}Discussion.}
%   \shortlipsum   
%  \hspace*{1.5ex}%
%
% \end{phfeaDocVirtualPage}
%
% Here are the main features of the \pkgname{phfextendedabstract} class:
% \begin{itemize}
% \item The class \pkgname{phfextendedabstract} loads the \pkgname{revtex4-2}
%   class, so you can directly use \RevTeX's features such as author
%   affiliations, etc.
%
% \item Only two sectioning commands are enabled: |\section| and |\paragraph|.
%   Both have run-in headings.  If you find yourself needing additional
%   sectioning levels, it might be that your extended abstract is too detailed
%   and you might want to think about how to keep it at a higher level.
%
%   By default, section and paragraphs have ``decorations'' (by default a symbol
%   in the margin) to guide the reader through the overall high-level structure
%   of the document.
%
% \item You can easily scale all vertical spacing dimensions (section and
%   paragraph spacing, vertical space around theorems and list environments,
%   etc.) by a common factor with a class option.  Do you have those extra two
%   lines that make you exceed your 3-page limit?  Try squeezing everything
%   together with something like |compressverticalspacing=0.7| class option.
%
% \item The \pkgname{phfnote} package is loaded in order to provide a set of
%   default \LaTeX\ packages and set up hyperlinks.  A generous default set of
%   standard LaTeX packages are loaded, including \pkgname{caption} and
%   \pkgname{enumitem}; see \pkgname{phfnote}'s documentation for the option
%   |pkgset=extended|).
%
%   You are expected to call |\usepackage{hyperref}| somewhere in your preamble.
%   We deliberately don't include \pkgname{hyperref} to give you greater control
%   of package loading order (most packages you might want to use must be loaded
%   before \pkgname{hyperref}).
%
% \item The page margins are tweaked with the \pkgname{geometry} package.
%   (Simply call |\geometry{...}| from the \pkgname{geometry} package if you'd
%   like to further change them.)
%
% \item Lists, i.e.\@ itemization and enumeration environments, are customized
%   so they take less space.  You also get a |enumerate*| environment that
%   typesets its items in-line, in a single paragraph.
%
% \item By default a customized theorem style so that it stands out but also so
%   that it contrasts well with the section and paragraph headings.
%
% \end{itemize}
%
%
% \section{Usage}
%
% Here's a minimal usage template:
% \begin{verbatim}
% \documentclass[papertype=a4paper]{phfextendedabstract}
% \usepackage{hyperref}
% \begin{document}
% \title{Title goes here}
% \author{Author 1}
% %\affiliation{...}
% \author{Author 2}
% %\affiliation{...}
% %\date{July 27, 2021}
% \maketitle
% 
% \section{Introduction}
% ...
%
% \end{document}
% \end{verbatim}
%
%
% \section{Class options}
%
% Here is a summary of class options:
% \begin{pkgoptions}
% \item[papertype=a4paper,letterpaper,\meta{paper type},\meta{empty}] Specify
%   the paper type to use (A4 or letter).  The argument given to this option is
%   directly specified as an option to the underlying \RevTeX\ class.
% \item[ptsize=10pt,11pt,12pt] Default font size to use.  Again the argument
%   given to this option is directly specified as an option to the underlying
%   \RevTeX\ class.
% \item[sectiondecorations=\meta{true or false}] Whether or not to ``decorate''
%   section headings, by default with a small black square typeset in the margin
%   of the section heading.
% \item[paragraphdecorations=\meta{true or false}] Whether or not to
%   ``decorate'' paragraph headings, by default with a small right-pointing
%   outlined triangle typeset in the margin of the section heading.
% \item[noheadingdecorations] Shorthand for
%   |sectiondecorations=false,paragraphdecorations=false|.
% \item[compressverticalspacing=\meta{factor}] Real factor by which to multiply
%   the vertical spacing between sections, paragraphs, theorems, and list
%   environments such as |itemize| and |enumerate|.  If you need to compress the
%   sections a bit to fit more content on a fixed number of pages (e.g.\@
%   because of page number constraints), you can set this option to a factor
%   that's less than one.  A \meta{factor} that's less than one compresses
%   sections together, a \meta{factor} equal to |1| does nothing, and a
%   \meta{factor} greater than one expands the sections further apart.
% \item[loadtheorems=\meta{true or false}] If |loadtheorems=true| (the default),
%   then the \pkgname{phfthm} package is loaded with some suitable default
%   options and a custom theorem style.  (The theorem style
%   |phfextendedabstractthm| is always defined, regardless of this package
%   option.)
% \item[sansstyle=\meta{true or false}] Use sans serif style by default for the
%   main title as well as for section and paragraph headings.  For greater
%   degree of control, see the ``customization macros'' section below.
% \item[usehypreref=\meta{true or false}] Should we set up the document for use
%   with the \pkgname{hyperref} package or not?  This option influences how we
%   load the \pkgname{phfnote} package.  If this option is set to |true| (the
%   default), then the \pkgname{phfnote} package is loaded with
%   |hyperrefdefs={defer,noemail}|.  This means that the document is set up for
%   use with \pkgname{hyperref}, although you will still need to say
%   |\usepackage{hyperref}| somewhere in your preamble.  If |usehyperref=false|,
%   then the \pkgname{phfnote} package is loaded without any hyperref options.
% \item[pkgset=\meta{\pkgname{phfnote} package set name}] The \pkgname{phfnote}
%   package (which we load internally) loads a bunch of standard packages for
%   your convenience, such as \pkgname{enumitem} or \pkgname{microtype}.  You
%   can influence this behavior by specifying a ``package set'' to load.  By
%   default, the |pkgset=extended| package set is loaded.  If you don't want to
%   load any additional packages beyond those that are strictly necessary, use
%   |pkgset=none| or |pkgset=minimal|.  See \pkgname{phfnote}'s documentation
%   for the package option |pkgset=...| for more information and a list of
%   possible package set names.
% \end{pkgoptions}
% 
%
% \section{Provided macros and environments}
%
% \subsection{Vertical spacing}
%
% \DescribeMacro{\phfeaVerticalSpacingCompressionFactor} \leavevmode\\ To
% conveniently globally adjust the vertical spacings in the document (including
% the section and paragraph vertical spacings, as well as the spacing above and
% below theorems, itemize, and enumerate environments), you can also use the
% |compressverticalspacing=X| class option.  Alternatively, you can redefine the
% macro |\phfeaVerticalSpacingCompressionFactor| to the desired compression
% factor:
% \begin{verbatim}
% \renewcommand\phfeaVerticalSpacingCompressionFactor{0.7}
% \end{verbatim}
%
% \DescribeMacro{\phfeaDisplayVerticalSpacingFactorWeight} \leavevmode\\ The
% vertical spacing factor also affects the vertical spacing around equations,
% but to a lesser extent. (Compressing the space around the equations by too
% much would not look nice.) For the spacing between equations, we take the
% weighted average of $1$ and the vertical spacing compression factor, where the
% weight is given in the macro |\phfeaDisplayVerticalSpacingFactorWeight|.  A
% weight of |0| means the vertical compression factor doesn't affect the
% vertical spacing around equations at all; a weight of |1| means the spacing
% around the equations is scaled exactly by the vertical compression factor.
%
% \DescribeMacro{\phfeaDefineTheoremStyle} Note: because the theorem style
% |phfextendedabstracthm| is defined when the class is loaded, any customization
% of |\phfeaVerticalSpacingCompressionFactor| and |\phfeaListsVerticalSkip| that
% happen later in the preamble aren't taken into account.  If you customize
% these quantities in your preamble, you should call |\phfeaDefineTheoremStyle|
% to redefine the theorem style after your customization.
%
%
% \DescribeMacro{\phfeaParskipVerticalSpacingFactorWeight} \leavevmode\\ A
% similar mechanism happens for how we adjust |\parskip|, the spacing between
% paragraphs.
%
%
% \subsection{List environments}
%
% \DescribeEnv{enumerate*} This class provides an |enumerate|-like
% environment which typesets its items inline, as a list.  For example, here is
% an inline paragraph with (a) one, (b) two, and (c) three points.
% \iffalse yeah, I cheated for this doc code, whatever \fi
%
% The |enumerate*| can be used exactly like you'd use the |enumerate|
% environment from the \pkgname{enumitem} package, for instance:
% \begin{verbatim}
% here is an inline paragraph with \begin{enumerate*}[label=(\alph*)]
% \item one,
% \item two,
% \item three
% \end{enumerate*}
% points.
% \end{verbatim}
%
% Check \pkgname{enumitem}'s documentation for inline lists.  You can specify
% for instance the keys |before={{}}|, |itemjoin={{ }}|, and \relax\relax\relax
% |itemjoin*={{ and }}| either as an optional argument to
% |\begin{enumerate*}| \iffalse\end{enumerate*}\fi or using
% |\setlist|:
% \begin{verbatim}
% \setlist[enumerate*]{%
%   itemjoin*={{ et }}%
% }
% \end{verbatim}
%
% \DescribeMacro{\phfeaListsVerticalSkip} \DescribeMacro{\phfeaListsItemSep}
% \DescribeMacro{\phfeaListsParSep} For (non-inline) list environments such as
% |itemize| and |enumerate|, you can redefine |\phfeaListsVerticalSkip|,
% |\phfeaListsItemSep|, and |\phfeaListsParSep| to set the |topsep|, |itemsep|
% and |parsep| properties of all of \pkgname{enumitem}'s list environments.
% These correspond to the vertical space above and below lists, the space
% between items, and the space between paragraphs within an item.  All these
% spacings automatically get compressed according to the
% |\phfeaVerticalSpacingCompressionFactor|.
%
% Note that |\phfeaListsVerticalSkip| is also used for the spacing above and
% below theorem environments.
%
%
% \subsection{Customization of the main title and general headings style}
%
% \DescribeMacro{\phfeaHeadingStyle} The macro |\phfeaHeadingStyle| is defined to be
% |\sffamily|, unless the |sansstyle=false| class option is provided
% in which case the macro expands to nothing.  You can redefine it to give a
% different general style to your main title and your section and paragraph
% headings.
%
% \DescribeMacro{\phfeaTitleStyle} The macro |\phfeaTitleStyle| sets the font
% style for the main document title.  By default the macro is defined to
% |\phfeaHeadingStyle\Large|.  Redefine this macro to change the title style.
%
% For instance, if you prefer \RevTeX's own simple boldface title, you can use:
% \begin{verbatim}
% \renewcommand\phfeaTitleStyle{\bfseries}
% \end{verbatim}
% 
% 
% \subsection{Customizing sections and paragraphs}
%
% You can customize section and paragraph headings, including spacing and style,
% by redefining the following macros.
%
% \DescribeMacro{\phfeaSectionBeforeSkip}
% \DescribeMacro{\phfeaSectionAfterHSkip} The macro |\phfeaSectionBeforeSkip|
% (it's a macro, not a length) is used to specify the vertical spacing before a
% new section heading.  The macro (not length) |\phfeaSectionAfterHSkip| is the
% horizontal space between the end of the section heading and the beginning of
% the section text contents.
%
% If you're keen to save more space and compress the vertical spacing between
% sections you can redefine the ``before skip'' macro (but see also the
% |compressverticalspacing=| class option and the
% |\phfeaVerticalSpacingCompressionFactor| macro):
% \begin{verbatim}
% \renewcommand\phfeaSectionBeforeSkip{0.3ex plus 0.2ex minus 0.1ex}
% \end{verbatim}
%
% \DescribeMacro{\phfeaSectionStyle} The macro |\phfeaSectionStyle| is used to
% set the style of the section headings.  By default, the default sans/heading
% style is used at the normal size and in bold face series.
%
% \DescribeMacro{\phfeaSectionDecoration}
% \DescribeMacro{\phfeaSectionDecorationSymbol} By default, section headings are
% ``decorated'' with a black square in the margin, to guide the reader's eye
% through the document's high-level structure.  You can customize this
% decoration by redefining the macros |\phfeaSectionDecoration| and
% |\phfeaSectionDecorationSymbol|.
%
% If the |sectiondecorations=false| class option was specified, the macros
% |\phfeaSectionDecoration| and |\phfeaSectionDecorationSymbol| are defined to
% be empty.
%
% If you simply want to change the decoration symbol (say change it to a star),
% you can redefine |\phfeaSectionDecorationSymbol| as follows:
% \begin{verbatim}
% \renewcommand{\phfeaSectionDecorationSymbol}{$\star$}
% \end{verbatim}
% If you want greater control over the decoration you can redefine
% |\phfeaSectionDecoration| instead (e.g., to change the spacing, vertical
% alignment, etc., of the decoration).  The contents of the macro
% |\pheaSectionDecoration| will be typeset in a zero-width right-aligned box
% which is anchored to the beginning of the section heading.  E.g.:
% \begin{verbatim}
% \renewcommand{\phfeaSectionDecoration}{%
%   \raisebox{-0.1ex}{{%
%     \Large\phfeaSectionDecorationSymbol}}%
%   \hspace*{2.5ex}%
% }
% \renewcommand{\phfeaSectionDecorationSymbol}{$\Rightarrow$}
% \end{verbatim}
%
% By default, the macro |\phfeaSectionDecoration| calls the macro
% |\phfeaSectionDecorationSymbol|, so if you redefine the former without calling
% the latter, the latter won't be used at all.
%
% \needspace{5\baselineskip} \DescribeMacro{\phfeaParagraphBeforeSkip}
% \DescribeMacro{\phfeaParagraphAfterHSkip} \DescribeMacro{\phfeaParagraphStyle}
% \DescribeMacro{\phfeaParagraphDecoration}
% \DescribeMacro{\phfeaParagraphDecorationSymbol} You can customize paragraphs
% in the exact same way as sections, with the corresponding macros
% |\phfeaParagraphBeforeSkip|, |\phfeaParagraphAfterHSkip|,
% |\phfeaParagraphStyle|, |\phfeaParagraphDecoration|,
% |\phfeaParagraphDecorationSymbol|.
%
% 
%
% \StopEventually{\vskip 3cm plus 2cm minus 2cm\relax\PrintChanges
%     \vskip 3cm plus 2cm minus 2cm\relax\PrintIndex}
%
%
%
% \section{Implementation}
%
% Here come the gory details.
%
%
% \paragraph{Class options}
% We process these first, to see which options we should pass on to
% \RevTeX.
%    \begin{macrocode}
\RequirePackage{kvoptions}
\SetupKeyvalOptions{%
  family=phfea,%
  prefix=phfeaopt@%
}
\DeclareStringOption[]{papertype}
\DeclareStringOption[11pt]{ptsize}
\DeclareBoolOption[true]{sectiondecorations}
\DeclareBoolOption[true]{paragraphdecorations}
\DeclareVoidOption{noheadingdecorations}{%
  \ifphfeaopt@sectiondecorationsfalse
  \ifphfeaopt@paragraphdecorationsfalse
}
\DeclareBoolOption[true]{loadtheorems}
\DeclareBoolOption[true]{sansstyle}
\DeclareStringOption[1]{compressverticalspacing}
\DeclareBoolOption[true]{usehyperref}
\DeclareStringOption[extended]{pkgset}
\ProcessKeyvalOptions*
%    \end{macrocode}
%
%
% \paragraph{Load \RevTeX, the base class}
%
%    \begin{macrocode}
\providecommand\phfea@revtexopts{%
  aps,pra,%
  notitlepage,reprint,%
  onecolumn,tightenlines,%
  superscriptaddress,%
  nofootinbib%
}
\PassOptionsToClass{%
  \phfea@revtexopts,%
  \phfeaopt@ptsize,%
  \phfeaopt@papertype,%
}{revtex4-2}
\LoadClass{revtex4-2}
%    \end{macrocode}
%
%
% \paragraph{Load \pkgname{phfnote} for the basic document setup}
%
%    \begin{macrocode}
\PassOptionsToPackage{%
  preset=reset,%
  pkgset=\phfeaopt@pkgset,%
  \ifphfeaopt@usehyperref
  hyperrefdefs={defer,noemail},%
  \fi
}{phfnote}
\RequirePackage{phfnote}
%    \end{macrocode}
%
%
% \paragraph{Page geometry}
%
% Set a default page geometry.  Works both for A4 paper and for letter paper.
% It's optimized for 11pt size, though.
%    \begin{macrocode}
\RequirePackage{geometry}
\geometry{hmargin=0.75in,vmargin=0.75in,marginparwidth=0.5in,marginparsep=0.125in}
%    \end{macrocode}
% 
%
% \paragraph{Default sans serif font}
%
% \begin{macro}{\phfeaHeadingStyle}
%   Unless instructed not to do so, \iffalse load the ``Source Sans Pro'' font
%   as default sans serif font, using the semibold glyphs in place of bold.  By
%   default, this font will \else set the sans serif font family to \fi be used
%   for section headings and the main title.
%    \begin{macrocode}
\ifphfeaopt@sansstyle
%%\RequirePackage[semibold]{sourcesanspro}
\def\phfeaHeadingStyle{\sffamily}
\else
\def\phfeaHeadingStyle{}
\fi
%    \end{macrocode}
% \end{macro}
%
%
% \paragraph{Default title format}
%
% \begin{macro}{\phfeaTitleStyle}
%   Change \RevTeX\ title format.  Title style can be customized by redefining
%   |\phfeaTitleStyle|.
%    \begin{macrocode}
\def\phfeaTitleStyle{\phfeaHeadingStyle\Large}
\def\frontmatter@title@format{\phfeaTitleStyle\centering\parskip\z@skip}
%    \end{macrocode}
% \end{macro}
%
%
% \paragraph{Vertical spacing compression factor}
%
% Define a general factor by which the section and paragraph spacings will be
% compressed.  This macro is set by the |compressverticalspacing=X| package
% option.
%    \begin{macrocode}
\edef\phfeaVerticalSpacingCompressionFactor{\phfeaopt@compressverticalspacing}
%    \end{macrocode}
%
% Tool for scaling glue expressions (for use with our vertical compression
% factor):
%    \begin{macrocode}
\def\phfea@scaleglue#1#2{% {factor}{glueexpr}
  \glueexpr#2*\numexpr\dimexpr#1pt\relax\relax/65536\relax
}
\def\phfea@scalegluedpt#1#2{% {factor given as dimexpr in pt}{glueexpr}
  \glueexpr#2*\numexpr#1\relax/65536\relax
}
%    \end{macrocode}
%
% Adjust spacing around display equations according to the vertical compression
% factor.  Do this only at the beginning of the document, since the user might
% still want to adjust |\phfeaVerticalSpacingCompressionFactor| in their
% preamble.
%
% \begin{macro}{\phfeaDisplayVerticalSpacingFactorWeight}
% We only apply a fraction of the scaling, because it's ugly if we compress
% equations too much.  Define |\phfeaDisplayVerticalSpacingFactorWeight| as $w$ and
% $\alpha$ as the vertical scaling factor.  The new skips are computed as
% \begin{equation*}
%   \phfverb{oldskip}\quad\rightarrow\quad
%     (1-w)\,\phfverb{oldskip} + w\,\alpha\,\phfverb{oldskip}\ .
% \end{equation*}
% (For $w=0$ we have $\phfverb{oldskip}\rightarrow\phfverb{oldskip}$.  For $w=1$
% the full scaling factor is applied,
% $\phfverb{oldskip}\rightarrow \alpha\,\phfverb{oldskip}$.)
%    \begin{macrocode}
\def\phfeaDisplayVerticalSpacingFactorWeight{.5}
%    \end{macrocode}
% \end{macro}
% Tool to compute the new spacing for each of the relevant display-related
% skips:
%    \begin{macrocode}
\def\phfea@adjustskipweighted#1#2{%
  #1=\glueexpr
    \phfea@scalegluedpt{%
      \dimexpr 1\p@ - #2\p@\relax
    }{#1}%
    +
    \phfea@scaleglue{%
      #2%
    }{%
      \phfea@scaleglue{%
        \phfeaVerticalSpacingCompressionFactor
      }{%
        #1
      }%
    }%
    \relax
}
%    \end{macrocode}
% And schedule this adjustment to be carried out at the beginning of the
% document.
%    \begin{macrocode}
\AtBeginDocument{%
  \phfea@adjustskipweighted\abovedisplayskip\phfeaDisplayVerticalSpacingFactorWeight
  \phfea@adjustskipweighted\belowdisplayskip\phfeaDisplayVerticalSpacingFactorWeight
  \phfea@adjustskipweighted\abovedisplayshortskip\phfeaDisplayVerticalSpacingFactorWeight
  \phfea@adjustskipweighted\belowdisplayshortskip\phfeaDisplayVerticalSpacingFactorWeight
}
%    \end{macrocode}
%
% A similar mechanism affects how we adjust the paragraph skip length
% |\parskip|.
%    \begin{macrocode}
\def\phfeaParskipVerticalSpacingFactorWeight{1}
\AtBeginDocument{%
  \phfea@adjustskipweighted\parskip\phfeaParskipVerticalSpacingFactorWeight
}
%    \end{macrocode}
% 
% \paragraph{Setup specific for sectioning}
%
% By design, there are only two sectioning levels in a
% \pkgname{phfextendedabstract} document: sections (|\section|) and paragraphs
% (|\paragraph|).  Any other sectioning command (e.g., |\subsection|) will
% produce an error.
%
% Neither of these two section levels is numbered.
%    \begin{macrocode}
\setcounter{secnumdepth}{0}
\setcounter{tocdepth}{1}
%    \end{macrocode}
%
% \begin{macro}{\phfeaSectionBeforeSkip}
% \begin{macro}{\phfeaSectionAfterHSkip}
% \begin{macro}{\phfeaSectionStyle}
% \begin{macro}{\phfeaSectionDecoration}
% \begin{macro}{\phfeaSectionDecorationSymbol}
%   Some helper and customization macros for |\section|.
%    \begin{macrocode}
\def\phfeaSectionBeforeSkip{1.5ex plus 0.8ex minus 0.25ex}
\def\phfeaSectionAfterHSkip{1em}
\def\phfeaSectionStyle{\normalfont\normalsize\phfeaHeadingStyle\bfseries}
\ifphfeaopt@sectiondecorations
  \def\phfeaSectionDecoration{%
    \raisebox{0.2ex}{{%
        \notesmaller[0.4]\phfeaSectionDecorationSymbol}}%
    \hspace*{1.5ex}%
  }
  \def\phfeaSectionDecorationSymbol{\ensuremath{\blacksquare}}
\else
  \def\phfeaSectionDecoration{}
  \def\phfeaSectionDecorationSymbol{}
\fi
%    \end{macrocode}
% \end{macro}
% \end{macro}
% \end{macro}
% \end{macro}
% \end{macro}
%
% \begin{macro}{\phfeaParagraphBeforeSkip}
% \begin{macro}{\phfeaParagraphAfterHSkip}
% \begin{macro}{\phfeaParagraphStyle}
% \begin{macro}{\phfeaParagraphDecoration}
% \begin{macro}{\phfeaParagraphDecorationSymbol}
%   Same helper and customization macros, now for |\paragraph|.
%    \begin{macrocode}
\def\phfeaParagraphBeforeSkip{0.6ex plus 0.4ex minus 0.1ex}
\def\phfeaParagraphAfterHSkip{0.75em}
\def\phfeaParagraphStyle{\normalfont\normalsize\phfeaHeadingStyle\small}
\ifphfeaopt@paragraphdecorations
  \def\phfeaParagraphDecoration{%
    \raisebox{0.2ex}{{%
        \notesmaller[0.6]\phfeaParagraphDecorationSymbol}}%
    \hspace*{1ex}%
  }
  \def\phfeaParagraphDecorationSymbol{\ensuremath{\triangleright}}
\else
  \def\phfeaParagraphDecoration{}
  \def\phfeaParagraphDecorationSymbol{}
\fi
%    \end{macrocode}
% \end{macro}
% \end{macro}
% \end{macro}
% \end{macro}
% \end{macro}
% 
% \begin{macro}{\section}
% \begin{macro}{\paragraph}
%   Redefine |\section| and |\paragraph| for formatting.  Observe that
%   |\section| and |\paragraph| have no optional argument and no starred
%   variant.
%    \begin{macrocode}
\renewcommand\section[1]{%
  \@startsection{section}%
  {1}% (level)
  {0pt}% (indent)
  {\phfea@scaleglue{\phfeaVerticalSpacingCompressionFactor}{%
    \glueexpr\phfeaSectionBeforeSkip\relax}}% (beforeskip)
  {-\glueexpr\phfeaSectionAfterHSkip\relax}% (afterskip)
  {\phfeaSectionStyle}% (style)
  [#1]% (toc title)
  {%
    \texorpdfstring{\makebox[0pt][r]{\phfeaSectionDecoration}}{}%
    #1%
  }% (title)
}
\renewcommand\paragraph[1]{%
  \@startsection{paragraph}%
  {2}% (level)
  {0pt}% (indent)
  {\phfea@scaleglue{\phfeaVerticalSpacingCompressionFactor}{%
      \glueexpr\phfeaParagraphBeforeSkip\relax}}% (beforeskip)
  {-\glueexpr\phfeaParagraphAfterHSkip\relax}% (afterskip)
  {\phfeaParagraphStyle}% (style)
  [#1]% (toc title)
  {%
    \texorpdfstring{\makebox[0pt][r]{\phfeaParagraphDecoration}}{}%
    #1%
  }% (title)
}
%    \end{macrocode}
% \end{macro}
% \end{macro}
%
%    \begin{macrocode}
\let\subsection\undefined
\let\subsubsection\undefined
\let\subparagraph\undefined
%    \end{macrocode}
%
%
% \paragraph{Set up itemization and enumeration environments}
%
% Provide customizable lengths for lists via macros (item sep, paragraph sep and
% vertical skip above and below list environments).
% \begin{macro}{\phfeaListsVerticalSkip}
% \begin{macro}{\phfeaListsItemSep}
% \begin{macro}{\phfeaListsParSep}
%    \begin{macrocode}
\def\phfeaListsVerticalSkip{0.6ex plus 0.4ex minus 0.1ex}
\def\phfeaListsItemSep{0.3ex plus 0.15ex minus 0.1ex}
\def\phfeaListsParSep{0.7\parskip}
%    \end{macrocode}
% \end{macro}
% \end{macro}
% \end{macro}
% Prepare the commands to run to configure \pkgname{enumitem} correctly in an
% internal macro which we will call if \pkgname{enumitem} is indeed loaded.
%    \begin{macrocode}
\def\phfea@setup@enumitem{%
%    \end{macrocode}
% Apply the spacings.
%    \begin{macrocode}
  \setlist{%
    itemsep={\phfea@scaleglue{\phfeaVerticalSpacingCompressionFactor}{\phfeaListsItemSep}},
    parsep={\phfea@scaleglue{\phfeaVerticalSpacingCompressionFactor}{\phfeaListsParSep}},
    topsep={\phfea@scaleglue{\phfeaVerticalSpacingCompressionFactor}{\phfeaListsVerticalSkip}},
  }%
%    \end{macrocode}
% \begin{environment}{enumerate*}
% Create the |enumerate*| list enumeration environment.
%    \begin{macrocode}
  \newlist{enumerate*}{enumerate*}{1}
  \setlist[enumerate*]{
    label={(\roman*)},
    before={},
    itemjoin={{ }},
    itemjoin*={{ and }}
  }%
}
%    \end{macrocode}
% \end{environment}
% And now, check if \pkgname{enumitem} is loaded and apply the definitions.
%    \begin{macrocode}
\@ifpackageloaded{enumitem}{\phfea@setup@enumitem}{}
%    \end{macrocode}
% 
%
% \paragraph{Setup specific for theorems}
%
% \emph{BUG: \phfverb{\phfeaListsVerticalSkip} is evaluated at this point when
% defining the new theorem style, ignoring any later redefinition by the user in
% their preamble.  I'm not sure how to fix this.}
%
% Create a new theorem style called |extendedabstracthm|.  Note that if
% \pkgname{amsthm} (or similar) wasn't loaded, then |\newtheoremstyle| isn't
% defined.  In that case, we simply won't define the new theorem style right
% now.
%    \begin{macrocode}
\def\phfeaDefineTheoremStyle{%
  \newtheoremstyle{phfextendedabstractthm}%
    {\phfea@scaleglue{\phfeaVerticalSpacingCompressionFactor}{\phfeaListsVerticalSkip}}%
    {\phfea@scaleglue{\phfeaVerticalSpacingCompressionFactor}{\phfeaListsVerticalSkip}}%
    {\itshape}%
    {0pt}%
    {\bfseries\itshape}%
    {:}%
    {0.8em}%
    {\thmname{##1\thmnumber{ ##2}\thmnote{ (##3)}}}%
}
%    \end{macrocode}
% Ensure we create our new theorem style (only if |\newtheoremstyle| is
% available) and load our \pkgname{phfthm} package.  Make sure we don't load
% \pkgname{phfthm} if we were asked not to load theorems in the class options.
%    \begin{macrocode}
\ifdefined\newtheoremstyle
  \phfeaDefineTheoremStyle
\fi
\ifphfeaopt@loadtheorems
  \ifdefined\newtheoremstyle\else
    \PackageError{phfextendedabstract}{Impossible to load theorems
      (loadtheorems=true) because there is no \string\newtheoremstyle\space
      command that was defined.  Consider setting pkgset= so that a theroems-related
      package (e.g., amsthm) is loaded! (e.g. pkgset=minimal, pkgset=rich or
      pkgset=extended)}{}
  \fi
  \PassOptionsToPackage{proofref=false,theoremstyle=phfextendedabstractthm}{phfthm}
  \RequirePackage{phfthm}
\fi
%    \end{macrocode}
%
% \iffalse
% \paragraph{References section}
%
% Have a simple section for references.  If you'd like to restore \RevTeX'
% ornament you can simply do |\let\bibsection\rtxapsbibsection|.
%    \begin{XXXmacrocode}
%% \let\rtxapsbibsection\bibsection
%% \def\bibsection{%
%%   \par\section{\refname}\leavevmode\par\addvspace{4pt}\relax
%% %%  \par\noindent\rule{6em}{.4pt}\par\addvspace{6pt}\relax
%% }
%    \end{XXXmacrocode}
% \fi
%
%
%\Finale
\endinput
