% \iffalse meta-comment
%
% Copyright (C) 2019 by Philippe Faist, philippe.faist@bluewin.ch
% -------------------------------------------------------
% 
% This file may be distributed and/or modified under the
% conditions of the LaTeX Project Public License, either version 1.3
% of this license or (at your option) any later version.
% The latest version of this license is in:
%
%    http://www.latex-project.org/lppl.txt
%
% and version 1.3 or later is part of all distributions of LaTeX 
% version 2005/12/01 or later.
%
% \fi
%
% \iffalse
%<*driver>
\ProvidesFile{phfcc.dtx}
%</driver>
%<package>\NeedsTeXFormat{LaTeX2e}[2005/12/01]
%<package>\ProvidesPackage{phfcc}
%<*package>
    [2020/04/02 v1.0 phfcc package]
%</package>
%
%<*driver>
\documentclass{ltxdoc}
\usepackage{xcolor}
\usepackage{lipsum}
\usepackage[preset=xpkgdoc]{phfnote}

\usepackage{needspace}


\def\eqsign@{=}
\def\eqsign{\protect\eqsign@}
\robustify\eqsign
\makeatother

\def\RevTeX{{\small R\raise-0.2ex\hbox{\textsc{ev}}}\TeX}

\EnableCrossrefs
\CodelineIndex
\RecordChanges

\begin{document}
  \DocInput{phfcc.dtx}
\end{document}
%</driver>
% \fi
%
% \CheckSum{0}
%
% \CharacterTable
%  {Upper-case    \A\B\C\D\E\F\G\H\I\J\K\L\M\N\O\P\Q\R\S\T\U\V\W\X\Y\Z
%   Lower-case    \a\b\c\d\e\f\g\h\i\j\k\l\m\n\o\p\q\r\s\t\u\v\w\x\y\z
%   Digits        \0\1\2\3\4\5\6\7\8\9
%   Exclamation   \!     Double quote  \"     Hash (number) \#
%   Dollar        \$     Percent       \%     Ampersand     \&
%   Acute accent  \'     Left paren    \(     Right paren   \)
%   Asterisk      \*     Plus          \+     Comma         \,
%   Minus         \-     Point         \.     Solidus       \/
%   Colon         \:     Semicolon     \;     Less than     \<
%   Equals        \=     Greater than  \>     Question mark \?
%   Commercial at \@     Left bracket  \[     Backslash     \\
%   Right bracket \]     Circumflex    \^     Underscore    \_
%   Grave accent  \`     Left brace    \{     Vertical bar  \|
%   Right brace   \}     Tilde         \~}
%
%
% \changes{v1.0}{2020/04/02}{Initial version}
%
% \GetFileInfo{phfcc.dtx}
%
% \iffalse Bypass indexing for following commands: \fi
% \DoNotIndex{\newcommand,\newenvironment,\renewcommand,\long,\def,\edef,\gdef,\xdef,\if,\else,\fi,\par,\relax,\vspace,\vskip,\hspace,\hskip,\vbox,\hbox}
% 
% \title{\phfqitltxPkgTitle{phfcc}}
% \author{Philippe Faist\quad\email{philippe.faist@bluewin.ch}}
% \date{\pkgfmtdate\filedate}
% \maketitle
%
% \begin{abstract}
%   \pkgname{phfcc}---A handy \LaTeX{} package for inline commenting in
%   collaborative LaTeX documents.
% \end{abstract}
%
% \phantomsection\label{sec:toc}
% \inlinetoc
%
% \section{Introduction}
%
% When elaborating documents, it is useful to leave inline comments and be able
% to mark text changes, especially when collaborating with a team.  Often in
% practice a simple macro definition will do, like this:
% \begin{verbatim}
%   \newcommand{\phf}[1]{{\color{blue}#1}} % Philippe Faist
% \end{verbatim}
% But because I kinda like more fancy stuff, and I definitely didn't have time
% to spare but still got distracted into programming \LaTeX{} macros, here is a
% way of defining a custom command such as |\phf| above, but that allows several
% syntax options that format the command differently.
% \DescribeMacro{\phfMakeCommentingCommand} This is done using the
% |\phfMakeCommentingCommand| macro.
%
% \phfMakeCommentingCommand[initials={AE}]{AlE}%
% \phfMakeCommentingCommand[initials={JB}]{JB}%
% \phfMakeCommentingCommand[initials={PhF}]{phf}%
% Let's start with a simple example. Suppose two authors, Albert Einstein and
% John Bell, are commenting on the same document. They might define the
% commenting commands |\AlE| and |\JB|:
% \begin{verbatim}
% \phfMakeCommentingCommand[initials={AE}]{AlE}
% \phfMakeCommentingCommand[initials={JB}]{JB}
% \end{verbatim}
% They can then comment on text like this:
% \begin{verbatim}
% Quantum mechanics developed quickly in the 1920's thanks to the
% works of Schr\"odinger, Heisenberg, and Dirac, who drew inspiration
% from earlier ideas by Planck \AlE{and Einstein} \AlE[@JB, you forgot
% about me!].  Quantum mechanics appears to allow spooky action at a
% distance via some strange thing called ``entanglement,'' \JB*{but
% this can be explained by hidden variables}\JB{which has revolutionized
% our understanding of local realism.} \JB[@Albert, we talked about
% this...]
% \end{verbatim}
% which gives:
%
% {\itshape%
% Quantum mechanics developed quickly in the 1920's thanks to the
% works of Schr\"odinger, Heisenberg, and Dirac, who drew inspiration
% from earlier ideas by Planck \AlE{and Einstein} \AlE[@JB, you forgot
% about me!].  Quantum mechanics appears to allow spooky action at a
% distance via some strange thing called ``entanglement,'' \JB*{but
% this can be explained by hidden variables}\JB{which has revolutionized
% our understanding of local realism.} \JB[@Albert, we talked about
% this...]%
% }
%
%
% \section{Syntax of commenting commands}
%
% Here is the full syntax of the commenting commands generated with
% |\phfMakeCommentingCommand|.  You can write:
%
%
% |\AlE{some text here}| --- \AlE{some text here} --- typeset the
% given text in color, perhaps to mark an addition by author |AlE| to
% the document.
%
% |\AlE[please revise this]| --- \AlE[please revise this] --- typeset an inline
% comment in a different font.  This is meant for comments that are not a
% portion of document text itself but rather an inline remark that refers to the
% text around it.
%
% |\AlE*{text to be removed}| --- \AlE*{text to be removed} --- typeset the
% given text in a way to indicate that the text should be suppressed.
%
% |\AlE!{important piece of text}| --- \AlE!{important piece of text}
% --- typeset the given text marking it as important.
%
% |\AlE![a really important comment]| --- \AlE![a really important comment] ---
% typeset the given comment and mark it as important.
%
% In each of these cases (except for comments in square brackets),
% you can use the syntax |\AlE .... \endAlE| instead of specifying the
% argument using curly brackets:\\
% |\AlE text here\endAlE| --- \AlE text here\endAlE \\
% |\AlE! important\endAlE| --- \AlE! important\endAlE \\
% |\AlE* remove\endAlE| --- \AlE* remove\endAlE
%
% For simple text, you can also use the |\begin|/|\end| syntax:\\
% |\begin{AlE} text here\end{AlE}| --- \begin{AlE} text here\end{AlE}
%
% \begin{pkgtip}
% Note that in contrast to optional
% arguments in most LaTeX macros, the argument is allowed to contain matching
% square brackets, for instance \relax%
% |\AlE[this works [as you'd expect]]| \AlE[this works [as you'd expect]]
%
% In any case you can protect the argument as usual with curly braces:
% |\AlE[{this is [weird}]| \AlE[{this is [weird}]
% \end{pkgtip}
%
% \section{Defining your commenting command}
%
% Now let's see exactly how to define your customized |\AlE| command.
% \DescribeMacro{\phfMakeCommentingCommand} The command
% |\phfMakeCommentingCommand| takes the following syntax:
%
% |\phfMakeCommentingCommand|\oarg{key=value,key2=value2,...}\marg{cmd name}
%
% The \marg{command name} should be given as a text name (e.g.,
% |{AlE}|), and not as a macro.  They key-value pairs may be one of the following:
%
%
% \phfMakeCommentingCommand[color=phfcolor,initials={PhF},formatinitials=box]{phfbox}
% \phfMakeCommentingCommand[color=phfcolor,initials={PhF},formatinitials=nobox]{phfnobox}
% \phfMakeCommentingCommand[color=phfcolor,initials={PhF},formatinitials=hide]{phfhideinitials}
% \phfMakeCommentingCommand[color=green!50!blue,initials={Philippe Faist},formatinitials=footnote]{phffootnote}
%
% \begin{cmdoptions}
% \item[color=\marg{color specification}] The color of the comments generated by
%   the new commenting command.  By default, a new suitable color is chosen for
%   each new commenting command defined.
%
%   The color specification may be a name such as |red|, |green|, etc., but also
%   a mixture like |blue!40!green| (\textcolor{blue!40!green}{which stands for
%   40\% of blue and 60\% of green}).  You can specify any argument you could
%   specify to |\colorlet| from the \pkgname{xcolor} package.  (The
%   \pkgname{xcolor} package is automatically loaded.)
%
% \item[initials=\marg{your initials}] Your name initials, which will by typeset
%   at the beginning of a comment (typically in a small box in the margin),
%   allowing to identify different comment colors with different people.  By
%   default, the initials are set to the name of the command itself.
%
% \item[formatinitials=\meta{keyword or command}] The command that will be used
%   to format the initials.  By default (|formatinitials=default| or
%   |formatinitials=margin|), the initials are displayed on the margin of the
%   paragraph the first time the change occurs on a given page \phf[like this
%   and like in the examples above].  You may also specify |formatinitials=box|
%   to typeset the initials inline immediately before your annotation
%   \phfbox[like this].  Specify |formatinitials=nobox| to remove the frame
%   \phfnobox[like this].  Or specify |formatinitials=hide| to hide the initials
%   altogether \phfhideinitials[like this].  You can also specify
%   |formatinitials=footnote| to hide the initials at the point of text but with
%   a footnote to associate colored comments with the name (for this to work
%   more nicely, specify your full name as argument to |initials={...}|)
%   \phffootnote[Here is an example].  [In the latter case, you may redefine
%   |\phfCCChangesBy|, e.g., \relax%
%   |\renewcommand{\phfCCChangesBy}{Changes in this color are by\ }| to change
%   the text in the footnote.]
%
%   You may also specify any \LaTeX{} command, which will then be used to format
%   the initials.
%
%   The default initials formatting style (|formatinitials=margin|) requires the
%   \pkgname{marginnote} package, which is loaded automatically.  If you do not
%   want to load the \pkgname{marginnote} package, specify the package option
%   |usemarginnote=false| and set a different |formatinitials=...|
%   style.
% \item[cfont=\marg{LaTeX commands}] The font commands to invoke when
%   typesetting comment. By default, use \phf[the default sans serif typeface,
%   like this].
% \item[cspacing=\meta{length}] Spacing to add around comments. This amount of
%   horizontal space is added before and after the comment.
% \item[cbegin=\marg{LaTeX text and/or commands}] Stuff to typeset at
%   the beginning of a commment.  By default, an opening square
%   bracket (with some spacing adjustments).
% \item[cend=\marg{LaTeX text and/or commands}] Stuff to typeset at
%   the end of a commment.  By default, a closing square bracket (with
%   some spacing adjustments).
% \item[rmfont=\marg{LaTeX commands}] The font commands to issue when
%   typesetting text that is to be removed. \phf*{By default, the text
%   is in italics like this}
% \item[rmspacing=\meta{length}] Spacing to add around text that is marked to be
%   removed. This amount of horizontal space is added before and after.
% \item[rmbegin=\marg{LaTeX text and/or commands}] Stuff to typeset at
%   the beginning of a piece of text to be removed.  By default, three
%   crosses.
% \item[rmend=\marg{LaTeX text and/or commands}] Stuff to typeset at
%   the end of a piece of text to be removed.  By default, three
%   crosses.
% \item[ifont=\marg{LaTeX commands}] The font commands to issue when
%   typesetting important text. \phf!{By default, the text is in
%   larger boldface format like this}
% \item[ispacing=\meta{length}] Spacing to add around important text. This
%   amount of horizontal space is added before and after.
% \item[ibegin=\marg{LaTeX text and/or commands}] Stuff to typeset at
%   the beginning of a piece of important text.  By default, three
%   exclamation marks.
% \item[iend=\marg{LaTeX text and/or commands}] Stuff to typeset at
%   the end of a piece of important text.  By default, three
%   exclamation marks.
% \end{cmdoptions}
% 
% After invoking |\phfMakeCommentingCommand|, the colors |xxxcolor|,
% |xxxrmcolor| and |xxxrmcolorlink| are automatically defined (where |xxx| is to
% be replaced by your commenting macro name).  These are used respectively for
% a usual comment color, text to be removed, and for links in text to be
% removed.  You may redefine these colors afterwards with |\colorlet| if you
% wish:
% \begin{verbatim}
% \colorlet{AlErmcolor}{gray}
% \end{verbatim}
%
% \DescribeMacro{\phfDisableCommentingCommands}
% The command |\phfDisableCommentingCommands| disables all commenting commands,
% and causes them to emit an error.  Use this command when approaching the final
% version of a long document to ensure that no commenting commands are left in
% the document.
%
% \section{Package Options}
%
% This package requires the \pkgname{marginnote} package to generate initials
% labels when using the |margin| initials style (the default style).  If you do
% not want to load the \pkgname{marginnote} package, you may use the package
% option |usemarginnote=false| (but then you cannot use the |margin| initials
% style):
% \begin{verbatim}
% \usepackage[usemarginnote=false]{phfcc}
% \end{verbatim}
%
% \section{Known limitations}
%
% If you place a comment inside a contruct that is typeset twice (e.g., an AMS
% equation, or a figure caption when using e.g. the \pkgname{caption} package,
% etc.), then the margin initials will not appear even if the comment is the
% first of the page.  This is because the margin label is produced the first
% time they are typeset, per page.  There's not much I can do about this, and
% unless there is a quick \& easy fix I don't intend to spend too much effort
% into trying to hack around this.
%
%
% \StopEventually{\PrintChangesAndIndex}
%
% \section{Implementation}
%
% Load these internally required packages.
%
%    \begin{macrocode}
\RequirePackage{xkeyval}
\RequirePackage{kvoptions}
\RequirePackage{etoolbox}
\RequirePackage{xparse}
%    \end{macrocode}
%
% Ensure we have the \pkgname{xcolor} package to manage text colors.  We need
% \pkgname{xcolor} and not \pkgname{color} because we use |\colorlet|.
%    \begin{macrocode}
\RequirePackage{xcolor}
%    \end{macrocode}
% 
% \begin{macro}{\phfCommentingDefault*}
% Provide sensible defaults for commenting formatting.
%
% Bold or semibold CFont sounds like a good idea, but comments would seem more
% aggressive like that, so keep them normal by default.
%    \begin{macrocode}
\def\phfCommentingDefaultCFont{\sffamily}
\def\phfCommentingDefaultCSpacing{0.2em}
\def\phfCommentingDefaultCBegin{[\,}
\def\phfCommentingDefaultCEnd{\,]}
\def\phfCommentingDefaultRmFont{\small\itshape}%\itshape\notesmaller[0.9]}
\def\phfCommentingDefaultRmSpacing{1em}
\def\phfCommentingDefaultRmBegin{\ensuremath{{\times}\!{\times}}~}
\def\phfCommentingDefaultRmEnd{~\ensuremath{{\times}\!{\times}}}
\def\phfCommentingDefaultIFont{\large\bfseries}
\def\phfCommentingDefaultISpacing{0.25em}
\def\phfCommentingDefaultIBegin{!\hspace*{0.1em}!\hspace*{0.1em}!~}
\def\phfCommentingDefaultIEnd{~!\hspace*{0.1em}!\hspace*{0.1em}!}
%    \end{macrocode}
% \end{macro}
%
% A default list of colors for commenting, and a minimal tool to select the next
% available color.  The macro |\phf@cc@usedcolors| stores a
% vertical-bar-separated list of color names that have been already used and
% which should not be used again.
%    \begin{macrocode}
\csdef{phf@cc@presetcolor@0}{red!25!orange}
\csdef{phf@cc@presetcolor@1}{blue}
\csdef{phf@cc@presetcolor@2}{purple}
\csdef{phf@cc@presetcolor@3}{blue!50!green}
\csdef{phf@cc@presetcolor@4}{violet}
\csdef{phf@cc@presetcolor@5}{green!60!black}
\csdef{phf@cc@presetcolor@6}{red}
\csdef{phf@cc@presetcolor@7}{orange!70!yellow}
\def\phf@cc@usedcolors{|}
\def\phf@cc@nextcolor@#1{%
  \ifcsname phf@cc@presetcolor@#1\endcsname% try preset
    \edef\phf@tmp@testxx{%
      \noexpand\in@{|\csname phf@cc@presetcolor@#1\endcsname|}{\phf@cc@usedcolors}}%
    \phf@tmp@testxx
    \ifin@% color already used, try next
      \expandafter\phf@cc@nextcolor@\expandafter{\the\numexpr#1+1\relax}%
    \else
      \def\phf@cc@thecolor{\csname phf@cc@presetcolor@#1\endcsname}% good, use
    \fi
  \else% out of colors, fallback to red
    \def\phf@cc@thecolor{red}%
  \fi
}
%    \end{macrocode}
% Read the argument.  If non-empty, set |\phf@cc@thecolor| to the argument
% value.  If empty, choose the next available preset color and set
% |\phf@cc@thecolor| to that.
%    \begin{macrocode}
\def\phf@cc@getcolor#1{%
  \edef\phf@tmp@xyz{#1}%
  \if\relax\detokenize\expandafter{\phf@tmp@xyz}\relax
    \phf@cc@nextcolor@{0}%
  \else
    \edef\phf@cc@thecolor{#1}%
  \fi}
%    \end{macrocode}
%
%
% Define the keys for |\setkeys| with \pkgname{xkeyval} package, and set the
% overall defaults.
%    \begin{macrocode}
\define@cmdkey{phfmkcc}{color}{}
\define@cmdkey{phfmkcc}{initials}{}
\define@cmdkey{phfmkcc}{formatinitials}{}
\define@cmdkey{phfmkcc}{cfont}{}
\define@cmdkey{phfmkcc}{cspacing}{}
\define@cmdkey{phfmkcc}{cbegin}{}
\define@cmdkey{phfmkcc}{cend}{}
\define@cmdkey{phfmkcc}{rmfont}{}
\define@cmdkey{phfmkcc}{rmspacing}{}
\define@cmdkey{phfmkcc}{rmbegin}{}
\define@cmdkey{phfmkcc}{rmend}{}
\define@cmdkey{phfmkcc}{ifont}{}
\define@cmdkey{phfmkcc}{ispacing}{}
\define@cmdkey{phfmkcc}{ibegin}{}
\define@cmdkey{phfmkcc}{iend}{}
\presetkeys{phfmkcc}{%
  color={},%
  initials={},%
  formatinitials={default},%
  cfont={\phfCommentingDefaultCFont},%
  cspacing={\phfCommentingDefaultCSpacing},%
  cbegin={\phfCommentingDefaultCBegin},%
  cend={\phfCommentingDefaultCEnd},%
  rmfont={\phfCommentingDefaultRmFont},%
  rmspacing={\phfCommentingDefaultRmSpacing},%
  rmbegin={\phfCommentingDefaultRmBegin},%
  rmend={\phfCommentingDefaultRmEnd},%
  ifont={\phfCommentingDefaultIFont},%
  ispacing={\phfCommentingDefaultISpacing},%
  ibegin={\phfCommentingDefaultIBegin},%
  iend={\phfCommentingDefaultIEnd}%
}{}
%    \end{macrocode}
%
% In the series of defaults, we also have the predefined initials formatting.
% Define the |hide|, |box|, and |nobox| styles first.
%    \begin{macrocode}
\def\phf@cc@formatinitialsstyle@box#1{%
  \hspace{0.25em}\relax
  \raisebox{1pt}{{\fboxsep=1pt\relax\fbox{\normalfont\sffamily\tiny#1}}}%
  \hspace{0.25em}}
\def\phf@cc@formatinitialsstyle@nobox#1{%
  \hspace{0.25em}\relax
  {\normalfont\sffamily\tiny#1}%
  \hspace{0.25em}}
\def\phf@cc@formatinitialsstyle@hide#1{}
%    \end{macrocode}
%
% Define the |margin| style now.  Define an auxiliary global macro to remember
% whenever we've already displayed the margin note on a given page.
%
% Strategy: use |\marginnote| all the time.  It would be nice to use
% |\marginpar| in normal paragraph text so that label don't overlap etc.,
% because we can't detect all cases where |\marginpar| is
% problematic. (|\ifinner| detects minipages/equations, value of |\@captype|
% detects figure captions, but I just can't detect footnotes, and I anticipate
% other problematic situations coming up).
%    \begin{macrocode}
\def\phf@cc@formatinitialsstyle@margin#1{%
%    \end{macrocode}
% 
% If the label was already displayed on this page (as witnessed by the following
% test macro being defined), don't repeat the label display and skip altogether.
%    \begin{macrocode}
  \ifcsname phf@cc@valfmtinitialsmargin@intlspage@\phf@cc@val@cur @\roman{page}\endcsname
  \else
%    \end{macrocode}
%
% Now we prepare the arguments to |\marginnote|/|\marginpar|, whichever we'll
% end up using.  In the ``argument tokens'' (but not in the actual arguments to
% |\marginpar|) we also include the command to register that we've displayed the
% label on this page.  Don't do it right away, because we might still abort
% displaying the label at this time (e.g. if in a figure caption and the user
% doesn't want to use |\marginnote|).
%    \begin{macrocode}
    %%\def\@captype{}\def\ftype@{}% fix for revtex (?!?)
    \edef\phf@cc@tmpxzxxwa{\unexpanded{\strut\color}{\phf@cc@val@cur color}%
      \unexpanded{\fboxsep=1pt\relax\fbox{\normalfont\sffamily\footnotesize #1}}}%
    \edef\phf@cc@tmp@marginparargtoks{%
      [{\noexpand\raggedleft\expandonce\phf@cc@tmpxzxxwa}]%
      {\noexpand\raggedright\expandonce\phf@cc@tmpxzxxwa}%
      \noexpand\csgdef{phf@cc@valfmtinitialsmargin@intlspage@\phf@cc@val@cur%
                       @\roman{page}}{1}%
    }%
    % 
%    \end{macrocode}
% Decide whether to use |\marginpar| or |\marginnote|.  We could have some
% complex decision process, but only if we can find a really reliable algorithm
% that won't generate LaTeX errors when collaborators want to comment on a
% document in a final hours before a deadline.  So for now, universally use
% |\marginnote|.  If you don't want to load |\marginnote|, use a different
% initials label style.
%    \begin{macrocode}
    \phf@cc@domarginnote\phf@cc@tmp@marginparargtoks%
  \fi
}
\robustify\phf@cc@formatinitialsstyle@margin
\def\phf@cc@domarginnote#1{%
  \edef\x{\noexpand\protect\noexpand\marginnote\expandonce#1}%
  \x
}
%    \end{macrocode}
%
% Define the |margin| initials style as the default:
%    \begin{macrocode}
\let\phf@cc@formatinitialsstyle@default\phf@cc@formatinitialsstyle@margin
%    \end{macrocode}
%
% Define the |footnote| style. Works very similarly to the |footnote| style.
% Note you can redefine |\phfCCChangesBy| to whatever you please, and it can
% even take the name as |#1| argument.
%    \begin{macrocode}
\def\phf@cc@formatinitialsstyle@footnote#1{%
  \ifcsname phf@cc@valfmtinitialsfootnote@intlspage@\phf@cc@val@cur @\roman{page}\endcsname
    % already displayed on this page, don't repeat
  \else
    \csgdef{phf@cc@valfmtinitialsfootnote@intlspage@\phf@cc@val@cur @\roman{page}}{1}%
    \def\phf@cc@tmp@zz{\gdef\@thefnmark{}\@footnotetext}%
    \expandafter\phf@cc@tmp@zz\expandafter{\expandafter{%
        \expandafter\color\expandafter{\phf@cc@val@cur color}\phfCCChangesBy{#1}}}%
  \fi
}
\robustify\phf@cc@formatinitialsstyle@footnote
\def\phfCCChangesBy{Changes by\ }
%    \end{macrocode}
%
% \begin{macro}{\phfMakeCommentingCommand}
%   Implementation of |\phfMakeCommentingCommand|.  See main package
%   documentation for usage details. The inner command definitions are kinda
%   tricky because of all the possible syntax options.
%   |\phfMakeCommentingCommand|\oarg{color=blue,...}\marg{cmdname}
%    \begin{macrocode}
\newcommand\phfMakeCommentingCommand[2][]{%
  \setkeys{phfmkcc}{#1}%
  \phf@cc@getcolor{\cmdKV@phfmkcc@color}%
  \colorlet{#2color}{\phf@cc@thecolor}%
  \edef\phf@tmp@xxappcmd{\noexpand\appto\noexpand\phf@cc@usedcolors{\phf@cc@thecolor|}}%
  \phf@tmp@xxappcmd
  \colorlet{#2rmcolor}{#2color!70!gray!55!white}%
  \colorlet{#2rmcolorlink}{blue!40!#2rmcolor}%
  \edef\phf@tmp@xxx{\cmdKV@phfmkcc@initials}%
  \if\relax\detokenize\expandafter{\phf@tmp@xxx}\relax
    \csedef{phf@cc@val@#2@initials}{#2}%
  \else
    \csedef{phf@cc@val@#2@initials}{\cmdKV@phfmkcc@initials}%
  \fi
  \ifcsname phf@cc@formatinitialsstyle@%
      \detokenize\expandafter{\cmdKV@phfmkcc@formatinitials}\endcsname%
    \expandafter\def\expandafter\phf@cc@tmp@xzx\expandafter{%
      \csname phf@cc@formatinitialsstyle@%
          \detokenize\expandafter{\cmdKV@phfmkcc@formatinitials}\endcsname}%
    \csedef{phf@cc@val@#2@formatinitials}{\expandonce\phf@cc@tmp@xzx}%
  \else
    \csedef{phf@cc@val@#2@formatinitials}{\expandonce\cmdKV@phfmkcc@formatinitials}%
  \fi
  \csedef{phf@cc@val@#2@cfont}{\expandonce\cmdKV@phfmkcc@cfont}%
  \csedef{phf@cc@val@#2@cspacing}{\expandonce\cmdKV@phfmkcc@cspacing}%
  \csedef{phf@cc@val@#2@cbegin}{\expandonce\cmdKV@phfmkcc@cbegin}%
  \csedef{phf@cc@val@#2@cend}{\expandonce\cmdKV@phfmkcc@cend}%
  \csedef{phf@cc@val@#2@rmfont}{\expandonce\cmdKV@phfmkcc@rmfont}%
  \csedef{phf@cc@val@#2@rmspacing}{\expandonce\cmdKV@phfmkcc@rmspacing}%
  \csedef{phf@cc@val@#2@rmbegin}{\expandonce\cmdKV@phfmkcc@rmbegin}%
  \csedef{phf@cc@val@#2@rmend}{\expandonce\cmdKV@phfmkcc@rmend}%
  \csedef{phf@cc@val@#2@ifont}{\expandonce\cmdKV@phfmkcc@ifont}%
  \csedef{phf@cc@val@#2@ispacing}{\expandonce\cmdKV@phfmkcc@ispacing}%
  \csedef{phf@cc@val@#2@ibegin}{\expandonce\cmdKV@phfmkcc@ibegin}%
  \csedef{phf@cc@val@#2@iend}{\expandonce\cmdKV@phfmkcc@iend}%
%    \end{macrocode}
%
% Once all the information was gathered and stored, define the commenting command
% itself (as well as its |\end*| counterpart).  Make them robust.
%    \begin{macrocode}
  \csdef{#2}{\phf@cc@do{#2}}%
  \expandafter\robustify\csname #2\endcsname%
  \csdef{end#2}{\phf@cc@end}%
  \expandafter\robustify\csname end#2\endcsname%
}
%    \end{macrocode}
% \end{macro}
%
% \begin{macro}{\phfDisableCommentingCommands}
%   This command causes any commenting command declared with
%   |\phfMakeCommentingCommand| to emit an error.
%    \begin{macrocode}
\newif\ifphf@cc@disabled
\phf@cc@disabledfalse
\def\phfDisableCommentingCommands{%
  \phf@cc@disabledtrue
}
%    \end{macrocode}
% \end{macro}
%
% Overall, the implementation of a commenting command goes as follows.  First,
% gather all the info about the command invocation (e.g., important? starred?
% comment?) and then ultimately call |\phf@cc@begin| and |\phf@cc@end| which
% execute the formatting options.
%
% Everything happens within |\begingroup|\ldots|\endgroup|.  The group is opened
% at |\phf@cc@do| (the commenting command itself), and is closed by |\phf@cc@end|.
%
% The macros |\phf@cc@val@grp*cmds| store the commands to be executed in
% |\phf@cc@begin| and |\phf@cc@end|.  By default, they are empty.  They are
% filled by relevant code when parsing the command invocation.
%    \begin{macrocode}
\def\phf@cc@val@grpcmds{}% by default no commands at group begin
\def\phf@cc@val@grpendcmds{}% by default no commands at group end
\def\phf@cc@val@spacing{0pt}% no spacing by default
%    \end{macrocode}
%
% \begin{macro}{\phf@cc@begin}
% Here we set the comment color and execute other opening commands stored
% earlier in the parsing stage. Observe that |\begingroup| \textbf{has already
% been issued} by |\phf@cc@do|.
%    \begin{macrocode}
\def\phf@cc@begin{%
  \color{\phf@cc@val@cur color}%
  \hspace{\phf@cc@val@spacing}%
  \phf@cc@showinitials%
  \phf@cc@val@grpcmds}
%    \end{macrocode}
% \end{macro}
% \begin{macro}{\phf@cc@end}
% Here we execute closing commands and close the group. |\phf@cc@end| is the
% last macro executed when the user invokes a comment, in any case.
%    \begin{macrocode}
\def\phf@cc@end{%
  \phf@cc@val@grpendcmds
  \hspace{\phf@cc@val@spacing}%
  \endgroup
}
%    \end{macrocode}
% \end{macro}
% 
% \begin{macro}{\phf@cc@showinitials}
% This helper macro prints out the initials, if any.
%    \begin{macrocode}
\def\phf@cc@showinitials{%
  \edef\phf@cc@tmpx{%
    \expandafter\expandonce\csname phf@cc@val@\phf@cc@val@cur @initials\endcsname}%
  \expandafter\notblank\expandafter{\phf@cc@tmpx}{%
    \csname phf@cc@val@\phf@cc@val@cur @formatinitials%
        \expandafter\endcsname\expandafter{\phf@cc@tmpx}}{}%
}
%    \end{macrocode}
% \end{macro}
%
% \begin{macro}{\phf@cc@do}
%   The user-instantiated commenting commands are an alias of this command, with
%   the user macro name as first argument.  Here's were everything starts.
%    \begin{macrocode}
\def\phf@cc@do#1{%
%    \end{macrocode}
%   First, we need to check if commenting commands were disabled with
%   |\phfDisableCommentingCommands|.  If this is the case, generate an error.
%    \begin{macrocode}
  \ifphf@cc@disabled
    \PackageError{phfnote}{Commenting commands have been disabled
      with \string\phfDisableCommentingCommands.}{}%
  \fi
%    \end{macrocode}
%   Now we start for real.  First, we open the \TeX\ group. Then we start
%   parsing the invocation syntax.
%    \begin{macrocode}
  \leavevmode\begingroup%
  \edef\phf@cc@val@cur{#1}%
  \@ifstar\phf@cc@star\phf@cc@nostar% star?
}
\def\phf@cc@nostar{\@ifnextchar!\phf@cc@important\phf@cc@grp}
%    \end{macrocode}
% At this point, we have determined whether the invocation is starred or important.
% The next execution points are |\phf@cc@star|, |\phf@cc@important|, or |\phf@cc@grp|.
%
% If starred or if important, set up the inner group execution commands to the
% corresponding settings, and defer to |\phf@cc@grp|.
%    \begin{macrocode}
\long\def\phf@cc@star{%
  \csname phf@cc@val@\phf@cc@val@cur @rmfont\endcsname%
  \color{\phf@cc@val@cur rmcolor}%
  \def\phf@cc@val@grpcmds{%
    \color{\phf@cc@val@cur rmcolor}%
    \colorlet{docnotelinkcolor}{\phf@cc@val@cur rmcolorlink}%
    \csname phf@cc@val@\phf@cc@val@cur @rmbegin\endcsname%
  }%
  \def\phf@cc@val@grpendcmds{%
    \csname phf@cc@val@\phf@cc@val@cur @rmend\endcsname%
  }%
  \def\phf@cc@val@spacing{%
    \csname phf@cc@val@\phf@cc@val@cur @rmspacing\endcsname%
  }%
  \phf@cc@grp}
\long\def\phf@cc@important!{%
  \csname phf@cc@val@\phf@cc@val@cur @ifont\endcsname%
%    \end{macrocode}
% Here we need a little hack, we want the important font to take precedence
% over any comment font.
%    \begin{macrocode}
  \appto\phfCommentingDefaultCFont{%
    \csname phf@cc@val@\phf@cc@val@cur @ifont\endcsname%
  }%
  \def\phf@cc@val@grpcmds{%
    \csname phf@cc@val@\phf@cc@val@cur @ibegin\endcsname%
  }%
  \def\phf@cc@val@grpendcmds{%
    \csname phf@cc@val@\phf@cc@val@cur @iend\endcsname%
  }%
  \def\phf@cc@val@spacing{%
    \csname phf@cc@val@\phf@cc@val@cur @ispacing\endcsname%
  }%
  \phf@cc@grp}
%    \end{macrocode}
% 
% The command |\phf@cc@grp| parses how the argument content is given. Is it
% a comment (``|[ ... ]|''), a single \LaTeX{} argument group (``|{ ... }|''),
% or should we expect it to end with |\end*|?
%    \begin{macrocode}
\def\phf@cc@grp{%
  \@ifnextchar[\phf@cc@comment\phf@cc@grpnocomment%]
}
\def\phf@cc@grpnocomment{%
  \@ifnextchar\bgroup{\phf@cc@grpwarg}{\phf@cc@begin}}
%    \end{macrocode}
%
% Now we determined how the argument is given, and we are calling one
% of |\phf@cc@comment|, |\phf@cc@grpwarg| (with argument), or
% |\phf@cc@begin| directly.
%    \begin{macrocode}
\long\def\phf@cc@grpwarg#1{\phf@cc@begin #1\phf@cc@end}
%    \end{macrocode}
%
% If the argument is a comment, set up the comment font, begin/end
% chars, and go.
%
% \changed[chg:phf-comment-accepts-long-comment-arg]{v3.0}{2020/01/24}{Comment
% command now accepts long argument when used in the form \phfverb{\phf}[...]}
%    \begin{macrocode}
\NewDocumentCommand\phf@cc@comment{+O{}}{%
  \normalfont%
  \csname phf@cc@val@\phf@cc@val@cur @cfont\endcsname%
  \preto\phf@cc@val@grpcmds{%
    \csname phf@cc@val@\phf@cc@val@cur @cbegin\endcsname%
  }%
  \appto\phf@cc@val@grpendcmds{%
    \csname phf@cc@val@\phf@cc@val@cur @cend\endcsname%
  }%
  \def\phf@cc@val@spacing{%
    \csname phf@cc@val@\phf@cc@val@cur @cspacing\endcsname%
  }%
  \phf@cc@begin%
  %\phf@cc@showinitials%
  #1%
  \phf@cc@end%
}
%    \end{macrocode}
% \end{macro}
%
%
% Initialization code for \pkgname{kvoptions} for our package options.
%    \begin{macrocode}
\SetupKeyvalOptions{
   family=phfcc,
   prefix=phfcc@opt@
}
\DeclareBoolOption[true]{usemarginnote}
%    \end{macrocode}
% 
% Process package options.
%    \begin{macrocode}
\ProcessKeyvalOptions*
%    \end{macrocode}
%
% Load the \pkgname{marginnote} unless requested not to.  This package is
% required for the |margin| initials style.
%    \begin{macrocode}
\ifphfcc@opt@usemarginnote
  \RequirePackage{marginnote}
\fi
%    \end{macrocode}
% 
%\Finale
\endinput
